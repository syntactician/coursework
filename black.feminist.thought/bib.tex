\documentclass[doc,12pt]{apa6}
\usepackage[colorlinks=false]{hyperref}
\usepackage{lmodern,amssymb,amsmath,apacite}
\linespread{1.5}

\begin{document}

\title{Preliminary Bibliography}
\shorttitle{Bibliography}
\author{Edward Hern\'{a}ndez}
\date{10 November}
\affiliation{College of William \& Mary}
\maketitle

\everypar{\hangindent2em \hangafter1}

\noindent
Davies, C. B. (2008). Left of Karl Marx: The political life of Black Communist Claudia Jones. Duke University Press.
\nocite{Davies08}

\noindent
Davies, C. B. (Ed.). (2011). Claudia Jones: Beyond containment. Ayebia Clarke Publishing.
\nocite{Davies11}

\noindent
Howard, W. T. (2013). We shall be free!: Black Communist protests in seven voices. Temple University Press.
\nocite{Howard13}

\noindent
Johnson, B. (1984). I think of my mother: Notes on the life and times of Claudia Jones. Karia Press.
\nocite{Johnson84}

\noindent
Jones, C. (1949a). An end to the neglect of the problems of the Negro woman! Political Affairs.
\nocite{Jones49a}

\noindent
Jones, C. (1949b). We seek full equality for women.
\nocite{Jones49b}

\noindent
McDuffie, E. S. (2011). Sojourning for freedom: Black women, American Communism, and the making of Black left feminism. Duke University Press.
\nocite{McDuffie11}

\noindent
O’Brien, K. (2014). Uncovering Black Marxist feminism. International Socialist Review, 90.
\nocite{OBrien14}

\noindent
Olende, K. (2014, January). Intersectionality and black communist women. International Socialism, 141.
\nocite{Olende14}

\noindent
Weigand, K. (2001). Red feminism: American Communism and the making of women’s liberation. Baltimore, MD: Johns Hopkins University Press.
\nocite{Weigand01}

% \nocite{Davies08,Davies11,Howard13,Johnson84,Jones49a,Jones49b,McDuffie11,Obrien14,Olende14,Washinton03,Weigand01}

% \clearpage
% \bibliography{black.feminist.thought,final}
% \bibliographystyle{apacite}

\end{document}
