\documentclass[doc,12pt]{apa6}
\usepackage{lmodern}
\usepackage{amssymb,amsmath}
\usepackage{ifxetex,ifluatex}
\usepackage{fixltx2e} % provides \textsubscript
\usepackage[T1]{fontenc}
\usepackage[utf8]{inputenc}
  
\usepackage[usenames,dvipsnames]{color}

\usepackage{graphicx,grffile}
\graphicspath{ {.resources/} }
\makeatletter
\def\maxwidth{\ifdim\Gin@nat@width>\linewidth\linewidth\else\Gin@nat@width\fi}
\def\maxheight{\ifdim\Gin@nat@height>\textheight\textheight\else\Gin@nat@height\fi}
\makeatother
% Scale images if necessary, so that they will not overflow the page
% margins by default, and it is still possible to overwrite the defaults
% using explicit options in \includegraphics[width, height, ...]{}
\setkeys{Gin}{width=\maxwidth,height=\maxheight,keepaspectratio}
\setlength{\parindent}{0pt}
\setlength{\parskip}{6pt plus 2pt minus 1pt}
\setlength{\emergencystretch}{3em}  % prevent overfull lines
\providecommand{\tightlist}{%
  \setlength{\itemsep}{0pt}\setlength{\parskip}{0pt}}
\setcounter{secnumdepth}{0}

\usepackage[colorlinks=false]{hyperref}

\begin{document}

\title{Final Project Proposal}
\shorttitle{Final Project Proposal}
\author{Edward Hern\'{a}ndez}
\date{29 October}
\affiliation{College of William \& Mary}
\maketitle

As we discussed in conference, I plan to write a traditional paper analyzing
Black Marxist feminist thought, using Claudia Jones' work as a starting point
for my research. I plan to compare this work to both white Marxism and feminism
and trace its effect on other Black feminist texts. I've read some Marxist
work, and my primary critique of the literature has been that it largely fails
to engage with the lived struggles of those oppressed along the axes of race
and gender. I think Black women are thus uniquely positioned to improve and
extend marxist critique, and from my preliminary reading, Claudia Jones seems
to do exactly that. She seems to write Marxist thought relevant to (liberating)
Black women. I would love to dedicate my research for this class to learning
more about her (and other Black Marxist feminists') work. I am most comfortable
with traditional academic papers, and, in engaging with a topic as new to me as
this one is, I think it would be a disservice to the topic and to me to try to
express it in some other medium.
%
% Consider the meaning of the phrase, "I plan to compare this work to both
% white Marxism and feminism and trace its effect on other Black feminist
% texts." and how it presumes influence/origins in white feminist texts on
% Black feminist texts. What are the intentions of such an assumption? Is it
% the most effective approach to exploring Black feminist texts?

\end{document}
