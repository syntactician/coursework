\documentclass[man,12pt,natbib]{apa6}
\usepackage[breaklinks=true,colorlinks=false]{hyperref}
\usepackage{times}

\begin{document}

\title{Claudia Jones: A Contested Legacy}
\shorttitle{Claudia Jones}
\author{Edward Hern\'{a}ndez}
\date{16 December 2015}
\affiliation{College of William \& Mary}
\maketitle

% Due at 1400 16 December 2015
\abstract{
	This paper will give a brief biography of Claudia Jones, attempt a summary
	of her theoretical work, and analyze her impact and the import of her
	positions.
}

\section{Introduction}

Claudia Jones was a Communist. That much seems clear from every
source. Past that, information is harder to come by than disagreement.
Much of her work is out of print. Many of her contributions to Black
Feminist thought and to Communism ``have simply disappeared from major
consideration in a range of histories'' \citep[Ch.~1]{Davies11}. Where
or with whom exactly she stood is hotly debated by those who bother to
write about her at all. Was she a revolutionary first? Black first?
Were her politics in line with Marx', or is her ideology separate
from, ``left of,'' \citep{Davies11} Marx?  To answer these questions,
some exploration of Jones' biography is necessary.

This paper will sketch Jones' biography insofar as it relates to
understanding her ideological positions and relations to other writers and
activists of other ideological bents. It will then use this biographical data
to place her ideology as precisely as possible, giving arguments against both
major camps of biographers and historians who address Jones and her legacy.

\section{Biography}

Claudia Jones was born Claudia Vera Cumberbatch, to Charles Bertrand
Cumberbatch and Sybil (Minnie Magdalene) Cumberbatch n{\'{e}}e Logan, 21
February 1915 in Belmont, Port-of-Spain, Trinidad
\citep[Chronology~section]{Davies08}. In 1924 when she was eight years old, she
emigrated to New York with her parents, sisters, and aunt. Her mother died of
meningitis in 1933, while Jones was still in high school. The next year, Jones
contracted tuberculosis due to poor living conditions, spending almost a year
in a sanatorium. Despite her ill health, she graduated high school in 1935.

Although she was a bright student, racism, sexism, and anti-immigrant attitudes
prevented Jones from attending college of finding high-paying work. After high
school she started working jobs in laundry, factory, and retail jobs. She also
began to write, starting with a column, \emph{Claudia's Comments}, in a Harlem
Black nationalist newspaper.

After the Scottsboro Boys were accused of raping two White women on a train, 
Jones became more politically active, organizing support for them. This
connected her with the Communist Party of the United States (CPUSA), who were
the most outspoken supporters of the Scottsboro Boys at the time. Within a 
year, she had joined the Young Communist League USA and begun to write for the
\emph{Daily Worker} \citep{Davis15}. In the next year, she joined the editorial
staff. From 1936 to 1947, she wrote prolifically for Communist papers, editing
several, including the \emph{Weekly Review} and \emph{Spotlight}.

In 1948, Jones was arrested under the Immigration Act of 1918 and the Alien
Registration Act of 1940, which make her political activity illegal because
she is an immigrant. She was threatened with deportation to Trinidad, but 
are postponed because Trinidad's government was unwilling to accept her and
no one was willing testify against her. Throughout her deportation process, she
continued touring the US, giving speeches at Communist rallies.

In 1953, the government finally succeeded in convicting Jones under the new
Smith Act. She was sentenced to one year and one day of jail time, after which
she was deported to Great Britain. In the UK, she was welcomed by the Communist
Party of Great Britain (CPGB) including members of the CPUSA who where deported
before her.

The CPGB never offered her a paid position the way the CPUSA had. Jones
remained something of an outsider in the party, despite the record of her work
in the US. Instead of publishing in the existing Communist papers, she founded
an anti-racist paper, the \emph{West Indian Gazette and Afro-Caribbean News},
in 1958 which she edited and managed until her death in 1964.

\section{Writings}

Jones wrote prolifically throughout her life, especially during the 30s and
40s.  Unfortunately, much of that writing appears to be lost, or in archives of
papers which no longer publish. This makes her work extremely difficult to
find.\footnote{Quite a lot of her writing has been curated and republished by
	\citet{Davies11}, but I was unable to get my hands on a copy of this book.}
This paper, as a result, will focus on her positions as expressed in her 1949
papers, \emph{An end to the neglect of the problems of the Negro woman!}
\citep{Jones49a} and \emph{We seek full equality for women} \citep{Jones49b}.

In \emph{We seek full equality for women}, \citet{Jones49b} discusses the
oppression of women. She cites Engels' writings on gender, establishing that
gender is an axis along which people are materially oppressed and their labor
is exploited. She calls for ``rooting'' the struggle for equality and the
actions of the CPUSA in the experience of women, claiming that Marxism in the
tool which will allow for the full equality of women---\emph{if} it is applied
correctly.

This work shows the way in which Jones writes as a Communist, but does not fall
squarely within the Marxist orthodoxy. She calls for the reformulation and
reapplication of Marxist thought and analysis, such that it includes women's
thought and fights for them explicitly.

\emph{An end to the neglect of the problems of the Negro woman!} \citep{Jones49a}
discusses Marxism-Leninism as it applies specifically to Black women. Her
analysis of the oppression which Black women face---as Black, as women, as
wokers---is groundbreaking for Marxist theory, which had historically suffered
from a problem of white universality, assuming that all oppression worked as it
did for white, male workers, as described by Marx \citep{Wilderson03}.

Jones' analysis of this ``triple oppression'' and the ``superexploitation''
\citep{Davies08} of Black women's labor is sometimes credited with paving the 
way for modern-day intersectional analysis. It is certainly unlike other
Marxist writings of the time. Jones' thought here is Black feminist thought
as much as---if not more than---it is classic Marxism.


\section{Legacy}

Claudia Jones is remembered quite differently by her various biographers. She
is called a Communist \citep{Lalkar, Davis15}, a Black Communist
\citep{Davies08, Howard13, McKittrick08}, a feminist \citep{Davis15,
	McClendon96}, a Socialist feminist \citep{Lynn14}, a Black Marxist feminist
\citep{OBrien14}, a Black feminist Marxist \citep{Johnson08}, an anti-racist
\citep{Davis15}, a Black nationalist \citep{McClendon96}, a journalist
\citep{Hinds08, McClendon96}, and a political activist \citep{McClendon96}.
While none of these are \emph{inaccurate} (with the possible exception of
`Black nationalist'), they reflect differences in opinion among her
biographers.

Jones seems to have identified herself quite distinctly as a Communist. At least,
she identified Marxism Leninism as ``the philosophy of [her] life''
\citep{Lalkar, Carter86}.  Her party membership, political activity, and 
publications seem to make her position as a Communist undeniable. Despite all this,
Jones is often argued to occupy an ideological position other than---separate
from---other contemporary Communists. \citet{Davies08} alludes to this
distinction in titling her biography, the most exhaustive one which exists,
\emph{Left of Marx}. They bolster their position by citing biographical
information about Jones' supposed exclusion from the CPGB, claiming that her fall
from grace in the party was due to an ideological split from Marxism.

Some Communists argue that making such a distinction, between the positions of
Marx and Jones, is unwarranted or even disingenuous. They point out that all
book-length biographies of her \citep{Davies08, Sherwood00} are written by
non-Communists. Some Communists explicitly characterize her biographers as
``anti-communists'' and black nationalists \citep{Lalkar}. \citet{Olende14}
accuses \citet{Davies08} and \citet{McDuffie11} of ``removing [her] from [her]
own tradition,'' and analyzing Jones' writings and political positions
ahistorically.

Essentially, these critiques are leveled by Socialists or Communists who argue 
that Marxism is not ``an unchanging block of ideas'' \citep{Olende14}, but rather
an evolving tradition comprised of various thinkers. On their view Jones does not
``move beyond Marx'' \citep{Olende14} by engaging with race and gender. Instead, 
she expands on Marx, and expands Marxism.

These positions---that Jones has ideas which are distinct from those of
\emph{Marx} and that her thought still falls squarely within
\emph{Marxism}---are not inherently contradictory.  However, it appears that
these two schools of thought have vested interests in characterizing Jones the
way they do. Communist writers tent to use Jones' affiliation with the
Communist Party to infer bad blood between her and the Black Nationalists
\citep{Lalkar, Olende14}.

Similarly, \citet{Sherwood00}, \citet{Davies08} and \citet{McDuffie11} seem to
want to paint the Communist party as unwelcoming of Jones (and of other women
and Black members---especially of Black women). That Black women were
relatively unwelcome is documented elsewhere, as well. \citet{Weigand01} in
particular discusses the discrimination and mistreatment faced by women in the
party at the time.

Regardless of whether Jones was well accepted among Communists or had bad blood
with Black nationalists, it is impossible to deny that she worked as a
Communist.  It is also undeniable that her ideas are not an exact copy of Karl
Marx'. Whether or not her work falls within the umbrella of Marxist thought 
depends, to a large extent, on what definition of Marxism is employed.
We may never know exactly what she thought on the subject.  What is clear,
however, is that she was shaped by Marx, but by more than Marx. She drew on
experience of triple oppression---both hers and her mothers---to write toward
the liberation of Black women. Her thought is Black feminist thought, through
and through.

\nocite{Azikiwe13}
\nocite{Haan13}
\nocite{Johnson84}
\nocite{Keith13}
\nocite{Mahamdallie04}
\nocite{Shabazz09}
\nocite{Taylor08}
\nocite{Thomson09}
\nocite{Washington03}

\clearpage
\bibliography{extra}

\end{document}
