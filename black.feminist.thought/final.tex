\documentclass[man,12pt,natbib]{apa6}
\usepackage[breaklinks=true,colorlinks=false]{hyperref}
\usepackage{times}
\linespread{1.5}

\begin{document}

\title{Claudia Jones: A Contested Legacy}
\shorttitle{Claudia Jones}
\author{Edward Hern\'{a}ndez}
\date{16 December 2015}
\affiliation{College of William \& Mary}
\maketitle

% Due at 1400 16 December 2015
\abstract{This paper will give a brief biography of Claudia Jones, attempt a
summary of her theoretical work, and analyze her impact and the import of her
positions.}

\section{Introduction}

Claudia Jones was a Communist. That much seems clear from every source. Past
that, information is harder to come by than disagreement.  Much of her work is
out of print. Many of her contributions to Black Feminist thought and to
Communism ``have simply disappeared from major consideration in a range of
histories.''\citep[Ch.~1]{Davies11} Where or with whom exactly she stood is
hotly debated by those who bother to write about her at all. Was she a
revolutionary first? Black first? Where her politics in line with Marx', or is
her ideology separate from, ``left of''\citep{Davies11} Marx?  To answer these
questions, some exploration of Jones' biography is necessary.

This paper will sketch Jones' biography insofar as it relates to understanding
her ideological positions and relations to other writers and activists of other
ideological bents. It will then use this biographical data to place her
ideology as precisely as possible, giving arguments against both major camps of
biographers and historians who address Jones and her legacy.

\section{Biography}

Claudia Jones was born Claudia Vera Cumberbatch, to Charles Bertrand
Cumberbatch and Sybil (Minnie Magdalene) Cumberbatch n{\'{e}}e Logan, 21
February 1915 in Belmont, Port-of-Spain, Trinidad
\citep[Chronology~section]{Davies08}. In 1924 when she was eight years old, she
emigrated to New York with her parents, sisters, and aunt.

\section{Writings}

Jones 

\section{Position}


\section{Legacy}

Claudia Jones is remembered quite differently by her various biographers. She
is called a Communist \cite{Lalkar, Davis15}, a Black Communist
\citep{Davies08, Howard13, McKittrick08}, a feminist \citep{Davis15,
McClendon96}, a Socialist feminist \cite{Lynn14}, a Black Marxist feminist
\citep{OBrien14}, a Black feminist Marxist \citep{Johnson08}, an anti-racist
\citep{Davis15}, a Black nationalist \citep{McClendon96}, a journalist
\citep{Hinds08, McClendon96}, and a political activist \citep{McClendon96}.
While none of these are \emph{inaccurate} (with the possible exception of
`Black nationalist'), they reflect differences in opinion among her
biographers.

Jones seems to have identified herself quite distinctly as a Communist. At least,
she identified Marxism Leninism as ``the philosophy of [her] life''
\citep{Lalkar, Carter86}.  Her party membership, political activity, and 
publications seem to make her position as a Communist undeniable. Despite all this,
Jones is often argued to occupy an ideological position other than---separate
from---other contemporary Communists. \citet{Davies08} alludes to this
distinction in titling her biography, the most exhaustive one which exists,
\emph{Left of Marx}.

Some Communists argue that making such a distinction, between the positions of
Marx and Jones, is unwarranted. They point out that all book-length biographies
of her \citep{Davies08, Sherwood00} are written by non-Communists. Some
Communists explicitly characterize her biographers as ``anti-communists''
\citep{Lalkar}.



\nocite{Azikiwe13}
\nocite{Davies08}
\nocite{Davies11}
\nocite{Davis15}
\nocite{Haan13}
\nocite{Hinds08}
\nocite{Howard13}
\nocite{Johnson08}
\nocite{Johnson84}
\nocite{Jones49a}
\nocite{Jones49b}
\nocite{Keith13}
\nocite{Lalkar}
\nocite{Lynn14}
\nocite{Mahamdallie04}
\nocite{McClendon96}
\nocite{McDuffie11}
\nocite{McKittrick08}
\nocite{OBrien14}
\nocite{Olende14}
\nocite{Shabazz09}
\nocite{Sherwood00}
\nocite{Taylor08}
\nocite{Thomson09}
\nocite{Washington03}
\nocite{Weigand01}

\clearpage
\bibliography{extra}

\end{document}
