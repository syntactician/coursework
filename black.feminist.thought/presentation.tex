\documentclass{beamer} % "Beamer" is a word used in Germany to mean video projector. 

\usetheme[everytitleformat=uppercase]{metropolis} % Search online for beamer themes to find your favorite or use the Berkeley theme as in this file.
\usecolortheme{seahorse}


\usepackage{color} % It may be necessary to set PCTeX or whatever program you are using to output a .pdf instead of a .dvi file in order to see color on your screen.
\usepackage{graphicx} % This package is needed if you wish to include external image files.
\usepackage[natbibapa]{apacite}

\theoremstyle{definition} % See Lesson Three of the LaTeX Manual for more on this kind of "proclamation."
\newtheorem*{dfn}{A Reasonable Definition}               

\title{{C}laudia {J}ones: A Legacy Contested}
\author{Edward Hern\'{a}ndez} 
\institute{}
\date{3 December 2015} 
% Remove the % from the previous line and change the date if you want a particular date to be displayed; otherwise, today's date is displayed by default.

% \AtBeginSection[]  % The commands within the following {} will be executed at the start of each section.
% {
% \begin{frame} % Within each "frame" there will be one or more "slides."  
% \frametitle{Presentation Outline} % This is the title of the outline.
% \tableofcontents[currentsection]  % This will display the table of contents and highlight the current section.
% \end{frame}
% } % Do not include the preceding set of commands if you prefer not to have a recurring outline displayed during your presentation.

\begin{document}

\begin{frame} 
\titlepage
\end{frame}

\begin{frame} 
	\frametitle{childhood}
	\begin{itemize}
		\item born Claudia Vera Cumberbatch in Trinidad
		\item emigrated to New York in the wake of the cocoa price crash
		\item poor living conditions in the US caused her to contract tuberculosis
	\end{itemize}

	\vfill
	\ \hfill \citep{Davies08}
\end{frame}

\begin{frame} 
	\frametitle{pre-political career}
	\begin{itemize}
		\item she was a good scholar, but as a black, immigrant woman in the '30s, her career options were limited
		\item worked at a laundry
		\item worked in retail
		\item after a few years she started writing for ``a Negro Nationalist newspaper'' \citep[ch.~2]{Davies08}
	\end{itemize}
\end{frame}

\begin{frame} 
	\frametitle{Communism}
	\begin{itemize}
		\item the Scottsboro Boys were Black teenagers accused of raping two white women on a train
		\item Claudia was looking for organizations that were supporting them, and found the Young Communist League USA
		\item within a year, she was on the editorial staff of the \emph{Daily Worker} and the \emph{Weekly Review}
		\item became an elected official of the Communist Party USA (CPUSA), writing, organizing and speaking at events
		\item relatively soon after joining up with Communists, she took on the name Jones as ``self-protective disinformation'' \citep{Taylor08P}
	\end{itemize}

	\vfill
	\ \hfill \citep{Davis15P}
\end{frame}

\begin{frame} 
	\frametitle{US writings}
	\begin{itemize}
		\item Jones wrote and edited Communist publications in the late '30s and throughout the '40s
		\item her most famous works appeared in \emph{Political Affairs}, including ``An End to the Neglect of the Problems of the Negro Woman!'' \citep{Jones49a}
		\item these works focus on the particular situation of Black women, which was often overlooked or misunderstood by contemporary Communists
	\end{itemize}

	\vfill
	\ \hfill \citep{Davies08}
\end{frame}

\begin{frame} 
	\frametitle{deportation}
	\begin{itemize}
		\item the Smith and McCarran Acts made Communist activity illegal
		\item Jones was arrested as early as 1948 (before the passage of the McCarran act), and threatened with deportation
		\item Trinidad refused her entry, making the deportation process complicated
		\item eventually she was offered deportation to the UK (where she had no ties) on ``humanitarian'' grounds, on the condition that she not protest her deportation
	\end{itemize}

	\vfill
	\ \hfill \citep{Davies08}
\end{frame}

\begin{frame} 
	\frametitle{UK career}
	\begin{itemize}
		\item the British African-Caribbean community was growing rapidly at the time
		\item Jones worked actively against racist immigration laws and practices in housing and employment
		\item she founded an anti-imperialist, anti-racist paper, \emph{The West Indian Gazette and Afro-Asian News}
		\item the newspaper ran for six years until her death
	\end{itemize}

	\vfill
	\ \hfill \citep{Davies08}
\end{frame}


\begin{frame} 
	\frametitle{Communism in the UK}
	\begin{itemize}
		\item the history of Jones' involvement with the Communist Party of Great Britain (CPGB) is not entirely clear
		\item she was not offered employment within the party as she had in the CPUSA
		\item Jones' relation to the party in contested
	\end{itemize}

	\vfill
	\ \hfill \citep{LalkarP}
\end{frame}

\begin{frame} 
	\frametitle{political ideology}
	\begin{itemize}
		\item there are two camps of biographers
		\item all book-length biographies that I can find of Jones are written by non-Communists
		\item Communists seem to claim that non-Communists misrepresent Jones' work and her ideologies
		\item this is complicated by the fact that her notebook disappeared following her death
	\end{itemize}

	\vfill
	\ \hfill \citep{LalkarP}
\end{frame}

\begin{frame} 
	\frametitle{black feminist thought}
	\begin{itemize}
		\item from what I can tell
		\item all book-length biographies that I can find of Jones are written by non-Communists
		\item Communists seem to claim that non-Communists misrepresent Jones' work and her ideologies
		\item they claim that Jones is a Communist through-and-through
		\item others claim that her work comprises an ideology distinct from that of Marx
		\item this is complicated by the fact that her notebook disappeared following her death
	\end{itemize}

	\vfill
	\ \hfill \citep{LalkarP}
\end{frame}

\begin{frame} 
	\frametitle{who's right?}
	\begin{itemize}
		% \item no-one claims that Jones was not a Communist, only that her thought was meaningfully separate from that of other Communists
		\item Marx did not talk about Black women
		\item Jones did
		\item that focus is important
		\item Jones' is distinct in that her work represents Black Feminist Thought (in a way that Marx's could not) with the express purpose of liberating Black women
	\end{itemize}
\end{frame}

\begin{frame}
% \frametitle{References}
\bibliography{final}
\bibliographystyle{apacite}
\end{frame}

\end{document}

