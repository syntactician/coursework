\documentclass[doc,12pt]{apa6}
\usepackage[colorlinks=false]{hyperref}
\usepackage{lmodern,amssymb,amsmath}
% \usepackage{apacite}
\linespread{1.5}

\begin{document}

\title{Preliminary Bibliography}
\shorttitle{Bibliography}
\author{Edward Hern\'{a}ndez}
\date{10 November}
\affiliation{College of William \& Mary}
\maketitle

\everypar{\hangindent2em \hangafter1}

\noindent
Davies, C. B. (2008). \emph{Left of Karl Marx: The political life of Black Communist Claudia Jones}. Duke University Press.
% \nocite{Davies08}

\noindent
Davies, C. B. (Ed.). (2011). \emph{Claudia Jones: Beyond containment}. Ayebia Clarke Publishing.
% \nocite{Davies11}
\vspace{6pt} \\ \-\hspace{2em}
Several sources I've run into so far have listed this as the premier "political
biography" of Claudia Jones. I expect that it will give me not only a picture
of her life (which I expect the other biographies in this list are capable of
doing as well) but an idea of the origins and development of her political
ideas. It also purports to analyze Jones' journalism and her place within
Carribean intellectual traditions, which no other source I've found has
discussed at all.

\noindent
Howard, W. T. (2013). \emph{We shall be free!: Black Communist protests in seven voices}. Temple University Press.
% \nocite{Howard13}

\noindent
Johnson, B. (1984). \emph{I think of my mother: Notes on the life and times of Claudia Jones}. Karia Press.
% \nocite{Johnson84}

\noindent
Johnson, C. (2008). \emph{Reclaiming Claudia Jones: When a Black feminist Marxist defies McCarthyism} (Vol. 22) (No. 1). Ann Arbor, MI: MPublishing, University of Michigan Library. Retrieved from \url{http://hdl.handle.net/2027/spo.ark5583.0022.102}
% \nocite{Johnson08}
\vspace{6pt} \\ \-\hspace{2em}
This is a relatively short essay that gives a great introduction to the life
and work of Claudia Jones. It has already introduced me to literature I
otherwise would not have found. It also offers a fairly technical
poststructuralist analysis that I think will be useful to think through once I
understand it.

\noindent
Jones, C. (1949a). An end to the neglect of the problems of the Negro woman! \emph{Political Affairs}.
% \nocite{Jones49a}

\noindent
Jones, C. (1949b). \emph{We seek full equality for women}.
% \nocite{Jones49b}

\noindent
McDuffie, E. S. (2011). \emph{Sojourning for freedom: Black women, American Communism, and the making of Black left feminism}. Duke University Press.
% \nocite{McDuffie11}

\noindent
O'Brien, K. (2014). Uncovering Black Marxist feminism. \emph{International Socialist Review, 90}. Retrieved from \url{http://isreview.org/issue/90/uncovering-black-marxist-feminism}
% \nocite{OBrien14}

\noindent
Olende, K. (2014, January). Intersectionality and black communist women. \emph{International Socialism, 141}. Retrieved from \url{http://isj.org.uk/intersectionality-and-black-communist-women/}
% \nocite{Olende14}
\vspace{6pt} \\ \-\hspace{1em}
This is a dissenting review of McDuffie (2011) and to a lesser extent of Davies
(2008). It seems to assert both that some of their analysis is ahistorical and
that their ideological positions are unsound. Because I am not well read in
Soviet history, I expect that this source (and the journal it is published in)
will help me broaden my understanding of the historical context of both Jones
and her biographers. I also expect that addressing the criticisms this author
raises will help me deepen my understanding of the biographies. In addition, it
has already pointed me toward literature I hadn't already found (Sherwood,
2000).

\noindent
Sherwood, M. (2000). \emph{Claudia Jones: A life in exile}. Lawrence and Wishart.
% \nocite{Sherwood00}

\noindent
Washington, M. H. (2003). Alice Childress, Lorraine Hansberry, and Claudia Jones: Black women write the popular front. In B. V. Mullen \& J. Smethurst (Eds.), \emph{Left of the color line: Race, radicalism, and twentieth-century literature of the United States} (p. 183-204). University of North Carolina Press.
% \nocite{Washington03}

\noindent
Weigand, K. (2001). Red feminism: American Communism and the making of women's liberation. Baltimore, MD: Johns Hopkins University Press.
% \nocite{Weigand0}
\vspace{6pt} \\ \-\hspace{1em}
According to Johnson (2008), Weigand "generously" cites Jones, in a way that
many other authors who have written on similar topics do not. I expect that
this work will help me to understand Jones' work specifically in the context of
Marxist Feminism. According to Johnson, this work is also useful for its
historical account of Jones' relation to the Communist Party of the United
States, which I imagine may be important to understanding her ideological
positions.

% \nocite{Davies08,Davies11,Howard13,Johnson08,Johnson84,Jones49a,Jones49b,McDuffie11,OBrien14,Olende14,Sherwood00,Washington03,Weigand01}

% \clearpage
% \bibliography{final}
% \bibliographystyle{apacite}

\end{document}
