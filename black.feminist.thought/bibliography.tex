\documentclass[man,12pt,natbib]{apa6}
\usepackage[breaklinks=true,colorlinks=false]{hyperref}
\usepackage{times}
\linespread{1.5}

\begin{document}

\title{Preliminary Bibliography}
\shorttitle{Bibliography}
\author{Edward Hern\'{a}ndez}
\date{10 November}
\affiliation{College of William \& Mary}
\maketitle

Several sources I've run into so far have listed \citet{Davies11} as the
premier ``political biography'' of Claudia Jones. I expect that it will give me
not only a picture of her life (which I expect the other biographies in this
list are capable of doing as well) but an idea of the origins and development
of her political ideas. It also purports to analyze Jones' journalism and her
place within Carribean intellectual traditions, which no other source I've
found has discussed at all.

\vspace{15pt}

\citet{Johnson08} offers a relatively short introduction to the life and work
of Claudia Jones. It has already introduced me to literature I otherwise would
not have found. It also offers a fairly technical poststructuralist analysis
that I think will be useful to think through once I understand it.

\vspace{15pt}

\citet{Olende14} offers a dissenting review of \citet{McDuffie11} and to a
lesser extent of \citet{Davies08}. It seems to assert both that some of their
analysis is ahistorical and that their ideological positions are unsound.
Because I am not well read in Soviet history, I expect that this source (and
the journal it is published in) will help me broaden my understanding of the
historical context of both Jones and her biographers. I also expect that
addressing the criticisms this author raises will help me deepen my
understanding of the biographies. In addition, it has already pointed me toward
literature I hadn't already found \citep{Sherwood00}.

\vspace{15pt}

According to \citet{Johnson08}, \citet{Weigand01} "generously" cites Jones, in a
way that many other authors who have written on similar topics do not. I expect
that this work will help me to understand Jones' work specifically in the
context of Marxist Feminism. According to Johnson, this work is also useful for
its historical account of Jones' relation to the Communist Party of the United
States, which I imagine may be important to understanding her ideological
positions.

\nocite{Davies08,Davies11,Howard13,Johnson08,Johnson84,Jones49a,Jones49b,McDuffie11,OBrien14,Olende14,Sherwood00,Washington03,Weigand01}

\clearpage
\bibliography{extra}

\end{document}
