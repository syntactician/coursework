\documentclass{article}
\usepackage[utf8]{inputenc}

\title{The Combahee River Collective Statement}
\author{Combahee River Collective}
\date{April 1977}

\begin{document}
\maketitle

We are a collective of Black feminists who have been meeting together since
1974.\footnote{This statement is dated April 1977.} During that time we have
been involved in the process of defining and clarifying our politics, while at
the same time doing political work within our own group and in coalition with
other progressive organizations and movements.  The most general statement of
our politics at the present time would be that we are actively committed to
struggling against racial, sexual, heterosexual, and class oppression, and see
as our particular task the development of integrated analysis and practice
based upon the fact that the major systems of oppression are interlocking. The
synthesis of these oppressions creates the conditions of our lives. As Black
women we see Black feminism as the logical political movement to combat the
manifold and simultaneous oppressions that all women of color face.

We will discuss four major topics in the paper that follows: (1) the genesis of
contemporary Black feminism; (2) what we believe, i.e., the specific province
of our politics; (3) the problems in organizing Black feminists, including a
brief herstory of our collective; and (4) Black feminist issues and practice.

\section{The genesis of Contemporary Black Feminism}

Before looking at the recent development of Black feminism we would like to
affirm that we find our origins in the historical reality of Afro-American
women's continuous life-and-death struggle for survival and liberation. Black
women's extremely negative relationship to the American political system (a
system of white male rule) has always been determined by our membership in two
oppressed racial and sexual castes. As Angela Davis points out in ``Reflections
on the Black Woman's Role in the Community of Slaves,'' Black women have always
embodied, if only in their physical manifestation, an adversary stance to white
male rule and have actively resisted its inroads upon them and their
communities in both dramatic and subtle ways. There have always been Black
women activists—some known, like Sojourner Truth, Harriet Tubman, Frances E. W.
Harper, Ida B. Wells Barnett, and Mary Church Terrell, and thousands upon
thousands unknown—who have had a shared awareness of how their sexual identity
combined with their racial identity to make their whole life situation and the
focus of their political struggles unique. Contemporary Black feminism is the
outgrowth of countless generations of personal sacrifice, militancy, and work
by our mothers and sisters.

A Black feminist presence has evolved most obviously in connection with the
second wave of the American women's movement beginning in the late 1960s.
Black, other Third World, and working women have been involved in the feminist
movement from its start, but both outside reactionary forces and racism and
elitism within the movement itself have served to obscure our participation. In
1973, Black feminists, primarily located in New York, felt the necessity of
forming a separate Black feminist group. This became the National Black
Feminist Organization (NBFO).

Black feminist politics also have an obvious connection to movements for Black
liberation, particularly those of the 1960s and I970s. Many of us were active
in those movements (Civil Rights, Black nationalism, the Black Panthers), and
all of our lives were greatly affected and changed by their ideologies, their
goals, and the tactics used to achieve their goals. It was our experience and
disillusionment within these liberation movements, as well as experience on the
periphery of the white male left, that led to the need to develop a politics
that was anti-racist, unlike those of white women, and anti-sexist, unlike
those of Black and white men.

There is also undeniably a personal genesis for Black Feminism, that is, the
political realization that comes from the seemingly personal experiences of
individual Black women's lives. Black feminists and many more Black women who
do not define themselves as feminists have all experienced sexual oppression as
a constant factor in our day-to-day existence. As children we realized that we
were different from boys and that we were treated differently. For example, we
were told in the same breath to be quiet both for the sake of being
``ladylike'' and to make us less objectionable in the eyes of white people. As
we grew older we became aware of the threat of physical and sexual abuse by
men. However, we had no way of conceptualizing what was so apparent to us, what
we knew was really happening.

Black feminists often talk about their feelings of craziness before becoming
conscious of the concepts of sexual politics, patriarchal rule, and most
importantly, feminism, the political analysis and practice that we women use to
struggle against our oppression. The fact that racial politics and indeed
racism are pervasive factors in our lives did not allow us, and still does not
allow most Black women, to look more deeply into our own experiences and, from
that sharing and growing consciousness, to build a politics that will change
our lives and inevitably end our oppression. Our development must also be tied
to the contemporary economic and political position of Black people. The post
World War II generation of Black youth was the first to be able to minimally
partake of certain educational and employment options, previously closed
completely to Black people. Although our economic position is still at the very
bottom of the American capitalistic economy, a handful of us have been able to
gain certain tools as a result of tokenism in education and employment which
potentially enable us to more effectively fight our oppression.

A combined anti-racist and anti-sexist position drew us together initially, and
as we developed politically we addressed ourselves to heterosexism and economic
oppression under capitalism.

\section{What We Believe}

Above all else, Our politics initially sprang from the shared belief that Black
women are inherently valuable, that our liberation is a necessity not as an
adjunct to somebody else's may because of our need as human persons for
autonomy. This may seem so obvious as to sound simplistic, but it is apparent
that no other ostensibly progressive movement has ever considered our specific
oppression as a priority or worked seriously for the ending of that oppression.
Merely naming the pejorative stereotypes attributed to Black women (e.g. mammy,
matriarch, Sapphire, whore, bulldagger), let alone cataloguing the cruel, often
murderous, treatment we receive, Indicates how little value has been placed
upon our lives during four centuries of bondage in the Western hemisphere. We
realize that the only people who care enough about us to work consistently for
our liberation are us. Our politics evolve from a healthy love for ourselves,
our sisters and our community which allows us to continue our struggle and
work.

This focusing upon our own oppression is embodied in the concept of identity
politics. We believe that the most profound and potentially most radical
politics come directly out of our own identity, as opposed to working to end
somebody else's oppression. In the case of Black women this is a particularly
repugnant, dangerous, threatening, and therefore revolutionary concept because
it is obvious from looking at all the political movements that have preceded us
that anyone is more worthy of liberation than ourselves. We reject pedestals,
queenhood, and walking ten paces behind. To be recognized as human, levelly
human, is enough.

We believe that sexual politics under patriarchy is as pervasive in Black
women's lives as are the politics of class and race. We also often find it
difficult to separate race from class from sex oppression because in our lives
they are most often experienced simultaneously. We know that there is such a
thing as racial-sexual oppression which is neither solely racial nor solely
sexual, e.g., the history of rape of Black women by white men as a weapon of
political repression.

Although we are feminists and Lesbians, we feel solidarity with progressive
Black men and do not advocate the fractionalization that white women who are
separatists demand. Our situation as Black people necessitates that we have
solidarity around the fact of race, which white women of course do not need to
have with white men, unless it is their negative solidarity as racial
oppressors. We struggle together with Black men against racism, while we also
struggle with Black men about sexism.

We realize that the liberation of all oppressed peoples necessitates the
destruction of the political-economic systems of capitalism and imperialism as
well as patriarchy. We are socialists because we believe that work must be
organized for the collective benefit of those who do the work and create the
products, and not for the profit of the bosses. Material resources must be
equally distributed among those who create these resources. We are not
convinced, however, that a socialist revolution that is not also a feminist and
anti-racist revolution will guarantee our liberation. We have arrived at the
necessity for developing an understanding of class relationships that takes
into account the specific class position of Black women who are generally
marginal in the labor force, while at this particular time some of us are
temporarily viewed as doubly desirable tokens at white-collar and professional
levels. We need to articulate the real class situation of persons who are not
merely raceless, sexless workers, but for whom racial and sexual oppression are
significant determinants in their working/economic lives. Although we are in
essential agreement with Marx's theory as it applied to the very specific
economic relationships he analyzed, we know that his analysis must be extended
further in order for us to understand our specific economic situation as Black
women.

A political contribution which we feel we have already made is the expansion of
the feminist principle that the personal is political. In our
consciousness-raising sessions, for example, we have in many ways gone beyond
white women's revelations because we are dealing with the implications of race
and class as well as sex. Even our Black women's style of talking/testifying in
Black language about what we have experienced has a resonance that is both
cultural and political. We have spent a great deal of energy delving into the
cultural and experiential nature of our oppression out of necessity because
none of these matters has ever been looked at before. No one before has ever
examined the multilayered texture of Black women's lives. An example of this
kind of revelation/conceptualization occurred at a meeting as we discussed the
ways in which our early intellectual interests had been attacked by our peers,
particularly Black males. We discovered that all of us, because we were
``smart'' had also been considered ``ugly,'' i.e., ``smart-ugly.''
``Smart-ugly'' crystallized the way in which most of us had been forced to
develop our intellects at great cost to our ``social'' lives. The sanctions In
the Black and white communities against Black women thinkers is comparatively
much higher than for white women, particularly ones from the educated middle
and upper classes.

As we have already stated, we reject the stance of Lesbian separatism because
it is not a viable political analysis or strategy for us. It leaves out far too
much and far too many people, particularly Black men, women, and children. We
have a great deal of criticism and loathing for what men have been socialized
to be in this society: what they support, how they act, and how they oppress.
But we do not have the misguided notion that it is their maleness, per se—i.e.,
their biological maleness—that makes them what they are. As Black women we find
any type of biological determinism a particularly dangerous and reactionary
basis upon which to build a politic. We must also question whether Lesbian
separatism is an adequate and progressive political analysis and strategy, even
for those who practice it, since it so completely denies any but the sexual
sources of women's oppression, negating the facts of class and race.

\section{Problems in Organizing Black Feminists}

During our years together as a Black feminist collective we have experienced
success and defeat, joy and pain, victory and failure. We have found that it is
very difficult to organize around Black feminist issues, difficult even to
announce in certain contexts that we are Black feminists. We have tried to
think about the reasons for our difficulties, particularly since the white
women's movement continues to be strong and to grow in many directions. In this
section we will discuss some of the general reasons for the organizing problems
we face and also talk specifically about the stages in organizing our own
collective.

The major source of difficulty in our political work is that we are not just
trying to fight oppression on one front or even two, but instead to address a
whole range of oppressions. We do not have racial, sexual, heterosexual, or
class privilege to rely upon, nor do we have even the minimal access to
resources and power that groups who possess anyone of these types of privilege
have.

The psychological toll of being a Black woman and the difficulties this
presents in reaching political consciousness and doing political work can never
be underestimated. There is a very low value placed upon Black women's psyches
in this society, which is both racist and sexist. As an early group member once
said, ``We are all damaged people merely by virtue of being Black women.'' We
are dispossessed psychologically and on every other level, and yet we feel the
necessity to struggle to change the condition of all Black women. In ``A Black
Feminist's Search for Sisterhood,'' Michele Wallace arrives at this conclusion:

\begin{quote}
	We exists as women who are Black who are feminists, each stranded for the
	moment, working independently because there is not yet an environment in
	this society remotely congenial to our struggle—because, being on the
	bottom, we would have to do what no one else has done: we would have to
	fight the world.\footnote{Wallace, Michele. ``A Black Feminist's Search for
		Sisterhood,'' The Village Voice, 28 July 1975, pp. 6-7.}
\end{quote}

Wallace is pessimistic but realistic in her assessment of Black feminists'
position, particularly in her allusion to the nearly classic isolation most of
us face. We might use our position at the bottom, however, to make a clear leap
into revolutionary action. If Black women were free, it would mean that
everyone else would have to be free since our freedom would necessitate the
destruction of all the systems of oppression.

Feminism is, nevertheless, very threatening to the majority of Black people
because it calls into question some of the most basic assumptions about our
existence, i.e., that sex should be a determinant of power relationships. Here
is the way male and female roles were defined in a Black nationalist pamphlet
from the early 1970s:

\begin{quote}
	We understand that it is and has been traditional that the man is the head
	of the house. He is the leader of the house/nation because his knowledge of
	the world is broader, his awareness is greater, his understanding is fuller
	and his application of this information is wiser... After all, it is only
	reasonable that the man be the head of the house because he is able to
	defend and protect the development of his home... Women cannot do the same
	things as men—they are made by nature to function differently. Equality of
	men and women is something that cannot happen even in the abstract world.
	Men are not equal to other men, i.e. ability, experience or even
	understanding. The value of men and women can be seen as in the value of
	gold and silver—they are not equal but both have great value. We must
	realize that men and women are a complement to each other because there is
	no house/family without a man and his wife. Both are essential to the
	development of any life.\footnote{Mumininas of Committee for Unified
		Newark, Mwanamke Mwananchi (The Nationalist Woman), Newark, N.J.,
		©1971, pp. 4-5.}
\end{quote}

The material conditions of most Black women would hardly lead them to upset
both economic and sexual arrangements that seem to represent some stability in
their lives. Many Black women have a good understanding of both sexism and
racism, but because of the everyday constrictions of their lives, cannot risk
struggling against them both.

The reaction of Black men to feminism has been notoriously negative. They are,
of course, even more threatened than Black women by the possibility that Black
feminists might organize around our own needs. They realize that they might not
only lose valuable and hardworking allies in their struggles but that they
might also be forced to change their habitually sexist ways of interacting with
and oppressing Black women. Accusations that Black feminism divides the Black
struggle are powerful deterrents to the growth of an autonomous Black women's
movement.

Still, hundreds of women have been active at different times during the
three-year existence of our group. And every Black woman who came, came out of
a strongly-felt need for some level of possibility that did not previously
exist in her life.

When we first started meeting early in 1974 after the NBFO first eastern
regional conference, we did not have a strategy for organizing, or even a
focus. We just wanted to see what we had. After a period of months of not
meeting, we began to meet again late in the year and started doing an intense
variety of consciousness-raising. The overwhelming feeling that we had is that
after years and years we had finally found each other. Although we were not
doing political work as a group, individuals continued their involvement in
Lesbian politics, sterilization abuse and abortion rights work, Third World
Women's International Women's Day activities, and support activity for the
trials of Dr. Kenneth Edelin, Joan Little, and Inéz García. During our first
summer when membership had dropped off considerably, those of us remaining
devoted serious discussion to the possibility of opening a refuge for battered
women in a Black community. (There was no refuge in Boston at that time.) We
also decided around that time to become an independent collective since we had
serious disagreements with NBFO's bourgeois-feminist stance and their lack of a
clear politIcal focus.

We also were contacted at that time by socialist feminists, with whom we had
worked on abortion rights activities, who wanted to encourage us to attend the
National Socialist Feminist Conference in Yellow Springs. One of our members
did attend and despite the narrowness of the ideology that was promoted at that
particular conference, we became more aware of the need for us to understand
our own economic situation and to make our own economic analysis.

In the fall, when some members returned, we experienced several months of
comparative inactivity and internal disagreements which were first
conceptualized as a Lesbian-straight split but which were also the result of
class and political differences. During the summer those of us who were still
meeting had determined the need to do political work and to move beyond
consciousness-raising and serving exclusively as an emotional support group. At
the beginning of 1976, when some of the women who had not wanted to do
political work and who also had voiced disagreements stopped attending of their
own accord, we again looked for a focus. We decided at that time, with the
addition of new members, to become a study group. We had always shared our
reading with each other, and some of us had written papers on Black feminism
for group discussion a few months before this decision was made. We began
functioning as a study group and also began discussing the possibility of
starting a Black feminist publication. We had a retreat in the late spring
which provided a time for both political discussion and working out
interpersonal issues. Currently we are planning to gather together a collection
of Black feminist writing. We feel that it is absolutely essential to
demonstrate the reality of our politics to other Black women and believe that
we can do this through writing and distributing our work. The fact that
individual Black feminists are living in isolation all over the country, that
our own numbers are small, and that we have some skills in writing, printing,
and publishing makes us want to carry out these kinds of projects as a means of
organizing Black feminists as we continue to do political work in coalition
with other groups.

\section{Black Feminist Issues and Projects}

During our time together we have identified and worked on many issues of
particular relevance to Black women. The inclusiveness of our politics makes us
concerned with any situation that impinges upon the lives of women, Third World
and working people. We are of course particularly committed to working on those
struggles in which race, sex, and class are simultaneous factors in oppression.
We might, for example, become involved in workplace organizing at a factory
that employs Third World women or picket a hospital that is cutting back on
already inadequate heath care to a Third World community, or set up a rape
crisis center in a Black neighborhood. Organizing around welfare and daycare
concerns might also be a focus. The work to be done and the countless issues
that this work represents merely reflect the pervasiveness of our oppression.

Issues and projects that collective members have actually worked on are
sterilization abuse, abortion rights, battered women, rape and health care. We
have also done many workshops and educationals on Black feminism on college
campuses, at women's conferences, and most recently for high school women.

One issue that is of major concern to us and that we have begun to publicly
address is racism in the white women's movement. As Black feminists we are made
constantly and painfully aware of how little effort white women have made to
understand and combat their racism, which requires among other things that they
have a more than superficial comprehension of race, color, and Black history
and culture. Eliminating racism in the white women's movement is by definition
work for white women to do, but we will continue to speak to and demand
accountability on this issue.

In the practice of our politics we do not believe that the end always justifies
the means. Many reactionary and destructive acts have been done in the name of
achieving ``correct'' political goals. As feminists we do not want to mess over
people in the name of politics. We believe in collective process and a
nonhierarchical distribution of power within our own group and in our vision of
a revolutionary society. We are committed to a continual examination of our
politics as they develop through criticism and self-criticism as an essential
aspect of our practice. In her introduction to Sisterhood is Powerful Robin
Morgan writes:

\begin{quote}
	I haven't the faintest notion what possible revolutionary role white
	heterosexual men could fulfill, since they are the very embodiment of
	reactionary-vested-interest-power.
\end{quote}

As Black feminists and Lesbians we know that we have a very definite
revolutionary task to perform and we are ready for the lifetime of work and
struggle before us.

\end{document}
