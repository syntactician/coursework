\documentclass{article}
\usepackage[utf8]{inputenc}

\title{The Master's Tools Will Never Dismantle the Master's House}
\author{Audre Lorde}
\date{1979}

\begin{document}
\maketitle

I agreed to take part in a New York University Institute for the Humanities conference a year ago, with the understanding that I would be commenting upon papers dealing with the role of difference within the lives of American women: difference of race, sexuality, class, and age. The absence of these considerations weakens any feminist discussion of the personal and the political. 

It is a particular academic arrogance to assume any discussion of feminist theory without examining our many differences, and without a significant input from poor women, Black and Third World women, and lesbians. And yet, I stand here as a Black lesbian feminist, having been invited to comment within the only panel at this conference where the input of Black feminists and lesbians is represented. What this says about the vision of this conference is sad, in a country where racism, sexism, and homophobia are inseparable. To read this program is to assume that lesbian and Black women have nothing to say about existentialism, the erotic, women's culture and silence, developing feminist theory, or heterosexuality and power. And what does it mean in personal and political terms when even the two Black women who did present here were literally found at the last hour? What does it mean when the tools of a racist patriarchy are used to examine the fruits of that same patriarchy? It means that only the most narrow perimeters of change are possible and allowable. 

The absence of any consideration of lesbian consciousness or the consciousness of Third World women leaves a serious gap within this conference and within the papers presented here. For example, in a paper on material relationships between women, I was conscious of an either/or model of nurturing which totally dismissed my knowledge as a Black lesbian. In this paper there was no examination of mutuality between women, no systems of shared support, no interdependence as exists between lesbians and women-identified women. Yet it is only in the patriarchal model of nurturance that women "who attempt to emancipate themselves pay perhaps too high a price for the results," as this paper states. 

For women, the need and desire to nurture each other is not pathological but redemptive, and it is within that knowledge that our real power is rediscovered. It is this real connection which is so feared by a patriarchal world. Only within a patriarchal structure is maternity the only social power open to women. 

Interdependency between women is the way to a freedom which allows the I to be, not in order to be used, but in order to be creative. This is a difference between the passive be and the active being. 

Advocating the mere tolerance of difference between women is the grossest reformism. It is a total denial of the creative function of difference in our lives. Difference must be not merely tolerated, but seen as a fund of necessary polarities between which our creativity can spark like a dialectic. Only then does the necessity for interdependency become unthreatening. Only within that interdependency of different strengths, acknowledged and equal, can the power to seek new ways of being in the world generate, as well as the courage and sustenance to act where there are no charters. 

Within the interdependence of mutual (nondominant) differences lies that security which enables us to descend into the chaos of knowledge and return with true visions of our future, along with the concomitant power to effect those changes which can bring that future into being. Difference is that raw and powerful connection from which our personal power is forged. 

As women, we have been taught either to ignore our differences, or to view them as causes for separation and suspicion rather than as forces for change. Without community there is no liberation, only the most vulnerable and temporary armistice between an individual and her oppression. But community must not mean a shedding of our differences, nor the pathetic pretense that these differences do not exist. 

Those of us who stand outside the circle of this society's definition of acceptable women; those of us who have been forged in the crucibles of difference—those of us who are poor, who are lesbians, who are Black, who are older—know that survival is not an academic skill. It is learning how to stand alone, unpopular and sometimes reviled, and how to make common cause with those others identified as outside the structures in order to define and seek a world in which we can all flourish. It is learning how to take our differences and make them strengths. For the master's tools will never dismantle the master's house. They may allow us temporarily to beat him at his own game, but they will never enable us to bring about genuine change. And this fact is only threatening to those women who still define the master's house as their only source of support. 

Poor women and women of Color know there is a difference between the daily manifestations of marital slavery and prostitution because it is our daughters who line 42nd Street. If white American feminist theory need not deal with the differences between us, and the resulting difference in our oppressions, then how do you deal with the fact that the women who clean your houses and tend your children while you attend conferences on feminist theory are, for the most part, poor women and women of Color? What is the theory behind racist feminism? 

In a world of possibility for us all, our personal visions help lay the groundwork for political action. The failure of academic feminists to recognize difference as a crucial strength is a failure to reach beyond the first patriarchal lesson. In our world, divide and conquer must become define and empower. 

Why weren't other women of Color found to participate in this conference? Why were two phone calls to me considered a consultation? Am I the only possible source of names of Black feminists? And although the Black panelist's paper ends on an important and powerful connection of love between women, what about interracial cooperation between feminists who don't love each other? 

In academic feminist circles, the answer to these questions is often, "We did not know who to ask." But that is the same evasion of responsibility, the same cop-out, that keeps Black women's art out of women's exhibitions, Black women's work out of most feminist publications except for the occasional "Special Third World Women's Issue," and Black women's texts off your reading lists. But as Adrienne Rich pointed out in a recent talk, white feminists have educated themselves about such an enormous amount over the past ten years, how come you haven't also educated yourselves about Black women and the differences between us—white and Black—when it is key to our survival as a movement? 

Women of today are still being called upon to stretch across the gap of male ignorance and to educate men as to our existence and our needs. This is an old and primary tool of all oppressors to keep the oppressed occupied with the master's concerns. Now we hear that it is the task of women of Color to educate white women—in the face of tremendous resistance—as to our existence, our differences, our relative roles in our joint survival. This is a diversion of energies and a tragic repetition of racist patriarchal thought. 

Simone de Beauvoir once said: "It is in the knowledge of the genuine conditions of our lives that we must draw our strength to live and our reasons for acting." 

Racism and homophobia are real conditions of all our lives in this place and time. I urge each one of us here to reach down into that deep place of knowledge inside herself and touch that terror and loathing of any difference that lives there. See whose face it wears. Then the personal as the political can begin to illuminate all our choices. 

\begin{quote}
	\emph{
		Prospero, you are the master of illusion.       \\
		Lying is your trademark.                        \\
		And you have lied so much to me                 \\
		(Lied about the world, lied about me)           \\
		That you have ended by imposing on me           \\
		An image of myself.                             \\
		Underdeveloped, you brand me, inferior,         \\
		That s the way you have forced me to see myself \\
		I detest that image! What's more, it's a lie!   \\
		But now I know you, you old cancer,             \\
		And I know myself as well.                      \\
	}

	Caliban, in Aime Cesaire's \emph{A Tempest}
\end{quote}

\end{document}
