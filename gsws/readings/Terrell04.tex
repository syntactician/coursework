\documentclass{article}
\usepackage[utf8]{inputenc}

\title{The Progress of Colored Women}
\author{Mary Church Terrell}
\date{1904}

\begin{document}
\maketitle

When one considers the obstacles encountered by colored women in their effort
to educate and cultivate themselves, since they became free, the work they have
accomplished and the progress they have made will bear favorable comparison, at
least with that of their more fortunate sisters, from whom the opportunity of
acquiring knowledge and the means of self-culture have never been entirely
withheld. Not only are colored women with ambition and aspiration handicapped
on account of their sex, but they are almost everywhere baffled and mocked
because of their race. Not only because they are women, but because they are
colored women, are discouragement and disappointment meeting them at every
turn. But in spite of the obstacles encountered, the progress made by colored
women along many lines appears like a veritable miracle of modern times. Forty
years ago for the great masses of colored women, there was no such thing as
home. Today in each and every section of the country there are hundreds of
homes among colored people, the mental and moral tone of which is as high and
as pure as can be found among the best people of any land.

To the women of the race may be attributed in large measure the refinement and
purity of the colored home. The immorality of colored women is a theme upon
which those who know little about them or those who maliciously misrepresent
them love to descant. Foul aspersions upon the character of colored women are
assiduously circulated by the press of certain sections and especially by the
direct descendants of those who in years past were responsible for the moral
degradation of their female slaves. And yet, in spite of the fateful heritage
of slavery, even though the safeguards usually thrown around maidenly youth and
innocence are in some sections entirely withheld from colored girls, statistics
compiled by men not inclined to falsify in favor of my race show that
immorality among the colored women of the United States is not so great as
among women with similar environment and temptations in Italy, Germany, Sweden
and France.

Scandals in the best colored society are exceedingly rare, while the
progressive game of divorce and remarriage is practically unknown.

The intellectual progress of colored women has been marvelous. So great has
been their thirst for knowledge and so Herculean their efforts to acquire it
that there are few colleges, universities, high and normal schools in the
North, East and West from which colored girls have not graduated with honor. In
Wellesley, Vassar, Ann Arbor, Cornell and in Oberlin, my dear alma mater, whose
name will always be loved and whose praise will always be sung as the first
college in the country broad, just and generous enough to extend a cordial
welcome to the Negro and to open its doors to women on an equal footing with
the men, colored girls by their splendid records have forever settled the
question of their capacity and worth. The instructors in these and other
institutions cheerfully bear testimony to their intelligence, their diligence
and their success.

As the brains of colored women expanded, their hearts began to grow. No Sooner
had the heads of a favored few been filled with knowledge than their hearts
yearned to dispense blessings to the less fortunate of their race. With
tireless energy and eager zeal, colored women have worked in every conceivable
way to elevate their race. Of the colored teachers engaged in instructing our
youth it is probably no exaggeration to say that fully eighty percent are
women. In the backwoods, remote from the civilization and comforts of the city
and town, colored women may be found courageously battling with those evils
which such conditions always entail. Many a heroine of whom the world will
never hear has thus sacrificed her life to her race amid surroundings and in
the face of privations which only martyrs can bear.

Through the medium of their societies in the church, beneficial organizations
out of it and clubs of various kinds, colored women are doing a vast amount of
good. It is almost impossible to ascertain exactly what the Negro is doing in
any field, for the records are so poorly kept. This is particularly true in the
case of the women of the race. During the past forty years there is no doubt
that colored women in their poverty have contributed large sums of money to
charitable and educational institutions as well as to the foreign and home
missionary work. Within the twenty-five years in which the educational work of
the African Methodist Episcopal Church has been systematized, the women of that
organization have contributed at least five hundred thousand dollars to the
cause of education. Dotted all over the country are charitable institutions for
the aged, orphaned and poor which have been established by colored women. Just
how many it is difficult to state, owing to the lack of statistics bearing on
the progress, possessions and prowess of colored women.

Up to date, politics have been religiously eschewed by colored women, although
questions affecting our legal status as a race are sometimes agitated by the
most progressive class. In Louisiana and Tennessee colored women have several
times petitioned the legislatures of their respective states to repel the
obnoxious Jim-Crow laws. Against the convict-lease system, whose atrocities
have been so frequently exposed of late, colored women here and there in the
South are waging a ceaseless war. So long as hundreds of their brothers and
sisters, many of whom have committed no crime or misdemeanor whatever, are
thrown into cells whose cubic contents are less than those of a good size
grave, to be overworked, underfed and only partially covered with vermin
infested rags, and so long as children are born to the women in these camps who
breathe the polluted atmosphere of these dens of horror and vice from the time
they utter their first cry in the world till they are released from their
suffering by death, colored women who are working for the emancipation and
elevation of their race know where their duty lies. By constant agitation of
this painful and hideous subject, they hope to touch the conscience of the
country, so that this stain upon its escutcheon shall be forever wiped away.

Alarmed at the rapidity with which the Negro is losing ground in the world of
trade, some of the farsighted women are trying to solve the labor question, so
far as it concerns the women at least, by urging the establishment of schools
of domestic science wherever means therefore can be secured. Those who are
interested in this particular work hope and believe that if colored women and
girls are thoroughly trained in domestic service, the boycott which has
undoubtedly been placed upon them in many sections of the country will be
removed. With so few vocations open to the Negro and with the labor
organizations increasingly hostile to him, the future of the boys and girls of
the race appears to some of our women very foreboding and dark.

The cause of temperance has been eloquently espoused by two women, each of whom
has been appointed national superintendent of work among colored people by the
Woman's Christian Temperance Union. In business, colored women have had signal
success. There is in Alabama a large milling and cotton business belonging to
and controlled by a colored woman, who has sometimes as many as seventy-five
men in her employ. Until a few years ago the principal ice plant of Nova Scotia
was owned and managed by a colored woman, who sold it for a large amount. In
the professions there are dentists and doctors whose practice is lucrative and
large. Ever since a book was published in 1773 entitled ``Poems on Various
Subjects, Religious and Moral by Phillis Wheatley, Negro Servant of Mr. John
Wheatley,'' of Boston, colored women have given abundant evidence of literary
ability. In sculpture we were represented by a woman upon whose chisel Italy
has set her seal of approval; in painting by one of Bouguereau's pupils and in
music by young women holding diplomas from the best conservatories in the land.
In short, to use a thought of the illustrious Frederick Douglass, if judged by
the depths from which they have come, rather than by the heights to which those
blessed with centuries of opportunities have attained, colored women need not
hang their heads in shame. They are slowly but surely making their way up to
the heights, wherever they can be scaled. In spite of handicaps and
discouragements they are not losing heart. In a variety of ways they are
rendering valiant service to their race. Lifting as they climb, onward and
upward they go struggling and striving and hoping that the buds and blossoms of
their desires may burst into glorious fruition ere long. Seeking no favors
because of their color nor charity because of their needs they knock at the
door of Justice and ask for an equal chance.

\end{document}
