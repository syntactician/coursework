\documentclass{article}
\usepackage[utf8]{inputenc}

\title{I Was Born a Boy}
\author{Janet Mock}
\date{18 May 2011}

\begin{document}
\maketitle

The flight to Bangkok's Don Muang Airport felt far longer than I'd imagined. It
was Christmas break during my freshman year at the University of Hawaii, and I
was 18, anxious, and alone. After high school graduation, many of my classmates
were throwing big graduation parties and buying new cars. Those kids went
looking for good times and great memories, but I was desperately searching for
one thing only: a chance to be in the right body for the first time in my
entire life. I had traveled more than 6,000 miles to have gender reassignment
surgery --- a sex change.

At the arrival gate, I was greeted by two smiling nurses who assured me that
everything was going to be OK. But I already knew that. I was the one who had
lived with the sheer torment of inhabiting a body that never matched who I was
inside, the one devastated by the quirk of fate that had consigned me to a life
of masked misery. By the time I set foot in Thailand, I knew there could be
nothing worse than living another day with a penis dangling between my legs.

Counting backward as the anesthesia took hold, I surrendered to what I believed
with certainty would be a better future. And then, just like that, I was awake
again. The sound of Muslim prayers rang through the air, echoing in my brightly
lit hospital room. Even though I'd spent the last three hours on the operating
table --- I could already feel the first tinges of pain in my lower body --- I
felt completely reborn. Though I had been born a boy to my native Hawaiian
mother and African-American father, I would never be a man. It was the birth of
my choosing this time. And now it was official: Charles had died so that Janet
could live.

Once, when I was 5-years-old, a little girl who lived next door to my
grandmother dared me to put on a muumuu and run across a nearby parking lot. So
I did. I threw it on, hiked it up in one hand, and ran like hell. It felt
amazing to be in a dress. But suddenly my grandmother appeared, a look of
horror on her face. I knew immediately that I had crossed some kind of line.
After yelling at me, she banished me to our patio, where I played quietly with
my sumo action figures for a while. I loved them because they had long hair,
and they were the only ``dolls'' OK for me, a boy, to play with.

It didn't take very long before the social cues got louder and clearer. My
parents started scolding me over the way I walked and held my hands. I learned
to hide aspects of my personality. Playing with girls was fine, for example,
but playing with their Barbies was something I could do only behind closed
doors. After my parents split, my mom said my younger brother and I needed a
strong male role model and sent us to live with our dad in Oakland, California.
Stern and critical, my father couldn't accept how feminine and dainty I was in
comparison to my rough-and-tumble brother. ``Get outside and play!'' he would
bark. One time, I pretended to be a girl named Keisha — I wasn't dressed like a
girl, but in my baggy jeans and colorful top and with my longish hair, I easily
passed for one. A boy who didn't know me told my cousin Mechelle that he
thought I was pretty. ``Isn't she?'' Mechelle said, playing along. \emph{She}.
It spoke to my soul.

It was my father who first dared to ask the question: \emph{You're not gay, are
you?} I was 8 and wasn't even sure what that meant, but I knew from his tone
that it was unacceptable. ``No!'' I shouted defensively.

When I was 12, my brother and I moved back to Honolulu to live with our mother.
Hawaii felt like another universe, and reflecting on it, I am struck by how
much more open and accepting it was. The searing social issues there had more
to do with locals versus ``foreigners'' (aka ``haoles'') than with kids like
me. In fact, I even found other boys like me there, and I eagerly gravitated to
them.  Together we envied girls, their ability to express their femininity
without shame; I admired the way their bodies bloomed and rounded out. Not
mine. I was beginning to loathe my shapeless body, the straight lines and hard
angles.

During recess one day, I met Wendi. A year older than me, she was part of a
small, tight-knit group of transsexuals who went around town wearing makeup and
skirts hitched up to the thigh. They congregated outside our school at night,
where they practiced the dance routines of Mariah Carey and Toni Braxton. They
were a revelation, and I was emboldened just watching them. Wendi lived with
her grandparents, who supported her and allowed her to wear girls' clothes and
makeup, a freedom I envied. I spent hours in her room, playing with her
cosmetics, plucking my eyebrows, trying on bras. The more time I spent with
Wendi, the more comfortable I grew expressing myself as a female. By the end of
my freshman year in high school, I was regularly wearing women's clothes to
school.

But the fallout was swift and merciless. \emph{Fag! I can see your balls!} The
insults reverberated off the lockers and echoed down the school hallways.
Though I was never physically threatened and never feared for my safety, the
harassment was relentless. Not a moment went by that wasn't accompanied by a
taunt, a slur, a cruel reminder that my classmates could not, would not, see me
as I saw myself.  ``You're making people uncomfortable,'' one vice principal
said while he looked me over with disdain. Soon he gave me an ultimatum: Wear a
skirt to school again and get sent home for the day. But it was too late to
turn back. I liked how I looked as a young woman, even though it meant exposing
myself to ridicule. After that, I held my head high as I strode through the
hallways in my miniskirts, past the haters who called me a freak, past the
teachers who looked on disapprovingly, and past the vice principal who
routinely sent me home. By the end of sophomore year, my mother, who condoned
my wardrobe, had had enough. Together, we decided it was time to transfer
schools.

Though most of the students at my new school had heard whispers about my past,
it was a much more open environment. There was even a Teen Center staffed with
social workers who counseled gay kids. One of them joined me as I introduced
myself to teachers as Janet and helped them get comfortable with calling me
that name instead of the one listed on the attendance sheets.

There are key moments in a person's life when you just know your destiny is
about to change. For me, this moment came when Wendi, whom I remained friends
with despite being in different schools, started taking female hormone pills.
When she graduated to injections a few months later, she sold me her pills for
\$1 a pop. The timing was divine, as I'd already begun to detect a hint of an
Adam's apple on my throat. The changes in my 15-year-old body horrified me.
Sometimes while showering, my thoughts got dark: \emph{What if I just cut this
thing off?} Wendi's pills were my savior. For three months, I took estrogen and
watched my body's slow metamorphosis: softer skin, budding breasts, a fuller
face.

But I knew that taking them without the supervision of a doctor was risky. I
needed someone to monitor my progress. That's when I finally confessed to my
mom what I'd been doing. A single, working mother, she didn't have the luxury
or will to micromanage my life and allowed me to do what I wanted so long as I
continued making honor roll. That was our unspoken deal. But the medical
changes were different --- she recognized that my desperation to be a woman was
not just teen angst or rebellion; it was a matter of life or death. ``If that's
what you want,'' she said, looking me straight in the eye, ``we're going to do
it the right way.'' So she signed off on a local endocrinologist's regimen of
treatments, which involved weekly hormone shots in the butt and daily estrogen
pills. For the first time, I could visualize heading off to college as a woman,
pursuing a career as a woman. No more dress-up, no more pretending.

While on these hormones, I lost my virginity at age 17 to a guy I met while I
was working at a boutique. He knew my background but said he didn't care. Even
though I trusted him, I couldn't relax and insisted on keeping the lights off.
I was a woman with the wrong parts, and tried to cover myself up. After that
awkward encounter, I knew that I could never share myself that way again. If I
was ever going to finally feel at ease with my body, I had to have a total sex
change.

I knew a woman, a friend of a friend, who had gone to Bangkok for gender
reassignment surgery. She told me that it cost only \$7,000, much cheaper than
getting it done in the U.S. Though that was still an extraordinary sum of money
for me at the time, I'd have paid any amount --- nothing was going to keep me
from my destiny. By year's end, I'd saved up enough to purchase my ticket to
Thailand.

I spent 10 days in the hospital recovery room, doped up on pain relievers.
During the operation, my surgeon had masterfully refashioned the tissue and
nerves from my male organs to construct a vagina. Finally, every part of me
made perfect sense. I didn't have to ``tuck'' anymore. Were I to change right
next to you in a locker room, you wouldn't think twice about my body, wouldn't
doubt for a second that you were in the company of a woman. A doctor signed off
on my gender reassignment papers, enabling me to legally change the sex on my
American birth certificate to female. With my male organ gone, I continued a
reduced hormone therapy regimen, which was ultimately phased out six months
later. If there was a secret now, it was mine to keep.

Two weeks after the surgery, I was in class at the University of Hawaii,
finally focusing on something other than my gender. Four years later, I left
Hawaii, a beautiful, confident woman armed with a journalism degree and bound
for graduate school and a career in New York City.

I was 25 minutes late and racked with nervous energy for my first date with
Aaron. We'd met at a Lower East Side bar --- he didn't know anything about me
when he approached me --- and our connection was so intense that it scared me.
He was good-looking but also, as I learned dating him over the next few weeks,
an open and thoughtful person. I decided that if the relationship was to go
further, if we were going to be intimate, I had to tell him my truth. One night
at his apartment, I took a deep breath. ``There's something about my past I
need to share with you,'' I calmly said. ``I was born a boy.'' I felt as though
the words were made of concrete, and I waited to hear them crash loudly to the
floor. Aaron looked at me with obvious concern, took my hand, and asked, ``Are
you OK?''

We spent the rest of the night talking. Slowly, I unpacked all the secrets and
shame I'd been dragging with me all these years. He was braver than I could've
dreamed. We didn't make love that night, but eventually we did, and I felt safe
with him. Revealing my story to Aaron was about finally embracing my authentic
self. Despite all the shit --- the childhood spent fearing my father's
judgments, the high school bullying, all those years mourning what I thought I
could never have --- here I was, in a blossoming relationship with a gorgeous,
astute, caring man. After 10 months of dating, we moved in together, and I've
never been more fulfilled.

Aaron is among just a handful of people who know about my unbelievable
adventure. I have a thriving career as a Web editor for a very popular
magazine. My coworkers don't know about my past, mostly because I never wanted
to be the poster child for transsexuals --- pre-op, post-op, or no op. But the
recent stories about kids who have killed themselves because of the secrets
they were forced to keep has shifted something in me.

That's why I decided to come out in the pages of \emph{Marie Claire}, why I'm
writing a memoir about my journey. It used to pain me to hear my birth name, a
heartbreaking insult classroom bullies would shout to get a rise out of me. But
talking and writing about my experiences have helped me finally accept the past
and celebrate the fact that I was once a big dreamer who happened to be born a
boy named Charles. I hope my story resonates with other big dreamers, lets them
know that no matter how huge, how insane, how unreasonable or unreachable your
goals may seem, nothing --- not even your own body --- can hold you back if you
are certain and fearless and, yes, even a little ballsy in your quest.

\end{document}
