\documentclass{article}
\usepackage[utf8]{inputenc}

\title{Oppression\\from \emph{The Politics of Reality}}
\author{Marilyn Frye}
\date{1983}

\begin{document}
\maketitle

It is a fundamental claim of feminism that women are oppressed. The word
``oppression'' is a strong word. It repels ant attracts. It is dangerous and
dangerously fashionable and endangered. It is much misused, and sometimes not
innocently.

The statement that women are oppressed is frequently met with the claim that
men are oppressed too. We hear that oppressing is oppressive to those who
oppress as well as those they oppress. Some men cite as evidence of their
oppression their much-advertised inability to cry. It is tough, we are told, to
be masculine. When the stresses and frustrations of being a man are cited as
evidence that oppressors are oppressed by their oppressing, the word
``oppression'' is being stretched to meaninglessness; it is treated as though
its scope includes any and all human experience of limitation or suffering, no
matter the cause, degree or consequence. Once such usage has been put over on
us, then if ever we deny that any person or group is oppressed, we seem to
imply that we think they never suffer and have no feelings. We are accused of
insensitivity; even of bigotry. For women, such accusation is particularly
intimidating, since sensitivity is on eof the few virtues that has been
assigned to us. If we are found insensitive, we may fear we have no redeeming
traits at all and perhaps are not real women. Thus are we silenced before we
begin: the name of our situation drained of meaning and our guilt mechanisms
tripped.

But this is nonsense. Human beings can be miserable without being oppressed,
and it is perfectly consistent to deny that a person or group is oppressed
without denying that they have feelings or that they suffer\ldots{}.

The root of the word ``oppression'' is the element ``press.'' \emph{The press
of the crowd; pressed into military service; to press a pair of pants; printing
press; press the button}. Presses are used to mold things or flatten them or
reduce them in bulk, sometimes to reduce them by squeezing out the gases or
liquids in them. Something pressed is something caught between or among forces
and barriers which are so related to each other that jointly they restrain,
restrict or prevent the thing's motion or mobility. Mold. Immobilize. Reduce.

The mundane experience of the oppressed provides another clue. One of the most
characteristic and ubiquitous features of the world as experienced by oppressed
people is the double bind -- situations in which options are reduced to a very
few and all of them expose one to penalty, censure or deprivation. For example,
it is often a requirement upon oppressed people that we smile and be cheerful.
If we comply, we signal our docility and our acquiescence in our situation. We
need not, then, be taken note of. We acquiesce in being made invisible, in our
occupying no space. We participate in our own erasure. On the other hand,
anything but the sunniest countenance exposes us to being perceived as mean,
bitter, angry or dangerous. This means, at the least, that we may be found
``difficult'' or unpleasant to work with, which is enough to cost one one's
livelihood; at worst, being seen as mean, bitter, angry or dangerous has been
known to result in rape, arrest, beating, and murder. One can only choose to
risk one's preferred form and rate of annihilation.

Another example: It is common in the United States that women, especially
younger women, are in a bind where neither sexual activity nor sexual
inactivity is all right. If she is heterosexually active, a woman is open to
censure and punishment for being loose, unprincipled or a whore. The
``punishment'' comes in the form of criticism, snide and embarrassing remarks,
being treated as an easy lay by men, scorn from her more restrained female
friends. She may have to lie to hide her behavior from her parents. She must
juggle the risks of unwanted pregnancy and dangerous contraceptives. On the
other hand, if she refrains from heterosexual activity, she is fairly
constantly harassed by men who try to persuade her into it and pressure her
into it and pressure her to ``relax'' and ``let her hair down''; she is
threatened with labels like ``frigid,'' ``uptight,'' ``man-hater,'' ``bitch,''
and ``cocktease.'' The same parents who would be disapproving of her sexual
activity may be worried by her inactivity because it suggests she is not or
will not be popular, or is not sexually normal. She may be charged with
lesbianism.  If a woman is raped, then if she has been heterosexually active
she is subject to the presumption that she liked it (since her activity is
presumed to show that she likes sex), and if she has not been heterosexually
active, she is subject to the presumption that she liked it (since she is
supposedly ``repressed and frustrated''). Both heterosexual activity and
heterosexual nonactivity are likely to be taken as proof that you wanted to be
raped, and hence, of course, weren't \emph{really} raped at all. You can't win.
You are caught in a bind, caught between systematically related pressures.

Women are caught like this, too, by networks of forces and barriers that expose
one to penalty, loss or contempt whether one works outside the home or not, is
on welfare or not, bears children or not, raises children or not, marries or
not, stays married or not, is heterosexual, lesbian, both or neither. Economic
necessity; confinement to racial and/or sexual job ghettos; sexual harassment;
sex discrimination; pressures of competing expectations and judgements about
\emph{women}, \emph{wives} and \emph{mothers} (in the society at large, in
racial and ethnic subcultures and in one's own mind); dependence (full or
partial) on husbands, parents or the state; commitment to political ideas;
loyalties to racial or ethnic or other ``minority'' groups; the demands of the
self-respect and responsibilities to others. Each of these factors exists in
complex tension with every other, penalizing or prohibiting all of the
apparently available options. And nipping at one's heels, always, is the
endless pack of little things. If one dresses one way, one is subject to the
assumption that one is advertising one's sexual availability; if one dresses
another way, one appears to ``not care about oneself'' or to be ``unfeminine.''
If one uses ``strong language,'' one invites categorization as a ``lady'' ---
one too delicately constituted to cope with robust speech or the realities to
which it presumably refers.

The experience of oppressed people is that the living of one's life is confined
and shaped by forces and barriers which are not accidental or occasional and
hence avoidable, but are systematically related to each other in such a way as
to catch one between and among them and restrict or penalize motion in any
direction. It is the experience of being caged in: all avenues, in every
direction, are blocked or booby trapped.

Cages. Consider a birdcage. If you look very closely at just one wire in the
cage, you cannot see the other wires. If your conception of what is before you
is determined by this myopic focus, you could look at that one wire, up and
down the length of it, and be unable to see why a bird would not just fly
around the wire any time it wanted to go somewhere.  Furthermore, even if, one
day at a time, you myopically inspected each wire, you still could not see why
a bird would gave trouble going past the wires to get anywhere. There is no
physical property of any one wire, \emph{nothing} that the closest scrutiny
could discover, that will reveal how a bird could be inhibited or harmed by it
except in the most accidental way. It is only when you step back, stop looking
at the wires one by one, microscopically, and take a macroscopic view of the
whole cage, that you can see why the bird does not go anywhere; and then you
will see it in a moment. It will require no great subtlety of mental powers. It
is perfectly obvious that the bird is surrounded by a network of systematically
related barriers, no one of which would be the least hindrance to its flight,
but which, by their relations to each other, are as confining as the solid
walls of a dungeon.

It is now possible to grasp one of the reasons why oppression can be hard to
see and recognize: one can study the elements of an oppressive structure with
great care and some good will without seeing the structure as a whole, and
hence without seeing or being able to understand that one is looking at a cage
and that there are people there who are caged, whose motion and mobility are
restricted, whose lives are shaped and reduced.

The arresting of vision at a microscopic level yields such common confusion as
that about the male door-opening ritual. This ritual, which is remarkably
widespread across classes and races, puzzles many people, some of whom do and
some of whom do not find it offensive. Look at the scene of the two people
approaching a door. The male steps slightly ahead and opens the door. The male
holds the door open while the female glides through. Then the male goes
through. The door closes after them.  ``Now how,'' one innocently asks, ``can
those crazy womenslibbers say that is oppressive? The guy \emph{removed} a
barrier to the lady's smooth and unruffled progress.'' But each repetition of
this ritual has a place in a pattern, in fact in several patterns. One has to
shift the level of one's perception in order to see the whole picture.

The door-opening pretends to be a helpful service, but the helpfulness is
false. This can be seen by noting that it will be done whether or not it makes
any practical sense. Infirm men and men burdened with packages will open doors
for able-bodied women who are free of physical burdens.  Men will impose
themselves awkwardly and jostle everyone in order to get to the door first. The
act is not determined by convenience or grace.  Furthermore, these very
numerous acts of unneeded or even noisome ``help'' occur in counter-point to a
pattern of men not being helpful in many practical ways in which women might
welcome help. What \emph{women} experience is a world in which gallant princes
charming commonly make a fuss about being helpful and providing small services
when help and services are of little or no use, but in which there are rarely
ingenious and adroit princes at hand when substantial assistance is really
wanted either in mundane affairs or in situations of threat, assault or terror.
There is no help with the (his) laundry; no help typing a report at 4:00 a.m.;
no help in mediating disputes among relatives or children. There is nothing but
advice that women should stay indoors after dark, be chaperoned by a man, or
when it comes down to it, ``lie back and enjoy it.''

The gallant gestures have no practical meaning. Their meaning is symbolic. The
door-opening and similar services provided are services which really are needed
by people who are for one reason or another incapacitated -- unwell, burdened
with parcels, etc. So the message is that women are incapable. The detachment
of the acts from the concrete realities of what women need and do not need is a
vehicle for the message that women's actual needs and interests are unimportant
or irrelevant. Finally, these gestures imitate the behavior of servants toward
masters and thus mock women, who are in most respects the servants and
caretakers of men. The message of the false helpfulness of male gallantry is
female dependence, the invisibility or insignificance of women, and contempt
for women.

One cannot see the meanings of these rituals if one's focus is riveted upon the
individual event in all its particularity, including the particularity of the
individual man's present conscious intentions and motives and the individual
woman's conscious perception of the event in the moment. It seems sometimes
that people take a deliberately myopic view and fill their eyes with things
seen microscopically in order not to see macroscopically. At any rate, whether
it is deliberate or not, people can and do fail to see the oppression of women
because they fail to see macroscopically and hence fail to see the various
elements of the situation as systematically related in larger schemes.

As the cageness of the birdcage is a macroscopic phenomenon, the oppressiveness
of the situations in which women live our various and different lives is a
macroscopic phenomenon. Neither can be \emph{seen} from a microscopic
perspective. But when you look macroscopically you can see it -- a network of
forces and barriers which are systematically related and which conspire to the
immobilization, reduction and molding of women and the lives we live\ldots

% From: Marilyn Frye, \emph{The Politics of Reality} (Trumansburg, N.Y.,:
% The Crossing Press, 1983).
\end{document}
