\documentclass{article}
\usepackage[utf8]{inputenc}
\usepackage{hyperref}

\title{\href{}{White Privilege: Unpacking the Invisible Knapsack}}
\author{Peggy McIntosh}
\date{1989}

\begin{document}
\maketitle

Through work to bring materials from Women's Studies into the rest of the
curriculum, I have often noticed men's unwillingness to grant that they are
over-privileged, even though they may grant that women are disadvantaged. They
may say they will work to improve women's status, in the society, the
university, or the curriculum, but they can't or won't support the idea of
lessening men's. Denials which amount to taboos surround the subject of
advantages which men gain from women's disadvantages. These denials protect
male privilege from being fully acknowledged, lessened or ended.

Thinking through unacknowledged male privilege as a phenomenon, I realized
that, since hierarchies in our society are interlocking, there was most likely
a phenomenon of white privilege that was similarly denied and protected. As a
white person, I realized I had been taught about racism as something that puts
others at a disadvantage, but had been taught not to see one of its corollary
aspects, white privilege, which puts me at an advantage.

I think whites are carefully taught not to recognize white privilege, as males
are taught not to recognize male privilege. So I have begun in an untutored way
to ask what it is like to have white privilege. I have come to see white
privilege as an invisible package of unearned assets that I can count on
cashing in each day, but about which I was ``meant'' to remain oblivious. White
privilege is like an invisible weightless knapsack of special provisions, maps,
passports, codebooks, visas, clothes, tools and blank checks.

Describing white privilege makes one newly accountable. As we in Women's
Studies work to reveal male privilege and ask men to give up some of their
power, so one who writes about white privilege must ask, ``Having described it,
what will I do to lessen or end it?''

After I realized the extent to which men work from a base of unacknowledged
privilege, I understood that much of their oppressiveness was unconscious. Then
I remembered the frequent charges from women of color that white women whom
they encounter are oppressive.

I began to understand why we are justly seen as oppressive, even when we don't
see ourselves that way. I began to count the ways in which I enjoy unearned
skin privilege and have been conditioned into oblivion about its existence.

My schooling gave me no training in seeing myself as an oppressor, as an
unfairly advantaged person, or as a participant in a damaged culture. I was
taught to see myself as an individual whose moral state depended on her
individual moral will. My schooling followed the pattern my colleague Elizabeth
Minnich has pointed out: whites are taught to think of their lives as morally
neutral, normative, and average, and also ideal, so that when we work to
benefit others, this is seen as work which will allow ``them'' to be more like
``us.''

I decided to try to work on myself at least by identifying some of the daily
effects of white privilege in my life. I have chosen those conditions which I
think in my case attach somewhat more to skin-color privilege than to class,
religion, ethnic status, or geographic location, though of course all these
other factors are intricately intertwined. As far as I can see, my African
American co-workers, friends, and acquaintances with whom I come into daily or
frequent contact in this particular time, place and line of work cannot count
on most of these conditions.
\begin{enumerate}
	\item I can if I wish arrange to be in the company of people of my race
		most of the time.

	\item If I should need to move, I can be pretty sure of renting or
		purchasing housing in an area which I can afford and in which I would
		want to live.

	\item I can be pretty sure that my neighbors in such a location will be
		neutral or pleasant to me.

	\item I can go shopping alone most of the time, pretty well assured that I
		will not be followed or harassed.

	\item I can turn on the television or open to the front page of the paper
		and see people of my race widely represented.

	\item When I am told about our national heritage or about ``civilization,''
		I am shown that people of my color made it what it is.

	\item I can be sure that my children will be given curricular materials
		that testify to the existence of their race.

	\item If I want to, I can be pretty sure of finding a publisher for this
		piece on white privilege.

	\item I can go into a music shop and count on finding the music of my race
		represented, into a supermarket and find the staple foods that fit with
		my cultural traditions, into a hairdresser's shop and find someone who
		can cut my hair.

	\item Whether I use checks, credit cards or cash, I can count on my skin
		color not to work against the appearance of financial reliability.

	\item I can arrange to protect my children most of the time from people who
		might not like them.

	\item I can swear, or dress in second-hand clothes, or not answer letters,
		without having people attribute these choices to the bad morals, the
		poverty, or the illiteracy of my race.

	\item I can speak in public to a powerful male group without putting my
		race on trial.

	\item I can do well in a challenging situation without being called a
		credit to my race.

	\item I am never asked to speak for all the people of my racial group.

	\item I can remain oblivious of the language and customs of persons of
		color who constitute the world's majority without feeling in my culture
		any penalty for such oblivion.

	\item I can criticize our government and talk about how much I fear its
		policies and behavior without being seen as a cultural outsider.

	\item I can be pretty sure that if I ask to talk to ``the person in
		charge,'' I will be facing a person of my race.

	\item If a traffic cop pulls me over or if the IRS audits my tax return, I
		can be sure I haven't been singled out because of my race.

	\item I can easily buy posters, postcards, picture books, greeting cards,
		dolls, toys, and children's magazines featuring people of my race.

	\item I can go home from most meetings of organizations I belong to feeling
		somewhat tied in, rather than isolated, out-of-place, outnumbered,
		unheard, held at a distance, or feared.

	\item I can take a job with an affirmative action employer without having
		co-workers on the job suspect that I got it because of race.

	\item I can choose public accommodations without fearing that people of my
		race cannot get in or will be mistreated in the places I have chosen.

	\item I can be sure that if I need legal or medical help, my race will not
		work against me.

	\item If my day, week, or year is going badly, I need not ask of each
		negative episode or situation whether it has racial overtones.

	\item I can choose blemish cover or bandages in ``flesh'' color and have
		them more less match my skin.
\end{enumerate}
I repeatedly forgot each of the realizations on this list until I wrote it
down.  For me, white privilege has turned out to be an elusive and fugitive
subject.  The pressure to avoid it is great, for in facing it I must give up
the myth of meritocracy. If these things are true, this is not such a free
country; one's life is not what one makes it; many doors open for certain
people through no virtues of their own.

In unpacking this invisible knapsack of white privilege, I have listed
conditions of daily experience that I once took for granted. Nor did I think of
any of these perquisites as bad for the holder. I now think that we need a more
finely differentiated taxonomy of privilege, for some of these varieties are
only what one would want for everyone in a just society, and others give
license to be ignorant, oblivious, arrogant and destructive.

I see a pattern running through the matrix of white privilege, a pattern of
assumptions that were passed on to me as a white person. There was one main
piece of cultural turf; it was my own turf, and I was among those who could
control the turf. My skin color was an asset for any move I was educated to
want to make. I could think of myself as belonging in major ways and of making
social systems work for me. I could freely disparage, fear, neglect, or be
oblivious to anything outside of the dominant cultural forms. Being of the main
culture, I could also criticize it fairly freely.

In proportion as my racial group was being made confident, comfortable, and
oblivious, other groups were likely being made inconfident, uncomfortable, and
alienated. Whiteness protected me from many kinds of hostility, distress and
violence, which I was being subtly trained to visit, in turn, upon people of
color.

For this reason, the word ``privilege'' now seems to me misleading. We usually
think of privilege as being a favored state, whether earned or conferred by
birth or luck. Yet some of the conditions I have described here work
systematically to overempower certain groups. Such privilege simply confers
dominance because of one's race or sex.

I want, then, to distinguish between earned strength and unearned power
conferred systemically. Power from unearned privilege can look like strength
when it is in fact permission to escape or to dominate. But not all of the
privileges on my list are inevitably damaging. Some, like the expectation that
neighbors will be decent to you, or that your race will not count against you
in court, should be the norm in a just society. Others, like the privilege to
ignore less powerful people, distort the humanity of the holders as well as the
ignored groups.

We might at least start by distinguishing between positive advantages, which we
can work to spread, and negative types of advantage, which unless rejected will
always reinforce our present hierarchies. For example, the feeling that one
belongs within the human circle, as Native Americans say, should not be seen as
privilege for a few. Ideally it is an unearned entitlement. At present, since
only a few have it, it is an unearned advantage for them. This paper results
from a process of coming to see that some of the power that I originally saw as
attendant on being a human being in the United States consisted in unearned
advantage and conferred dominance.

The question is: ``Having described white privilege, what will I do to end
it?'' I have met very few men who are truly distressed about systemic, unearned
male advantage and conferred dominance. And so one question for me and others
like me is whether we will be like them, or whether we will get truly
distressed, even outraged, about unearned race advantage and conferred
dominance, and, if so, what will we do to lessen them. In any case, we need to
do more work in identifying how they actually affect our daily lives. Many,
perhaps most, of our white students in the U.S. think that racism doesn't
affect them because they are not people of color, they do not see ``whiteness''
as a racial identity.  In addition, since race and sex are not the only
advantaging systems at work, we need similarly to examine the daily experience
of having age advantage, or ethnic advantage, or physical ability, or advantage
related to nationality, religion, or sexual orientation.

Difficulties and dangers surrounding the task of finding parallels are many.
Since racism, sexism, and heterosexism are not the same, the advantages
associated with them should not be seen as the same. In addition, it is hard to
disentangle aspects of unearned advantage which rest more on social class,
economic class, race, religion, sex, and ethnic identity than on other factors.
Still, all of the oppressions are interlocking, as the Combahee River
Collective Statement of 1977 continues to remind us eloquently.

One factor seems clear about all of the interlocking oppressions. They take
both active forms, which we can see, and embedded forms, which as a member of
the dominant group one is taught not to see. In my class and place, I did not
see myself as a racist because I was taught to recognize racism only in
individual acts of meanness by members of my group, never in invisible systems
conferring unsought racial dominance on my group from birth.

Disapproving of the systems won't be enough to change them. I was taught to
think that racism could end if white individuals changed their attitudes. But a
``white'' skin in the United States opens many doors for whites whether or not
we approve of the way dominance has been conferred on us. Individual acts can
palliate, but cannot end, these problems.

To redesign social systems, we need first to acknowledge their colossal unseen
dimensions. The silences and denials surrounding privilege are the key
political tool here. They keep the thinking about equality or equity
incomplete, protecting unearned advantage and conferred dominance by making
these taboo subjects. Most talk by whites about equal opportunity seems to me
now to be about equal opportunity to try to get into a position of dominance
while denying that systems of dominance exist.

It seems to me that obliviousness about white advantage, like obliviousness
about male advantage, is kept strongly inculturated in the United States so as
to maintain the myth of meritocracy, the myth that democratic choice is equally
available to all. Keeping most people unaware that freedom of confident action
is there for just a small number of people props up those in power and serves
to keep power in the hands of the same groups that have most of it already.

Although systemic change takes many decades, there are pressing questions for
me and I imagine for some others like me if we raise our daily consciousness on
the perquisites of being light-skinned. What will we do with such knowledge? As
we know from watching men, it is an open question whether we will choose to use
unearned advantage to weaken hidden systems of advantage, and whether we will
use any of our arbitrarily awarded power to try to reconstruct power systems on
a broader base.

\end{document}
