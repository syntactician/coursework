\documentclass[man,12pt,natbib]{apa6}
\usepackage{times}

\begin{document}

\title{Exploitation of Female Labor}
\shorttitle{Exploitation of Female Labor}
\author{Edward Hern\'{a}ndez}
\date{24 July 2017}
\affiliation{College of William \& Mary}
\maketitle

\abstract{
	This paper will explore the theoretical motivations for a Marxist analysis
	of sex/gender oppression, explore the ideas of Marxist feminism and its
	application to both reproductive rights and sex work. I will analyze the
	work of Radical Women through this lens, concluding in an evaluation of
	their work and how it may be continued and extended.
}

\section{Introduction}

This paper will explore the theoretical motivations for a Marxist analysis of
sex/gender oppression, explore the ideas of Marxist feminism and its
application to both reproductive rights and sex work. I will analyze the work
of Radical Women through this lens, concluding in an evaluation of their work
and how it may be continued and extended.

\section{Theoretical Background}

\citet{Engels84} argues that gender\footnote{Although Engels largely conflates
	gender and sex, I hear use gender to express his position, given his view
	of oppression as arising socially, rather than biologically} oppression is
like class oppression; men oppress women as the bourgeoisie the
proletariat.\footnote{This is not to say that the oppression of women does not
	predate captalism. On the contrary, Engels' traces historically the
	transition from feudalism to capitalism, and how it produced the
	\emph{modern} system of sex oppression.} On this view, the subordination of
women is maintained because it serves the interest of capital.

The motivation for such an analysis can be found in the work of
\citet{Kollontai09}:
\begin{quote}
	For what reason\ldots should the woman worker seek a union with the
	bourgeois feminists? Who, in actual fact, would stand to gain in the event
	of such an alliance? Certainly not the woman worker.
\end{quote}
Clearly, proletarian women face different struggles than bourgeois women.
Bourgeois women are not necessarily invested (nor even interested) in the
emancipation of proletarian women (let alone the entirety of the proletariat)
from capitalist systems of oppression.

\section{(Re)productive Labor}

Marxist feminists posit two sorts of labor: productive and reproductive.
Productive labor produces capital \citep{Vogel13}. Reproductive labor comprises
that labor which does not produce capital but rather people do for themselves.
This includes having and raising children,\footnote{There can be no labor (from
	which to profit) if there are no new laborers.} but also cooking and
cleaning; these functions are essential to life (and to the continued
generation of capital). Such labor, since it does not directly produce capital,
is often uncompensated.

Women, in being relegated to the domestic sphere, are yolked with reproductive
labor, for which they are both uncompensated and unrecognized. This is no
accident. This relegation allows men (and institutions) access to free labor.
Thus capital (and men, who are relatively privileged under capitalism by this
system) is incentivized to prevent women from seeking (compensated) labor in
the public sphere.

\section{Reproductive Rights}

A Marxist lens is of great use in analyzing the restriction of reproductive
rights. Historically, having a child has further restricted a women to the
domestic sphere, to tend to its raising. Having a child may make it harder for
a woman to access paid work, ensuring that she continues to engage in
uncompensated reproductive labor. This makes clear the capitalist motivation
for restricting contraceptives, abortion, and other reproductive resources.

Marxist feminists (predictably) fight for free access to abortions, allowing
women a greater freedom over their bodies and over whether or not they enter
the public sphere (and engage in the compensated labor that entails). In
addition, some fight for reproductive labor to be compensated, so that women
who do have children and/or do remain in the domestic sphere are not left
reliant on men for access to goods and services
\citep{Bromberg97}.\footnote{This approach is controversial, as many Marxists
	view wage earning as inherently oppressive.}

\section{(Non-reproductive) Sexual Labor}

Clearly, not all sorts of labor available to women are reproductive. While
sexual labor within the institutions of marriage and the nuclear family may be
purely reproductive,\footnote{Even if heterosexual intercourse is not for the
	purpose of procreation, it does not generate capital and is not
	compensated. Labor devoted to pleasure is reproductive in that it is for
	the benefit of the laborers, not for the production of capital.} there
exists compensated sex work. Opinions on sex work vary widely, and are no more
unified among (Marxist) feminists.

While all feminists, Marxists or otherwise, are sure to agree that being forced
into sex work is harmful to women, opinions are divided on whether willing sex
work is harmful (or possible). Marx himself was of the opinion that it was:
``postitution is only a \emph{specific} expression of the \emph{general}
prostitution of the \emph{labourer}\ldots'' \citep{Marx44}, but there is no
question as to whether Marx found wage earning inherently oppressive. On his
view, sex work is yet another way that the laborer is alienated from her labor
and from her nature. Feminists have added additional arguments to his,
including that sex work (especially pornography) circulates ideas and
representations of women that are damaging to women as a group.
\citet{Lerner86} argues additionally that sex markets are historically and
intrinsically linked to the subjugation and literal enslavement of women, while
\citet{Rubin75} links sex markets to historical kinship systems and patriarchal
modes of authority.

Some feminists hold the opposite view: that the sex worker does not permanently
alienate her sexual capacity, but rather merely contracts her sexual labor for
compensation. This may even be compatible with Marx's view on wage earning, if
the sex worker is conceived of as an independent contractor rather than a wage
laborer in the Marxian sense \citep{Bromberg97}. However, even feminists who
hold this view (and argue that laws which prohibit sex work undermine women's
autonomy, e.g. \citealp{Almodovar02}) seem to tend toward supporting
\emph{some} regulation of the sex market, in order to avoid enslavement, human
trafficking, and forced sex work.

\section{Radical Women}

Perhaps the most famous contemporary group of Marxist feminists is Radical
Women, a group operating out of Seattle, WA. The group was formed in 1967, and
has, in the intervening years, worked on anti-war, anti-poverty, anti-racist,
pro-union, and pro-abortion causes. They are affiliated with other feminist and
other Marxist groups, and attempt grassroots political activity aimed at both
institutional and cultural change \citep{RadicalWomen}.

According to their published material, much of their effort is spent on
organizing with other radical groups, forming coalitions and alliances. They
use these alliances to mobilize protests and rallies, push for legislation,
fight for union causes, among other types of political action.

Reproductive rights are prioritized in many of their actions. Their website
documents dozens of political actions in the last two decades seeking and
defending abortion rights. These action include rallies for legislation or
visibility, vigils, solidarity events, and protests. Their goal in all of these
events seems to be to raise group consciousness, inspiring more women of all
backgrounds to join together and effect political change from the ground up.

Despite their prolific body of work, I cannot find any documentation of Radical
Women events about sex work, either about protection of sex workers,
legalization, prevention, or solidarity. It forms a conspicuous absence in the
repertoire of an organization with activism on many fronts. I do not know
whether the issue is avoided because it is divisive (and might hurt the group
if they were to organize around their views), or because there is not consensus
among the women who make up the organization, or whether it is simply not a
priority. All I know is I cannot find a record of their action on the subject.

\section{Conclusion}

While Radical Women shows the viability of radical intersectional politics for
building solidarity and effecting change, they seem conspicuously quiet on the
issue of sex work. While I understand that sex work is a controversial topic,
protecting women from harmful institutions (slavery, patriarchal dominance,
economic dependence, etc.) is tantamount, and I hold that, at the very least,
effort should be taken by the radical left to eradicate the trafficking and
enslavement of women, an issue often dealt with in conjunction with the
regulation of sex markets.

\clearpage
\bibliography{course,extra}

\end{document}
