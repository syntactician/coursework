\documentclass[man,12pt,natbib]{apa6}
\usepackage[colorlinks=false]{hyperref}
\usepackage[super]{nth}
\usepackage{times}

\begin{document}

\title{Feminism and Difference}
\shorttitle{Feminism and Difference}
\author{Edward Hern\'{a}ndez}
\date{14 July 2017}
\affiliation{College of William \& Mary}
\maketitle

% \emph{Due Friday, July 14, by 11:59 pm.}
%
% In this essay, use the readings from the first two weeks of class to define
% feminism and explain some of the complexity and changes within the feminist
% movement over time. In the formulation of your thesis and essay make sure to
% answer the following questions: What is feminism? How has feminism changed
% over time? How have feminists deal with issues of difference (gender, class,
% race, sexuality, political goals, and/or political strategies, etc.)? Please
% use specific examples from at least 2 different ``waves'' of feminist
% activism in answering these questions.
%
% The essay should be 4--5 pages, double spaced, and should include a formal
% thesis and proper citations. Be sure to include references to at least 3
% different readings from class.

\section{Introduction}

Feminism is generally thought of as a movement that organized around particular
causes (suffrage, etc.). I here argue that those waves are as well (if not
better) divided by their conceptions and definitions.

\section{Waves}

The ``first wave'' of feminism is generally thought of as comprising feminist
activities in the \nth{19} and \nth{20} centuries. In general, these women
worked towards legal reform to address women's issues: temperance, suffrage,
and legal rights.

First-wave feminism, since the term was coined, has been defined in opposition
to later waves. \citet{Lear68}, who coined the term, coined ``second-wave'' in
the same article.

Third-wave (as a term) was coined in 1992. \citet{Walker92} used it to refer to
the activities of queer and non-white women.

\section{Difference}

% \nocite{*}
\clearpage
\bibliography{course,extra}

\end{document}
