\documentclass[man,12pt,natbib]{apa6}
\usepackage[colorlinks=false]{hyperref}
\usepackage[super]{nth}
\usepackage{times}

\begin{document}

\title{Feminism and Difference}
\shorttitle{Feminism and Difference}
\author{Edward Hern\'{a}ndez}
\date{14 July 2017}
\affiliation{College of William \& Mary}
\maketitle

% \emph{Due Friday, July 14, by 11:59 pm.}
%
% In this essay, use the readings from the first two weeks of class to define
% feminism and explain some of the complexity and changes within the feminist
% movement over time. In the formulation of your thesis and essay make sure to
% answer the following questions: What is feminism? How has feminism changed
% over time? How have feminists deal with issues of difference (gender, class,
% race, sexuality, political goals, and/or political strategies, etc.)? Please
% use specific examples from at least 2 different ``waves'' of feminist
% activism in answering these questions.
%
% The essay should be 4--5 pages, double spaced, and should include a formal
% thesis and proper citations. Be sure to include references to at least 3
% different readings from class.

\section{Introduction}

``Feminism'' generally refer to movements for the rights of women. More
generally it can refer to any resistance to patriarchal power, to the
oppression of any persons due to their sex, gender, or position under
patriarchy. In the United States, the focus of feminist activism has shifted
several times in the last century.

\section{Waves}

Often, histories of feminism use the metaphor of waves, to describe change
within the movement. The ``first wave'' consists of activism in the \nth{19}
and \nth{20} centuries, largely focused on issues of legal reform, such as
temperance and suffrage. The first wave is often considered as having ended
with the passage of the Nineteenth Amendment to the United States Constitution
in 1919, granting women the right to vote in all states (a right that feminists
had won one year earlier in the UK).

The ``second wave'' describes activities significantly later, beginning in the
1960s. Second-wave feminists oriented themselves toward social as well as legal
equality, concerning themselves not only with passage of laws such as the Equal
Rights Act (ERA), but with social issues such as the objectification of women,
including a landmark protest of the Miss America pageant \citep{Morgan69}.
This pair of concerns is exemplified in the slogan ``the personal is
political'' \citep{Hanisch70,CombaheeRiverCollective77,Lorde84}.

Though many of the concerns facing second-wave feminists persist, a new,
``third'' wave began in the 1990s. Feminists in this wave challenge the
essentialism which persisted through the first two waves, encompassing new
understandings of sex and gender, as well as attempting to address the
experiences of queer and non-white women.

\section{Difference}

Suffragists, in their rhetoric, tended towards essentialism and universalism,
dismissing difference outright. \citet{Stanton48}, in writing the
\emph{Declaration of Sentiments} typifies this approach in her alterations to
the Declaration of independence. She simply extends the Declaration's
provisions to women as well as men, arguing that rights are universally
inalienable, regardless of sex.  Rather than suggesting that only women need to
be liberated, \citet{Harper95} suggests that everyone, male or female, will
benefit from feminism: ``So close is the bond between men and women that you
can not raise one without lifting the other.''

The issue of suffrage is framed by many of these feminists (Stanton among them)
as an issue of universality. Despite this approach, the movement fractured
around the issue. There was much dispute among feminists about strategy. The
American Women's Suffrage Association (AWSA) split from the National Women's
Suffrage Association (NWSA) over (among other disputes), a difference of
opinion over the Fifteenth Amendment. Additionally, established feminists (like
AWSA) sought relatively slow reform, pushing for the right to vote on a
state-by-state basis. A younger generation of feminists (AWSA, the NWP, etc.),
influenced by the approach of the British suffragettes, pushed for total reform
at a national level: a constitutional amendment.

While these groups used universalist rhetoric, they were unable to agree. Women
of many backgrounds, with many identities, tried to agree, to work together.
While they did eventually accomplish their ultimate goal, their fragmentation
showed that the variety of women's experiences cannot be adequately captured by 
a single narrative, nor their concerns addressed by a single approach.

``Second-wave'' feminists, responding to largely different concerns, used
different rhetorical and political approaches. Very little of their rhetoric
equates men and women. The highlight how the conditions under which men and
women live are fundamentally different. Men and women are presented with
different messages \citep{Morgan68}, they were faced with different
expectations, different labor \citep{Susan70}, and their experiences were
additionally different based on race and class
\citep{CombaheeRiverCollective77}. Consciousness-raising in particular dealt
explicitly with difference. Though it was still a goal to identify common
struggles around which to organize, there was a realization that to find truly
common experiences, there would have to be conversation among diverse
communities of women \citep{CombaheeRiverCollective77}. While this approach
does deal with difference (whereas the first wave avoided or ignored it), the
differences to be explored were the differences between men and women.
Essential differences among women (and their respective struggles) were still
emphasized.

% In addition to fighting against objectification and other cultural
% inequalities, they also sought legal reform. The ERA (and its near passage) 

Feminists of the ``third wave'' break away from this essentialism.
\citet{Butler88} writes about gender being performative, implying, among other
things, that there is nothing in particular essential to women. If `woman' is a
role to be played, what women have in common is their performance, not
something about their being. This is a radical departure from the ideas of the
first and second wave, and allows for much more radical activism. Rather than
resisting the \emph{position} of women under patriarchy, as the first and
second waves had done, this literature poises feminists to resist patriarchy
\emph{itself} from the root; if gender is in question, on what grounds can
patriarchy operate? 

Trans activists and academics, like \citet{Stryker07}, resist patriarchy in
just this way (both with their bodies and their writings). Their work directly
breaks down the boundaries of categories like ``woman,'' radically expanding
expanding the possibilities of lived experience.

However, the third wave has had problems with difference as well. There are of
course, those who still espouse essentialist views, who find themselves at odds
with these new views. There are also debates over sex positivity, disputes over
sex work, pornography, and any number of strategic differences.

\section{Conclusion}

While feminism may have started as a movement to win women the right to vote,
it has expanded in ways that were then unforeseen (and unforeseeable). It now
encompasses women from all backgrounds, races, sexualities, classes, and
bodies. It questions patriarchy and gender at its core. It questions the very
concept of personal identity, invalidates patriarchy and other oppressive
systems at their core. It fights for the rights of persons oppressed under
patriarchy, and its boundaries and definitions are constantly changing,
constantly in flux. And though it has historically suffered from the
unacknowledged difference among its members, and now suffers to find common
ground on which to fight, its flexibility, its malleability, may be its
greatest strength, especially now that it contains such a wealth of lived
experience.

% \nocite{*}
\clearpage
\bibliography{course,extra}

\end{document}
