\documentclass[man,12pt,natbib]{apa6}
\usepackage[colorlinks=false]{hyperref}
\usepackage{amssymb,amsmath,times,CJKutf8}
\linespread{1.5}

\begin{document}

\title{Linguistic Anthropology Final}
\shorttitle{Linguistic Anthropology Final}
\author{Edward Hern\'{a}ndez}
\date{17 December 2015}
\affiliation{College of William \& Mary}
\maketitle

\begin{quote}
	% The goal of the exam is to get you to take some perspective on what we
	% have read and discussed this semester and to give you the opportunity to
	% display your understanding of the general theoretical issues raised in
	% the course and to apply that understanding in a final essay.

	% Essay length 1500-2000 words.  You have until 12 noon on the 17th to
	% email your essay to me.  No extensions are possible.

	% Choose one of the following prompts for your essay:

	% 1.
	% \citet{Agar95} presents the discipline of linguistic anthropology as
	% based on `a way of seeing' which contrasts with the way of seeing in core
	% linguistics (what he calls ``inside-the-circle'' linguistics). Core
	% linguistics views language as a formal system consisting in arbitrary
	% patterns (phonological, lexical, morphological, syntactic, etc.) and
	% views knowledge of such a system as what knowing a language consists in
	% (i.e., knowing English, Japanese, Swahili, Apache, etc.).  How do our
	% readings and class discussions this semester support the comparative
	% validity of linguistic anthropology's way of seeing and studying
	% language, as compared to that of core, ``inside-the-circle'' linguistics?
	% To what extent is Agar's conclusion justified that ``the circle is a
	% lie'' (p.~16) and ``the circle has to go'' (p.~30).  Support your answer
	% with extensive discussion of at least three of the articles we have read
	% since Fall Break (not including the article you presented). 

	% 2.
	Several times this semester it has been suggested that the properties of
	a particular culture's language are inseparable from the ways which that
	culture thinks and talks about its language (i.e., from their language
	ideology) and that, therefore, there is no objective, ideology-free way for
	a linguist to describe a language.  Do you agree?  Why?  Support your
	answer with discussion of at least three of the articles we have read since
	Fall Break (not including the article you presented). 

	% 3.
	% In his Coral Gardens and their Magic, % Malinowski \citet{Malinowski35}
	% says that ``the real linguistic fact is the full utterance within its
	% context of situation'' On page 60 of the section entitled ``Meaning as a
	% function of words'', he continues ``In order to show the meaning of words
	% we must not merely give sound of utterance and equivalence of
	% significance. We must above all give the pragmatic context in which they
	% are uttered, the correlation of sound to context, to action and to
	% technical apparatus; and incidentally, in a full linguistic description,
	% it would be necessary also to show the types of cultural drill or
	% conditioning or education by which words acquire meaning.'' Do you think
	% the readings we have discussed this semester have given sufficient
	% support for these claims by Malinowski?  Why?  Support your answer with
	% extensive discussion of at least three of the articles we have read since
	% Fall Break (not including the article you presented).  
\end{quote}
\clearpage

Language ideology is inseparable from language. Cultures and language are
invariably tied up in thought and talk about language. To fully describe a
language---and to describe at all a \emph{languaculture} as \citet{Agar95}
conceives of it---a linguist(ic anthropologist) must make reference to
ideology.

\citet{Morford97} describes the use of the French pronouns \emph{tu} and
\emph{vous}. She describes who is more likely to use each in particular
situations. Some of that description can be done without explicit reference to
ideology. Young speakers are more likely to use \emph{tu}. Speakers are likely
to use the same pronominal form as their interlocutor. Social class and
political orientation predict which form will be used. However, to truly
understand \emph{why} these facts are true, to understand the use of \emph{tu}
and \emph{vous} robustly, one must appeal to ideology.

Why does a French speaker use one pronominal form and not another? Why do some
speakers feel strongly about the declining use of \emph{tu}? Why does Morford
feel justified in claiming that the use of the T or V form is linked to power
dynamics? The answer to all of these questions is language ideology. Ideology
informs our choice of words. It disposes us to prefer some words over forms
over others. It allows for value judgements like morality to be attached to the
use of particular forms. To understand why a French speaker uses a particular
form, we must understand what it means \emph{to them}, referentially and 
pragmatically---something we cannot know without taking into account the 
ideologies which inform the speaker's understandings.

This is extremely visible in Japanese. \citet{Clancy87} discusses the
acquisition by Japanese speakers of what is called ``communicative style.''
Communicative style, as defined by \citet{Barlund75}, includes not only the
words that speakers use, but what topics speakers choose to discuss, what form
of interaction they choose to engage in, how they convey information, and how
attuned they are to listeners. As with the French T/V distinction, it is
possible to explore parts of this topic without making reference to ideology.
I might, for instance, study which topics young speakers avoid, which speakers
use particular words over others. I could answer questions about statistical
prevalence. I could not, however, get at why these phenomena exist.

In order to answer these questions, \citet{Clancy87}, like \cite{Barlund75}, 
refers quite explicitly to ideology. She discusses \emph{omoiyari}
\begin{CJK}{UTF8}{min}(思いやり)\end{CJK}, a trait widely held to be a cultural
value in Japanese society. It consists, she argues, in practicing empathy, but
also expecting that others will be empathetic. She holds that it effects the
style of Japanese communication, allowing for such phenomena as negation at the
very end of an utterance. If I could not expect listeners to be benevolent and
empathetic, the argument goes, I could not negate an assertion at the end. If
this is true, then the very structure of Japanese is influenced by Japanese
language ideology.  Whether or not this is literately true, the ideology of
\emph{omoiyari} informs Japanese language behavior and communicative choices,
and without an understanding of this culturally held language ideology, a
linguist would be hard-pressed to explain Japanese word choice without
appealing to this shared understanding.

\citet{Nevins04} describes a case in which language ideologies becomes quite
explicit: a clash of ideology between Apache speakers and linguists over this
very issue. The linguists, in their language maintenance program, teach the
Apache language as a formal system. We, as linguists, may think of this as an
``ideology-free'' way of describing a language. As far as we're concerned, a
language formally has particular characteristics: it is synthetic or analytic,
it has, say, VSO or SVO word order, ect. These, on our view, are simply facts.
However, as the White Mountain Apache teach \citet{Nevins04}, this sort of
description stems from the particular language ideologies which we have as
linguists, which do not always line up with the language ideologies of the
cultures whose languages we study.

In particular, the Apache balk at our way of teaching a language. While we
might feel comfortable saying that \emph{nakih} means two or \emph{shash} means
bear, some Apache speakers feel that this misses the point. \emph{Shash} does
not mean bear, to them. It means a ``vital presence moving  . . . in the
mountains'' \citep[p.~283]{Nevins04}. Whether or not the Apache have some 
drastically different mental representation of word meanings is, to a large 
extent, irrelevant. What matters here is that they feel that their language
is incompatible with our tools for teaching it, that our pedagogical methods
are insufficient or inappropriate. To learn Apache is something completely 
other than learning the formal system we would teach. It is chopping wood, or
baking bread. It is participating in Apache culture. While these things seem to
us unrelated to language proficiency, the Apache see them as crucial.

To describe the Apache language as a formal system only \emph{seems}
ideology-free. If you understand that this description is at odds with Apache
understandings, that it runs counter to their pedagogical tradition, but you
continue to record it formally, as if there were no ideologies at play, you are
exercising your own ideologies. This shows an ideology not just that language
is a system that can be formally described, that our scientific view of
language is objectively correct, etc., but also that our ways of knowing, our
ways of thinking about language are more valuable, more worthy than native ways
of knowing. This is, I think, a slippery slope. If we are to disregard language 
ideology in our account of language, where do we stop? Do we exclude all of
culture from our account, because it is in some sense not identical to what we
think knowledge of a language consists in? If we think linguistic anthropology
has value in addition to core linguistics (syntax, phonetics, etc.), then we
must include language ideology in our accounts of languages.

\clearpage
\bibliography{course,extra}

\end{document}
