\documentclass[man,12pt,natbib]{apa6}
\usepackage[colorlinks=false]{hyperref}
\usepackage{amssymb,amsmath,times,CJKutf8}
\linespread{1.5}

\begin{document}

\title{Linguistic Anthropology Final}
\shorttitle{Linguistic Anthropology Final}
\author{Edward Hern\'{a}ndez}
\date{17 December 2015}
\affiliation{College of William \& Mary}
\maketitle

\begin{quote}
	The goal of the exam is to get you to take some perspective on what we have
	read and discussed this semester and to give you the opportunity to display
	your understanding of the general theoretical issues raised in the course
	and to apply that understanding in a final essay.

	Essay length 1500-2000 words.  You have until 12 noon on the 17th to email
	your essay to me.  No extensions are possible.

	Choose one of the following prompts for your essay:

	1. % Agar
	\citet{Agar95} presents the discipline of linguistic anthropology as based
	on `a way of seeing' which contrasts with the way of seeing in core
	linguistics (what he calls ``inside-the-circle'' linguistics). Core
	linguistics views language as a formal system consisting in arbitrary
	patterns (phonological, lexical, morphological, syntactic, etc.) and views
	knowledge of such a system as what knowing a language consists in (i.e.,
	knowing English, Japanese, Swahili, Apache, etc.).  How do our readings and
	class discussions this semester support the comparative validity of
	linguistic anthropology's way of seeing and studying language, as compared
	to that of core, ``inside-the-circle'' linguistics?  To what extent is
	Agar's conclusion justified that ``the circle is a lie'' (p.16) and ``the
	circle has to go'' (p.30).  Support your answer with extensive discussion
	of at least three of the articles we have read since Fall Break (not
	including the article you presented). 

	2. Several times this semester it has been suggested that the properties of
	a particular culture's language are inseparable from the ways which that
	culture thinks and talks about its language (i.e., from their language
	ideology) and that, therefore, there is no objective, ideology-free way for
	a linguist to describe a language.  Do you agree?  Why?  Support your
	answer with discussion of at least three of the articles we have read since
	Fall Break (not including the article you presented). 

	3. In his Coral Gardens and their Magic, % Malinowski
	\citet{Malinowski35} says that ``the real linguistic fact is the full
	utterance within its context of situation'' On page 60 of the section
	entitled ``Meaning as a function of words'', he continues ``In order to
	show the meaning of words we must not merely give sound of utterance and
	equivalence of significance. We must above all give the pragmatic context
	in which they are uttered, the correlation of sound to context, to action
	and to technical apparatus; and incidentally, in a full linguistic
	description, it would be necessary also to show the types of cultural drill
	or conditioning or education by which words acquire meaning.'' Do you think
	the readings we have discussed this semester have given sufficient support
	for these claims by Malinowski?  Why?  Support your answer with extensive
	discussion of at least three of the articles we have read since Fall Break
	(not including the article you presented).  
\end{quote}

\clearpage
\bibliography{course,extra}

\end{document}
