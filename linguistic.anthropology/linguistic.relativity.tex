\documentclass[doc,12pt]{apa6}
\usepackage{lmodern}
\usepackage{amssymb,amsmath}
\usepackage{ifxetex,ifluatex}
\usepackage{fixltx2e} % provides \textsubscript
\ifnum 0\ifxetex 1\fi\ifluatex 1\fi=0 % if pdftex
  \usepackage[T1]{fontenc}
  \usepackage[utf8]{inputenc}
  
\usepackage[usenames,dvipsnames]{color}

\usepackage{graphicx,grffile}
\graphicspath{ {.resources/} }
\makeatletter
\def\maxwidth{\ifdim\Gin@nat@width>\linewidth\linewidth\else\Gin@nat@width\fi}
\def\maxheight{\ifdim\Gin@nat@height>\textheight\textheight\else\Gin@nat@height\fi}
\makeatother
% Scale images if necessary, so that they will not overflow the page
% margins by default, and it is still possible to overwrite the defaults
% using explicit options in \includegraphics[width, height, ...]{}
\setkeys{Gin}{width=\maxwidth,height=\maxheight,keepaspectratio}
\setlength{\parindent}{0pt}
\setlength{\parskip}{6pt plus 2pt minus 1pt}
\setlength{\emergencystretch}{3em}  % prevent overfull lines
\providecommand{\tightlist}{%
  \setlength{\itemsep}{0pt}\setlength{\parskip}{0pt}}
\setcounter{secnumdepth}{0}

\usepackage[colorlinks=false]{hyperref}
\usepackage{dirtytalk}
\usepackage{apacite}

\newcommand\blfootnote[1]{%
  \begingroup
  \renewcommand\thefootnote{}\footnote{#1}%
  \addtocounter{footnote}{-1}%
  \endgroup
}

\begin{document}

\title{There Is No Question of Linguistic Relativity}
\shorttitle{linguistic relativity}
\author{Edward Hern\'{a}ndez}
\date{2015-09-20}
\affiliation{College of William \& Mary}
\maketitle

\blfootnote{This is, of course, to steal the title and central idea of Crane \& Mellor (1990).}
\nocite{Crane90}

For clarity, a few key terms will be defined here, and their use throughout this paper 
will not deviate from these definitions. \say{A language} will here mean the a particular 
spoken or signed system of human communication which in shared by a group of people who 
also share a culture.\footnote{By this definition, most variations usually described as 
dialects are languages (e.g. African American English is a language distinct from other 
variations of English).} To \say{think} will here mean to form and use concepts or 
categories, and will not concern such primitive cognitive events as sensory perception 
itself. \say{Linguistic Relativity} is the claim that a speaker's thought is influenced 
by the language which she speaks. The \say{weak} form of this claim is simply that 
features of a speaker's particular language is correlated
% …features….are correlated
with her thoughts. Neither of these claims is identical to \say{Linguistic Determinism,} 
the claim that a speaker's language limits or determines her thoughts.

Many linguists make the claim that Linguistic Relativity is true. 
% Actually, ‘in the Circle’ linguists (grammarians, phonologists) on the whole argue 
% against it, whereas linguistic anthropologists and sociolinguists accept it.
They offer a variety of evidence. Evidence includes that speakers of Pirah\~{a} do not 
speak of cardinal numbers, and they also have difficulty on tasks requiring the 
transposition or recollection of numerosity \cite{Everret12}. Similarly, the 
Guugu-Yimidhirr do not speak of egocentric directions like \say{left} or \say{right,} 
but rather of cardinal directions like \say{north} and \say{south,} regardless of scale 
\cite{Foley97}.\footnote{I'm not entirely sure this citation is correct, but I did the 
best I could based on the excerpt I had.} Both Pirah\~{a} and Guugu-Yimidhirr appear 
to limit their speakers to these modes of thought. Pirah\~{a} does not have lexical 
items identical to numbers as we commonly understand them, and Guugu-Yimidhirr lacks 
words for left and right. 
% And front and back.
Less extreme examples can be seen in other languages as well; for instance, English 
lacks formality and number distinctions in the second person, while Spanish has both. 
% Are you suggesting that there are differences in how English and Spanish speakers 
% think that are the consequence of these linguistic differences in their 2nd person 
% pronouns? Given where you have placed this remark, that would seem to be implied. 
% But if it is, then you should make the point explicitly.
This evidence in the literature makes it clear that speakers of different languages do, 
in fact, think in different ways. It seems to suggest that these ways of thinking are 
\textbf{caused} by the differences in the languages. The argument is that 
Guugu-Yimdihirr lack the concepts of left and right due to their absence in their 
language, and this causes them 
% …and, as the experimental evidence appears to show, this causes them….
to utilize instead their concepts of east and west.

While this seems like controvertible evidence, 
% I think you mean ‘incontrovertible’: impossible to deny.
it seems so only because the authors have begged the question. They have made the claim 
that these differences in thought arise from the differences in the language, rather 
than other differences. The Pirah\~{a} are hunter-gatherers in an environment which 
largely does not require counting or numeration. A proponent of Linguistic Relativity 
would claim that for the Pirahã, their cultural lack of numeration tasks caused a lack 
of numeration terms in the language, which causes Pirahã speakers to think in other 
terms. Likewise, the utility of cardinal directions in open countryside caused 
Guugu-Yimidhirr to develop as it is, and the language causes its speakers to think of 
direction cardinally.
% Well put.

If this is the case, if differences in the language are truly what cause differences 
in thought, then introducing speakers of Pirahã to words for numbers might be sufficient 
to allow them to think of numerosity. Certainly Pirah\~{a} who were exposed to Portuguese 
would have concepts of numbers, and be able to employ them. However, \citeA{Gordon04} 
suggests that Pirah\~{a} who have contact with Portuguese speakers and understand it to 
some degree do not understand the embedded number words (p. 497), and \citeA{Everett12}, 
despite attempting to supplement the Pirah\~{a} language with number words, were largely 
unsuccessful at training the Pirah\~{a} to transpose or remember numbers. 
% E&M only administered the quantity-matching tests (although this included the ‘brief 
% presentation’ or ‘hidden match’ test).  They did not administer the ‘nuts in a can’ or 
% ‘candy in a box’ tasks, which are the ones which, according to Frank et al, depend the 
% most on memory and transpositional abilities.
Strikingly, these Pirah\~{a} have access to a language\footnote{It is a pidgin, but by 
the above definition, no less a language.} with number words, but still do not think 
numerically. 
% Although they did perform better on the quantity-matching tasks than members of 
% villages where no such number training had occured.
This shows that there must be something beyond the limits of the language to which they 
have access causing this particular difference in their mode of thought. I argue that 
it is culture, influencing thought directly, without language as a mediating variable; 
rather than culture influencing language and language subsequently influencing thought, 
culture directly affects both language and thought, making the influence of language on 
thought trivial.
% I like this argument, although, to be complete, it needs development in the form of an 
% explanation of the means by which culture affects thought.  What, indeed, is meant by 
% ‘culture’ here?  What is certainly true is that ‘culture’ is a potentially confounding 
% variable in the Piraha studies, whose presence weakens the interpretation of the 
% experimental results as caused by the influence of language. 

For \citeA{Agar95}, language cannot be separated from culture. This means that a 
difference in the way two groups of people think may be a result of differences between 
their respective languages or differences between their respective cultures, but it is 
impossible to tell which (or, perhaps, meaningless to do so). If we adopt this view, 
that language and culture are necessarily intertwined, the question of Linguistic 
Relativity is absurd and the research on the topic even more so. It is meaningless to 
conduct a study about perceptions of the color blue and attribute findings to the 
difference between Russian and English as languages. The differences obviously have just 
as much to do with cultural discourses of color: speakers of Russian are raised with 
differing descriptions of light and dark blue as a part of their culture. It is not 
possible to disentangle those experiences from their understanding of language and their 
formation of mental categories. Similarly, claims that Guugu-Yimidhirr, the language, 
prevents its speakers from thinking in terms of left and right miss that it may be the 
culture itself which prevents them, if anything does, from thinking in that way.

In conclusion, there is no non-trivial sense in which Linguistic Relativity is true. The 
language and thought of a particular person are correlated, as proven in the literature. 
Thus, the weakest possible formulation of Linguistic Relativity is true. However, it is 
trivially true. As a finding, it tells us nothing about the working of thought or 
language. There is no way to establish from this correlation a causal mechanism and no 
hope of eliminating confounds. 
% Good point.
The only way to salvage Linguistic Relativity as a claim is to incorporate culture as an 
inalienable element of language, which is to change completely what is meant by the term
. Thus, I reject Linguistic Relativism as a useful or accurate hypothesis. This does not 
mean that I accept pre-Boas understandings of the world, or think that all humans have 
practically identical thoughts labelled differently by their languages. Instead, I 
advocate a hypothesis and a research agenda what I will, in honor of Agar, call 
Languacultural Relativity.\footnote{Even though it sounds really horrible} That is to say
, I acknowledge that different groups of people have vastly different cultures. They 
speak a wide variety of languages which possess dissimilar structures and lexicons. I 
also think that they experience incomparable cultures. Those cultures undoubtably 
%undoubtedly
impact their languages but likewise influence their thought directly. To tease apart 
these causes is impossible. They must be studied together, and that is how their study 
will be most valuable. To tell us about the variety of human language and human thought, 
the linguistic anthropologist must not simply do linguistics; she must do anthropology.

% Excellent, Edward: indeed, the best paper I think I’ve ever received on linguistic 
% relativity – and I’ve had lots of them over the years.  Moreover, most linguistic 
% anthropologists today would agree with you, that is, those who, like Agar, don’t buy 
% into the ‘inside the Circle’ concept of a language.  
%
% Grade: A+
%
% You’re doing a good job in class discussion as well.

\clearpage

\bibliography{linguistic.anthropology}
\bibliographystyle{apacite}

\end{document}
