\documentclass[man,12pt,natbib]{apa6}
\usepackage[colorlinks=false]{hyperref}
\usepackage{amssymb,amsmath,times,CJKutf8}
\linespread{1.5}

\begin{document}

\title{Critique of \emph{The acquisition of communicative style in Japanese} by
	P. Clancy (1987)}
\shorttitle{Critique of Clancy (1987)}
\author{Edward Hern\'{a}ndez}
\date{7 December 2015}
\affiliation{College of William \& Mary}
\maketitle

% Each student will choose two of the articles assigned after Fall Break (other
% than an article that you presented in class) and provide a written critique
% (3pps.)   A critique includes a summary of the main argument of a text as
% well as discussion of an author’s use of particular terms or concepts, and a
% reasoned evaluation of the overall strength of their argument in light of
% other readings.  The first critiques is due within one week after the class
% in which the article was discussed.  The second critique is due on the last
% day of classes.  Please either submit critiques on paper or as an emailed
% Word attachment. 

\section{Summary}

\citet{Clancy87} claims that ``[t]he particular communicative style of a
culture arises from shared beliefs about people, what they are like, and how
they should relate to one another'' \cite[p.~245]{Clancy87}, and, further, that
this communicative style ``perpetuat[es] those beliefs \cite[p.~245]{Clancy87}.
From her data, she argues that in the case of Japanese children, this style is
acquired largely from the speech of the mother, and that, in this case, style
is often taught explicitly and intentionally.  She claims that it is largely
communicative style which is used to transmit ``cultural values to children''
\cite[p.~246]{Clancy87}.

\section{Terms}

Clancy relies on the idea of ``communicative style,'' as discussed by
\citet{Barlund75} (in another of Japanese speech) According to Clancy,
Barlund's definition ``[includes] the topics people discuss, their favorite
forms of interaction, the depth of involvement sought, the extent to which they
rely upon the same channels for conveying information, and the extent to which
they are tuned to the same level of meaning, such as factual versus emotional
competence'' \citep[p.~213]{Clancy87}.
Clancy also ``loosely'' defines the term as she uses it: ``the way language is
used and understood in a particular culture'' \citep[p.~213]{Clancy87}.

It seems to me that this is not as clear as it could (or should) be.  ``The way
language is used'' is, obviously, a topic worthy of linguistic study. However,
it is not clear to me that this is comprised of ``the way language is used and
understood'' is comprised of such disparate elements as ``the depth of
involvement sought'' and ``factual versus emotional competence.'' At least, I
don't think these aspects of language use are identifiable through (purely)
linguistic inquiry, even with an anthropological bent. Some of these questions
seem to me to require cognitive and (inter/ethno)cultural psychology in
addition to methods and theories in linguistics. To what extent is
conflict-avoidance an element of \emph{communicative} style? It is well
attested that avoiding conflict and maintaining face are culturally important
to the Japanese \citep{Benedict46,Ohbuchi94}. If a businessman avoids a
conversation altogether to avoid the conflict it might contain, is that part of
communicative \emph{style}? I find it hard to conceptualize the boundaries of
the concept as it is described here.

Clancy also claims that communicative style is ``obviously'' ``one aspect of
`communicative competence''' \citep[p.~213]{Clancy87}, which she links to the
idea of ``rules for use'' \citep{Hymes72}, which ``govern speakers' production
. . .  of language appropriately in context'' \citep[p.~213]{Clancy87}.  I am
somewhat confused by this claim. I am not entirely sure to what effect it is
being employed. Is this just to say that communicative style is socially
consequential?  She similarly invokes the ideas of social cognition, world view
\citep{Whorf56}, and ``reality sets[s]'' \citep{Scollon81}, without making
their use in her analysis particularly clear.

In the context of Japanese communicative style (and world view) she discusses
\emph{omoiyari} \begin{CJK}{UTF8}{min}(思いやり)\end{CJK} which she defines
simply as ``empathy'' \citep[p.~214]{Clancy87}. She holds that \emph{omoiyari}
is ``emphasize[d]'' by ``a set of cultural values'' which ``[set] the basis of
[the] style'' of Japanese communication \cite[pp.~213-214]{Clancy87}. She
identifies these values as dispreference for verbosity, indirection, conflict
avoidance, and unanimity \citep[p.~214-216]{Clancy87}.

This is less than ideal. \emph{Omoiyari} is not a particularly simple word (to
translate), as it carries a lot of cultural weight. ``Empathy'' is not an
\emph{incorrect} translation \citep{Travis98}, but it would be just as correct
to say sympathy, or compassion, or ``altruistic sympathy'' \citep{Hara06}.
While \emph{omoiyari} is indeed regarded as a central cultural value in Japan,
it is far from the only value, and it would be a mistake to attribute the whole
of the Japanese communicative style to it. In particular, I think it would be a
mistake to attribute conflict-avoidance to an expression of this trait. Other
studies have cited conceptions of interdependence or collectivism as a probable
cause for this behavior \citep{Ohbuchi94}, and I think it would be a mistake to
conflate those concepts with \emph{omoiyari} without a compelling reason.

\section{Evaluation}

I am concerned that this paper does not account for gender differences.
Japanese is spoken quite differently by male and female speakers. Particular
words are considered more masculine or more feminine than others, and there are
gendered sentence-final particles \citep{Siegal03}. However, these are not the
only differences. Men tend to more assertive (not omitting the copula
\emph{da}, using fewer polite forms, etc.), exhibit less exhibit fewer
instances of backchannel communication (\emph{aizuchi} forms;
\citealp{Kita07,Tanaka04}), and display a ``self-oriented'' speaking style
(contrary to Clancy's claims about \emph{omoiyari}; \citealp{Itakura04}).
Obviously, these traits all fall within the umbrella of communicative style.
How do the female participants look different from the male one? Did they in
this study? I think that data is of interest, and is certainly important to
report.

Additionally, I am uncomfortable with discussion of Japan as a homogeneous
society. Conformity is often discussed as a primary Japanese cultural value,
often foregrounding \emph{omoiyari} as central to Japanese uniformity. Clancy
says: ``the individual is seen primarily as a member of a social group, with a
responsibility to uphold the interests of that group. Thus arises the need for
empathy and conformity, which help to preserve group harmony and group
values.'' I am not so sure that Japanese conformity is a benevolent, empathetic
phenomenon.  Despite the strong national discourse of homogeneity, Japanese in
a multi-ethnic society with strong minority communities \citep{Lie01} This
discourse is mobilized against these minorities in racist, xenophobic ways
which serve to marginalize those who do not conform \citep{Diene06}. To take
this discourse at face value and accept the common attitude that it as rooted
in empathy is, I think, to disregard the social realities of Japan.

\clearpage
\bibliography{course,extra}

\end{document}
