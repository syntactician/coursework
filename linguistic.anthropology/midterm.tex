\documentclass[doc,12pt]{apa6}
\usepackage{lmodern}
\usepackage{amssymb,amsmath}
\usepackage{ifxetex,ifluatex}
\usepackage{fixltx2e} % provides \textsubscript
\usepackage[T1]{fontenc}
\usepackage[utf8]{inputenc}
  
\usepackage[usenames,dvipsnames]{color}

\makeatletter
\def\maxwidth{\ifdim\Gin@nat@width>\linewidth\linewidth\else\Gin@nat@width\fi}
\def\maxheight{\ifdim\Gin@nat@height>\textheight\textheight\else\Gin@nat@height\fi}
\makeatother
% Scale images if necessary, so that they will not overflow the page
% margins by default, and it is still possible to overwrite the defaults
% using explicit options in \includegraphics[width, height, ...]{}
\setkeys{Gin}{width=\maxwidth,height=\maxheight,keepaspectratio}
\setlength{\parindent}{0pt}
\setlength{\parskip}{6pt plus 2pt minus 1pt}
\setlength{\emergencystretch}{3em}  % prevent overfull lines
\providecommand{\tightlist}{%
  \setlength{\itemsep}{0pt}\setlength{\parskip}{0pt}}
\setcounter{secnumdepth}{0}

\usepackage[colorlinks=false]{hyperref}
\usepackage{dirtytalk}
\usepackage{apacite}

\begin{document}

\title{Linguistic Anthropology Midterm}
\shorttitle{Midterm}
\author{Edward Hern\'{a}ndez}
\date{2015-09-20}
\affiliation{College of William \& Mary}
\maketitle

\subsection{Required}\label{required}

\begin{enumerate}
\def\labelenumi{\arabic{enumi}.}
\tightlist
\item
  List the 6 factors of the speech event and their corresponding
  functions. For each one, supply an illustrative example and an
  explanation of how the example fulfills the function in question.

  \begin{itemize}
  \tightlist
  \item
    Context --- Referential

    \begin{itemize}
    \tightlist
    \item
      ``I'm sitting in a chair.''
    \item
      Referential meaning has to do with semantic meaning, and how the
      utterance refers to things. Here, ``I'' refers to a person,
      ``chair'' refers to an object, and ``sitting'' refers to our
      relation.
    \end{itemize}
  \item
    Message --- Poetic

    \begin{itemize}
    \tightlist
    \item
      ``Dont sing,\\
      don't do anything,\\
      just drive.''
    \item
      This utterance has parallel structures and rhyme between the first
      two clauses, which serve to unify them and distinguish the third
      and final clause, which contains the message. It uses properties
      of language to produce or accentuate the message.
    \end{itemize}
  \item
    Addresser --- Expressive

    \begin{itemize}
    \tightlist
    \item
      ``Oh''
    \item
      In the ``Trip to Syracuse'' phone call, Ilene expresses her
      emotional state upon realizing that Charlie is not, in fact,
      taking her to Syracuse.
    \end{itemize}
  \item
    Addressee --- Conative

    \begin{itemize}
    \tightlist
    \item
      ``How are you?''
    \item
      A conative utterance involves the addressee. All second-person
      speech qualifies.
    \end{itemize}
  \item
    Mode --- Phatic

    \begin{itemize}
    \tightlist
    \item
      ``Hello?'' -- in a phone call
    \item
      During a phone call, it is necessary (or at least customary) to
      establish that your interlocutor is connected to the line and able
      to hear you clearly. This is a primary function of the phone
      ``hello,'' which is particular to the phone as a mode of
      communication. Both Charlie and Ilene do this during their phone
      call.
    \end{itemize}
  \item
    Code --- Metalinguistic

    \begin{itemize}
    \tightlist
    \item
      ``I talked to the girl.''
    \item
      The example discusses language itself. Charlie is discussing a
      previous linguistic activity.
    \end{itemize}
  \end{itemize}
\end{enumerate}

\subsection{Answer 6 of the following:}\label{answer-6-of-the-following}

\begin{enumerate}
\def\labelenumi{\arabic{enumi}.}
\setcounter{enumi}{1}
\tightlist
\item
  By reference to an example of a sign, explain the difference between
  an utterance which has a metasemantic function and an utterance with a
  metapragmatic function.

  \begin{itemize}
  \tightlist
  \item
    ``What do you mean?''

    \begin{itemize}
    \tightlist
    \item
      I might ask someone this if I fail to understand the propositional
      or referential content of their last statement. I'm asking about
      the semantic meaning of that utterance.
    \end{itemize}
  \item
    A: ``Can you make it tomorrow morning?''\\
     B: ``I'm busy.''\\
     A: ``So, you can't make it?''

    \begin{itemize}
    \tightlist
    \item
      The final question in this exchange is about not the propositional
      content of the second, but rather what it was meant to convey
      pragmatically. ``I'm busy.'' does not just convey business, but
      rather a denial of the question of the first utterance; the third
      utterance metapragmatically addresses that meaning.
    \end{itemize}
  \end{itemize}
\item
  By reference to an example, explain how it signifies by more than one
  sign mode.

  \begin{itemize}
  \tightlist
  \item
    ``Woof woof''
  \item
    This sign signifies both as icon and as index. It signifies as icon
    the sound of a dog, suggesting the dog itself. It also (like much
    onomatopoeia) indexes childishness.
  \end{itemize}
\item
  By reference to an example, explain why deictic indexes are called
  `shifters.'

  \begin{itemize}
  \tightlist
  \item
    Deictic indexes (e.g. ``here,'' ``now'') are context dependent. To
    understand what you mean by ``I am here, now,'' I have to know both
    where and when the utterance occurred. The meaning changes (or
    ``shifts'') if it is uttered elsewhere or at a different time.
  \end{itemize}

\setcounter{enumi}{5}
\item
  Explain the difference between what Malinowski calls the `pragmatic
  function' and the `intellectual function' of the speech event.

  \begin{itemize}
  \tightlist
  \item
    For Malinowski, the intellectual function of a speech event consists
    in its semantico-referential function, or what it conveys as
    information to the hearer. Its pragmatic function is ``the change
    produced by the sound in the behavior of people,'' of what it is
    that the speech act does, rather than conveys.
  \end{itemize}
\item
  How does Agha define a ``register?''

  \begin{itemize}
  \tightlist
  \item
    Agha defines a register as ``a linguistic repertoire that is
    associated, culture-internally, with particular social practices and
    persons who engage in such practices.''
  \end{itemize}
\item
  Explain the difference between an individual's register range and the
  social domain of a register.

  \begin{itemize}
  \tightlist
  \item
    An individual's register range is the variety of registers to which
    they have access. The social domain of a register is the variety of
    social contexts or situations in which it might be appropriate or
    useful to use that register. For instance, I have access to border
    Mexican drug slang (it is in my register range), but talking to my
    suburban friends at dinner, it is not useful or appropriate to use
    it (that context is outside the social domain of that register).
  \end{itemize}
\end{enumerate}

\subsection{Paragraph Answers: Do 2 of
3.}\label{paragraph-answers-do-2-of-3.}

\begin{enumerate}
\def\labelenumi{\arabic{enumi}.}
\setcounter{enumi}{9}
\tightlist
\item
  By reference to an example of a particular register, explain how
  institutions of register replication cause social asymmetries.
\end{enumerate}

Schools and grammar texts replicate Standardized English, and together
comprise an institution of register replication. This replication is
most effective in wealthy, white students who attend affluent schools.
Obviously, access to these institutions is differential along race and
class lines. Particular sorts of students receive extensive training in
the Standardized English register required for school, while others
receive little to no training in it. This causes social (and economic)
asymmetry by way of affecting the college and career opportunities of
speakers. Those with access to these institutions, and therefore
Standardized English, have more and better college and white-collar
career opportunities open to theme.

\begin{enumerate}
\def\labelenumi{\arabic{enumi}.}
\setcounter{enumi}{11}
\tightlist
\item
  Explain an example of a contrary-to-stereotype use of a register's
  forms which has a tropic significance.
\end{enumerate}

White boys and men use African American English (or what they think
African American English is) in a variety of tropic ways. White boys are
outside of the stereotype associated with the register, so their use is
necessarily contrary-to-stereotype. African American English is often
used in an attempt to index masculinity or sports aptitude, evoking
obvious tropes about black men. Gay white men often use a particular
register stereotypically associated with Black women, which draws on
tropes of ``angry'' or ``sassy'' Black women, to build a particular
social image for the speaker.


\end{document}
