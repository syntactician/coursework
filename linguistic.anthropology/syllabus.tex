\documentclass[doc,12pt]{apa6}
\usepackage{lmodern}
\usepackage{amssymb,amsmath}
\usepackage{fixltx2e} % provides \textsubscript
\usepackage[T1]{fontenc}
\usepackage[utf8]{inputenc}
  
\usepackage[usenames,dvipsnames]{color}

\usepackage{graphicx,grffile}
\graphicspath{ {.resources/} }
\makeatletter
\def\maxwidth{\ifdim\Gin@nat@width>\linewidth\linewidth\else\Gin@nat@width\fi}
\def\maxheight{\ifdim\Gin@nat@height>\textheight\textheight\else\Gin@nat@height\fi}
\makeatother
% Scale images if necessary, so that they will not overflow the page
% margins by default, and it is still possible to overwrite the defaults
% using explicit options in \includegraphics[width, height, ...]{}
\setkeys{Gin}{width=\maxwidth,height=\maxheight,keepaspectratio}
\setlength{\parindent}{0pt}
\setlength{\parskip}{6pt plus 2pt minus 1pt}
\setlength{\emergencystretch}{3em}  % prevent overfull lines
\providecommand{\tightlist}{%
  \setlength{\itemsep}{0pt}\setlength{\parskip}{0pt}}
\setcounter{secnumdepth}{0}

\usepackage[colorlinks=false]{hyperref}
\usepackage{ragged2e}

\begin{document}

\title{Linguistic Anthropology Syllabus}
\shorttitle{Syllabus}
\author{Talbot Taylor}
\date{}
\affiliation{College of William \& Mary}
\maketitle

\center
Tucker 214: T 11-12, W 5-6 \\
txtayl@wm.edu \\
(757) 810-4428

\justify

\noindent
\subsubsection{Course Goals}\label{course-goals}

\begin{itemize}
\tightlist
\item
  To provide students with the tools to understand and evaluate
  scholarly papers in contemporary linguistic anthropology
\item
  To present an overview of the historical development of linguistic
  anthropology
\item
  To introduce some of the main research topics active in linguistic
  anthropology today
\item
  To provide a grounding in the analytical framework used in linguistic
  anthropological research
\item
  To help students improve written and oral communication skills
\end{itemize}

\subsubsection{Course Materials}\label{course-materials}

Agar, M., \emph{Language Shock,} Quill, 1996 (ISBN 978-0688149499)

\emph{Most readings for the course will be accessible on the course's
Blackboard site.}

\subsubsection{Course Requirements}\label{course-requirements}

\begin{itemize}
\tightlist
\item
  Response paper on Agar: 5\%
\item
  Linguistic relativity essay: 10\%
\item
  Midterm test: 15\%
\item
  Class presentation: 15\%
\item
  Article critiques (x 2): 10\% each
\item
  Reading preparation and class participation: 15\%
\item
  Final exam: 20\%
\item
  Extra credit (5\% max.): Reasoned, clearly articulated, productive
  disagreement in class with a substantive claim made by the professor
  or by student presenters of an assigned reading.
\end{itemize}

\subsubsection{Reading Questions}\label{reading-questions}

This is a discussion-based seminar. So, you must undertake to do the
reading for each class on time; for the success of class discussion will
depend on your having read the material and prepared the Reading
Questions (RQs) for that day. Reading Questions will be posted on the
course website under Assignments.

\subsubsection{Class participation}\label{class-participation}

This is a crucial requirement for the success of the seminar and is thus
an obligation for each student. All students will be expected to
contribute actively to class discussion of assigned readings and Reading
Questions, including those assigned for their fellow students' class
presentations. Students will receive a provisional evaluation of their
class participation halfway through the semester.

\subsubsection{Attendance}\label{attendance}

Because of the nature of the seminar, attendance is mandatory. Any
unexcused absences (``excused'' = either having prior permission or
producing a medical note) will be subtracted from your final grade: 1/3
letter grade per absence.

Note: the College's add-drop deadline is Sept 4. The deadline for
withdrawal from a course is Oct 23.

\subsubsection{Class presentations}\label{class-presentations}

After Fall Break, each student will co-present one of the assigned
readings and lead class discussion for that day. The goal of this
assignment is for you to work together to deepen your understanding of a
particular topic and to work on your presentational skills. At least one
week prior to their scheduled presentation, each two-person team is
required to meet in person with me to discuss their plans for the
presentation and discussion. (It is your responsibility to schedule this
meeting.) The evening before this meeting, a one-page draft outline of
the presentation must be submitted to me by email, along with a
provisional list of at least five Reading Questions to be used in
guiding class discussion. (The finalized list must be distributed by
email to the class 5 days before the scheduled presentation, to give the
other class members sufficient time to prepare their responses for
discussion in class.) Pairings and assigned articles will be announced
by the end of September; the first presentation is provisionally
scheduled for October 27th.

In your presentation you should (1) give a resume of the assigned
article, focusing on its central themes, questions, data, methods,
argument, and conclusions, including, where appropriate, how these are
related to other readings in the course, and (2) guide class discussion
of the article by working one-by-one through each of your distributed
Reading Questions.

Keep in mind: you are teaching this article to the class, both by means
of the resume and by leading discussion of well-crafted Reading
Questions. Accordingly, you will be evaluated for (i) clarity of the
presentation, during both its resume and RQ portions, (ii) quality and
accuracy of the information and explanations provided, (iii) effective
and helpful guidance of class discussion. Each member of the pair is
expected to be equally involved in preparing and actively contributing
to both parts of the presentation. Therefore, each student will be
assigned the same grade, except if there is a broadly noticeable
difference in the quality of their contribution. 5\% of the grade for
your presentation will be based on your classmates' written assessments
of the presentations and 10\% on my assessment. My assessment will also
take into account your preparation for the meeting in the week before
the presentation.

Use of paper handouts or visual aids in the presentation is permitted
but not required. Presentations (including class discussion) should take
a minimum of 45 minutes of classroom time.

\subsubsection{Article critique}\label{article-critique}

Each student will choose two of the articles assigned after Fall Break
(other than an article that you presented in class) and provide a
written critique (3pps.) Students sign up for their article on
Blackboard and post their critiques there as well. A critique includes a
summary of the main argument of a text as well as discussion of an
author's use of particular terms or concepts, and a reasoned evaluation
of the overall strength of their argument in light of other readings.
The first critiques is due within one week after the class in which the
article was discussed. The second critique is due on the last day of
classes.

\subsubsection{Student Accessibility
Services}\label{student-accessibility-services}

It is the policy of The College of William and Mary to accommodate
students with disabilities and qualifying diagnosed conditions in
accordance with federal and state laws. Any student who feels s/he may
need an accommodation based on the impact of a learning, psychiatric,
physical, or chronic health diagnosis should contact me privately to
discuss your specific needs. Students will also need to contact Student
Accessibility Services staff at 757-221-2509 or at sas@wm.edu to
determine if accommodations are warranted and to obtain an official
letter of accommodation. For more information, please see
www.wm.edu/sas.

\vfill

\textbf{Students have the responsibility to check the course Blackboard
site regularly for important announcements, posted assignments and
readings, Reading Questions, changes to the schedule or syllabus, etc.
The official syllabus for the course is that which is posted (and
updated from time to time) on Blackboard.}

\end{document}
