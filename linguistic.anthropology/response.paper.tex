\documentclass[man,12pt,natbib]{apa6}
\usepackage{amssymb,amsmath,times}
\usepackage[T1]{fontenc}
\usepackage[utf8]{inputenc}
\linespread{1.5}
  
\renewcommand*{\thefootnote}{\fnsymbol{footnote}}

\begin{document}

\title{\textit{Languacultural} Autobiography}
\shorttitle{\textit{Languacultural} Autobiography}
\author{Edward Hern\'{a}ndez}
\date{30 August 2015}
\affiliation{College of William \& Mary}
\maketitle

My parents both speak Spanish. My father grew up on both sides of the US-Mexico
border, speaking exclusively Spanish in ethnically Mexican households. English
is his second language, and he did not begin to learn it until he was required
to in a United States elementary school. My mother grew up in an
English-speaking family, learning Spanish only in classrooms. Both of them
remain at least conversationally fluent. As a result, I grew up hearing
multiple varieties of both English and Spanish at home.

Despite speaking Spanish to each other, neither of my parents made any sort of
effort to speak it to me or teach it to me. I was never really fluent as a
child; the best I could ever do was string a halting sentence or two together
to greet my aunts and uncles when we visited. Eventually, they started speaking
English to me as well, and I entirely lost all of my Spanish ability. At least,
I thought I did. When I got to high school and took a Spanish class, it quickly
became apparent that I knew more than I thought.

None of my friends really had a problem ``learning Spanish'' in the traditional
sense.  They learned the phonology, the vocabulary, and everything else in the
textbook. But when it came to speaking conversationally, they had some issues
that I never ran into.  While my friends were struggling with word order, I got
it right every time, not because I was a better student, but because I thought
of adjectives as following nouns. I thought of love as a reflexive verb.
Because I had a window into the culture behind the language, or at least into
the manner of thinking which accompanied the use of the language, I was a step
ahead. As my Spanish teacher put it, they were ``speaking English with Spanish
words'' and I was speaking Spanish.

That is not to say that I am any sort of expert. I am not ever fluent, and what
knowledge I have only really extends to particular Peninsular and Mexican
varieties. One of my closest friends is a fluent bilingual speaker of English
and Spanish. However, it\footnote{My friend is queer and prefers to be referred
to using the pronoun ``it.''} is from Argentina and speaks only castellano
rioplatense and United Kingdom Standard English. When we first met, we
communicated mostly in English, but we found that a lot of words had different
meanings for us, especially newer words. Recently, I have been trying to learn
more Spanish, and I have started speaking with it in Spanish.  Nothing in my
learning prepared me for the differences between Mexican and Argentinian
Spanish. The second person pronouns are different and the concepts of respect
and familiarity that dictate their use are completely unfamiliar to me. The
vocabulary is full of Italian loanwords. If we didn't share a second language
as well, it would be extremely hard for us to communicate. Even though I have
academically ``learned Spanish,'' there is a huge disconnect between my
learning and the competence(s) I need to communicate with other Spanish
speakers from other cultures and dialects.

\end{document}
