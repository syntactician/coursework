\documentclass[doc,12pt]{apa6}
\usepackage{lmodern}
\usepackage{amssymb,amsmath}
\usepackage{fixltx2e} % provides \textsubscript
\usepackage[T1]{fontenc}
\usepackage[utf8]{inputenc}
  
\usepackage[usenames,dvipsnames]{color}

\usepackage{graphicx,grffile}
\graphicspath{ {.resources/} }
\makeatletter
\def\maxwidth{\ifdim\Gin@nat@width>\linewidth\linewidth\else\Gin@nat@width\fi}
\def\maxheight{\ifdim\Gin@nat@height>\textheight\textheight\else\Gin@nat@height\fi}
\makeatother
% Scale images if necessary, so that they will not overflow the page
% margins by default, and it is still possible to overwrite the defaults
% using explicit options in \includegraphics[width, height, ...]{}
\setkeys{Gin}{width=\maxwidth,height=\maxheight,keepaspectratio}
\setlength{\parindent}{0pt}
\setlength{\parskip}{6pt plus 2pt minus 1pt}
\setlength{\emergencystretch}{3em}  % prevent overfull lines
\providecommand{\tightlist}{%
  \setlength{\itemsep}{0pt}\setlength{\parskip}{0pt}}
\setcounter{secnumdepth}{0}


% Redefines (sub)paragraphs to behave more like sections
\ifx\paragraph\undefined\else
\let\oldparagraph\paragraph
\renewcommand{\paragraph}[1]{\oldparagraph{#1}\mbox{}}
\fi
\ifx\subparagraph\undefined\else
\let\oldsubparagraph\subparagraph
\renewcommand{\subparagraph}[1]{\oldsubparagraph{#1}\mbox{}}
\fi

\begin{document}

\title{Linguistic Relativity and the Boasian Tradition}
\shorttitle{Linguistic Relativity}
\author{8 September}
\affiliation{Linguistic Anthropology}
\maketitle

\begin{enumerate}
\def\labelenumi{\arabic{enumi}.}
\tightlist
\item
  Franz Boas is usually identified as the founder of linguistic
  anthropology. He was born in Germany but did his research and teaching
  in the USA, focusing on Native American languages. What was Boas' view
  on the relation between (1) mental/cognitive categories and concepts
  (i.e., thought), (2) a particular language's structures and
  categories, and (3) that culture's interests and emphases? How does
  the notion of Versuch fit into this?
\item
  Why did Boas believe that studying a culture's linguistic features is
  an undistorted way of learning about the culture itself?
\item
  Boas' student, Edward Sapir taught linguistics at Chicago and
  anthropology at Yale before the second world war. In what way, as
  Foley says, does Sapir reverse Boas' view on the relationship between
  language and thought? Why does Foley, like John Lucy (1992), use the
  expression ``channels'' to describe Sapir's position?

  \begin{itemize}
  \tightlist
  \item
    Language mediates conceptual thinking (or, `the form of thought').
    On p.~198, Foley says that ``For Sapir, it is only in language that
    the full potential of thought is unfolded; true conceptual thinking
    is impossible without language because it is symbolically
    mediated\ldots{}.'' What do you think (pun intended) Foley and Sapir
    mean in saying that language symbolically mediates thought? Please
    come to class with an example of a kind of thinking or a mental
    operation that could be said to depend on the mediation of language
    and would be, as Sapir says, impossible without it?
  \end{itemize}
\item
  What is meant by the ``mapping view'' that Boas, Sapir, and Whorf all
  are opposed to? Please explain Sapir's discussion (pp.~198-199) of the
  Nootka sentence about the falling stone. How is it intended to show
  the error of the mapping view?
\item
  Foley says that Whorf introduced an important distinction between a
  language's overt and covert categories or classes. (The latter are
  also called `cryptotypes'.) Can you explain this distinction by
  reference to some illustrative examples drawn from English (but not
  those mentioned by Foley)? Why did Whorf take cryptotypes to be of
  special importance?

  \begin{itemize}
  \tightlist
  \item
    Can you explain the ``vacillation'' in Whorf''s famous passage
    quoted and discussed by Foley on p.~201, between two different kinds
    of linguistic relativity?
  \end{itemize}
\item
  Foley says that Whorf demonstrated his Principle of Linguistic
  Relativity in two ways. The first is exemplified in the English \&
  Shawnee example on pp.~202-203. How is this example intended to
  support the Principle?
\item
  The second way that Whorf demonstrated the Principle of Linguistic
  Relativity is by reference to what has come to be called ``cognitive
  appropriation''. Can you explain what cognitive appropriation is by
  reference to the famous example of `empty gasoline drums'
  (pp.~203-204)?
\item
  How does Whorf apply this notion of cognitive appropriation to the
  example of the differences between English and Hopi ways of speaking
  about time?
\end{enumerate}

\end{document}
