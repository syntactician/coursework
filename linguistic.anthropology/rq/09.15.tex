\documentclass[]{article}
\usepackage{lmodern,amssymb,amsmath,hyperref}
\usepackage[T1]{fontenc}
\usepackage[utf8]{inputenc}
% use upquote if available, for straight quotes in verbatim environments
\IfFileExists{upquote.sty}{\usepackage{upquote}}{}
% use microtype if available
\IfFileExists{microtype.sty}{%
\usepackage{microtype}
\UseMicrotypeSet[protrusion]{basicmath} % disable protrusion for tt fonts
}{}
\usepackage[usenames,dvipsnames]{color}
\hypersetup{breaklinks=true,
            bookmarks=true,
            pdfauthor={},
            pdftitle={},
            colorlinks=true,
            citecolor=blue,
            urlcolor=blue,
            linkcolor=magenta,
            pdfborder={0 0 0}}
\urlstyle{same}  % don't use monospace font for urls
\setlength{\parindent}{0pt}
\setlength{\parskip}{6pt plus 2pt minus 1pt}
\setlength{\emergencystretch}{3em}  % prevent overfull lines
\providecommand{\tightlist}{%
  \setlength{\itemsep}{0pt}\setlength{\parskip}{0pt}}
\setcounter{secnumdepth}{0}

\date{}

% Redefines (sub)paragraphs to behave more like sections
\ifx\paragraph\undefined\else
\let\oldparagraph\paragraph
\renewcommand{\paragraph}[1]{\oldparagraph{#1}\mbox{}}
\fi
\ifx\subparagraph\undefined\else
\let\oldsubparagraph\subparagraph
\renewcommand{\subparagraph}[1]{\oldsubparagraph{#1}\mbox{}}
\fi

\begin{document}

\section{Piraha Numerical Cognition}\label{piraha-numerical-cognition}

\subsection{15 September}\label{september}

\begin{itemize}
\item [h\'{o}i]
  What is the research question which Gordon formulates in the 2nd main
  paragraph in col. 2, p.496? He says that there are two parts to the
  Linguistic Determinism Hypothesis (p.496, col.1). Which of these two
  parts does his research question directly address?
\item [ho\'{i}]
  Where do the Pirahã live? What are the features of their cultural life
  which seem most relevant to Gordon's research on their numerical
  abilities? With regard to their language, what number words to the
  Pirahã use? How might these be translated into English? Is it a
  recursive, base-2 system?
\item [baagi]
  What observations did Gordon derive from his initial studies with
  Keren Everett (i.e., ``Year 1'')?
\item [baagi]
  Please explain the difference between the 2 kinds of `primitive'
  numerical abilities (also referred to as ``the dual model of mental
  enumeration''): i.e., the difference between parallel individuation
  (also called `exact enumeration') and analog estimation.
\item [baagi]
  Gordon's conclusion (p.498, col.3) is the following: ``The analog
  estimation abilities exhibited by the Pirahã are a kind of numerical
  competence that appears to be immune to numerical language
  deprivation. (\ldots{}) The present experiments allow us to ask
  whether humans who are not exposed to a number system can represent
  exact quantities for medium-size sets of four or five. The answer
  appears to be negative.'' Please explain how the data from his
  experiments (presented in Fig.1 A-H and Fig 2 A \& B) support this
  conclusion.
\item [baagi]
  Can you explain the differences between the two Whorfian claims which
  the Frank et al paper investigates? What do the authors mean by the
  concept of exact quantity?
\item [baagi]
  What was the format of Frank et al's first experiment on numeral
  elicitation (Experiment 1)? Are its results in agreement with Gordon's
  claim that Pirahã has number words? Do the results agree with Gordon's
  account of the meanings of (what he thought to be) number vocabulary?
\item [baagi]
  Which of Gordon's original experimental tasks did Frank et al have
  their subjects perform in their Experiment 2?
\item [baagi]
  Did the results from Frank et al's Experiment 2 replicate the results
  which Gordon obtained? In what respects and on which experimental
  tasks were their results the same or different? What do Frank et al
  suggest as the source of the difference in results?
\item [baagi]
  Explain the relevance that the authors take their research to have for
  the Linguistic Determinism Hypothesis (``the strong Whorfian claim'')
  and the Linguistic Relativity Hypothesis? How do they take their
  conclusions to relate to the studies of color and navigation discussed
  in the paper's final two paragraphs and the ``compressive'' role which
  number words (and language more generally) is said to play in
  cognitive processes?
\end{itemize}

\end{document}
