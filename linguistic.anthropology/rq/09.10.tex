\documentclass[]{article}
\usepackage{lmodern}
\usepackage{amssymb,amsmath}
\usepackage{ifxetex,ifluatex}
\usepackage{fixltx2e} % provides \textsubscript
\ifnum 0\ifxetex 1\fi\ifluatex 1\fi=0 % if pdftex
  \usepackage[T1]{fontenc}
  \usepackage[utf8]{inputenc}
\else % if luatex or xelatex
  \ifxetex
    \usepackage{mathspec}
    \usepackage{xltxtra,xunicode}
  \else
    \usepackage{fontspec}
  \fi
  \defaultfontfeatures{Mapping=tex-text,Scale=MatchLowercase}
  \newcommand{\euro}{€}
\fi
% use upquote if available, for straight quotes in verbatim environments
\IfFileExists{upquote.sty}{\usepackage{upquote}}{}
% use microtype if available
\IfFileExists{microtype.sty}{%
\usepackage{microtype}
\UseMicrotypeSet[protrusion]{basicmath} % disable protrusion for tt fonts
}{}
\ifxetex
  \usepackage[setpagesize=false, % page size defined by xetex
              unicode=false, % unicode breaks when used with xetex
              xetex]{hyperref}
\else
  \usepackage[unicode=true]{hyperref}
\fi
\usepackage[usenames,dvipsnames]{color}
\hypersetup{breaklinks=true,
            bookmarks=true,
            pdfauthor={},
            pdftitle={},
            colorlinks=true,
            citecolor=blue,
            urlcolor=blue,
            linkcolor=magenta,
            pdfborder={0 0 0}}
\urlstyle{same}  % don't use monospace font for urls
\setlength{\parindent}{0pt}
\setlength{\parskip}{6pt plus 2pt minus 1pt}
\setlength{\emergencystretch}{3em}  % prevent overfull lines
\providecommand{\tightlist}{%
  \setlength{\itemsep}{0pt}\setlength{\parskip}{0pt}}
\setcounter{secnumdepth}{0}

\date{}

% Redefines (sub)paragraphs to behave more like sections
\ifx\paragraph\undefined\else
\let\oldparagraph\paragraph
\renewcommand{\paragraph}[1]{\oldparagraph{#1}\mbox{}}
\fi
\ifx\subparagraph\undefined\else
\let\oldsubparagraph\subparagraph
\renewcommand{\subparagraph}[1]{\oldsubparagraph{#1}\mbox{}}
\fi

\begin{document}

\section{\texorpdfstring{Foley ``Space'' and Lucy ``Toward a New
Approach,''
pp.~48-52}{Foley Space and Lucy Toward a New Approach, pp.~48-52}}\label{foley-space-and-lucy-toward-a-new-approach-pp.48-52}

\subsection{10 September}\label{september}

\begin{enumerate}
\def\labelenumi{\arabic{enumi}.}
\item
  What are the strong universalist claims about spatial conception,
  which Foley discusses on pages 215-216? If these universalist claims
  about spatial conception are true, what does this imply about spatial
  terms/vocabulary in the languages of the world?
\item
  Foley (p.~216ff) discusses research at the Max Planck Institute in the
  Netherlands which has led to a conclusion that is opposed to
  universalist claims about spatial conception and spatial vocabulary, a
  conclusion that is more consonant with the Principle of Linguistic
  Relativity. What is this conclusion? (We'll get to the evidence for it
  below.) Does it deny that there are any universals of spatial
  conception?
\item
  Explain the use of spatial terms in Guugu-Yimidhirr. Using an invented
  example, explain how this contrasts with English use of terms like
  left, right, front, and back. Why is one pattern called `absolute' and
  the other `egocentric' (or `relative')? Why is the former better
  suited than the latter to navigation in open country?
\item
  Explain Brown and Levinson's example (p.217-218) of a dinner party at
  the top of the Centerpoint Tower in Sydney.
\item
  Explain the experiments that Levinson (p.~218ff) designed to explore
  differences in the cognitive consequences for speakers using absolute
  or egocentric spatial language.
\item
  Explain the spatial system used by the Tzeltal in the Mayan community
  in Mexico and the experiments by Brown and Levinson exploring the
  cognitive influence of this system.
\item
  Why are these experiments said to provide ``powerful support for the
  Principle of Linguistic Relativity'' (p.~221)? Are they evidence
  against the assumption of the psychic unity of humanity?
\item
  In ``Towards a new approach'', Lucy identifies 4 ``requirements for an
  improved approach to research'' on the linguistic relativity
  hypothesis (48-49). What are these 4 requirements? How well do you
  think Lucy's study of number marking in Yucatec Maya and American
  English, discussed pp.~49-52 meets these requirements?
\end{enumerate}

\end{document}
