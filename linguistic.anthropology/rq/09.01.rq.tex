\documentclass[doc,12pt]{apa6}
\usepackage{lmodern}
\usepackage{amssymb,amsmath}
\usepackage{fixltx2e} % provides \textsubscript
\usepackage[T1]{fontenc}
\usepackage[utf8]{inputenc}
  
\usepackage[usenames,dvipsnames]{color}

\usepackage{graphicx,grffile}
\graphicspath{ {.resources/} }
\makeatletter
\def\maxwidth{\ifdim\Gin@nat@width>\linewidth\linewidth\else\Gin@nat@width\fi}
\def\maxheight{\ifdim\Gin@nat@height>\textheight\textheight\else\Gin@nat@height\fi}
\makeatother
% Scale images if necessary, so that they will not overflow the page
% margins by default, and it is still possible to overwrite the defaults
% using explicit options in \includegraphics[width, height, ...]{}
\setkeys{Gin}{width=\maxwidth,height=\maxheight,keepaspectratio}
\setlength{\parindent}{0pt}
\setlength{\parskip}{6pt plus 2pt minus 1pt}
\setlength{\emergencystretch}{3em}  % prevent overfull lines
\providecommand{\tightlist}{%
  \setlength{\itemsep}{0pt}\setlength{\parskip}{0pt}}
\setcounter{secnumdepth}{0}


% Redefines (sub)paragraphs to behave more like sections
\ifx\paragraph\undefined\else
\let\oldparagraph\paragraph
\renewcommand{\paragraph}[1]{\oldparagraph{#1}\mbox{}}
\fi
\ifx\subparagraph\undefined\else
\let\oldsubparagraph\subparagraph
\renewcommand{\subparagraph}[1]{\oldsubparagraph{#1}\mbox{}}
\fi

\begin{document}

\title{Agar 31-60}
\shorttitle{Agar}
\author{1 September}
\affiliation{Linguistic Anthropology}
\maketitle


\textbf{Note:} There are no RQs for pp.~13-30.

\begin{enumerate}
\def\labelenumi{\arabic{enumi}.}
\tightlist
\item
  What was different about Saussure's approach to linguistics from that
  of the historical linguistics in which he was trained?
\item
  Why does Agar call Saussure ``the founder of
  inside-the-Circle-linguistics''? What does he mean by ``the Circle''
  and how does it relate to Saussure's distinction between language
  (langue) and speech (parole)? What might Agar mean by saying that a
  language in Saussure's sense is ``an idea as old as Plato's cave''?
\item
  On pages 42-46, Agar discusses Saussure's notion of symbolic systems
  (language being one example) as built on two kinds of relations
  between signs: paradigmatic and syntagmatic. Can you give an example
  of each of these two kinds of relations in a symbolic system -- your
  own example, not one drawn from the book?\\
\item
  In what way was the anthropology that Boas was reacting against based
  on ethnocentrism? How did it relate to Darwinian evolutionary theory?
\item
  What does Agar mean by saying that Boas wanted the anthropological
  description of culture to be objectively about ``them'', yet he
  couldn't help but make it about ``his-theirs'' (p.50)?
\item
  According to Agar (pp.~55-56), how did Bloomfield's 1933 book on
  linguistics transform the linguistic anthropology of Boas?
\item
  What did Boas mean by the principles of linguistic relativity and
  cultural relativity? How do the principles of linguistic and cultural
  relativity contrast with the notion of moral relativity?
\end{enumerate}

\end{document}
