\documentclass[doc,12pt]{apa6}
\usepackage[colorlinks=false]{hyperref}
\usepackage{amssymb,amsmath,times,tipa,phonrule,gb4e}
\linespread{1.5}
\renewcommand{\labelenumi}{\alph{enumi}.}

\begin{document}

\title{Homework 8: Somali}
\shorttitle{Homework 8}
\author{Edward Hern\'{a}ndez}
\date{22 March 2016}
\affiliation{with Sora Edwards-Thro}
\maketitle

\section{Part 1: Obstruents}

Somali has several phonological rules affecting obstruent consonants.  In our
data, /lt/ becomes \textipa{S}, stops become fricatives
intervocalically\footnote{Below, I call this process lenition just to have
	something short to call it. I'm not sure whether spirantization is more
accurate.}, and /dt/ clusters are reduced to /d/.

All of the data we have is on nouns. These nouns seem to have as their root the
singular form, and are derived via affixes into the singular definite and
plural forms. Some of these roots have word-final stops, which are realized as
fricatives in the plural form. This appears to happen because the plural affix
/-o/ is a vowel, which makes these previously word-final stops intervocalic.

The singular definite forms of the nouns in our data end in /-ta/, with a few
notable exceptions. When adding /-ta/ would produce /lt/, it is changed to
/\textipa{S}/, and when it would produce /dt/, it is changed to /d/.  It would
be easy to write a rule \phon{dt}{d}, of course, but I think that there is no
reason to think that such a change would not happen in phonetically similar
cases (which our data does not provide).  Since /dt/ is the only cluster of
coronal sounds in the data, I've opted to write the rule as a deletion of the
second sound in any cluster of two coronals. I realize that this is the
broadest way that I could write the rule, but I feel that it is true to the
phonetic motivation of reducing back-to-back sounds articulated with the same
part of the tongue.\footnote{It is just as reasonable to write a more focused
rule, two stops in a row at the same place of articulation become a single
stop. I wasn't sure how to write such a rule without an $\alpha$ for each
possible place of articulation.}

\begin{exe}

	\ex \begin{tabbing}
		coronal cluster reduction (CCR) \= \kill
		intervocalic lenition (IR) \>
		\phonb{\phonfeat{-son}}{
		       \phonfeat{+cont}}{
		       \phonfeat{+syl}}{
		       \phonfeat{+syl}}
		\end{tabbing}

	\ex \begin{tabbing}
		coronal cluster reduction (CCR) \= \kill
		coronal cluster reduction (CCR) \>
		\phonl{\phonfeat{+cor\\-son}}{
		       $\emptyset$}{
		       \phonfeat{+cor\\-son}}
		\end{tabbing}

	\ex \begin{tabbing}
		coronal cluster reduction (CCR) \= \kill
		lt to \textipa{S} \>
		\phon{lt}{\textipa{S}}
	\end{tabbing}

\end{exe}

As you can see in the derivation below, coronal cluster reduction
\emph{counterfeeds} intervocalic lenition, resulting in intervocalic stops.
If the rules were in a different order (i.e. coronal cluster reduction fed
intervocalic lenition) {[}bada{]} would instead be produced
{[}ba\textipa{D}a{]}.

\begin{exe}
	\ex \begin{tabbing}
		CCR \hspace{2em} \= /bad/ \hspace{2em}
		\= /bad-ta/ \hspace{2em} \= /bad-o/ \kill
		UR  \> /bad/ \> /bad-ta/ \> /bad-o/          \\
		IL  \> /bad/ \> /badta/  \> /ba\textipa{D}o/ \\
		CR \> /bad/ \> /bada/   \> /ba\textipa{D}o/ \\
		SR  \> {[}bad{]} \> {[}bada{]} \> {[}ba\textipa{D}o{]}
		\end{tabbing}
\end{exe}

\section{Part 2: Nasals}

Somali appears to have two nasal sounds, {[}m{]} and {[}n{]}. While there are
no minimal pairs in the data, it does not appear that these two sounds could be
allomorphs of the same phoneme, because when each occurs is not fully
predictable by rule. Since it is not predictable by rule, there must be two 
nasal phonemes in the underlying representation of the words.

Given that /m/ and /n/ are distinct phonemes in the UR, it is still difficult
to write up a rule that predicts when each will be realized. Words whose UR
contains /m/ are still realized with /n/ in some cases.  In our data, /m/
appears to become /n/ word finally nad before /t/ (the only consonant it
appears before in our data. While it is relatively intuitive for the place to
assimilate before an alveolar stop, the change word-finally does not seem to
have a phonetic motivation. Instead of writing two rules, only one of which is
motivated, I opted to write a single rule: /m/ becomes /n/ when it is a
syllable coda.\footnote{I still don't understand the phonetic motivation here,
but it's the only way I see to handle it in a single rule.} Assuming that
Somali breaks up consonant clusters between syllables (\phon{VCCV}{VC.CV}), this
captures the change occurring in the data:

\begin{exe}
	\ex \begin{tabbing}
		coronal cluster reduction (CCR) \= \kill
		nasal movement \>
		\phonr{\phonfeat{+nas\\-syl}}{
		       \phonfeat{-lab\\+cor}}{
		       {]}_\sigma}
		\end{tabbing}
\end{exe}

Here are the derivations of two singular nouns, /laam/ `branch' and /sim/
`hide', into their definite /-ta/ and plural /-o/ forms to show how the rule
acts:

\begin{exe}
	\ex \begin{tabbing}
		UR \hspace{2em} \= /laam/ \= /laam.ta/ \= /laa.mo/ \\
		NM \> /laan/ \> /laan.ta/ \> /laa.mo/ \\
		SR \> {[}laan{]} \> {[}laan.ta{]} \> {[}laa.mo{]}
		\end{tabbing}
	\ex \begin{tabbing}
		UR \hspace{2em} \= /sim/ \= /sim.ta/ \= /si.mo/ \\
		NM \> /sin/ \> /sin.ta/ \> /si.mo/ \\
		SR \> {[}sin{]} \> {[}sin.ta{]} \> {[}si.mo{]}
		\end{tabbing}
\end{exe}

\end{document}
