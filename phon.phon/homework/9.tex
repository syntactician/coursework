\documentclass[doc,12pt]{apa6}
\usepackage[colorlinks=false]{hyperref}
\usepackage{amssymb,amsmath,times,tipa,phonrule,gb4e}
\linespread{1.5}
\renewcommand{\labelenumi}{\alph{enumi}.}

\begin{document}

\title{Homework 9: Feature Organization}
\shorttitle{Homework 9}
\author{Edward Hern\'{a}ndez}
\date{29 March 2016}
\affiliation{College of William \& Mary}
\maketitle

\section{Part 1: Greek}

\section{Part 2: Sanskrit}


Trying to express this process without feature geometry, we are left with truly
unwieldy rules:\footnote{Properly, there should be two for laryngeal features
	and one for every place of stops in Sanskrit. I guessed about place
	features based on a gestural account at
	\url{http://www.sanskritweb.net/deutsch/ipa_sans.pdf}.}

\begin{exe}
	\ex \phonr{
			\phonfeat{- syllabic\\
				      - sonorant\\
					  -contuant}}{
			\phonfeat{$\alpha$ voice\\
				      $\beta$ aspiration}}{
			\phonfeat{- syllabic\\
				      - sonorant\\
					  - continuant\\
					  $\alpha$ voice\\
					  $\beta$ aspiration}}
	\ex \phonr{
			\phonfeat{- syllabic\\
				      - sonorant\\
					  - continuant}}{
			\phonfeat{$\alpha$ labial\\
				      $\beta$ coronal\\
					  $\gamma$ distributed\\
					  $\delta$ anterior\\
				      $\epsilon$ dorsal\\
				      $\zeta$ back}}{
			\phonfeat{- syllabic\\
				      - sonorant\\
					  - continuant\\
					  $\alpha$ labial\\
				      $\beta$ coronal\\
					  $\gamma$ distributed\\
					  $\delta$ anterior\\
				      $\epsilon$ dorsal\\
				      $\zeta$ back}}
				      
\end{exe}

\end{document}
