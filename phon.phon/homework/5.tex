\documentclass[doc,12pt]{apa6}
\usepackage[colorlinks=false]{hyperref}
\usepackage{amssymb,amsmath,times,graphicx,tipa,phonrule,gb4e}
\linespread{1.5}

\begin{document}

\title{Homework 5: Dravidian Languages}
\shorttitle{Homework 5}
\author{Edward Hern\'{a}ndez}
\date{23 February}
\affiliation{with Sora Edwards-Thro}
\maketitle

\section{Tamil}

I propose that {[}\textipa{p b B}{]}, {[}\textipa{t d D}{]}, and {[}\textipa{k
g G}{]} are, respectively, allophones of three phonemes.  For each phoneme,
which allophone is produced can be predicted by two rules, one specifying
behavior in the word-initial position and one specifying behavior following
nasal consonants.
\begin{exe}
	\ex \phonl{\textipa{B}}{p}{\#}
	\ex \phonl{\textipa{B}}{b}{nasal C}
	\ex \phonl{\textipa{D}}{t}{\#}
	\ex \phonl{\textipa{D}}{d}{nasal C}
	\ex \phonl{\textipa{G}}{k}{\#}
	\ex \phonl{\textipa{G}}{g}{nasal C}
\end{exe}
Following nasal consonants, the sounds become voiced stops. This most likely
results from the release of the closures made to produce the nasal airflow.
In word-initial positions, they become voiceless stops. I am less sure of
why this occurs, but I expect that it is easier to produce that way.

These three phonemes are not the only sounds behaving in this way in the data.
Both retroflex stops and post-alveolar affricates are also becoming voiced
after nasal consonants.
\begin{exe}
	\ex \phonl{\textrtailt}{\textrtaild}{nasal C}
	\ex \phonl{\texttslig}{\textdyoghlig}{nasal C}
\end{exe}

The nasal sounds all seem to be undergoing place assimilation. They all seem to
be produced in the place of the immediately proceeding consonant, if there is
one. I'm not sure how to write this out as a rule.

\section{Malayalam}

In the Malayalam data, vowel length is contrastive. There are three minimal
pairs (with different meanings) based on vowel length: {[}\textipa{ciri}{]} and
{[}\textipa{ci:ri}{]}, {[}\textipa{ke{\textrtailt}:u}{]} and
{[}\textipa{ke:{\textrtailt}:u}{]}, and {[}\textipa{ka{\textrtailt}i}{]} and
{[}\textipa{ka{\textrtailt}i}{]}. Since these pairs exist, long and short
vowels constitute different phonemes in Malayalam (rather than being allophones
of the same phoneme), so there cannot be a meaning-independent rule that
predicts which of them will appear in which environments. There are also
minimal pairs which establish that at minimum, {[}i{]}, {[}e{]}, and {[}a{]}
are phonemically distinct from /u/. I suspect that each of these vowels is its
own phoneme, making the vowel inventory of Malayalam somithing like
{[}\textipa{i u e e: a a:}{]}. There is not enough data on {[}\textipa{@}{]} to
decide whether or not it is phonemically distinct from other vowels.

\end{document}
