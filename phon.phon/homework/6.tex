\documentclass[doc,12pt]{apa6}
\usepackage[colorlinks=false]{hyperref}
\usepackage{amssymb,amsmath,times,tipa,phonrule,gb4e}
\linespread{1.5}
\renewcommand{\labelenumi}{\alph{enumi}.}

\begin{document}

\title{Homework 6: Distinctive Features}
\shorttitle{Homework 5}
\author{Edward Hern\'{a}ndez}
\date{23 February}
\affiliation{with Sora Edwards-Thro}
\maketitle

\section{1. Distinctive Features}
\begin{enumerate}
	\item high, low, back
	\item high, low, back, round
	\item high, low, back, round, ATR
\end{enumerate}

\section{2. Turkish Suffixes}

In Turkish, the genative case is produced by adding a suffix to the nominative
form of the noun. That affix has four allomorphs:
\begin{exe}
	\ex -in
	\ex -{\textbari}n
	\ex -yn
	\ex -un
\end{exe}
The allomorphs differ only in which vowel is produced. Which allomorph is
produced is not predictable based on the preceding phoneme, but it is
predictable by the previous vowel. In all cases, the affix is produced such
that it contains a high vowel with the same backness and roundness as the
preceding vowel.
The rule governing which allomorph is produced can be formalized: \\
\phonl{\phonfeat{+ high}n}{{\phonfeat{{$\alpha$}  back\\{$\beta$} back}}n}{\phonfeat{{$\alpha$}  back\\{$\beta$} back}...}

The plural suffix also assimilates to information from the preceding vowel. It
has two allophones:
\begin{exe}
	\ex -lar
	\ex -ler
\end{exe}
The vowel is always unrounded, but varies in backness and height. This
information is picked up from (and is therefore predictable by) the preceding
vowel. If the preceding vowel is back or central, the suffix is produced with a
low back vowel. If the preceding vowel is front, the suffix is produced with a
mid front vowel.  This can be formalized: \\
\phonl{l\phonfeat{- high\\-round}r}{l\phonfeat{$\alpha$ back}r}{\phonfeat{$\alpha$ back}...}


\end{document}
