\documentclass[doc,12pt]{apa6}
\usepackage[colorlinks=false]{hyperref}
\usepackage{amssymb,amsmath,times,graphicx,tipa,gb4e}
\linespread{1.5}

\begin{document}

\title{Homework 4: Acoustic Phonetics}
\shorttitle{Homework 4}
\author{Edward Hern\'{a}ndez}
\date{18 February}
\affiliation{with Sora Edwards-Thro}
\maketitle

\section{Part 1: Vowel Mapping}

I mapped my vowel space (Fig. 1) based on uttering the list of words on the
assignment sheet. The results for single vowels (Fig. 2) were a bit surprising.
My /i/ is lower than expected, almost colliding with my /\textipa{I}/ and /e/,
my /\textipa{E}/ is low and my /\textipa{A}/ is fronted, making them almost
adjacent.  My /\textipa{U}/ is quite centralized, sounding almost like
/\textipa{@}/.

My dipthongs (Fig. 3) are not at all what I expected. Few of the component
vowels appear to be in their correct positions. I do not know whether this is
standard, or whether I'm measuring my vowels at the wrong part of the dipthong,
but I'm quite surprised by this. I re-recorded and re-plotted all of them, and
they seemed to be relatively consistent.

I had some trouble plotting some of my back vowels, especially /\textipa{O}/,
because the F\textsubscript{1} and F\textsubscript{2} were so close together,
and there was so much noise is the spectrogram. I also noticed that most of
these vowels came out creaky when I was reading the list, which I think may
have caused me some problems.

\section{Part 2: Greek Again}

Originally, when I transcribed the Greek passage, I had quite a lot of trouble
deciding on vowels. I listened to it a lot, and basically wrote a guess after a
few listens. The original transcript is included here (1).

\begin{exe}
	\ex \textipa{\{E|e\}nAs fotino:s pu{\textturnr}Asnos piDAkA sikise piDi s@pot@mi: tu{\textturnr}avil du sne:p kIx tipsIka d@st@ dun d@mb\textsyllabic{l}do{\textturnr}}
\end{exe}

With the spectrogram, I could measure the vowels. I chose the strongest, most
obvious syllables, and plotted them as examplars to establish the speaker's
vowel space (Fig. 4), and compared all other vowels to those exemplars. In most
cases, this resolved confusion, but some fall between my exemplars, which made
it hard to make a call. In those cases, I have included the two vowels I think
it fell between.

\begin{exe}
	\ex \textipa{enAs foteno:s po{\textturnr}Asnos piDAkA sikise piDe s\{@|o\}poteme: t\{u|o\}{\textturnr}@vil du sneip k\textsuperscript{h}ix tipsek\{e|a\} dasT@ t\{u|o\}n dAmb\textsyllabic{l}do{\textturnr}}
\end{exe}

\begin{figure}
	\includegraphics[width=\textwidth]{4a.png}
	\caption{My vowel plot}
\end{figure}

\begin{figure}
	\includegraphics[width=\textwidth]{4b.png}
	\caption{My monopthongs}
\end{figure}

\begin{figure}
	\includegraphics[width=\textwidth]{4c.png}
	\caption{My dipthongs}
\end{figure}

\begin{figure}
	\includegraphics[width=\textwidth]{4d.png}
	\caption{Dimitrios' vowel plot}
\end{figure}

\end{document}
