\documentclass[doc,12pt]{apa6}
\usepackage[colorlinks=false]{hyperref}
\usepackage{amssymb,amsmath,times,tipa,phonrule,gb4e}
\linespread{1.5}
\renewcommand{\labelenumi}{\alph{enumi}.}

\begin{document}

\title{Homework 7: Fijian}
\shorttitle{Homework 7}
\author{Edward Hern\'{a}ndez}
\date{23 February}
\affiliation{with Sora Edwards-Thro \\ \& Connor Symons}
\maketitle

Fijian appears to delete final consonants:
\phonl{\phonfeat{+consonantal}}{$\emptyset$}{\#}. Notably, there seem to be many
verbs whose underlying representations contain final consonants.  While they
are not produced in the surface representation of the root form of the verb,
when affixes are added, the consonants appear.

These consonants are not part of the affixes themselves, nor are they
epentesized. If they were, we should be able to predict which consonant appears
when an affix is added. We can see this is not the case if we look at (1) and
(2). Clearly, we cannot predict from the root /ula/ which consonant will arise
if we add the suffix /-ia/. The only analysis available to us is that the
consonants are contained in the underlying representation: that (1) is mentally
represented /ulaf/ and (2) is /ulag/.

\begin{exe}
	\ex \begin{tabbing}
		ulafia \= `be smoked' \kill
		ula \> `smoke' \\
		ulafia \> `be smoked'
		\end{tabbing}
	\ex \begin{tabbing}
		ulagia \= `be smoked' \kill
		ula \> `joke' \\
		ulagia \> `be joked'
		\end{tabbing}
\end{exe}

To illustrate how this phonological rule works in practice, I'll present the 
derivation of the roots and affixed forms of three verbs, two of which have 
final consonants in their underlying representations:

\begin{exe}
	\ex /sulu/ \\
		sulu
	\ex /sulu-ia/ \\
		suluia
	\ex /taul/ \\
		\phonl{\phonfeat{+consonantal}}{$\emptyset$}{\#} \\
		tau
	\ex /taul-ia/ \\
		taulia
	\ex /alof/ \\
		\phonl{\phonfeat{+consonantal}}{$\emptyset$}{\#} \\
		alo
	\ex /alof-ia/ \\
		alofia
\end{exe}

\end{document}
