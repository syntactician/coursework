\documentclass[man,12pt,natbib]{apa6}
\usepackage[colorlinks=false]{hyperref}
\usepackage{amssymb,amsmath,times}
\linespread{1.5}

\begin{document}

\title{Literature Review}
\shorttitle{Lit Review}
\author{Edward Hern\'{a}ndez}
\date{2015-11-16}
\affiliation{College of William \& Mary}
\maketitle

\noindent
\emph{
I haven't really settled on a topic enough to write a targeted literature
review. I'll write a little about some of the stuff I'm reading.
}
% That's how the magic happens!

\vspace{12pt}

Right now, I'm entering the political science literature on speech and
communication via Druckman's work on framing. According to \citet{Chong07},
``framing refers to the process by which people develop a particular conception
of an issue or reorient their thinking about an issue'' (p.~104). Frames, on
this view, ``organize everyday reality'' \citep[p.~193]{Tuchman78} by
encouraging ``particular definitions and interpretations of political issues''
\citep[p.~343]{Shah02}. Political scientists view individual speech acts as
emphasizing particular considerations which constitute a frame
\citep[p~106]{Chong07}, communicating that frame to others and priming them to
use that frame in evaluating the topic of the speech event (p.~110).
% why do you have random hyphens in here—whatever you’re converting from isn’t
% working that well.

This literature contains the valuable concept of ``culturally available
frames'' \citep[p.~144]{Gamson87}, the set of frames which are used by political
elites in discussing a particular issue (on both sides), and thus become
familiar to listeners. Once these frames have been heard, they are stored in
the memory of listeners, and individual speakers can then tap into them in
their own speech on those issues.
% Tie to theories of advertising and marketing- that’s who they’re using for
% help these days as well as political scientists
%
% http://www.forbes.com/sites/kimberlywhitler/2015/05/31/the-marketing-of-a-political-candidate-insight-from-a-marco-rubio-campaign-insider/ 
% http://www.tandfonline.com/loi/wplm20#.VkpPocpD0o8 
% http://www.political-marketing.org/ 
% http://www.amazon.com/Political-Marketing-Applications-Jennifer-Lees-Marshment/dp/0415431298 
%
% which of course leads to propaganda research that intersects again with
% psychology
Using a frame which is in the audience's memory, then, activates that memory,
influencing the audience to use those memories to evaluate the issue at hand,
producing what is called a framing effect. Frames which are not available
(which is to say that they are not stored in the memory of the audience of a
speech event) cannot cause a framing effect, because the consideration(s) which
it emphasizes are not present in the mind(s) of the audience
\citep[p.~110]{Chong07}.

Political scientists, in this literature, at least, seem to conceive of
decision-making as drawing on ``the set of available beliefs''
\citep[p~111]{Chong07}, some of which are more salient than others, due largely
to recency of activation.
% So then how are new beliefs created?
By priming certain beliefs and considerations through utilizing a particular
frame in speech, one is able to influence the decision-making of a listener.
I'm interested in the way that this sort of framing effect plays out in debate
--- not among political elites --- both academic and everyday. How does framing
an issue in a particular way or expressing it in particular terms effect the
way that it is evaluated? How able are we to resist these effects?
% Definitely look at marketing research if you're focusing in on people who
% aren't considered elites

There seems to be a fairly well developed strain of research on these effects
as they pertain to race. This literature seems to prefer the terms \emph{media
effects} and \emph{priming} over \emph{framing}, but the suppositions and
questions seem to be largely similar. I'm just beginning to read this
literature, starting with \citet{Mendelberg08a,Mendelberg08b},
\citet{Hutchings09}, and \citet{Yadon12}. I'll also be starting to read
\emph{Articulate While Black} \citep{Alim12} soon, which should help me iron
out what (sorts of) questions I want to be asking for my honors thesis.
% Great

\clearpage
\bibliography{course,extra}

\end{document}
