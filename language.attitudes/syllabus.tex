\documentclass[]{article}
\usepackage{lmodern,amssymb,amsmath,hyperref}
\usepackage[T1]{fontenc}
\usepackage[utf8]{inputenc}
% use upquote if available, for straight quotes in verbatim environments
\IfFileExists{upquote.sty}{\usepackage{upquote}}{}
% use microtype if available
\IfFileExists{microtype.sty}{%
\usepackage{microtype}
\UseMicrotypeSet[protrusion]{basicmath} % disable protrusion for tt fonts
}{}
\usepackage[usenames,dvipsnames]{color}
\hypersetup{breaklinks=true,
            bookmarks=true,
            pdfauthor={},
            pdftitle={},
            colorlinks=true,
            citecolor=blue,
            urlcolor=blue,
            linkcolor=magenta,
            pdfborder={0 0 0}}
\urlstyle{same}  % don't use monospace font for urls
\setlength{\parindent}{0pt}
\setlength{\parskip}{6pt plus 2pt minus 1pt}
\setlength{\emergencystretch}{3em}  % prevent overfull lines
\providecommand{\tightlist}{%
  \setlength{\itemsep}{0pt}\setlength{\parskip}{0pt}}
\setcounter{secnumdepth}{0}

\date{}

% Redefines (sub)paragraphs to behave more like sections
\ifx\paragraph\undefined\else
\let\oldparagraph\paragraph
\renewcommand{\paragraph}[1]{\oldparagraph{#1}\mbox{}}
\fi
\ifx\subparagraph\undefined\else
\let\oldsubparagraph\subparagraph
\renewcommand{\subparagraph}[1]{\oldsubparagraph{#1}\mbox{}}
\fi

\begin{document}

\title{Language Attitudes Syllabus}
\author{Anne H. Charity Hudley}
\date{Fall 2015}

\maketitle

\textbf{Syllabus information is subject to change at Prof.~Charity
Hudley's discretion based on the needs of the class yet within College
guidelines.}

\textbf{Professor:} \href{http://annecharityhudley.com/}{Dr.~Anne H.
Charity Hudley}\\
\textbf{Email:}
\href{maito:acharityhudley@wm.edu}{acharityhudley@wm.edu}\\
\textbf{Skype:} acharityhudley\\
\textbf{Phone:} (757) 221-3930\\
\textbf{Cell:} (804) 304-3493 (email before calling to set up a time)\\
\textbf{Place:} Tucker 325\\
\textbf{Time:} Tuesday 1530-1820\\
\textbf{Office Hours:} Monday 1500-1700, Wednesday 1300-1500, just after
all 1600-1800 \href{}{WMSURE events}, and by appointment

\textbf{Undergraduate Teaching Fellows:}
\href{mailto:ealambert@email.wm.edu}{Ebony Lambert} \&
\href{mailto:ehernandez@email.wm.edu}{Edward Hernández}

The optional class
\href{https://www.facebook.com/groups/WMlanguageattitudes/}{Facebook
group}.

\subsubsection{Course Objectives}\label{course-objectives}

This community based research seminar will examine the social, economic,
and educational ramifications of language attitudes including: the
linguistic intersection of race, gender, and social class; comparisons
of standardized and Standard English; and the role of linguistics in the
formation of language policy. We will have an emphasis on language
assessment in U.S. schools and the educational ramifications of
linguistic discrimination. Our approach will be hands-on and students
will be involved in research design and data analysis. Opportunities for
continued research participation and internships related to the topic
are available upon completion of the course.

\subsubsection{Course Requirements}\label{course-requirements}

Community Studies students (in the Sharpe Program, Community Studies
minor, and all affiliated Community Studies courses) will be expected
to:

\begin{enumerate}
\def\labelenumi{\arabic{enumi}.}
\tightlist
\item
  Develop and articulate clearly the scholarly and learning intentions
  associated with a community engagement project, partnership, or
  endeavor.
\item
  Collaborate effectively and responsively with faculty, fellow
  students, community individuals, or relevant others in forming shared
  goals, organizing collective resources, and designing appropriate
  approaches to community issues, according to intercultural and group
  partnering dynamics.
\item
  Facilitate learning within the classroom and outside of it; produce
  knowledge for further examination, use, or development in other
  educational and community capacities.
\item
  Evaluate their experiences in engaged scholarship, in terms of
  scholastic productivity, community impact, and other pre-determined
  project-specific goals and outcomes.
\end{enumerate}

Learning objectives for the Community Studies curriculum at the College
of William and Mary were developed collaboratively by faculty, students,
administrators, and co-educating community partners in order to advance
the core values and best practices in higher educational
service-learning, as articulated by the Carnegie Foundation. (See
www.carnegiefoundation.org for more information on national goals and
standards for community engagement and scholarship at colleges and
universities.)

Attendance in class and at the school is \textbf{MANDATORY}.
\textbf{Your overall course grade will be lowered by 5 points for each
unexcused absence and by 1 point for each unexcused lateness.} Other
people are depending on us including many students at William and Mary
who need our support. Please see professor Charity Hudley directly if
you need an excused absence, as proper documentation is required
(i.e.~from the Dean of Students, or Health Services). Religious and
spiritual related absences will be facilitated.

\textbf{Bring a computer to class if at all possible! If you do not have
a laptop or it is hard to bring, please see me right away.}

\subsubsection{Turning in Assignments}\label{turning-in-assignments}

All writing assignments are to be turned in electronically as Word
documents. The title of the document should be
yourlastname.CMST.250.MMDDYY.docx (ex:
Yourlastname.CMST250.09.08.2015.docx). Assignments should be double
spaced in 12 point Times New Roman font with 1-inch margins on all
sides.

\subsubsection{Grading Breakdown}\label{grading-breakdown}

\begin{itemize}
\tightlist
\item
  Class \& elsewhere participation (including class project questions
  and integration of community based research concepts) 20\%
\item
  Weekly Assignments 30\%
\item
  Take-home midterm essay 10\%
\item
  Final cumulative paper (15 pages) 35\%
\item
  Final oral presentation (10 minutes w/ 5 minutes for questions) 5\%
\end{itemize}

\subsubsection{Grading:}\label{grading}

Grades will be predicated on the overall quality of the work submitted
throughout the semester. Your grades will reflect your mastery of the
course material and your ability to critically analyze the social,
cultural, philosophical, and historical issues of our topics. The
following is a framework to assist you in understanding my expectations
regarding work for the course.

\begin{itemize}

\item[\textbf{A}] An A is awarded in recognition of exemplary work, reflecting
a high level of proficiency. An A- designation denotes exemplary work with some
minimal mechanical or organizational challenges.

\item[\textbf{B}] A B+ designation denotes satisfactory work with some evidence
of exemplary analysis. In most cases, B+ work offers intriguing and original
analysis, but may have some notable mechanical and/or organizational problems.
A B is awarded in recognition of satisfactory work, reflecting an acceptable
level of proficiency. A B- designation denotes satisfactory work with
substantial mechanical and/or organizational problems. In most cases, B- work
has focus and demonstrates a basic understanding of the relevant concepts and
arguments, but has limited evidence of originality or depth.

\item[\textbf{C}] A C is awarded for developing work, reflecting limited
evidence of proficiency. In most cases, C work fails to offer a coherent,
original or compelling thesis; the work has excessive mechanical and/or
organizational problems; and the author has demonstrated a basic understanding
of key concepts and arguments.

\item[\textbf{D}] A D is awarded for work that is superficial, demonstrates
very little effort, and limited in depth regarding the course materials and
student’s overall thinking. In addition, D work also fails to adhere to
instructions in the syllabus and contains numerous conceptual errors.

\item[\textbf{F}] In addition to being work that is superficial, it
demonstrates very little effort, and is limited in depth regarding the course
materials and student’s overall thinking. In most instances, F work also fails
to adhere to instructions in the syllabus and contains numerous conceptual
errors. In addition, the student fails to address or meet the basic
requirements of the assignment.

\end{itemize}

\subsubsection{Criteria For Evaluating Written
Assignments}\label{criteria-for-evaluating-written-assignments}

In addition to the above grading scale, written assignments will be
assessed according to the following criteria:

\begin{itemize}
\tightlist
\item
  Author expresses an explicit, detailed, and coherent argument, as well
  as, demonstrates critical thinking. A mere summary of the reading does
  not meet this criterion.
\item
  Author supports position with appropriate examples from the reading
  and research based evidence. Anecdotes and/or unsupported opinions do
  not satisfy this criterion.
\item
  Author incorporates additional empirical based claims and positions to
  support argument. Additional support from scholarly and/or credible
  sources satisfies this criterion.
\end{itemize}

\subsubsection{\texorpdfstring{\href{http://www.wm.edu/offices/deanofstudents/services/studentaccessibilityservices/}{Accommodation
for Students with Accessibility
Needs}}{Accommodation for Students with Accessibility Needs}}\label{accommodation-for-students-with-accessibility-needs}

\paragraph{Procedure for Requesting
Accommodations}\label{procedure-for-requesting-accommodations}

\begin{enumerate}
\def\labelenumi{\arabic{enumi}.}
\item
  The student will verify disability with appropriate documentation.

  Please refer to specific
  \href{http://www.wm.edu/offices/deanofstudents/services/studentaccessibilityservices/requesting-accommodations/documentation\%20/documentation\%20guidelines/index.php}{Guidelines
  for Documentation}. In all cases, documentation that supports
  reasonable accommodation must clearly indicate:

  \begin{itemize}
  \tightlist
  \item
    The existence of a disability (as defined by state and federal
    regulations);
  \item
    That the disability substantially limits a major life activity
    including a statement of the nature and extent of the limitations;
    and
  \item
    A statement of what accommodation(s) is recommended
  \end{itemize}

  If this step has already been completed, move to step 2.
\item
  The student will make an appointment with Student Accessibility
  Services (SAS) staff to review documentation provided and assist in
  determining appropriate accommodation.

  Reasonable and appropriate accommodation is often determined through a
  flexible, interactive process involving the student and SAS staff.
  Final responsibility for selection of the most appropriate
  accommodation rests with the College. Early consultation regarding
  accommodation is essential whenever questions of compliance and/or
  funding for accommodation are involved. Late requests will be honored
  to the best of our ability, but could result in delay, substitution,
  or inability to complete.
\item
  SAS staff will prepare an Accommodation Letter outlining the
  accommodations for which the student has been approved.

  Letters regarding student-specific accommodations are prepared for
  relevant faculty members/instructors or for that student's designated
  contact at the graduate program. SAS staff is available to explain the
  procedures and assist students as needed. If difficulties arise in
  obtaining accommodations or there are concerns related to this
  process, it is the student's responsibility to contact SAS
  immediately.

  Step 3 will be repeated each semester (or at other times, as
  appropriate) to review accommodations needed.
\end{enumerate}

\textbf{Note:} Students engaged in academic work off site (such as
internships for course credit and study away) are not generally eligible
for accommodation by the College of William \& Mary. The ability to
accommodate a student with a disability should be a precondition to any
business, agency, or organization that wants to participate in an
internship or externship agreement with the College. In the case of
academic course work taken at another institution of higher learning,
SAS staff will serve as consult and provide information~as needed~for
students arranging accommodations at these locations.

\subsubsection{Note on Sustainability}\label{note-on-sustainability}

I support sustainability initiatives on the WM campus. To reduce paper
use, most of your course documents (including the syllabus, readings,
and most assignments) will be provided on the course Wiki. Please try to
save paper by reading documents and text online whenever possible. If
you must print out documents, please consider printing double-sided
and/or with two sheets per page. Assignments will be turned in
electronically. For more information, please see the
\href{http://www.wm.edu/sustainability}{Sustainability at W\&M website}.

\subsubsection{\texorpdfstring{Modified from Tolerance.org's
``Declaration of
Tolerance''}{Modified from Tolerance.org's Declaration of Tolerance}}\label{modified-from-tolerance.orgs-declaration-of-tolerance}

\url{http://www.tolerance.org/}

\url{http://www.shepherd.edu/alliesweb/resources/Ten_Ways_Campus.pdf}

``Tolerance is a personal decision that comes from a belief that every
person is a treasure. I believe that America's diversity is its
strength. I also recognize that ignorance, insensitivity and bigotry can
turn that diversity into a source of prejudice and discrimination.

To help keep diversity a wellspring of strength and make America a
better place for all, I pledge to have respect for people whose
abilities, beliefs, culture, race, sexual identity or other
characteristics are different from my own.

After examining hundreds of cases involving thousands of students,
tolerance.org found this: Although administrators, faculty and staff are
vital players in any response, it is the student activist who makes the
most difference.

Because things improve only when people like you take action.

\textbf{Because each student has the power to make a difference.}

And because apathy, in some ways, is as dangerous as hate."

\url{http://www.wm.edu/about/diversity/}

\subsubsection{Course Texts}\label{course-texts}

\textbf{Course Textbooks are required.} Please bring textbooks to class
at each meeting unless otherwise indicated. You may purchase the paper
or online edition of the texts. If you use the texts online, please
bring a laptop to class at each meeting. I really advise using the
online versions! Save your back and a tree! There are copies for your
references in Blow 236.

\paragraph{Required Texts}\label{required-texts}

Cress, C., Collier, P., \& Reitenauer, V. (2013). \emph{Learning through serving: A student guidebook for service-learning and civic engagement across academic disciplines and cultural communities}. New York: Stylus.

Lippi-Green, R. (2011). \emph{English with an accent: Language, ideology, and discrimination in the United States}. United Kingdom: Routledge.

\paragraph{Texts that will be given to
you}\label{texts-that-will-be-given-to-you}

Charity Hudley, A. H., \& Mallinson, C. (2010). \emph{Understanding English language variation in U.S. schools} (J. A. Banks, Ed.). New York: Teacher’s College Press.

\paragraph{Required Articles}\label{required-articles}

Charity, A. H. (2013). Sociolinguistics and social activism. In R. Bayley, R. Cameron, \& C. Lucas (Eds.), \emph{The Oxford Handbook of Sociolinguistics}. Oxford University Press.

Dunstan, S., \& Jager, A. (2015). Dialect and influences on the academic experiences of college students. \emph{Journal of Higher Education}, 86(5).

Simmons, R. J. (1998). My mother’s daughter: Lessons I learned in civility and authenticity. \emph{Texas Journal of Ideas}, History and Culture(20), 20-29.

Heymann, L. (2008). The reasonable person in trademark law. \emph{St. Louis University Law Journal}.

Heymann, L. (2011). The grammar of trademarks. \emph{Lewis and Clark Law Review}, 14 (4).


Labov, W. (1978). \emph{How i got into linguistics, and what i got out of it}. Retrieved from \url{http://www.ling.upenn.edu/~wlabov/HowIgot.html}

Linneman, T. (2013). Gender in Jeopardy!: Intonation variation on a television game show. \emph{Gender in Society}, 27(1), 82-105.

Monteith, M., \& Winters, J. (2002). Why we hate. \emph{Psychology Today}, 35(3).

Zentella, A. C. (2002). Latin@ languages and identities. In M. Su\'{a}rez-Orozco 7 \& M. P\'{a}ez (Eds.), \emph{Latinos! an agenda for the 21st century}. Berkeley: University of California Press.

\paragraph{Suggested Articles}\label{suggested-articles}

Charity, A. (2008). African-American English: An overview. \emph{ASHA Division 14 newsletter}.

Charity, A. H. (2008). African American English. In H. Neville, B. Tynes, \& S. Utsey (Eds.), \emph{Handbook of African-American psychology}. Sage.

Eckert, P., \& McConnell-Gineth, S. (2003). \emph{Language and gender}. Cambridge University Press.

Fryer, R. G., \& Levitt, S. D. (2004). The causes and consequences of distinctively Black names. \emph{The Quarterly Journal of Economics}, 119(3), 767-805.

Labov, W. (1995). Can reading failure be reversed. In V. Gadsden \& D. Wagner (Eds.), \emph{Literacy among African-American youth}. Cresskill, NJ: Hampton Press.

Levon, E. (2006). Hearing gay: Prosody, interpretation and the affective judgments of men’s speech. \emph{American Speech}, 81, 56-78.

Markley, D. E. (2000). \emph{Regional accent discrimination in the hiring process: A language attitude study} (Unpublished master’s thesis). University of North Texas.

Matsuda, M. (1991). Voices of America: Accent, antidiscrimination law, and a jurisprudence for the last reconstruction. \emph{Yale Law Journal}.

\subsubsection{Writing Resources}\label{writing-resources}

The \href{http://www.wm.edu/wrc/}{Writing Resources Center} (Swem
Library, 1st Floor, 757-221-3925) is dedicated to helping William \&
Mary students improve their writing and oral communication skills.

Additional online resources are available at:
\url{http://owl.english.purdue.edu/}

\end{document}
