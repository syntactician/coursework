\documentclass[man,12pt,natbib]{apa6}
\usepackage[colorlinks=false]{hyperref}
\usepackage{amssymb,amsmath,times}
\linespread{1.5}

\begin{document}

\title{14 September Response Paper}
\shorttitle{14 September 2015}
\author{Edward Hern\'{a}ndez}
\date{2015-09-14}
\affiliation{College of William \& Mary}
\maketitle

The first time I read this chapter, I was basically fresh out of high school. I
had only been judging debate for a couple of months. I saw the communicative
burden and language subordination as abstractly relevant to debate, I think,
but I was worried about people other than be not accepting their communicative
burden. Now, I've come to realize that I need to be worried about how these
things effect me as a judge. 
% Good example for why the reflection aspect of this work is key

Among judges there seems to be an ideology that the better debater is the one
who makes them try less. If the judge doesn't have to try hard to understand
the argumentation, the debater did a good job. Conversely, if the judge has a
hard time following the speech or understanding the points being made, the
debater is bad. High value is placed on “roadmapping” and “signposting.” A
roadmap is an initial overview of the points to be made in a speech. Signposts
are utterances internal to the speech which announce when the speaker has moved
onto a new major point. I've heard judges say that roadmaps and signposts
should be so extensive that they should never have to think or make a decision
about how to record an argument; it should always come with an introduction and
a heading. This ideology is so strong that I've even heard of a number of
judges refusing to take notes on arguments simply because they weren't
roadmapped or signposted adequately. I've been effected by this ideology as
well; I've been habitually asking for students to roadmap and signpost. 
% Good to note

Now I wonder whether this has anything to do with skill in debate.  Does the
ability to take on more of the burden of communication, as
\citet{Lippi-Green11} describes it, make a debater better? 
% Great question—tie to the descriptions of discourse patterns of AAE and SAE  
I want to say no. It seems like saying yes would mean that every debater who
doesn't speak SAE is inherently worse. Even with this sort of motivation to say
no, it's hard for me to say it wholeheartedly. My entire idea of what makes
debate good is tied up in this idea of communicative effectiveness and taking
more than your share of the communicative burden.

For some students, specifically students who speak in non-standardized ways,
judges who are willing to accept their half of the communicative burden are a
necessity. If a judge refuses to do their best to listen, to accept their
responsibility in the communicative act, only debaters who speak like them, or
who are protected by dominant language ideologies, will have a fair chance.
But, if we somehow remove the distribution of the communicative burden as a
criterion for judging, what would judging look like? I'll admit I'm entirely
lost on this question right now. I don't have any answers just yet.

Luckily, I'm meeting with Isaiah Moore tomorrow to discuss exactly this! 
% Awesome! 
Additionally, we'll be discussing the possibility of working together on future
work for this class (and possibly beyond). I'm really excited. Of course, I'll
keep writing updates on that as it develops. 
%Great plan!

\clearpage
\bibliography{course,extra}

\end{document}
