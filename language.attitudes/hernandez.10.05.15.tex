\documentclass[doc,12pt]{apa6}
\usepackage[colorlinks=false]{hyperref}
\usepackage{lmodern,amssymb,amsmath,apacite}
\linespread{1.5}

\begin{document}

\title{5 October Response Paper}
\shorttitle{5 October 2015}
\author{Edward Hern\'{a}ndez}
\date{2015-10-05}
\affiliation{College of William \& Mary}
\maketitle


\citeA{Linneman13} presents a perfect example of what I've been noting
throughout the semester: tendencies to ascribe qualities to speakers based on
their language variations. He highlights specifically our tendency to view
uptalk as marking uncertainty. This is directly relevant to debate.  
% Good – ask him about it tomorrow 
It pays to sound confident. I'd be willing to bet that a debater would be
scored measurably lower if she used uptalk when delivering her main contentions
(especially if the debater were not male or traditionally masculine). I think
that few judges would specifically cite uptalk as a reason for grading the
debater down, but many would caution the debater to sound more confident. This
is exactly the sort of reaction I'm worried about. 
% And why this type of research is so important—it has very direct and visceral
% consequences.

Even if we understand that uptalk is just how some people talk, and that this
variation isn't inherently linked to uncertainty and certainly isn't inherently
bad, we may still perceive a speaker who uses it as uncertain. 
% That’s exactly the conundrum 
I don't see a way to remove this sort of impression, which I believe largely
happens below the level of conscious introspection, from the decision calculus
of judging. 
% What about the use of education? 
Even if one could, it's not clear that all judges would be willing to do so. I
anticipate that some would argue that students should be trained not to use
uptalk since for many listeners it indexes uncertainty, which they should avoid
doing to appear more confident or assertive. That assertion makes me
uncomfortable. To say that students should avoid using gendered styles of
speech to avoid indexing perceived negative traits seems awfully similar to
asking students not to use their native cultural variations (like African
American English). 
% Bingo. 

On the other hand, training debaters not to speak with styles that judges will
not appreciate may be of utility to the debater. If I encourage students to use
their own varieties as they are, without changing how debate is judged, I may
well be setting them up for failure. 
% The \$100,000,000 issue – here’s the abstract for the latest paper I
% submitted to present at AERA: Valuing African-American English in the
% Classroom
%
% Anne H. Charity Hudley The College of William and Mary The National Council
% of Teachers of English's 1974/2003 Resolution on the Students' Right to Their
% Own Language supports the idea that students should be able to use their home
% languages and language varieties in educational contexts. Yet, much
% information for educators about African-American English (AAE) employs a
% ``code-switching'' model that is designed to encourage AAE speakers to be
% adept at switching between AAE and the language variety of the school or
% dominant culture. 
%
% At their worst, code-switching models help speakers of AAE acquire
% standardized language while demeaning AAE in the process. At their best,
% code-switching models help students use their knowledge of AAE and build on
% it while helping students acquire standardized English. Either way, the
% ideology of code-switching, while touted as practical and effective in
% classrooms, is highly racialized. The message that students glean from the
% hidden curriculum of code switching is that students and educators are best
% served by leaving AAE at the classroom door. The code switching ideology
% promotes internalized racism as well as linguistic insecurity for both
% students and educators.
%
% In order to create a more socially just teaching context, I demonstrate
% examples of ways to work with educators to help them both acquire knowledge
% about and value for AAE. The model encourages educators to become language
% learners with their students. I focus on classroom practice and share ways of
% promoting the study and use of AAE in classrooms through oral and written
% exercises. 
%
% My research findings show that the misinterpretation of messages about code
% switching happens because even educators who possess positive sentiments
% about African-American English may not have the linguistic skills to
% effectively use and teach about AAE in classrooms. It is important for
% education researchers to work with educators to face their fears of not
% wanting to use AAE in the classroom because they do not want to be perceived
% as mocking African-Americans because they are not fully fluent in AAE.
% Educators report that they need explicit help with knowing what to say to
% students about AAE and why it is used in some contexts and not others. They
% also wish to know how to have “courageous conversations” about how language
% and race intersect in educational contexts. 
%
% I conclude with introspective research on how education researchers ourselves
% perpetuate code-switching practices through the ways in which we structure
% our own classes. Audre Lorde (1984) reminded us ``The Master's Tools Will
% Never Dismantle the Master's House.'' Education researchers must also provide
% spaces and places for students to use and to acquire African-American English
% so that we educate linguistically and communicatively competent researchers. 
%
% Such an approach has implications for how researchers navigate our positions
% in higher education more broadly including how we advocate for students in
% the admissions process and what we value in our own students’ oral and
% written expression. Will there ever be a time at AERA when I could both
% submit the proposal for and give this paper freely in African-American
% English? If not, what does that say about our overarching objectives?
%
% References:
%
% Lorde, Audre. “The Master’ s Tools Will Never Dismantle the Master’s House.”
% 1984. Sister Outsider: Essays and Speeches. Ed. Berkeley, CA: Crossing Press.
% 110-114. 2007. Print.
%
% National Council of Teachers of English. (2003). Resolution on affirming the
% CCCC “Students’ right to their own language.” Retrieved from
% www.ncte.org/positions/statements/affirmingstudents

While I think that their varieties are important and valuable, they may well
remain damaging to their performance in debate. If I go in to do this work to
improve debate, and I inadvertently hurt the win/loss record of every debater I
work with, I will have failed the community I set out to help. This project is
a failure if I don't help the underrepresented debaters currently in debate,
not just vaguely improve judging norms for future cohorts of debaters. I have a
responsibility to the needs of the community I am working with, and damaging
their chances in debate certainly constitutes a failure to uphold that
responsibility \cite[p.~140]{Cress13}. 
% So how do we work towards the greater change and preserve the immediate
% individual?

Re-reading \citeA{Linneman13}, I have some other concerns with his assumptions.
A pair of sentences contain quite a lot of them, so I'll quote them here:
\begin{quote} If uptalk indeed signals uncertainty and submission, we would
	expect that those from the lower classes would be more likely to use
	uptalk as a signal of their lower-class status. In a study of uptalk
	use in Australia, some researchers have found that uptalk use is
	related to social class: Middle-class speakers used it the least, while
lower– working class speakers used it the most (Guy et al. 1986).
\cite[p.~86]{Linneman13} \end{quote} Linneman here questions that uptalk
signals uncertainty or submission, but he does not question that ``the lower
classes'' are both uncertain and submissive.  This assumption is huge, and, I
think, unfounded. It certainly doesn't seem to be the case that anyone is apt
to display either uncertainty or submission in every part of her life. I also
see no reason to assume that people of lower classes are uncertain more
frequently or pervasively than people of higher classes. I have a
methodological complaint with his mixing of studies of Australian and New
Zealand English with his study of a North American game show. 
% Ask him about why he did so? 
Who's to say that uptalk indexes precisely the same thing in all the relevant
dialects? Additionally, assuming the gender of participants (which it appears
he did) is dangerous, especially when you are studying performative gendered
behaviors. Besides my complaints, the paper is interesting, and I'm looking
forward to hearing him talk it through again.

\clearpage

\bibliography{cmst}
\bibliographystyle{apacite}

\end{document}
