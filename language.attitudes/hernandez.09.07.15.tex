\documentclass[doc,12pt]{apa6}
\usepackage{lmodern}
\usepackage{amssymb,amsmath}
\usepackage{ifxetex,ifluatex}
\usepackage{fixltx2e} % provides \textsubscript
\usepackage[T1]{fontenc}
\usepackage[utf8]{inputenc}
  
\usepackage[usenames,dvipsnames]{color}

\usepackage{graphicx,grffile}
\graphicspath{ {.resources/} }
\makeatletter
\def\maxwidth{\ifdim\Gin@nat@width>\linewidth\linewidth\else\Gin@nat@width\fi}
\def\maxheight{\ifdim\Gin@nat@height>\textheight\textheight\else\Gin@nat@height\fi}
\makeatother
% Scale images if necessary, so that they will not overflow the page
% margins by default, and it is still possible to overwrite the defaults
% using explicit options in \includegraphics[width, height, ...]{}
\setkeys{Gin}{width=\maxwidth,height=\maxheight,keepaspectratio}
\setlength{\parindent}{0pt}
\setlength{\parskip}{6pt plus 2pt minus 1pt}
\setlength{\emergencystretch}{3em}  % prevent overfull lines
\providecommand{\tightlist}{%
  \setlength{\itemsep}{0pt}\setlength{\parskip}{0pt}}
\setcounter{secnumdepth}{0}

\usepackage[colorlinks=false]{hyperref}
\usepackage{dirtytalk}
\usepackage{apacite}

\begin{document}

\title{7 September Response Paper}
\shorttitle{7 September 2015}
\author{Edward Hern\'{a}ndez}
\date{2015-09-07}
\affiliation{College of William \& Mary}
\maketitle


% 100
One of the most essential issues for me to address this year comes from
\citeA{Cress13}: community partnerships. All the high school students that I
personally debated with have graduated and gone on to colleges around the
country. Most of them are no longer available to work with me in person.  I am
sure some of them would be willing to be interviewed on their experiences, and
I would love to still gather data about them. 
%That’d be awesome 
However, they are no longer members of the community I want to serve
% —according to what definition of community
. They are no longer present to help with or participate in my project
%—it sounds like they are- just from afar. 
I need to form new bonds with current debaters. 
% Good
As \citeA{Cress13} define them, I have \textit{\textbf{assets}}---resources and
qualities that I bring to the table---and I have \textit{\textbf{interest}} in
effecting change. However, I do not have the same \textit{\textbf{needs}} as
the community, and I cannot \emph{a priori} or by introspection find them out.
I need to forge new community partnerships to identify the current needs of the
community (p. 24).
% excellent

This is especially necessary since I know that the debate leagues are no longer
as I remember them. There is a new set of judges, several of whom recently did
policy debate at big schools. They seem to me to be more accepting. New judges,
of course, bring new issues, and I'm sure that there is still more work to be
done, but I don't yet know what work needs to be done. Even if I did know, I
couldn't do it without new community partners.
% Good to re-calibrate your investigation in this way

The reading also brought up another question that I've struggled with in the
past in the context of debate, but never found a satisfactory answer to. It's
troubling me more now that I have more judging experience under my belt.
Sometimes, I think argumentation is just bad. It might simply not connect the
premises to the conclusion. It may seem unorganized. It may seem
ill-researched. I still think that some debaters do a bad job, in that they
don't convey ideas well, or don't respond adequately to their opponents
assertions. But I worry that some (or much) of what I consider to be bad
argumentation may simply be argumentation performed in a way I'm not familiar
or comfortable with. 
% Crucial to note in the context of notions of standard and standardized
% argumentation 
\citeA{Lippi-Green11} gives examples of two semantically equivalent strings:
\say{he doesn't go there anymore,} and \say{he don't go there no more} (p. 55).
I'd like to think I'd never be swayed by this sort of difference, and in a case
this obviously equivalent, I don't t{hink I would. I do worry, though, that a
	non-linear presentation, perhaps in a \say{topic-associating}
	\cite[p.~102]{CharityHudley10} style characteristic of African American
	English, might seem less organized or focused to me. I'm concerned that
	I might evaluate it negatively, or at least less positively that a
	linear presentation of the same ideas. 
% Very important to note—you can even test it out! 
I need to think this through more thoroughly, as I don't see an immediate or
easy answer as to how to evaluate argumentation without allowing my implicit
standardization ideologies influence my appraisals. I would love to discuss
this with you at some point, perhaps in the context of Education literature. 
% Sure thing! 

This year I am very interested in working with Isaiah Moore. His work with
middle schoolers seems great to me. 
% I’ll have y’all work in a group together tomorrow
I don't have a ton of experience working with children below high school age
(besides coaching recreational soccer), and I need to at least try to
anticipate how judging and working with middle schoolers would be different
than high schoolers. I also plan to meet with Isaiah to talk about working
collaboratively, either on a project for this class or in general. 
% Good plan!

\clearpage

\bibliography{cmst}
\bibliographystyle{apacite}

\end{document}
