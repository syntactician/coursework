\documentclass[man,12pt]{apa6}
\usepackage[colorlinks=false]{hyperref}
\usepackage{lmodern,amssymb,amsmath}
\usepackage[natbibapa]{apacite}
\usepackage{times}
\linespread{1.5}

\begin{document}

\title{Expanded Action Plan \& Midterm Planning}
\shorttitle{Expanded Action Plan}
\author{Edward Hern\'{a}ndez}
\date{2015-10-19}
\affiliation{College of William \& Mary}
\maketitle

% 95

\subsection{Experience}\label{experience}

\begin{quote}
	Describe ways in which you have observed issues of language and
	communication manifest in your previous volunteer, service-learning
	experience, civic engagement, or community-based research experiences.
\end{quote}

I work with debaters. I judge debate. I teach debaters. I recruit and sometimes
train judges. I have occasionally trained new and inexperienced coaches. Most
of the time a volunteer. Frequently I'm asked for help. Rarely, I'm paid for my
help. It's always community service and service learning.
% Indeed

Every issue in debate is (at least partially) an issue of language and
communication. I often find myself faced with problems discussed in
\citet{Lippi-Green11} and \citet{Cress13} in the real world. Language
subordination is active just as \citet{Lippi-Green11} discusses it (ch.~5). I
also find myself in a position to foster and use community partnerships. I need
community partners both to help me in my service --- other judges to staff
events, students and faculty to hone my ideas --- and to keep me in touch with
community \emph{needs} and \emph{interests} \cite[ch.~2]{Cress13} --- current
debaters, coaches, officials, and other judges --- since the debate community
is no longer the way I remember it from my high school years.
% Good summary to build a paper or a thesis on!

(There is one issue with which I am constantly confronted, in working in
debate: how to fairly evaluate students and their work.) In debate, this is
almost a completely a language issue. The student work which it is my job to
evaluate \emph{is} language. Students speak, and I am to assess it. Not only am
I to assess it, I am supposed to say which of two students (or teams of
students) is \emph{better} --- largely better at using language --- which is a
giant responsibility. I feel vastly unprepared for this, and I know other
judges are unprepared as well. I still think that this needs to be changed.
% You can do it!

\subsection{Goals}\label{goals}

\begin{quote} 
	What are some of the specific goals you wish to accomplish through
	participation in Language Attitudes this fall?
\end{quote}

I have a few different sorts of goals for Language Attitudes this year.  I want
to get started on my honors thesis.
% Now's the time
I hope to use the remaining papers for the class to narrow possible topics and
develop ones that seem promising. I hope to use my joint office hours with you
to do this as well. I hope to have a considerable chunk of at least the
planning of my thesis done by the end of the semester.

I intend to keep judging debate. As I've said, I need to find new community
partners, and identify current needs. I plan to do that, and I intend to talk
about those experiences in my papers for this class, hopefully working toward
honors topics. I'm still not sure whether I want to focus on debate for my
honors project, but I know that debate always gets me thinking about the sorts
of things that I \emph{do} want to study. Even if I don't end up focusing on
debate, I think I will probably want to focus on some linguistic topic relevant
to education.
% You can tie your interests together—let’s work on mashing your ideas up this
% Wednesday.

At the same time, I want to work with Isaiah on his ideas for middle schoolers.
I think forensics is a great tool for teaching necessary skills, and I think
that middle school is probably a better age to start than high school.
% I think so as well
Both forensics and middle schoolers are a bit out of my comfort zone, so I'm
happy to have someone else to work with. I doubt I would do this sort of work
alone, so I'm very excited to be doing it with him. I suppose it's also a
personal goal to get Isaiah to see the light about gender roles.
% Good reflection
I think that having the sort of ideologies that he holds about the ways that
boys and girls should be treated differently is detrimental to the kids, and
not good practice for judging forensics or doing education.

Lastly, I want to get an idea of how I ought to be assessing students.  As I've
discussed in earlier papers, I do not want to enforce a code-switching model
that in any way incentivizes Standardized English over other varieties. On the
other hand, I don't want to set these kids up to fail with other judges or
educators, who I can't expect to assess the way I do.
% So how will you address that tension? What do organizations that support
% English education say about the matter? See:
%	http://www.ncte.org/positions/assessment
%	http://www.ncte.org/positions/language

I also struggle with identifying and correcting my biases.  Just tonight, I was
watching a documentary \cite{Lee06} which contained a spoken word poem that I
assume was good, given that Spike Lee thought to include it. However, it didn't
\emph{sound} good to me.
% Why not? Be descriptive
I don't have context for other New Orleans poets, or spoken word poets in
general, so I don't know how to assess it. I worry that this sort of
unfamiliarity with particular varieties, or deeply ingrained language
subordination ideologies, will lead me to inadvertently prefer Standardized or
White debate, speech styles, ideas, etc.. I want to work toward a system (or at
least an ideological position or some other sort of goal) for assessing
students' work fairly.
% Good

Maybe some subset of the assessment question will lead me to an honors topic.

\subsection{Schedule}\label{schedule}

\begin{quote}
	Include an update about your schedule planning progress. Share what method
	you are using to schedule and what your major successes and challenges to
	schedule are.
\end{quote}

I use a Google Calendar to schedule nearly everything in my life. I share
calendars with my partner and my mom, so I can make sure to keep time open for
real life outside school.
% Great!
I have a shared Google Drive folder with my coworkers in the SNaPP Lab, to make
sure we can efficiently coordinate our time and avoid duplicating work (and
comparable Google Docs with Prof.~Cochrane and her other research assistant are
in the works). This system works quite well for me, since it's shared among
multiple devices and people effortlessly. It's been good for scheduling fixed
events. However, I still have trouble scheduling more nebulous time
commitments, like studying and sleeping.

This semester I've missed a single scheduled event, and I only missed it
because I slept through my alarm.
% That's a tough one
My calendar system definitely works to get me to events. My primary problem
right now is that I don't record my commitments to studying, eating, and
sleeping.
% Do it so that you have the balance
I consistently find myself rushed to read or struggling to find time to sleep.
I think I'm going to start scheduling meal times and trying to regulate my
sleep schedule using my calendar.
% Good SLEEP RULES
Hopefully that helps. Also, I plan to have a talk with Prof.~Settle, about my
involvement in the SNaPP Lab. My interests only tangentially intersect with the
work being done in the lab, and while I enjoy working there, I fully intend to
reduce the amount of work I do there.
% It's important to focus on your own goals at this point
That should help my scheduling process immensely.

\clearpage

\bibliography{cmst}
\bibliographystyle{apacite}

\end{document}
