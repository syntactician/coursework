\documentclass[man,12pt,natbib]{apa6}
\usepackage[colorlinks=false]{hyperref}
\usepackage{amssymb,amsmath,times}
\linespread{1.5}

\begin{document}

\title{9 November Response Paper}
\shorttitle{9 November 2015}
\author{Edward Hern\'{a}ndez}
\date{2015-11-09}
\affiliation{College of William \& Mary}
\maketitle

\noindent
\emph{Starting this week, I will be writing my papers about possible honors
topics (and the development of my ideas about those topics), rather than the
exact readings or assignment for the week (though I did still do the reading).}
% https://global.oup.com/academic/product/articulate-while-black-9780199812981?cc=us&lang=en&

\vspace{12pt}

As we've discussed, I want my honors thesis to be solidly focused on
Linguistics, and I want to make some effort to incorporate Political Science
(and I may want to incorporate methods from Psychology as well). This leads me
to want to investigate political rhetoric. I've talked about possible topics
with Professor Settle and with a group of Linguistics students and professors,
who gave me vastly different ideas.

A few weeks ago, I attended a meeting of the new Computational and Experimental
Linguistics Lab (CELL), during which we all tossed around ideas for new
studies. I brought up my honors thesis, and my desire to mix linguistics with
political science. We ended up talking extensively about Donald Trump and the
particular way that he speaks. We reflected specifically that we have seen
non-linguists describe the way he speaks as ``simple'' \citep{Katz15,Byrne15},
and that he and his his speech, according to some news sources, are perceived
as ``authentic'' \citep{Burns15,Green15}. We find it interesting that he is not
universally perceived as authentic \citep{Reingold15}, and that, from what
we've seen, those who are characterizing him as authentic seem to be doing so
at least partially based on his speech. They say things like that Trump is a
``straight talker'' \citep[para.~1]{Green15}, or that his words are his own,
``dangling participles and all'' \citep[para.~15]{Burns15}. At that meeting, we
speculated that we might be able to test whether it was Trump's \emph{speech}
which caused him to be perceived as authentic, and, if it was, to identify what
features of his speech led listeners to perceive him that way.

We conceived of a few possible studies. The most developed of our ideas was to
find politically innocuous topics about which to write speeches in the
linguistic styles of Trump and other candidates. We imagined constructing a
speech with Trump's sentence length and his common words and constructions, and
another speech (or potentially several other speeches) with characteristics of
another politician (or several other politicians) who we think are more typical
and less authentic. We could, we thought, with a design like that, show that
there was some effect of Trump's language specifically, on his being perceived
as authentic.

We ran into a lot of trouble thinking about how we would control for things.
What if the linguistic features that led to his being perceived as authentic
were prosodic? Would we have to manufacture recordings of actors doing speeches
prosodically differently? That sounds like a horrible headache. We also
couldn't conceive of controlling for the effects of things like gesture or
facial expression, let alone clothing or appearance, which we think are almost
necessarily meaningful and significant in an analysis of candidate perception.

I have also talked to Professor Settle about possible topics. Her inclination
is that the perceived authenticity of a political candidate has as much, if not
more, to do with their positions as their way of speaking. She noted that she
couldn't imagine Trump being perceived as authentic in the same way if his
positions were exactly in line with the rest of the candidates, even if his
speeches were linguistically identical. She thinks it's something particular
about the extremity of his ideas. She noted that Bernie Sanders is also
frequently perceived as authentic\citep{Burns15}, but his speech is considered
to be much more complex \citep{Katz15}.

Currently, I don't know whether that sort of study would even be feasible. If
it is, there's another student studying under Professor Settle, Jacob Nelson (a
former Sharpe Scholar) who is working on a similar idea. I met with him this
week to talk about our research ideas. He's also working with idea of
complexity of speeches and the effect that simpler or more complex speeches
might have on audiences. He's working currently on speeches by Mike Huckabee
and Elizabeth Warren, which he notes are stylistically very different. I'm not
sure how useful he'll be in developing my ideas, but I was able to help him
develop his quite a lot.

After we discussed Trump, Settle and I started to gravitate toward ideas of
racialized rhetoric. She has directed me to literature on ``racial priming,''
which seems promising
\citep{Junn06,Mendelberg95,Mendelberg97,Mendelberg01,Mendelberg08a,Mendelberg08b}.
I haven't had a lot of time to really read up and immerse myself in the
literature yet; I hope to start this week. I'll keep reporting in on new
literature and ideas as I run into them. Right now I think that racialized
rhetoric is the direction I'm going. I don't yet know what questions I want to
ask or how I might answer them, but I hope to use the political science
literature (which I've cited here) as a starting point to get there.

\clearpage
\bibliography{course,extra}

\end{document}
