\documentclass[man,12pt]{apa6}
\usepackage[colorlinks=false]{hyperref}
\usepackage{lmodern,amssymb,amsmath,apacite}
\usepackage{times}
\linespread{1.5}

\begin{document}

\title{28 September Response Paper}
\shorttitle{28 September 2015}
\author{Edward Hern\'{a}ndez}
\date{2015-09-28}
\affiliation{College of William \& Mary}
\maketitle

% 100
Earlier this year, I noted that it will be essential for me to form new
community partnerships as discussed by \citeA{Cress13}. 
% Good to note 
My high school partners have all graduated and gone off to college. Most of
them are no longer available to work with me, at least in person.  I need to
form new bonds with current debaters, and with middle school students and
faculty, if I am going to help Isaiah realize his ideas. As I discussed earlier
this semester I have \textit{\textbf{assets}}—--resources, skills, and
qualities that I bring to the table—and I have \textit{\textbf{interest}} in
effecting change. However, I do not have the same \textit{\textbf{needs}} as
the community, and I cannot \emph{a priori} or by introspection find them out.
Up until this year, I had a good idea of some of the needs of the high school
policy debate community, but now the judging pool is vastly different, and I
have no idea what the debaters need. 
% Crucial to make a plan for 
I know even less about the needs of the middle schoolers that Isaiah and I want
to help. I need to forge new community partnerships pretty immediately
\cite{Cress13}.

I bring this up again because it is a necessary step to evaluation.
\citeA{Cress13} discuss evaluation as structured reflection on a service
project (p. 164). When I was reading this section, it struck me that I cannot
structure any sort of evaluation plan yet, since my objectives aren't decided.
I can't reflect on how well I've met objectives of there aren't any objectives
to meet. That led me to try to come up with some tentative objectives, only to
hit a brick wall; I really don't know what people need. 
% So start with planning questions 
I need to figure this out from them. My project used to have the goal of
training judges, and that would difficult, but possible, to evaluate.  But now
there is a new set of judges, several of whom are fresh out of big debate
schools. The goals I set originally are no longer useful. I really do need to
find new community partners if I want to identify new needs and work towards
filling them in local debate.

I think Isaiah does have some specific goals in mind though, from our
discussions. I also think that those goals lend themselves to evaluation.  He
has specifically targeted middle school learning objectives, in line with the
curriculum and the SOL standards, and hopes that his program will help students
reach those learning outcomes. 
% Good –these are concrete and will get buy in 
It seems to me that that sort of objective is the easiest to evaluate
theoretically, but I'm not sure how we would do it practically. It could be
measured on a test, like an SOL, since that is the sort of assessment we would
theoretically prepare them for. But, how do we ask a kid in an optional (and
hopefully fun) after-school activity to take a test for us? 
% By offering incentives! 
I have to think this through a lot more before I have an answer.

Of course, these questions bring me back to my internal dilemmas about student
assessment. I don't know what good assessment looks like. I know that African
American English is a language, that it has predictable grammar, etc. But
people who may be assessing students down the road may not. It may hurt these
kids to not prepare them for other sorts of assessment. 
% That’s the ongoing conundrum 
But does that mean I should assess them as if I didn't have this understanding?
I can't imagine that that would be a good thing to do.  To encourage kids to
prepare for graders with negative attitudes toward their language seems
identical to buying into that ideology myself. But, if I don't, am I doing them
some sort of disservice? It's not an easy question, actually. 
% Not at all—because we should be wary of anyone who wants something different
% for us than what has gotten us our privilege—so it gets really complex 
Even having come to a decision that I ought to value students' language
varieties and allow them to speak the way they want, I worry that I still have
negative attitudes toward varieties like African American English as
\citeA{Lippi-Green11} describes. I don't know how to make sure that they do not
influence my judging of debate or forensics, or my assessment of any sort of
academic work or activity. The readings from this week left me even more unsure
of how to do this, honestly. I need to talk this out with you, probably, and
dive into the relevant literature. 
% Sure thing! We can start in class tomorrow and continue on Wednesday

\nocite{CharityHudley10}

\clearpage

\bibliography{cmst}
\bibliographystyle{apacite}

\end{document}
