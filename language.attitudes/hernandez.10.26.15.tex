\documentclass[doc,12pt]{apa6}
\usepackage{lmodern}
\usepackage{amssymb,amsmath}
\usepackage{ifxetex,ifluatex}
\usepackage{fixltx2e} % provides \textsubscript
\usepackage[T1]{fontenc}
\usepackage[utf8]{inputenc}
  
\usepackage[usenames,dvipsnames]{color}

\usepackage{graphicx,grffile}
\graphicspath{ {.resources/} }
\makeatletter
\def\maxwidth{\ifdim\Gin@nat@width>\linewidth\linewidth\else\Gin@nat@width\fi}
\def\maxheight{\ifdim\Gin@nat@height>\textheight\textheight\else\Gin@nat@height\fi}
\makeatother
% Scale images if necessary, so that they will not overflow the page
% margins by default, and it is still possible to overwrite the defaults
% using explicit options in \includegraphics[width, height, ...]{}
\setkeys{Gin}{width=\maxwidth,height=\maxheight,keepaspectratio}
\setlength{\parindent}{0pt}
\setlength{\parskip}{6pt plus 2pt minus 1pt}
\setlength{\emergencystretch}{3em}  % prevent overfull lines
\providecommand{\tightlist}{%
  \setlength{\itemsep}{0pt}\setlength{\parskip}{0pt}}
\setcounter{secnumdepth}{0}

\usepackage[colorlinks=false]{hyperref}
\usepackage{dirtytalk}
\usepackage{apacite}

\begin{document}

\title{Language Attitudes Midterm}
\shorttitle{Midterm}
\author{Edward Hern\'{a}ndez}
\date{2015-10-26}
\affiliation{College of William \& Mary}
\maketitle

\subsection{Programs}
\begin{quote}
	Find information about 2-3 actual programs similar to the one that you have
	proposed to work with or create for your final project. First summarize in
	what ways the programs you have found are similar to the one that you have
	proposed to work with or create. Then describe how you will (1) learn from
	and (2) improve on the design of the programs that you have found. 
	(6-8 pages)
\end{quote}

I intend to create a program for training judges for policy debate.
% Good
I want the program to be as painless as possible, as I do not want to
discourage the few new faces who volunteer by overloading them with
information.
% Very important
I also do not want to let them into a round completely unprepared. Policy can
be a little shocking if you aren't prepared. I am also inclined to believe that
judges without training are likely to disprefer non-traditional styles of
debate which are largely used by underrepresented students.
% Good -- now you have Shannon Dutchie and the student we met from JMU who can
% give you good, fresh input about your ideas

From what I can tell, \citeA{Bellon06} is the most widely circulated
instructional material in policy debate right now. Nearly everyone I have
talked to has been given a copy at some time or another. Even though a lot of
the people I have talked to have joined debate since after the publication of a
revised version \cite{Bellon08}, most people I have talked to are using the old
2006 version. I think that this has a lot to do with length. The 2006 version
is much shorter (56 pages, as opposed to 121), making it much easier to read.
It is also a lot easier to find. If you google ``Policy Debate Manual,'' the
2006 edition comes up first. I've also seen it linked to from many more
debate-related sites.

I think that the best thing about \citeA{Bellon06}, though, is its style. It is
very easy to read. It is not intimidating, even though the subject is
potentially scary. It presents everything systematically and clearly, with
definitions and explanations. It has helpful diagrams and illustrations. I
think that it is great example of an approachable, readable text. I need to
learn from this; I have the feeling that most of my work is not approachable in
this way, especially when I try to engage with debate as a topic.
% Test it out on people!

That said, the Policy Debate Manual is not what I am trying to produce. Even 56
pages is too many. I want a quick program which judges can be run through on
the morning of a tournament to give them an idea of what they're doing. This
means I need a different medium. Printed words won't cut it. It needs to be a
planned presentation or a video. There might be handouts involved, but
certainly not 56 pages of them.
% What might be you target length

I also need to target a different audience. The Manual is definitely for
student debaters. It briefly presents the debate theory pertinent to each
topic, but its focus is winning. Like \citeA{Bruschke95} and
\citeA{Snowball94}, the other book-length instructional texts in wide
circulation, it is intended to teach the reader how to win debates. It does not
attempt to explain the topics in a way that helps you judge them. Knowing how
to argue does not necessarily mean that you know how to assess arguments. To
teach judges, I need a completely different approach.

One thing I do like about this approach, however, is that it does not emphasize
rules. It instead emphasizes strategies. Instead of saying that anything is
against the rules, the Manual says that it can be argued that it is against a
rule or a norm. 
% good--tie to themes from Lippi-Green
I think that this is a useful way of thinking about debate, and about teaching
it. Few leagues have codified rules (Christian Communicators of America being
the only exception I know about), and judges vary wildly in what they will
accept. To try to focus on rules is useless, and I think largely serves to
restrict debaters creativity. To teach judges to think similarly, in terms of
norms rather than rules, will allow for more speech styles and creative debate
performances.


The Chicago Debate League has an ``Online Training'' program for students and
coaches. The program is expansive, intended to take about 12 hours to complete.
It is composed of modules on particular topics which are aligned with a system
of developmental benchmarks for debaters. The benchmarks are currently under
revision, and the links on the website are dead, so I will talk only about the
online training resources for now.

Each module consists of a Prezi or video presentation of a topic and a Google
Forms assessment quiz accompanying it. If I understand correctly, debaters are
intended to read through or watch each presentation, then take the quiz. I'm
not sure who gets the quiz results, or what is done with them, but the
presentations seem informative. I find their Prezis to be highly readable,
inviting, and easy to absorb. They also have helpful voice-overs. I think that
if I had known about these as a debater, I would have learned a lot faster.
Having this sort of resource available for judges who wanted to learn more
would be amazing.
% What might be the challenges with Prezi—do you have internet access at all
% debate sites?

Centrally hosting resources on a website is a strategy that I have been toying
with for a while. I was enamored with wikis for a while, but I have begun to
doubt that they are good for everything. While they are almost certainly the
right tool for fostering student collaboration on informative resources,
presenting a single easy-to-use website to judges would probably be more
useful.
% Good to note
I think the Chicago Debate League website is well-designed for this purpose
\cite{ChicagoURL}.

The linear structure of the Chicago training modules is also worth considering.
I think it might be good to have extra lessons or modules for judges who were
interested in getting more involved.
% Great! You might also have video examples for the more involved as you
% describe below
I know that last year I had two judges who were interested in learning more
about policy debate and asked me questions about its mechanics and theory
nearly every tournament. For motivated judges like that, online resources would
be useful.

Like \citeA{Bellon06}, the Chicago modules are mostly about strategy. They are
more about how to win than the theory behind the activity. This is because they
are still targeted at student debaters, rather than judges. To improve, I will
have to focus specifically on judges' needs.


For my third program, I've found a judge training resource. There is a training
video hosted on YouTube by the Cristian Communicators of America on how to
judge policy debate \cite{CCAvideo}. The final seconds of the video thank
viewers for their effort in ``this tournament,'' suggesting that the video is
(or is at least designed to be) presented at tournaments, to prospective
judges. This sort of presentation is exactly what I had originally planned to
produce, so this is definitely a project to consider.

The CCA video is a good resource, and it does some of what I want to do, but it
is designed specifically for those intending to judge CCA tournaments. It is
not perfect for prospective judges in other leagues. It starts with a quick
breakdown on the structure of a debate, with information like how many people
are on a time, how long speeches are, how cross examination works, etc. The
video also explains what is expected of the judge, including how to take notes
on arguments, how to award wins, and how to fill out the paperwork. This sort
of information is valuable, and is exactly what prospective judges at TDL and
PDL tournaments are given about other debate events (Lincoln-Douglas and Public
Forum). I think that a good training program should include this information,
but I don't think it is enough.

Since this video is produced and distributed by the CCA, it also discusses
things particular to their tournaments. It discusses particular pieces of
paperwork which do not have analogs in PDL or TDL. Apparently, CCA judges are
given "flow sheets" on which to take notes on (or "flow") the arguments given
during the round. It is not made clear whether these sheets are to be turned
in, or whether they are simply to help the judge visualize the round. Either
way, I think that this form may be useful to new judges, and I think I may try
to find or create a similar form for my program.

While locally we have no equivalent to their flow form, we have very similar
ballots. Ballots are forms turned into the tournament officials, which detail
who won a round and what scores each participant received. I had always assumed
that these were pretty universal. They just need lines on which to print
winners and scores. However, CCA has rules which other leagues do not have
(possibly partially because of being a Christian league). Their ballots, in
addition to the fields I'm used to seeing, have a line for ``Ethics,'' and the
possibility of awarding a ``double loss,`` a condition in which neither team
wins the round. Double losses are to be given only when the team who otherwise
would have won commits an ethics violation. There is no explanation in the
video of what constitutes an ethics violation, an oversight I find confusing.

I would need to create a resource that is not specific to the rules of anther
league. However, I don't know what rules I would make the training specific to.
PDL and TDL do not have codified rules about how to judge.
% So set the guidelines, but make them inclusive and flexible!
There is no official word on what constitutes a win. The CCA video implies that
there are rules in their league, and, so far as I can tell, those rules are
relatively extensive \cite{CCA15}. This is the first time I have seen any rules
on how to judge, and the rules don't look like what I understand policy debate
to be. They require that speeches be delivered at less than 185 words per
minute, and they do not allow for kritiks or non-traditional cases. Only three
sorts of cases are allowed \cite[p.~5]{CCA15}. This is not the debate I know at
all. These rules prevent everything I love about policy.
% Which is?

This presents a problem that I have not really thought about before. I want to
make judging more approachable, but I do not want to limit debaters in order to
do so. I want debaters to be able to run the cases they want to run, to utilize
kritiks, and to argue for the rules and norms that are best for them and best
for debate.
% So lay that out in a document as the model rather than a limitation or
% deviance
I think that the lack of codified rules keeps policy debate fresh and allows
for creative activity that is not possible in other events. The CCA program
prevents all of this, for reasons I do not understand. While I plan to use
their instructional video as a starting point for a short lesson on judging, I
do not want to incorporate any of their ideas about what policy debate is or
how it functions. This makes my job much more difficult. How do I tell people
what debate is and how it works, if those things are, themselves, up for
debate? How do I train judges without that training limiting what debate can
be? I need to work through these questions, but my plan for bettering the CCA
model is to strike some sort of balance, rather than just codifying the rules
and then explaining them as they are written.

\clearpage
\subsection{Faculty}
\begin{quote}
	Research 2-3 tenured or tenure track William and Mary faculty, as well as 2
	community members who might be able to help you with the research and
	planning you will need to advance your project ideas.
	(2-4 pages)
\end{quote}

I will be working with Professor Charity Hudley. I already know her, and she is
already familiar with previous versions of this project. Given her familiarity
with the project, with me, and with the pertinent education and linguistics
literature, working with her is an obvious choice. I will be meeting with her
weekly throughout the semester, which should help tremendously in sharpening my
ideas and planning for the rest of the semester. 
% Indeed! Now teach it all to me!

Lindy Johnson, at the school of education, publishes on teacher education
\cite{Johnson13}, critical discourse analysis \cite{Johnson14b}, and secondary
English education, and adolescent literacy \cite{Johnson12}, all of which are
relevant to this project, and debate more generally. Working with her would be
especially useful for teacher education, the part of this project with which I
am least familiar. It would be invaluable to have someone to talk to who has a
grasp of the relevant research. I'm also quite interested in a book chapter she
co-authored last year \cite{Johnson14a}, titled ``Creating Critical Spaces for
Youth Activists.'' That topic seems incredibly pertinent to this project (and
my interests), and it sounds like a great thing to go talk to her about.
% She'd love to!

My community partners will be debaters, coaches, and officials in the Tidewater
and Peninsula Debate Leagues. Unfortunately, A lot of the people I knew are
gone. Almost every debater I knew from my days as a competitor has graduated.
The presidents of Tidewater Debate League (TDL) and Peninsula Debate League
(PDL) have left their positions and I do not know who is replacing them.
Warwick High School's debate coach is new, and I do not yet know how receptive
he will be to this sort of work. I don't know a lot about how my community
partnership situation will look, but I hope to partner with the Warwick coach
and at least one of the PDL or TDL presidents, in order to have access to both
students and new judges. I cannot hope to do this project without both.
% Turnover is a big issue throughout education

I expect that the Warwick debate coach will be Eddie Krohn, a Warwick graduate.
He is not especially open-minded (about debate or much of anything else) and I
am concerned that he may not be open to working on a project like this. If he
is not, I am somewhat close with the new Hampton High School coach (or at
least, I think he will be the coach), who I expect would be slightly more open
(although he still dislikes kritik). If neither of them is up for this work, I
can reach out to an assistant coach in Virginia Beach, who is a recent graduate
of JMU and was hugely influential in moving this area toward the kritik. If he
is not willing, there is a new coach in Williamsburg, at Lafayette, who would
be willing to help me, I think, but is not especially knowledgeable. Regardless
of who it is, I expect I will be able to work with at least one coach and their
team.
% Also try Warhill as The William and Mary School of education is working on a
% partnership to work more closely with the school and Dr. Lindy Johnson is
% part of that effort.

Working with a league president is trickier. I knew the previous presidents
relatively well, and I think that both of them would have been amenable to
working with me. However, I have absolutely no idea who is replacing them. I
will have to wait and see who they are and how they will react to this sort of
project.

\clearpage
\subsection{Schedule}
\begin{quote}
	Draft a timeline for your project for the rest of the semester. The
	timeline is meant to help you with your final paper proposal, so you can
	use whatever timeline format best helps you plan (i.e. short narratives
	with dates, plans on a Google or paper Calendar/Planner) but you must
	provide dates for research and action as well as objectives.
	(1-2 pages)
\end{quote}

I plan to start judging tournaments again starting next week.
% Great!
This will let me get back into contact with my debate acquaintances. This
should allow me to start forming new community partnerships as well. I will
meet the new PDL and TDL presidents, and I will begin to learn whether they are
amenable to the sorts of work I am trying to do. What sorts of community
partnerships I am able to form and how receptive the new presidents seem will
determine how my schedule for the rest of the semester looks. I can do very
little without partners and a receptive league.

I still plan on working with Isaiah, but our schedules have been hard to
reconcile.
% Good to note
I plan on meeting with him more and finding out what I can do to help him out.
I don't know exactly what he's planning to do right now, so it's difficult to
plan that as well. 
% Catch up with him in class this week

Aside from this debate project, my major goal for the semester is to plan my
honors thesis. I still do not have a definite topic, but I am trying to work in
political analysis in some meaningful way. To that end, I'm meeting with Prof.
Settle this Monday, 26 October, to discuss possible topics. I will continue to
discuss topics with Prof. Charity Hudley during office hours as well, to ensure
that my ideas get solidified quickly.
% Report on the readings she's given you.
I aim to have a topic picked by Thanksgiving Break, so that I can start the
researching in earnest during my time off. I aim to have a literature review
drafted by semester break, so that I can start designing any studies or
experiments over the break.

I plan to meet with Prof. Charity Hudley this week to talk about papers for
this class, as well. I have some questions about what exactly I should be
writing. Should I continue with this debate topic? Should I start writing
exclusively about my honors thesis topic? Should I split papers with Isaiah?
I'm somewhat lost about what to do right now.

% I think you can keep going with the debate project a bit—it’s good to have
% several well developed ideas heading into graduate school. But now is also
% the time to start working in earnest on your honors ideas so split the
% difference—writing about different possible honors ideas will help you pick a
% theme or mix ideas together—have you ever considered doing an honors thesis
% on political debates? Who sets those rules and guidelines and what are the
% politics behind them—who coaches the candidates and what are the discourse
% patterns and models involved? How do they tie to the issues you’ve seen in
% high school debate and how are they different?

\clearpage

\bibliography{cmst}
\bibliographystyle{apacite}

\end{document}
