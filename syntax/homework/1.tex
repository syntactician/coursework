\documentclass[doc,12pt,natbib]{apa6}
\usepackage[colorlinks=false]{hyperref}
\usepackage{amsmath}
\usepackage{amssymb}
\usepackage{times}
\usepackage{tipa}
\usepackage{threeparttable}

\usepackage{gb4e}

\linespread{1.5}

\begin{document}

\title{Homework 1}
\shorttitle{Homework 1}
\author{Edward Hern\'{a}ndez}
\date{5 September 2016}
\affiliation{College of William \& Mary}
\maketitle

\setcounter{secnumdepth}{3}

\section{Nootka}

\begin{exe}

	\ex 
	\gll \textipa{mamu:ck-ma} \textipa{qu:Pas-Pi.} \\
	working-\textsc{pres} man-\textsc{def} \\
	\trans `The man is working.'

	\ex 
	\gll \textipa{qu:Pas-ma} \textipa{mamu:k-Pi.} \\
	man-\textsc{pres} working-\textsc{def} \\
	\trans `the working one is a man.'

\end{exe}


\begin{enumerate}

	\item In (1), \textit{\textipa{qu:Pas}} functions as a noun.
	\item In (1), \textit{M\textipa{amu:k}} functions as a verb.
	\item In (2), \textit{Q\textipa{u:Pas}} functions as a verb.
	\item In (2), \textit{\textipa{mamu:k}} functions as a noun.
	\item I assumed that a word acting as a predicate was a verb, and that an
		argument to that predicate was a noun.
	\item If a single root might variously act as a noun or a verb, as in in
		the above examples, I despair at the thought of identifying syntactic
		category by means of purely semantic criteria.

\end{enumerate}

\section{Nominal Prenominal Modifiers}

\begin{exe}
	\ex the \underline{leather} couch
	\ex the \underline{water} spout
\end{exe}

In (3) and (4), leather and water appear to act as adjectives. Both leather and
water modify the proceeding noun. Leather is the type or the material of couch.
Water is the type or substance of spout. We might expect that such behavior
makes the words definitionally adjectives in this case.

However, these words do not always function as adjectives.
\begin{exe}
	\ex{the leather}
	\ex{the water}
	\ex[?]{the very leather couch (cf. the very red couch)}
	\ex[?]{the very water spout (cf. the very big spout)}
	\ex[*]{the more leather couch / *The letherer couch (cf. the bigger couch)}
	\ex[*]{the more water spout}
	\ex[*]{the waterest spout}
\end{exe}
In (4) and (5), both words clearly act as nouns. Additionally, when we attempt
to use them in other ways common to adjectives, the results are unacceptable.
They may not be modified with adverbs or combined with derivational affixes
common to adjectives.  It may be best to conceive of these words not as
adjectives but as noun adjuncts --- nouns directly modifying other nouns.

\section{Subcategories of Adverbs}

To facilitate categorizing English adverbs, Table \ref{table:AdvPos} marks in
which sentence positions particular adverbs are unacceptable (*) or doubtful
(?). This analysis will consider four possible positions: before the subject
(front), between a modal or auxilary and a main verb (mid.), after the object
or at the end of a sentence (end), and between an object and a prepositional
phrase in a ditransitive sentence (ditrans.).\footnote{The names ``front,''
	``middle,'' and ``end'' have been adapted from nearly identical concepts in
	English grammar texts \citep{AdverbPlacement}.}

\begin{table}
\begin{threeparttable}
\begin{tabular}{ l c c c c }
	\toprule
	adverb & front & mid. & end & ditrans. \\
	\midrule
	\textit{luckily} & & & ? & ? \\
	\textit{earnestly} & * & & &  \\
	\textit{intently} & * & & & \\
	\textit{hopefully} & & & & \\
	\textit{probably} & * & & & \\
	\textit{certainly} & & & ? & * \\
	\textit{frequently} & & & & \\
	\textit{patiently} & & & & \\
	\textit{always} & * & & ? & ? \\
	\textit{completely} & * & & & \\
	\textit{almost} & * & & ? & ? \\
	\textit{again} & ? & * & & \\
	\textit{evidently} & & ? & & * \\
	\textit{frankly} & & & & \\
	\textit{demandingly} & * & ? & & \\
	\textit{yesterday} & & * & & * \\
	\textit{necessarily} & ? & & ? & * \\
	\bottomrule
\end{tabular}
\begin{tablenotes}
	\small
	\item * = unacceptable.
	\item ? = doubtful.
\end{tablenotes}
\end{threeparttable}
\caption{Acceptability by Position}
\label{table:AdvPos}
\end{table}

Based on the positions in which they are acceptable, there appear to be at
least 3 distinct types of adverbs represented in Table \ref{table:AdvPos}.
Adverbs of certainty (e.g.  certainly, probably) are acceptable in middle
position, or in end position after a comma.  Adverbs of degree (e.g. almost)
appear to be most acceptable in middle position.  Adverbs of evaluation (e.g.
luckily, evidently) are acceptable in front, middle, and end positions.
Adverbs of manner (e.g. earnestly, intently) are acceptable in the middle, end,
ditransitive positions.  Adverbs of time (e.g. yesterady, always) are
acceptable in front and end positions.  Table \ref{table:AdvClass} gives a full
classification of the adverbs from Table \ref{table:AdvPos}.

\begin{table}
\begin{threeparttable}
\begin{tabular}{ c c c c c }
	\toprule
	certainty & degree & evaluation & manner & time \\
	\midrule
	\textit{certainly} & \textit{almost} & \textit{evidently} & \textit{completely} & \textit{again} \\
	\textit{necessarily} & \textit{completely} & \textit{hopefully} & \textit{demandingly} & \textit{always} \\
	\textit{probably} & & \textit{luckily} & \textit{earnestly} & \textit{frequently} \\
	& & & \textit{frankly} & \textit{yesterday} \\
	& & & \textit{intently} & \\
	& & & \textit{patiently} & \\
	\bottomrule
\end{tabular}
\end{threeparttable}
\caption{Adverbs by Type}
\label{table:AdvClass}
\end{table}

\clearpage
\bibliography{../extra.bib}

\end{document}
