\documentclass[doc,12pt,natbib]{apa6}
\usepackage[colorlinks=false]{hyperref}
\usepackage{amsmath}
\usepackage{amssymb}
\usepackage{times}
\usepackage{tipa}
\usepackage{threeparttable}

\usepackage{gb4e}

\linespread{1.5}

\begin{document}

\title{Homework 2}
\shorttitle{Homework 2}
\author{Edward Hern\'{a}ndez}
\date{14 September 2016}
\affiliation{College of William \& Mary}
\maketitle
	\small

\begin{exe}

	\ex
	\begin {xlist}

		\ex {[}The {[}very {[}orange {[}pumpkin{]]]]} {[}squashed {[}the
		{[[}tiny {[}ant{]]} with {[}the {[}broken {[}leg{]]]]]]} \\

		\ex {[}The {[}handsome {[}young hooligan{]]]} will {[}smash {[}the
		{[}car{]]} {[}with {[}a {[}big {[}hammer{]]]]} {[}tommorow{]]}

	\end{xlist}
\end{exe}

\begin{exe}
	\ex
	\begin{xlist}

		\ex Terry saw the elephant with a telescope.
		\begin{xlist}
			\ex Terry saw [the elephant with a telescope]
			\ex Terry saw [the elephant] [with a telescope]
			\ex{ [With a telescope] Terry [saw the elephant] }
		\end{xlist}

		\ex Robin broke the bottle on the table.
		\begin{xlist}
			\ex Robin broke [the bottle on the table]
			\ex Robin broke [the bottle] [on the table]
			\ex{ [On the table] Robin [broke the bottle] }
		\end{xlist}

		\ex Leslie stole the letter from the boss.
		\begin{xlist}
			\ex Leslie stole [the letter from the boss]
			\ex Leslie stole [the letter] [from the boss]
			\ex{ [From the boss] Leslie [stole the letter] }
		\end{xlist}

		\ex Lee sold the presents for the boss.
		\begin{xlist}
			\ex Lee sold [the presents for the boss]
			\ex Lee sold [the presents] [for the boss]
			\ex{ [For the boss] Lee [sold the presents] }
		\end{xlist}

		\ex Sandy looked over the table.
		\begin{xlist}
			\ex Sandy [looked over] the table
			\ex Sandy looked [over the table]
			\ex{ [Over the table] Sandy looked }
		\end{xlist}

	\end{xlist}
\end{exe}

Sentences (2a-d) are ambiguous in that they each have a  prepositional phrases
which may be interpreted either as forming or not forming a constituent with
the preceding noun phrase (as shown in sub-examples i.-ii.). In the case that the
prepositional phrase does not form a constituent with the noun phrase, the
ambiguity may be removed by moving the prepositional phrase to the beginning of
the sentence (as in sub-examples iii.).

In (2e), it is ambiguous whether the verb is ``looked'' or ``looked over'',
making it unclear whether ``the table'' is the object of a preposition or of a
verb. The differences in the constituency structures of these two analyses are
sketched in i.-ii. In the case that ``over the table'' is a constituent (a
prepositional phrase), it may be moved to the front of the sentence (as in
iii.) to disambiguate.

\begin{exe}
	\ex
	\begin{xlist}
		\ex I don't like my coffee black
		\begin{xlist}
			\ex I don't like my black coffee
			\ex I don't like my [coffee] [black]
		\end{xlist}

		\ex I don't like the people present
		\begin{xlist}
			\ex[*]{I don't like the present people}
			\ex I don't like the [people present]
		\end{xlist}
	\end{xlist}
\end{exe}

In (3a), ``black'' can comfortably be moved forward. In (3b), ``present'' cannot. I would argue that this is evidence of a differing constituency structure (shown in sub-examples ii.).

\end{document}
