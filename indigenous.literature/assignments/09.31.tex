\documentclass[12pt,letterpaper]{article}

\usepackage[colorlinks=false]{hyperref}
\usepackage{ifpdf}
\usepackage{mla}

\usepackage[american]{babel}
\usepackage{csquotes}
\usepackage[style=mla]{biblatex}
\addbibresource{../course.bib}

\begin{document}
\begin{mla}{Edward}{Hernandez}{Thompson}{Indigenous Literature}{August 31, 2016}{Creative Nonfiction}

\begin{quote}

	Write an essay that locates your relationship to Native America.  By
	\emph{relationship} and \emph{Native America} I mean your engagement and
	experience with, or knowledge of, American Indian history, politics,
	culture, land, people, and so on. You may choose a very specific
	experience, or you may think more broadly.  Perhaps you will write about
	where you are from and how this space constructed your sense of Native
	America (this would be a broad approach), or perhaps you once visited a
	tourist attraction that capitalizes on Native America and you want to write
	about that experience (a more specific approach). Have you been to, or
	lived on, a reservation? Do you identify as American Indian, Native, or
	Indigenous? Do you know people who identify as American Indian, Native, or
	Indigenous? You may also choose to write about a cultural production, such
	as a film or novel, that was formative in your understanding of Native
	America.

	You may even think at first that you do not have a ``relationship to Native
	America,'' but the main objective of this first essay is to realize
	that---even indirectly---we all do by the virtue of our presence on this
	land. Therefore, you may even write about what you do not know, what you
	have not experienced directly and how this impacts you. To get started, you
	may consider the following questions:

	\begin{itemize}
		\item What can we learn about Native America through individual,
			personal experiences?
		\item What can't we learn?
		\item How do memory and identity shape our present understanding of
			history or a specific experience?
	\end{itemize}

	\textbf{Secondary Sources:}
	No outside research is required, but I certainly encourage you to bring in
	other sources if they help you or your readers. I have called this essay,
	``creative nonfiction'' because this should not read as a personal journal
	entry, but instead a critical engagement with a personal history or
	experience. Put differently, do not write about something you don't want
	exposed to comment or critique by your reader.

	\textbf{Objective:}
	This paper will give me a better sense of your background knowledge and
	experiences that will inform your overall approach to this course. Second,
	this is an opportunity for you to reflect on an important term for us this
	semester, \emph{accountability}. Questions you should consider (but do not
	need necessarily to answer directly) should include: What does
	accountability mean? For Native/American Indian-identified folks, how can I
	be accountable to my family and my community? For non-Native/American
	Indian/Indigenous folks, how can I be an ally? What are the limits of
	allyship?

\end{quote}

\clearpage
\printbibliography

\end{mla}
\end{document}
