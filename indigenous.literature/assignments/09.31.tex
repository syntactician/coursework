\documentclass[12pt,letterpaper]{article}

\usepackage[colorlinks=false]{hyperref}
\usepackage{ifpdf}
\usepackage{mla}

\usepackage[american]{babel}
\usepackage{csquotes}
\usepackage[style=mla,showmedium=false]{biblatex}
\addbibresource{../course.bib}
\addbibresource{../extra.bib}

\begin{document}
\begin{mla}{Edward}{Hernandez}{Thompson}{Indigenous Literature}{August 31, 2016}{Creative Nonfiction}

	% Write an essay that locates your relationship to Native America.  By
	% \emph{relationship} and \emph{Native America} I mean your engagement and
	% experience with, or knowledge of, American Indian history, politics,
	% culture, land, people, and so on. You may choose a very specific
	% experience, or you may think more broadly.  Perhaps you will write about
	% where you are from and how this space constructed your sense of Native
	% America (this would be a broad approach), or perhaps you once visited a
	% tourist attraction that capitalizes on Native America and you want to write
	% about that experience (a more specific approach). Have you been to, or
	% lived on, a reservation? Do you identify as American Indian, Native, or
	% Indigenous? Do you know people who identify as American Indian, Native, or
	% Indigenous? You may also choose to write about a cultural production, such
	% as a film or novel, that was formative in your understanding of Native
	% America.

	% You may even think at first that you do not have a ``relationship to Native
	% America,'' but the main objective of this first essay is to realize
	% that---even indirectly---we all do by the virtue of our presence on this
	% land. Therefore, you may even write about what you do not know, what you
	% have not experienced directly and how this impacts you. To get started, you
	% may consider the following questions:

	% \begin{itemize}
	% 	\item What can we learn about Native America through individual,
	% 		personal experiences?
	% 	\item What can't we learn?
	% 	\item How do memory and identity shape our present understanding of
	% 		history or a specific experience?
	% \end{itemize}

	% \textbf{Secondary Sources:}
	% No outside research is required, but I certainly encourage you to bring in
	% other sources if they help you or your readers. I have called this essay,
	% ``creative nonfiction'' because this should not read as a personal journal
	% entry, but instead a critical engagement with a personal history or
	% experience. Put differently, do not write about something you don't want
	% exposed to comment or critique by your reader.

	% \textbf{Objective:}
	% This paper will give me a better sense of your background knowledge and
	% experiences that will inform your overall approach to this course. Second,
	% this is an opportunity for you to reflect on an important term for us this
	% semester, \emph{accountability}. Questions you should consider (but do not
	% need necessarily to answer directly) should include: What does
	% accountability mean? For Native/American Indian-identified folks, how can I
	% be accountable to my family and my community? For non-Native/American
	% Indian/Indigenous folks, how can I be an ally? What are the limits of
	% allyship?

My relationship with Native America is almost entirely secondhand.  I grew up
on the Virginia Peninsula, an area rife with the history of the early
colonization of North America. Our public libraries are filled with picture
books of Native American people, many of which I checked out and read through.
School field trips were often to historic settlement sites, including Yorktown
and Jamestown. My earliest memory of learning about Native history outside of a
picture book was being led by a man with a loincloth and suspiciously light
hair telling us about the Powhatan while leading us on a tour of (what I
understood to be) an Iriquois-style longhouse.

I went to a Christian school for much of my life, where I was taught curricula
which glorified European settlers, not only for ``civilizing'' the New World,
but also for Christianizing it. I was taught explicitly that the white
colonizers had done the Native peoples a service by taking away their gods. I
was concerned about this, but I did not know how to object. After several years
of soaking it in, I objected that even if their souls were saved, much of their
cultures were tragically lost. My teacher was not pleased.

At church, my family pressured me into participating in a program called Royal
Rangers, ``an activity-based, small group church ministry for boys and young
men in grades K-12 [intended] to evangelize, equip and empower the next
generation of Christlike men and lifelong servant leaders''
\autocite{RoyalRangers}. The program was in theory an intensive vacation bible
school and in practice Christian-brand Boy Scouts. When I joined, everything
was cowboys-and-indians themed. Boys would play cowboys and indians and fight.
Cowboys were the preferable side, as they usually won. The youngest boys were
called ``Straight Arrows,'' and participated in Indian-themed events. When you
reached second grade, you graduated to a Buckaroo, which we understood to be
equivalent to white cowboys. This was very exciting. It meant you were on the
winning team by default. I was uncomfortable with nearly everything about this
experience. So were many parents. The Straight Arrows and Buckaroos are now the
``Ranger Kids'' and ``Discovery Rangers.''

It was not until high school that I had a friend who I was aware had Native
ancestry. He was from a tribe in Mexico, and I knew it meant a lot to him, but
he never talked much about it. No one at could pronounce his name, and I know
it made him sad. Even though he was our valedictorian, he could not go to
college. He did not have the money, and he did not have citizenship to seek
financial aid. I was (and remain) disgusted that he faced such citizenship
complications with an occupying government.

Now that I have read more, and I have an idea of what I would like to say to
that guide, to my teachers, to my scout leaders, and to my friend. But I also
increasingly realize that I do not---and cannot---understand. Additionally,
most of my reading has been historical or political, about the ways in which
and effects to which Native Americans have been displaced and oppressed. I feel
remiss in having read so much \emph{about} Native Americans and having read so
little by them. I realize that I have been guilty of thinking of learning about
Native Americans as learning \emph{history}. I see this class as an opportunity
to explore Native American literature as ongoing, and to begin to orient myself
to be an ally to Native American people in the present day.

\clearpage
\printbibliography

\end{mla}
\end{document}
