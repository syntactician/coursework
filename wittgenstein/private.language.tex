\documentclass[doc,12pt,apacite,biblatex]{apa6}
\usepackage{amssymb,amsmath,latexsym,apacite,dirtytalk}

% Page length commands go here in the preamble
%\setlength{\oddsidemargin}{-0.25in} % Left margin of 1 in + 0 in = 1 in
%\setlength{\textwidth}{7in}   % Right margin of 8.5 in - 1 in - 6.5 in = 1 in
%\setlength{\topmargin}{-.75in}  % Top margin of 2 in -0.75 in = 1 in
%\setlength{\textheight}{9.2in}  % Lower margin of 11 in - 9 in - 1 in = 1 in
\renewcommand{\baselinestretch}{1.5} % 1.5 denotes double spacing. Changing itwill change the spacing
\setlength{\parindent}{0in}

\usepackage{dirtytalk}

\begin{document}
\title{Wittgenstein on Private Language}
\author{Edward Hern\'{a}ndez}
\date{\today}
\affiliation{College of William \& Mary}
\shorttitle{Wittgenstein on Private Language}

\maketitle

\vspace{-10pt}
\begin{quote}
	Consider \S 270 and \S 271.  Imagine that you are the person who discovers
	that whenever you have a particular sensation (which you note in your diary
	by writing ``S'') a medical apparatus shows that your blood-pressure rises.
	So you can say when your blood-pressure is rising without having to use
	that or any other apparatus. Wittgenstein asks ``And what reason do we have
	for calling `S' the name of [your] sensation?'' \\
	Of course ``S'' is the name of your sensation, right?  It's that (privately
	felt) sensation which you make ``S'' stand for in your diary---and whose
	occurrence is objectively indicated by the medical apparatus---isn't it?
	If your answer to these questions is `no', then aren't you left with the
	conclusion that a sensation word like ``S'' doesn't really stand for/mean
	your felt sensation?  That ``the pain in my stomach I'm feeling right now''
	doesn't really stand for the pain that I'm feeling in my stomach right now?
	What does it stand for then?  My groaning?  My bending over and holding my
	stomach?  What if I feel it but don't do those things?  Is its meaning
	different then? \\
	As McGinn points out, Wittgenstein's ``Private Language Argument'' has been
	characterized as advocating a form of behaviorism.  Indeed, in \S 281 an
	interlocutor asks ``But doesn't what you say come to this: that there is no
	pain, for example, without pain-behavior?'' \\ 
	Is this a fair characterization of what Wittgenstein is doing (or saying)?
\end{quote}
\clearpage

Let us suppose that I consciously attend to all of my sensations, marking them
in my diary. Let us also suppose that I keep track of my blood pressure
constantly. After some time, I might reasonably find that every time I have a
particular sensation (for which I might mark ``S'' in my diary) my blood
pressure rises.  After some amount of time, I will feel comfortable saying that
my blood pressure rises when I feel \textit{S}, even without checking my blood
pressure in any way.

It seems (to me) obvious and natural that I am here using ``S'' as the name for
my sensation---the sensation which accompanies my increased blood pressure.
\citeA{Wittgenstein53}, however, asks what my justification here is for calling
``S'' the name of a sensation (\S 270). As he often does, he provides a
possible answer immediately following his question: ``Perhaps the kind of way
this sign is employed in this language-game'' \citeA[\S 270]{Wittgenstein53}.
This is a sort of answer which seems unintuitive at first glance --- is having
sensations a language-game? \emph{Naming} sensations, at the very least, seems
like it must be. Therefore, it would seem I would have to answer Wittgenstein:
I am justified in calling ``S'' the name of my sensation because it is employed
in a particular kind of way in the language-game of naming a sensation.

This answer at first seems thoroughly unsatisfying, especially with ``Private
Language'' in mind. Naming a sensation---insofar as it is a distinct
language-game---requires quite a lot of foreknowledge. I might imagine that
naming a sensation is simply a matter of making up a sign, ``S,'' which I
record at future occurrences of the sensation. Wittgenstein complicates this in
several ways. First, he asserts that it is not sufficient to merely focus on
the sensation and record a sign. If that is all I do, all I have accomplished
is committing something about the previous instances to memory; I do not
establish criteria for the sign's application in the future. This means that I
cannot apply the sign meaningfully in the future.  In addition, to do all of
this, I need to have an understanding of what it is to name. Otherwise, the act
of concentrating my attention on a sensation and recording some sign in my
diary while experiencing that sensation, the ``ceremony'' I have performed
amounts to nothing. As he says, ``much must be prepared in the language for the
mere naming to make sense'' \cite[\S 257]{Wittgenstein53}.

With all this in mind, it seems less natural that ``S'' is, in fact, the name
of my sensation. However, it still seems tenable. Suppose that what I need
\emph{is} prepared in the language: I have foreknowledge of some language-game
wherein I name sensations by recording signs in a diary. As \citeA{McGinn97}
puts it, I have used ``S'' in ``the distinctive grammar of sensation words.''
Then the concern is setting up criteria for the application of that sign in
future instances.  Here, I might be tempted to say that my blood pressure will
serve as a criterion. If, at some future time, my blood pressure rises
\emph{and} I report ``S,'' that is a valid report---I have named my sensation
successfully.

This account leads me somewhere, philosophically, that I am not sure I am
comfortable. I am now tempted to say that ``S'' does not name my sensation
(some internal object) but rather the incidence of my blood pressure rising. At
the very least, I feel compelled to say that my use of ``S'' can only be
meaningful if an independent means of verification (like my blood pressure) is
available to prove my usage correct. If I follow this line of thinking I am
similarly tempted to say that ``pain'' refers not to some internal sensation
but rather to pain-behavior---groaning, crying, etc.---or, at least, cannot
occur in its absence, which is thoroughly at odds with my commonsense view of
pain (and all sensations). Obviously, this reads like fairly classic
behaviorism (a criticism to which Wittgenstein responds in \S 307).
It seems that pain is identified via behavior and that ostensive definition of
the word does, in fact, rely on the exhibition of pain-behavior. Likewise, it
seems intuitive that a genius child who names pain herself will not be able to
communicate its meaning without pain-behavior. This, as \citeA{McGinn97} points
out, shows us that the act of private definition ``has no connection with our
ordinary criteria for mastery of the concept of pain.'' This leads us toward a
behaviorist account of language and of word-meaning, but, so far as I can tell,
cannot account for signs in ``private language'' to mean anything whatsoever. 

Seemingly to expound on this point, Wittgenstein invites us to imagine another
scenario, which \cite{McGinn97} claims is to show us that ``an allegedly
private act of identifying what is introspected as `the same again' ... plays
no role'' in the language-game of naming the sensation or identifying my rise
in blood-pressure. I do not fully understand what it is that Wittgenstein
intends me to imagine, nor what I am supposed to gather from having imagined
it. McGinn is of some help here, but not nearly enough.  Wittgenstein invites
us to imagine that ``I regularly identify [``S''] wrong'' \cite[\S
270]{Wittgenstein53}, and asserts that this mistake ``does not matter in the
least'' (\S 270). McGinn seems to take this situation to be one in which the
``private ceremony of identification'' of ``S'' no longer accords with my
``public practice'' of detecting a rise in blood pressure. I suppose, then,
that this would mean that I begin to detect my blood pressure rising without
introspecting and marking ``S'' in my diary. She asserts that this goes to show
that the meaning of ``S'' is derived from its use in some particular
language-game (though I do not know which one she means), rather than from its
connection with a rise in blood pressure. I confess that I do not understand
the implications here.

I fear that it may be beyond my ability to determine whether Wittgenstein
\emph{intends} to endorse a behaviorist view in these sections of the
\emph{Investigations}. Additionally, I have become hesitant to say that
Wittgenstein ever advocates any one view in particular.  Rather, as throughout
the \emph{Investigations}, he seems here to be reacting against (his)
temptation(s) to philosophize in particular ways. It may just be that the ways
in which he is inviting us to think about sensation \emph{look} behaviorist.
Those positions may not be his conclusions (if he has conclusions at all). This
is the view which \citeA{McGinn97} seems to take, and I am inclined to agree.

Even if Wittgenstein does not intend to advocate for a position or make clear
conclusions, there are still conclusions which it is reasonable to draw from
his comments. I think that Wittgenstein argues against our commonsense views on
private language in these sections of the \emph{Investigations}. It seems to me
that he stands in opposition to (or at least intends to show us alternatives
to) a view on which signs in private language can correspond easily to
sensations, without the use of language-games which we use to refer to
sensations generally. I also agree with McGinn that these arguments stand in
opposition to \citeA{James90} and his claim that there can be some internal
language of psychology, divorced cleanly from objective reference or outside
verification. Insofar as we need language-games to name sensations, James'
proposed language is inoperable. I do not, however, take this to be a rejection
on Wittgenstein's part, of mental events or an embrace of behaviorism. I think
in \S 307 he is stating that he means to illuminate grammatical fictions
\emph{as opposed to mental ones}. This means, as McGinn puts it, that ``The
distinction between psychological states and behaviour, which the picture of
`inner states' is meant to capture, is, at bottom, a grammatical one.'' This
does not necessarily imply that we are to bound to embrace behaviorism (though
I leave open the possibility that it points toward behaviorism as a likely
source for a resolution of the problems he raises here), but rather that we
ought examine the language-games involved with sensation lest we be taken in by
this grammatical distinction to believe that these sets---the psychological and
the behavioral---are cleanly separable.

% \begin{quote}
% 	\S 270. Let us now imagine a use for the entry of the sign ``S'' in my
% 	diary. I discover that whenever I have a particular sensation a mano-meter
% 	shews that my blood-pressure rises. So I shall be able to say that my
% 	blood-pressure is rising without using any apparatus. This is a useful
% 	result. And now it seems quite indifferent whether I have recognized the
% 	sensation \textit{right} or not. Let us suppose I regularly identify it
% 	wrong, it does not matter in the least. And that alone shews that the
% 	hypothesis that I make a mistake is mere show. (We as it were turned a knob
% 	which looked as if it could be used to turn on some part of a machine; but
% 	it was a mere ornament, not connected with the mechanism at all.)

% 	And what is our reason for calling ``S'' the name of a sensation here?
% 	Perhaps the kind of way this sign is employed in this language-game,---And
% 	why a ``particular sensation,'' that is, the same one every time? Well,
% 	aren't we supposing that we write ``S'' every time?
% \end{quote}

% \begin{quote}
% 	\S 271. ``Imagine a person whose memory could not retain \emph{what} the
% 	word `pain' meant---so that he constantly called different things by the
% 	name---but nevertheless used the word in a way fitting in with the usual
% 	symptoms and presuppositions of pain''---in short he uses it as we all do.
% 	Here I should like to say: a wheel that can be turned though nothing else
% 	moves with it, is not part of the mechanism.
% \end{quote}

\nocite{Wittgenstein53}

\clearpage

\bibliography{wittgenstein}{} \bibliographystyle{apacite} \printindex[autx]

\end{document}
