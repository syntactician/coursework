\documentclass[doc,12pt,apacite,biblatex]{apa6}
\usepackage{amssymb,amsmath,latexsym,apacite,dirtytalk}
\linespread{1.5}

% Page length commands go here in the preamble
%\setlength{\oddsidemargin}{-0.25in} % Left margin of 1 in + 0 in = 1 in
%\setlength{\textwidth}{7in}   % Right margin of 8.5 in - 1 in - 6.5 in = 1 in
%\setlength{\topmargin}{-.75in}  % Top margin of 2 in -0.75 in = 1 in
%\setlength{\textheight}{9.2in}  % Lower margin of 11 in - 9 in - 1 in = 1 in
\renewcommand{\baselinestretch}{1.5} % 1.5 denotes double spacing. Changing itwill change the spacing
\setlength{\parindent}{0in}

\begin{document}
\title{Wittgenstein on Experience}
\author{Edward Hern\'{a}ndez}
\date{\today}
\affiliation{College of William \& Mary}
\shorttitle{Wittgenstein on Experience}

\maketitle

\vspace{-20pt}
\begin{quote}
	Please review the Investigation's sections on understanding, `reading', and
	the experience of being guided (\S\S 150-176). In the ``Philosophy of
	Psychology'' fragment (a.k.a Part II), read the sections on aspect-shifting
	(\S\S 111-147) and on `the elasticity of the concept of representing what
	is seen' (secs. 160-161).  How is Wittgenstein's discussion of concepts of
	experience in these sections an example of ``the work of the philosopher''
	as he describes this in the following two sections from what is called the
	`methodological' part of the Investigations?

	\S 127. The work of the philosopher consists in marshalling recollections
	for a particular purpose.

	\S 129. The aspects of things that are most important for us are hidden
	because of their simplicity and familiarity. (One is unable to notice
	something  -- because it is always before one's eyes.) The real foundations
	of their inquiry do not strike people at all. Unless that fact has at some
	time struck them. -- And this means: we fail to be struck by what, once
	seen, is most striking and powerful.
\end{quote}
\clearpage


\nocite{Wittgenstein53}

\clearpage
\bibliography{wittgenstein}
\bibliographystyle{apacite}

\end{document}
