\documentclass[man,12pt,natbib]{apa6}
\usepackage{amssymb,amsmath,latexsym,times}
\linespread{1.5}

% Page length commands go here in the preamble
% \setlength{\oddsidemargin}{-0.25in} % Left margin of 1 in + 0 in = 1 in
% \setlength{\textwidth}{7in}   % Right margin of 8.5 in - 1 in - 6.5 in = 1 in
% \setlength{\topmargin}{-.75in}  % Top margin of 2 in -0.75 in = 1 in
% \setlength{\textheight}{9.2in}  % Lower margin of 11 in - 9 in - 1 in = 1 in
% \renewcommand{\baselinestretch}{1.5} % 1.5 denotes double spacing. Changing itwill change the spacing
\setlength{\parindent}{0in}

\begin{document}
\title{Wittgenstein on Experience}
\author{Edward Hern\'{a}ndez}
\date{\today}
\affiliation{College of William \& Mary}
\shorttitle{Wittgenstein on Experience}

\maketitle

% \vspace{-20pt}
\begin{quote}
	Please review the Investigation's sections on understanding, `reading', and
	the experience of being guided (\S\S 150-176). In the ``Philosophy of
	Psychology'' fragment (a.k.a Part II), read the sections on aspect-shifting
	(\S\S 111-147) and on `the elasticity of the concept of representing what
	is seen' (\S\S 160-161).  How is Wittgenstein's discussion of concepts of
	experience in these sections an example of ``the work of the philosopher''
	as he describes this in the following two sections from what is called the
	`methodological' part of the Investigations?

	\S 127. The work of the philosopher consists in marshalling recollections
	for a particular purpose.

	\S 129. The aspects of things that are most important for us are hidden
	because of their simplicity and familiarity. (One is unable to notice
	something  -- because it is always before one's eyes.) The real foundations
	of their inquiry do not strike people at all. Unless that fact has at some
	time struck them. -- And this means: we fail to be struck by what, once
	seen, is most striking and powerful.
\end{quote}
\clearpage

\citet{Wittgenstein53} holds that ``the work of the philosopher consists in
marshalling recollections for a particular purpose'' (\S 127). I understand
this to mean that he seeks to lead us, his readers, to critically evaluate our
own memories and sensations to understand them better.
% Yet what he most often discusses—reminds us of—is how we talk about ...
% sensations, experiences, memories, mental processes, etc.  So he is
% marshalling recollections of our uses of language (what we say) – in concert
% with doing things, with experiencing [e.g., being guided, seeing an
% aspect-shift].
%
% The philosophical problems he identifies (“what troubles us”) and attempts to
% treat are not problems with our experiences or sensations or memories, but
% with how, when we do philosophy/theorize, we feel inclined to talk about our
% experiences, sensations, memories.
%
% The work of the philosopher (for LW) is arrange ‘what we already know’ (about
% how to talk about ...) so that we can get “a clear view” of it – and so not
% be led down unproductive paths.
Because of his idea that philosophy ``simply puts everything before us'' (\S
127), this sort of recollections is central to his method.  As he says, ``it is
the business of philosophy, not to resolve a contradiction by means of a ...
discovery, but to make it possible for us to get a clear view of [what]
troubles us'' (\S 125).  To get this clear view, we have to recall and
reexamine our experiences. This method is most strikingly used in his
discussions of reading and being guided.

In Part II of the \emph{Investigations}, like Part I, a large part of
Wittgenstein's method consists in offering the reader language-games, in which
to imagine the workings of language.
% Yes: ... of language.
In \S 130, he refers to these language games as ``\emph{objects of
comparison},'' which he says are ``meant to throw light on the facts of our
language by way not only of similarities, but also of dissimilarities.''
Through them, he attempts to prompt the reader to examine the workings of
language in a particular, circumscribed setting or activity, to reveal ways in
which language functions in other settings.
% Why?
Unlike some of the examples from early in Part I, like the builders' game,
which could only take place in an imagined community, many of the examples in
Part II are drawn from real uses of language which the reader is likely to have
experienced.  Additionally, Part I often invited the reader to imagine that
language worked a particular way (as in the grocer's and builders' games), to
compare that system with the system of language with we operate. This section,
instead of inviting us to imagine the language system, invites us to imagine
experiences. This section's language games focus, I would argue, instead of on
the language itself, on the ``form[s] of life'' (\S 19) which they suggest.
% What do you mean by the forms of life that language-games suggest?

The central example in \S\S 156-176 is reading.
% How do you mean ‘central’?  In what way?  And, example of what?  Of a
% language game focusing on forms of life?  Not clear.
Wittgenstein calls us to imagine people reading (\S 156), imagine ourselves
reading (\S 162), and to actually perform reading (\S 169) in various settings
and circumstances.  He shows us a disconnect between the first-personal
experience we call reading (of being guided by the letters on the page) with
our conditions for saying of someone else that they are reading.
% DISCUSS: These sections again and again address our inclination to speak of
% reading as a “mental process”.  “But surely — we’d like to say — reading is a
% quite particular process!” (...)  So what does the characteristic thing about
% the experience of reading consist in?” (165)
%
% “But what in all this is essential to reading as such?  Not any one feature
% that occurs in all cases of reading.” (168).
%
% “But when we read, don’t we feel the look of the words somehow causing our
% utterance?” (...) But why do you say that we felt a causing?  (...) One might
% rather say that the letters are the reason why I read such-and-such.  For if
% someone asks me “why do you read this way?’ – I justify it by the letters
% which are there.” (169)
%
% “We imagine that a feeling enables us to perceive...a connecting mechanims
% between the look of the wod and the sound that we utter.....so to say, feel
% the movement of the levers...” (170)
%
% “I’d like to say ‘I experience the because’”...(177)
He is only able to do this because he is able to remind his reader of the
experiences which we have when we read. He is able to prompt us to recall
previous instances in which we had these experiences, and in this case perhaps
more than any other, he is able to induce those experiences in us. He is able
to do this because of our familiarity with reading, a thing so \emph{simple and
familiar} (\S 129) that it no longer strikes us. He marshals our recollections,
and makes it striking again.

In \S 139, Wittgenstein discusses being ``guided''. He calls into question our
inclination to say that a picture \emph{suggests} a particular use or
interpretation to us (as a letter or a word might).  One of his examples is a
picture representing ``an old man walking up a steep path leaning on a stick.''
I find it easy to imagine such a picture. However, as Wittgenstein points out,
a Martian might as easily describe the image as one of the man sliding down the
hill. This is no less valid an interpretation of the picture, but we do not at
all experience an inclination to choose this application of the picture.
Whether or not pictures do suggest or force particular applications of
themselves is not the topic here (as is what Wittgenstein has to say about
whether they do). What I mean to discuss is his method and his motivation in
choosing \emph{this} example.

There are other examples that Wittgenstein could choose here. Arguably, he has
already said enough earlier in \S 139, in his discussion of cubes, and does not
need another example.  However, the cube example is not easily accessible to
me, and I imagine it is that way for many other readers. I struggle to imagine
(experiencing) methods of projection of an image of a cube, let alone ``that
not merely the picture of the cube, but also the method of projection comes
before our mind,'' (\S 141).
% The connection between (talking about) grasping in a flash the meaning of a
% word AND the word’s understanding and meaningful use over time.  “What really
% comes before our mind when we understand a word?” (139).
%
% Cf. What really is in the box you keep referring to as your beetle?   A wheel
% that can be turned without engaging the mechanism.....
This, I would argue, is because I have many more recollections relevant to
pictures of men walking, even though I have no specific memory of looking at a
picture of an old man climbing a steep hill with a stick.\footnote{At least, I
didn't before I began writing this paper.} As a result, Wittgenstein is able to
``marshal'' those recollections to bring me to an understanding of his point.
Likewise, I find it easy to imagine choosing between the words ``imposing,''
``dignified,'' ``proud,'' and ``venerable,'' (\S 139), because I decide between
similar words frequently. It does not matter whether I can recall choosing
between \emph{exactly} those words.

These examples of reading, looking at a picture, and choosing among competing
words \emph{marshal} related, similar examples in the readers' lives. This
allowing them to get at the experiences that trouble Wittgenstein. Through
them, he is able to do ``the work of a philosopher'' (\S 127) ``despite the
``limits of language'' (\S 119) in expressing this sort of idea about language
or experience (\S 120). They allow him to make those things which are to us too
familiar and simple to be striking strike us.


% While these sections do indeed concern our experience, the philosophical
% issues they address are those arising from philosophical concepts of
% experience.  The marshalling of recollections—recollections of ‘what we
% say’—is intended to remind us of the source of these philosophical
% concepts—these philosophical ways of talking about experience—in the diverse
% and heterogeneous ways of talking about experience within our ordinary
% behavioral contexts of our cultural forms of life.

\nocite{Wittgenstein53}

\clearpage
\bibliography{course,extra}

\end{document}
