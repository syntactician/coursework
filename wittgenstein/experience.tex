\documentclass[doc,12pt,apacite,biblatex]{apa6}
\usepackage{amssymb,amsmath,latexsym,apacite,dirtytalk}
\linespread{1.5}

% Page length commands go here in the preamble
%\setlength{\oddsidemargin}{-0.25in} % Left margin of 1 in + 0 in = 1 in
%\setlength{\textwidth}{7in}   % Right margin of 8.5 in - 1 in - 6.5 in = 1 in
%\setlength{\topmargin}{-.75in}  % Top margin of 2 in -0.75 in = 1 in
%\setlength{\textheight}{9.2in}  % Lower margin of 11 in - 9 in - 1 in = 1 in
\renewcommand{\baselinestretch}{1.5} % 1.5 denotes double spacing. Changing itwill change the spacing
\setlength{\parindent}{0in}

\begin{document}
\title{Wittgenstein on Experience}
\author{Edward Hern\'{a}ndez}
\date{\today}
\affiliation{College of William \& Mary}
\shorttitle{Wittgenstein on Experience}

\maketitle

\vspace{-20pt}
\begin{quote}
	Please review the Investigation's sections on understanding, `reading', and
	the experience of being guided (\S\S 150-176). In the ``Philosophy of
	Psychology'' fragment (a.k.a Part II), read the sections on aspect-shifting
	(\S\S 111-147) and on `the elasticity of the concept of representing what
	is seen' (secs. 160-161).  How is Wittgenstein's discussion of concepts of
	experience in these sections an example of ``the work of the philosopher''
	as he describes this in the following two sections from what is called the
	`methodological' part of the Investigations?

	\S 127. The work of the philosopher consists in marshalling recollections
	for a particular purpose.

	\S 129. The aspects of things that are most important for us are hidden
	because of their simplicity and familiarity. (One is unable to notice
	something  -- because it is always before one's eyes.) The real foundations
	of their inquiry do not strike people at all. Unless that fact has at some
	time struck them. -- And this means: we fail to be struck by what, once
	seen, is most striking and powerful.
\end{quote}
\clearpage

% the work of the philosopher
% is to 'do philosophy`
% which is to say...
% to conjure 

% he calls us to recall things which are familiar
% because we wouldn't reflect on them this way otherwise!

\citeA{Wittgenstein53} holds that ``the work of the philosopher consists in
marshalling recollections for a particular purpose'' (\S 127). I understand
this to mean that he seeks to lead us, his readers, to critically evaluate our
own memories and sensations to understand them better. Because of his idea that
philosophy ``simply puts everything before us'' (\S 127), this sort of
recollections is central to his method.  As he says, ``it is the business of
philosophy, not to resolve a contradiction by means of a ... discovery, but to
make it possible for us to get a clear view of [what] troubles us'' (\S 125).
To get this clear view, we have to recall and reexamine our experiences. This
method is most strikingly used in his discussions of reading and being guided.
% reorder to match canonical order!

In Part II of the \emph{Investigations}, like Part I, a large part of
Wittgenstein's method consists in offering the reader language-games, in which
to imagine the workings of language. In \S 130, he refers to these language
games as ``\emph{objects of comparison},'' which he says are ``meant to throw
light on the facts of our language by way not only of similarities, but also of
dissimilarities.'' Through them, he attempts to prompt the reader to examine
the workings of language in a particular, circumscribed setting or activity, to
reveal ways in which language functions in other settings. Unlike some of the
examples from early in Part I, like the builders' game, which could only take
place in an imagined community, many of the examples in Part II are drawn from
real uses of language which the reader is likely to have experienced.
Additionally, Part I often invited the reader to imagine that language worked a
particular way (as in the grocer's and builders' games), to compare that system
with the system of language with we operate. This section, instead of inviting
us to imagine the language system, invites us to imagine experiences. This
section's language games focus, I would argue, instead of on the language
itself, on the ``form[s] of life'' (\S 19) which they suggest.

The central example in \S\S 156-176 is reading. Wittgenstein calls us to
imagine people reading (\S 156), imagine ourselves reading (\S 162), and to
actually perform reading (\S 169) in various settings and circumstances.  He
shows us a disconnect between the first-personal experience we call reading (of
being guided by the letters on the page) with our conditions for saying of
someone else that they are reading.  He is only able to do this because he is
able to remind his reader of the experiences which we have when we read. He is
able to prompt us to recall previous instances in which we had these
experiences, and in this case perhaps more than any other, he is able to induce
those experiences in us. He is able to do this because of our familiarity with
reading, a thing so \emph{simple and familiar} (\S 129) that it no longer
strikes us. He marshals our recollections, and makes it striking again.

In \S 139, Wittgenstein discusses being ``guided''. He calls into question our
inclination to say that a picture \emph{suggests} a particular use or
interpretation to us (as a letter or a word might).  One of his examples is a
picture representing ``an old man walking up a steep path leaning on a stick.''
I find it easy to imagine such a picture. However, as Wittgenstein points out,
a Martian might as easily describe the image as one of the man sliding down the
hill. This is no less valid an interpretation of the picture, but we do not at
all experience an inclination to choose this application of the picture.
Whether or not pictures do suggest or force particular applications of
themselves is not the topic here (as is what Wittgenstein has to say about
whether they do). What I mean to discuss is his method and his motivation in
choosing \emph{this} example.

There are other examples that Wittgenstein could choose here. Arguably, he has
already said enough earlier in \S 139, in his discussion of cubes, and does not
need another example.  However, the cube example is not easily accessible to
me, and I imagine it is that way for many other readers. I struggle to imagine
(experiencing) methods of projection of an image of a cube, let alone ``that
not merely the picture of the cube, but also the method of projection comes
before our mind,'' (\S 141). This, I would argue, is because I have many more
recollections relevant to pictures of men walking, even though I have no
specific memory of looking at a picture of an old man climbing a steep hill
with a stick.\footnote{At least, I didn't before I began writing this paper.}
As a result, Wittgenstein is able to ``marshal'' those recollections to bring
me to an understanding of his point. Likewise, I find it easy to imagine
choosing between the words ``imposing,'' ``dignified,'' ``proud,'' and
``venerable,'' (\S 139), because I decide between similar words frequently. It
does not matter whether I can recall choosing between \emph{exactly} those
words.

These examples of reading, looking at a picture, and choosing among competing
words \emph{marshal} related, similar examples in the readers' lives. This
allowing them to get at the experiences that trouble Wittgenstein. Through
them, he is able to do ``the work of a philosopher'' (\S 127) ``despite the
``limits of language'' (\S 119) in expressing this sort of idea about language
or experience (\S 120). They allow him to make those things which are to us too
familiar and simple to be striking strike us.

\nocite{Wittgenstein53}

\clearpage
\bibliography{wittgenstein}
\bibliographystyle{apacite}

\end{document}
