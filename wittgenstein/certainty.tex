\documentclass[man,12pt,natbib]{apa6}
\usepackage[breaklinks=true,colorlinks=false]{hyperref}
\usepackage{amssymb,amsmath,latexsym,times,mathptmx}
\linespread{1.5}

\begin{document}
\title{Wittgenstein on Certainty}
\author{Edward Hern\'{a}ndez}
\date{\today}
\affiliation{College of William \& Mary}
\shorttitle{Wittgenstein on Certainty}

\maketitle

\begin{quote}
	Wittgenstein rejects both Moore's `common-sense' (or `realist') argument
	and the sceptical doubt of the idealist which Moore's argument is designed
	to refute.  Why?  What is Wittgenstein's own (positive) view of certainty,
	knowing, doubting, justificational grounds, and `the bottom of the language
	game'?

	% \S 3.  If e.g. someone says ``I don't know if there's a hand here'' he
	% might be told ``Look closer''.  --- This possibility of satisfying
	% oneself is part of the language-game.  It is one of its essential
	% features.

	% \S 10. I know that a sick man is lying here?  Nonsense! (...) So I don't
	% know, then, that there is a sick man lying here?  Neither the question
	% nor the assertion makes any sense.

	% \S 24. The idealist's question would be something like: ``What right have
	% I not to doubt the existence of my hands?'' (And to that the answer can't
	% be: I know that they exist.) But someone who asks such a question is
	% overlooking the fact that a doubt about existence only works in a
	% language-game.  Hence, that we should first have to ask: what would such
	% a doubt be like?, and don't understand this straight off.

	% \S 56. And everything descriptive of a language-game is part of logic.

	% \S 59.  ``I know'' is here a logical insight.  Only realism can't be
	% proved by means of it.

	% \S 87. Can't an assertoric sentence, which was capable of functioning as
	% an hypothesis, also be used as a foundation for research and action?
	% i.e., can't it simply be isolated from doubt, though not according to any
	% explicit rule?  It simply gets assumed as a truism, never called in
	% question, perhaps not even ever formulated?

	% \S 105. All testing, all confirmation and disconfirmation of a hypothesis
	% takes place already within a system.  And this system is not a more or
	% less arbitrary and doubtful point of departure for all our arguments: no,
	% it belongs to the essence of what we call an argument. The system is not
	% so much the point of departure, as the element in which arguments have
	% their life.

	% \S 114. If you are not certain of any fact, you cannot be certain ofo the
	% meaning of your words either.

	% \S 115. If you tried to doubt everything you would not get as far as
	% doubting anything.  The game of doubting itself presupposes certainty.  

	% \S 127.  For how do I know that someone is in doubt?  How do I know that
	% he uses the words ``I doubt it'' as I do?

	% \S 151 - 152.  I should like to say: Moore does not know what he asserts
	% he knows, but it stands fast for him, as also for me; regarding it as
	% absolutely solid is part of our method of doubt and enquiry.  I do not
	% explicitly learn the propositions that stand fast for me.  I can discover
	% them subsequently like the axis around which a body rotates.  This axis
	% is not fixed in the sense that anything holds it fast, but the movement
	% around it determines its immobility.  \emph{cf}. \S\S~109, 110, 141, 142,
	% 213

	% \S 204.  Giving grounds, however, justifying the evidence, comes to an
	% end;
	% --- but the end is not certain propositions' striking us immediately as
	% true, i.e. it is not a kind of seeing on our part; it is our acting,
	% which lies at the bottom of the language-game.

	% \S 341. [T]he questions that we raise and our doubts depend on the fact
	% that some propositions are exempt from doubt, are as it were like hinges
	% on which those turn.

	% \S 358 - 359.  Now I would like to regard this certainty ... as a form of
	% life.  But that means I want to conceive it as something that lies beyond
	% being justified or unjustified; as it were, as something animal.

	% \S 166.  The difficulty is to realize the groundlessness of our
	% believing.  

	% \S 455.  Every language-game is based on words and `objects' being
	% recognized again.  We learn with the same inexorability that this is a
	% chair as that $2 \times 2 = 4.$

	% \S 527.  An Englishman who calls this color ``red'' is not `sure it is
	% called ``red'' in English'. A child who has mastered the use of the word
	% is not `sure that in his language this color is called...'.  Nor can one
	% say of him that when he is learning to speak he learns that the color is
	% called that in English; nor yet: he knows this when he has learned the
	% use of the word.  (...) And in spite of this ... one child, for example,
	% will say, of another or of himself, that he already knows what
	% such-and-such is called.

	% \S 538.  The child, I should like to say, learns to react in
	% such-and-such a way; and in so reacting doesn't so far know anything.
	% Knowing only begins at a later level.
\end{quote}
\clearpage

\citet{Moore39} famously claims to prove the existence of an external world on
the basis of his certainty that he has hands. The skeptical argument, to which
Moore is responding\footnote{I am not entirely clear on how the ``skeptical''
and ``idealist'' views to which he seems to be responding differ, though I'm
sure they do.} is of the following form:
\begin{quote}
	(P1) I don't know that not-$H$, then I don't know that $CS$ \\
	(P2) I don't know that not-$H$ \\
	$\therefore$ I don't know that $CS$
	\vspace{6pt} \\
	where: \\
	$H$ is a skeptical hypothesis \\
	$CS$ is any common sense proposition or `Moorean fact'\\
	\citep{DeRose99}
\end{quote}
This is formally a simple \emph{modus ponens}:
\begin{quote}
	(P1) $\neg$P $\to~\neg$Q \\
	(P2) $\neg$P \\
	$\therefore~\neg$Q
	\vspace{6pt} \\
	where: \\
	P = I know that not-$H$ \\
	Q = I know that $CS$ \\
	\citep{Robinson10}
\end{quote}
Moore responds by claiming that he \emph{does} know $q$, that ``here is one
hand''. Since he couldn't know this if the skeptical possibility were true, it
must be false.  He reverses the skeptical argument's \emph{modus ponens} into a
\emph{modus tollens} in a logical maneuver (apparently) now called a Moorean
Shift \citep{Preston}:
\begin{quote}
	(P1) If I don't know that not-$H$, then I don't know that $CS$ \\
	(P2) I knows that $CS$ \\
	$\therefore$ not-$H$
	\vspace{6pt} \\
	where: \\
	I am G. E. Moore \\
	$H$ is any skeptical possibility which denies the external world \\
	$CS$ is ``here is one hand''
\end{quote}
Or, formally:
\begin{quote}
	(P1) $\neg$P $\to~\neg$Q \\
	(P2) Q \\
	$\therefore$ P
\end{quote}
This is the argument to which \citet{Wittgenstein69} responds in \emph{On
Certainty}.
% Nicely clarified.

Wittgenstein concedes that \emph{if} Moore does know that \emph{here is one
hand}, the argument is sound (\S 1). However, he is not convinced as Moore is,
that this proposition is certain simply from its \emph{seeming} true.

Wittgenstein asks whether ``it can make sense to doubt'' (\S 2) a proposition
like `here is one hand.' Outside of philosophy, we would never think to do such
a ridiculous thing as doubt the existence of our own hands.\footnote{And,
	indeed, non-philosophy majors to whom I've talked about this paper were
surprised and somewhat confused that this was even a topic up for debate.}
% They must think you pretty strange asking them if they know whether they have
% any hands!  See what your college education has done for you?
It also seems that this doubt is not the same as our ordinary doubts.
% One might say that W tries to shift the discussion from being about doubts
% (commonsense doubts, philosophical doubts – and knowledge, certainty, etc.)
% to the language game(s) in which expressions of doubt (and knowledge and
% certainty) occur. But why should this satisfy the sceptic?  Or anyone for
% that matter?
If I doubt whether or not `there's a hand here', I might be told to ``Look
closer'' (\S 3). The possibility of satisfaction, on Wittgenstein's view, is a
part of our language game of doubt. But the skeptic's doubt cannot be satisfied
by looking harder, or at all within this language game.
% Very good.  And one might ask what exactly is being talked about by the
% expression ``the sceptic's doubt''? Is it a (private) feeling or attitude
% s/he has, like a pain or a fondness [for x]?  For sure, s/he talks about
% ``doubting'' that P and that s/he has ``doubts'' about P – just as, e.g.,
% Moore talks about ``knowing'' that P and  being ``certain'' that P (or his
% ``knowledge'' and ``certainty'' that P).  
%
% How might they support/attempt to justify their claims of doubting, knowing,
% being certain?  Do they pull out their doubts, etc., and display them for
% public examination?  No, they say (and/or act within discourse: I'm thinking
% here of Moore waving his hand in front of his audience), ``I doubt/know/am
% certain about this because....'' .  And how do they support/attempt to
% justify those claims?  (I'm reminded here of Margaret who always follows your
% answer to her 1st `why' query by addressing a 2nd `why'' to your answer.  And
% so on....)  Where does it stop? 

Like
% Just as this `doubt'...
this `doubt' is not the same as our ordinary use of `doubt' in everyday
language games, `knowing' for Moore is different from our ordinary `knowing.'
% Careful, this is tricky ground: in the second `doubt' you are referring to
% the use of a word, but in the first the reference is not to a word but to....
% what?  And how are the scare quotes meant to function here: metapragmatic
% presentationally?  And what do the two instances of `knowing' refer to?  Or
% the scare-quoted `do not know' later in the paragraph?
I would never say that I \emph{know} that I have a hand. It cannot make sense
for me to doubt this, so it would never make sense for me to say I know it.
% As James Bond says, ``never say never'' (for it invites the reader's reply
% ``Really? What about .....?'')
That, however, does not mean that I `do not know' that I have a hand. It means
that these assertions are nonsense (\S 10). They, if I understand correctly,
fall outside our language games, outside our grammar of `knowing.'

In \S\S 57-59, Wittgenstein further discusses the grammar of ``I know,'' making
a distinction between knowing and surmising. For him, ``I know'' is in some
sense equivalent to ``There is no such thing as a doubt in this case'' (\S 58).
If the grammar of ``I know'' is conceived of in this way, it does not (and
cannot) matter who utters ``I know...''; it cannot be contingent on the
speaker. ``I know'' is then ``a logical insight'' (\S 59), a statement about
the world rather than about our convictions. However, this sort of statement,
on his view, cannot be used to prove realism (\S 59).  This seems to be a
direct rejection of Moore's \emph{modus tollens}, asserting that such
statements cannot prove realism at all.\footnote{I am not completely sure what,
	in particular, motivates this rejection. I am not sure whether it is his
	conviction that Moore's argumentation is a misapplication of the language
	games of knowing, or that ``I know'' is a hypothesis in need of some sort
	of testing (\S 60?). Either seems plausible to motivate such a rejection.}

In his addition to rejecting Moore's argument Wittgenstein admits that the
skeptical hypothesis is ``not nonsense'' (\S 37),
% What he says is that ``for them'' it is not nonsense and that [his
% grammatical remarks about the language-games of knowing, doubting, etc.] is
% not ``the end of the matter''.  These are important points.  I think what he
% says in 38 can be seen as addressing the first, but I don't know where he
% ever addresses the 2nd.  Although I don't believe he would agree to
% characterizing it as ``admitting that we cannot prove the existence of an
% external world''.
arguably inhabiting the position \citet{Wright91} calls the ``Russellian
Retreat,'' of admitting that we cannot prove the existence of an external world
\citep{Robinson}.  However, unlike \citet{Russell12}, Wittgenstein does not
give an account of knowing which appeals to evidence or likelihood.  Instead,
Wittgenstein takes as his starting point that some propositions are exempt from
doubt (\S 341). It is not that they are any less worthy of being doubted, or
that they are in some way less susceptible to the skeptic's argumentation.
Rather, we are not capable of doubting them.
% What do you mean by ``not capable'' (or for that matter, ``we'')?
We cannot avoid our conviction that they are true,
% We cannot avoid our conviction...''.  But doesn't the sceptic do a pretty
% good job of not-not-avoiding this conviction?
and, moreover, even if we were to doubt them in some philosophical sense, we
would still \emph{act} as if we held them true.  (If I, as a skeptic, doubted
my hands, I would still use them to write my doubts.)
% Well-put.  Contrasting conviction with action is crucial here.
This acting, Wittgenstein claims, lies at the bottom of the language game (\S
204).\footnote{But at the bottom of which language game? Of certainty?  Of
``knowing''? All language games?} Even to `doubt,' there are things I must hold
certain (\S 115) to construct the language games of doubt. Without the language
game ``[I] would not get as far as doubting anything,'' (\S 115), and if you
could somehow doubt privately, ``how [could] I know that someone is in doubt''
without some presuppositions?
% ???

On his view, these propositions ``stay put'' (\S 343) and act as a hinge on
which other propositions (\S 341), other doubts, other disputes (\S 655), and
our very lives (\S 344) turn (if those things are to function, which they do).
It is impossible, as he points out, to play chess without accepting that the
pieces are not moving without one's noticing (\S 346), let alone without
accepting the existence of an external world. These things form the
``starting-point'' (\S 209) for the chess game, which must be fixed for the
players to participate.
% Very nice passage here, with the references to particular passages to ground
% your reading of the text in the text. It also demonstrates your thorough
% familiarity with the text.  
Likewise, if my friend tells me he has certain tree buds in New York (\S 208)
and I am convinced, I must also hold that his tree exists, that the earth
exists...
% Not how I think these remarks should be read.
Moore, then, (on Wittgenstein's view) when he says he `knows'
% I think you need to be more careful with scare quotes.  They are ambiguous.
% What, for instance, do you mean by using them here?  Why not ``knows''?  Or
% with no quotes at all?  And the italics.....?
that \emph{there exists an external world} on the basis of his certainty
% Shouldn't the `knowing' come first and the `certainty' second in this `on the
% basis of' relation?
that \emph{here is one hand} is expressing that the existence of an external
world is a hinge (or set of hinges), which must stay put, on which his
certainty about his hands turns (\S 136).

Wittgenstein seems to hold that we cannot act as if these hinge propositions
were false nor truly
% ...sensibly
hold them to be false. This avoids giving
% Interesting choice of expression here. Why do you say ``giving in to''?  Is
% it a force pulling one?  Or a weakness or addiction?  Or a bad habit?  I
% think it is indeed an appropriate expression in talking about Wittgenstein's
% perspective on such things, but all on its own, without explanation, the
% reason for the choice of expression is unclear.
into radical skepticism. However, I am not sure how this is a satisfying
alternative.
% How it is? Or for whom it is (might be)?  Or whether it is?

\clearpage
\bibliography{course,extra}

\end{document}
