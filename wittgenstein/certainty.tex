\documentclass[doc,12pt,apacite,biblatex]{apa6}
\usepackage{amssymb,amsmath,latexsym,apacite,dirtytalk}
\linespread{1.5}

% Page length commands go here in the preamble
% \setlength{\oddsidemargin}{-0.25in} % Left margin of 1 in + 0 in = 1 in
% \setlength{\textwidth}{7in}   % Right margin of 8.5 in - 1 in - 6.5 in = 1 in
% \setlength{\topmargin}{-.75in}  % Top margin of 2 in -0.75 in = 1 in
% \setlength{\textheight}{9.2in}  % Lower margin of 11 in - 9 in - 1 in = 1 in
% \renewcommand{\baselinestretch}{1.5} % 1.5 denotes double spacing. Changing itwill change the spacing
\setlength{\parindent}{0in}

\begin{document}
\title{Wittgenstein on Certainty}
\author{Edward Hern\'{a}ndez}
\date{\today}
\affiliation{College of William \& Mary}
\shorttitle{Wittgenstein on Certainty}

\maketitle

\vspace{-20pt}
\begin{quote}
	Wittgenstein rejects both Moore's `common-sense' (or `realist') argument
	and the sceptical doubt of the idealist which Moore's argument is designed
	to refute.  Why?  What is Wittgenstein's own (positive) view of certainty,
	knowing, doubting, justificational grounds, and `the bottom of the language
	game'?
\end{quote}
\clearpage

\citeA{Moore39} famously claims to prove the existence of an external world on
the basis of his certainty that he has hands. The skeptical argument, to which
Moore is responding\footnote{I am not entirely clear on how the ``skeptical''
and ``idealist'' views to which he seems to be responding differ, though I'm
sure they do.} is of the following form:
\begin{quote}
	(P1) I don't know that not-$H$, then I don't know that $CS$ \\
	(P2) I don't know that not-$H$ \\
	$\therefore$ I don't know that $CS$
	\vspace{6pt} \\
	where: \\
	$H$ is a skeptical hypothesis \\
	$CS$ is any common sense proposition or `Moorean fact'\\
	\cite{DeRose99}
\end{quote}
This is formally a simple \emph{modus ponens}:
\begin{quote}
	(P1) $\neg$P $\to~\neg$Q \\
	(P2) $\neg$P \\
	$\therefore~\neg$Q
	\vspace{6pt} \\
	where: \\
	P = I know that not-$H$ \\
	Q = I know that $CS$ \\
	\cite{Robinson10}
\end{quote}
Moore responds by claiming that he \emph{does} know $q$, that ``here is one
hand''. Since he couldn't know this if the skeptical possibility were true, it
must be false.  He reverses the skeptical argument's \emph{modus ponens} into a
\emph{modus tollens} in a logical maneuver (apparently) now called a Moorean
Shift \cite{Preston}:
\begin{quote}
	(P1) If I don't know that not-$H$, then I don't know that $CS$ \\
	(P2) I knows that $CS$ \\
	$\therefore$ not-$H$
	\vspace{6pt} \\
	where: \\
	I am G. E. Moore \\
	$H$ is any skeptical possibility which denies the external world \\
	$CS$ is ``here is one hand''
\end{quote}
Or, formally:
\begin{quote}
	(P1) $\neg$P $\to~\neg$Q \\
	(P2) Q \\
	$\therefore$ P
\end{quote}
This is the argument to which \citeA{Wittgenstein69} responds in \emph{On
Certainty}.

Wittgenstein concedes that \emph{if} Moore does know that \emph{here is one
hand}, the argument is sound (\S 1). However, he is not convinced as Moore is,
that this proposition is certain simply from its \emph{seeming} true.

Wittgenstein asks whether ``it can make sense to doubt'' (\S 2) a proposition
like `here is one hand.' Outside of philosophy, we would never think to do such
a ridiculous thing as doubt the existence of our own hands.\footnote{And,
indeed, non-philosophy majors to whom I've talked about this paper were
surprised and somewhat confused that this was even a topic up for debate.}
It also seems that this doubt is not the same as our ordinary doubts. If I
doubt whether or not there `there's a hand here', I might be told to ``Look
closer'' (\S 3). The possibility of satisfaction, on Wittgenstein's view, is a
part of our language game of doubt. But the skeptic's doubt cannot be satisfied
by looking harder, or at all within this language game.

Like this `doubt' is not the same as our ordinary use of `doubt' in everyday
language games, `knowing' for Moore is different from our ordinary `knowing.' I
would never say that I \emph{know} that I have a hand. It cannot make sense for
me to doubt this, so it would never make sense for me to say I know it. That,
however, does not mean that I `do not know' that I have a hand. It means that
these assertions are nonsense (\S 10). They, if I understand correctly, fall
outside our language games, outside our grammar of `knowing.'

In \S\S 57-59, Wittgenstein further discusses the grammar of ``I know,''
making a distinction between knowing and surmising. For him, ``I know'' is in
some sense equivalent to ``There is no such thing as a doubt in this case'' (\S
58). If the grammar of ``I know'' is conceived of in this way, it does not (and
cannot) matter who utters ``I know...''; it cannot be contingent on the
speaker. ``I know'' is then ``a logical insight'' (\S 59), a statement about
the world rather than about our convictions. However, this sort of statement,
on his view, cannot be used to prove realism (\S 59).  This seems to be a
direct rejection of Moore's \emph{modus tollens}, asserting that such
statements cannot prove realism at all.\footnote{I am not completely sure what,
in particular, motivates this rejection. I am not sure whether it is his
conviction that Moore's argumentation is a misapplication of the language games
of knowing, or that ``I know'' is a hypothesis in need of some sort of testing
(\S 60?). Either seems plausible to motivate such a rejection.}

In his addition to rejecting Moore's argument Wittgenstein admits that the
skeptical hypothesis is ``not nonsense'' (\S 37), arguably inhabiting the
position \citeA{Wright91} calls the ``Russellian Retreat,'' of admitting that
we cannot prove the existence of an external world \cite{Robinson}.  However,
unlike \citeA{Russell12}, Wittgenstein does not give an account of knowing
which appeals to evidence or likelihood.  Instead, Wittgenstein takes as his
starting point that some propositions are exempt from doubt (\S 341). It is not
that they are any less worthy of being doubted, or that they are in some way
less susceptible to the skeptic's argumentation. Rather, we are not capable of
doubting them. We cannot avoid our conviction that they are true, and,
moreover, even if we were to doubt them in some philosophical sense, we would
still \emph{act} as if we held them true.  (If I, as a skeptic, doubted my
hands, I would still use them to write my doubts.) This acting, Wittgenstein
claims, lies at the bottom of the language game (\S 204).\footnote{But at the
bottom of which language game? Of certainty?  Of ``knowing''? All language
games?}
Even to `doubt,' there are things I must hold certain (\S 115) to construct the
language games of doubt. Without the language game ``[I] would not get as far
as doubting anything,'' (\S 115), and if you could somehow doubt privately,
``how [could] I know that someone is in doubt'' without some presuppositions?

On his view, these propositions ``stay put'' (\S 343) and act as a hinge on
which other propositions (\S 341), other doubts, other disputes (\S 655), and
our very lives (\S 344) turn (if those things are to function, which they do).
It is impossible, as he points out, to play chess without accepting that the
pieces are not moving without one's noticing (\S 346), let alone without
accepting the existence of an external world. These things form the
``starting-point'' (\S 209) for the chess game, which must be fixed for the
players to participate. Likewise, if my friend tells me he has certain tree
buds in New York (\S 208) and I am convinced, I must also hold that his tree
exists, that the earth exists...
Moore, then, (on Wittgenstein's view) when he says he `knows' that \emph{there
exists an external world} on the basis of his certainty that \emph{here is one
hand} is expressing that the existence of an external world is a hinge (or set
of hinges), which must stay put, on which his certainty about his hands turns
(\S 136).

Wittgenstein seems to hold that we cannot act as if these hinge propositions
were false nor truly hold them to be false. This avoids giving into radical
skepticism. However, I am not sure how this is a satisfying alternative.

\clearpage
\bibliography{wittgenstein}
\bibliographystyle{apacite}

\end{document}
