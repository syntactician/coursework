\documentclass[man,12pt,natbib]{apa6}
\usepackage[breaklinks=true,colorlinks=false]{hyperref}
\usepackage{amssymb,amsmath,latexsym,times}
\linespread{1.5}

% Page length commands go here in the preamble
%\setlength{\oddsidemargin}{-0.25in} % Left margin of 1 in + 0 in = 1 in
%\setlength{\textwidth}{7in}   % Right margin of 8.5 in - 1 in - 6.5 in = 1 in
%\setlength{\topmargin}{-.75in}  % Top margin of 2 in -0.75 in = 1 in
%\setlength{\textheight}{9.2in}  % Lower margin of 11 in - 9 in - 1 in = 1 in
\renewcommand{\baselinestretch}{1.5} % 1.5 denotes double spacing. Changing itwill change the spacing
\setlength{\parindent}{0in}

\begin{document} \title{Wittgenstein on Augustine}
\author{Edward Hern\'{a}ndez}
\date{\today}
\affiliation{College of William \& Mary}
\shorttitle{Wittgenstein on Augustine}

\maketitle

% \vspace{-20pt}
\begin{quote}
	In \S 32, Wittgenstein says that ``Augustine describes the learning of
	human language as if the child came into a foreign country and did not
	already understand the language of the country; that is, as if he already
	had a language, only not this one.''

	How can the sections prior to 32 be seen as leading to this point?  What
	role does Wittgenstein's use of invented language-games, especially the
	grocer's game (in \S 1) and the builder's game (in \S\S~2, 8, etc.) play in
	the development of this point?

	% In your response, you might consider incorporating the points Wittgenstein
	% makes in \S 13 (``If we say, `Every word in the language signifies
	% something', we have so far said nothing whatever''), in \S 19 (``...to
	% imagine a language means to imagine a form of life''), in \S 22 (Frege's
	% opinion that every assertion contains an assumption...), and/or in \S 31
	% (re. explaining chess to someone).  
\end{quote}
\clearpage

\citet{Wittgenstein53} opens his \emph{Philosophical investigations} with a
passage from Augustine:\footnote{I am taking the liberty of producing a recent
translation, rather than the one Anscombe provides.}
\begin{quote}
	When they named any thing, and as they spoke turned towards it, I saw and
	remembered that they called what they would point out by the name they
	uttered. And that they meant this thing and no other was plain from the
	motion of their body, the natural language, as it were, of all nations,
	expressed by the countenance, glances of the eye, gestures of the limbs,
	and tones of the voice, indicating the affections of the mind, as it
	pursues, possesses, rejects, or shuns. And thus by constantly hearing
	words, as they occurred in various sentences, I collected gradually for
	what they stood; and having broken in my mouth to these signs, I thereby
	gave utterance to my will. \citep[\S 1.8.13]{Pusey09}
\end{quote}
Wittgenstein then asserts that Augustine's view, as expressed here,
rests upon some assumptions: ``Every word has a meaning. This meaning is
correlated with the word. It is the object for which the word stands.''
According to \citet{McGinn97}, these assumptions tempt not just Augustine, but
also Frege, Russell, and even Wittgenstein (in the \emph{Tractatus}. 
% More generally, he views it as a temptation for anyone doing philosophy or
% theorizing about language. And why should it be thought to be a false
% assumption?  Doesn’t the word ‘essay’ in “I’m looking at Edward’s essay on my
% computer” stand for what I am looking at?  Doesn’t the word “computer” stand
% for/mean the thing that I’m typing on?  Wouldn’t it be bizarre to say that,
% no, even if most people would say those words do in fact stand for those
% things, they’re wrong?
Here, Wittgenstein centers 
% I think “focuses on” would be better here.  If “center” , then I believe it
% should be “centers on”.
Augustine's account in order to resist and react against
this temptation. He does so by constructing ``language-games'', or situations
in which speakers use language situated in other activity.

The first language-game Wittgenstein offers is the grocer's game. It involves
sending an order to the grocer for ``five red apples''. He imagines that upon
receiving this order, the grocer:
\begin{quote}
	opens the drawer marked "apples"; then he looks up the word "red" in a
	table and finds a colour sample opposite it; then he says the series of
	cardinal numbers---I assume that he knows them by heart---up to the word
	"five" and for each number he takes an apple of the same colour as the
	sample out of the drawer. \citep[\S 1]{Wittgenstein53}
\end{quote}
Already, we are moved away from the assumption that what is important is what
is signified by each word. 
% I think something more specific than “signified” would be better here.  “…
% what is important is the object denoted by each word” (or “the objects for
% which the words stand”).  
%
% A surrogational theory of meaning: Cf sec.120 “People say: it’s not the word
% that counts, but its meaning, thinking of the meaning as a thing of the same
% kind as the word, even though different from the word. Here the word, there
% the meaning.  The money and the cow one can buy with it.”
We may be tempted to ask what the word ``five'' means here, but Wittgenstein
tells us that ``No such thing was in question here, only how the word "five"
is used.''
% Couldn’t one respond “Sure – since how it was used was to stand for the
% number five”?

Wittgenstein's next language-game, the builder's game, offers a small language
in which to test 
% “test” is misleading, since he simply says “Let’s imagine a language for
% which A is right” The language-game is an object of comparison. Cf. sec 130: 
%
% Our clear and simple language-games are not preliminary studies for a future
% regimentation of language -- as it were, first approximations, ignoring
% friction and air resistance. Rather, the language- games stand there as
% objects of comparison which, through similarities and dissimilarities, are
% meant to throw light on features of our language.
%
% So how does the grocer’s or builder’s language-games “throw light on features
% of our language”?  And, as objects of comparison, how do they help us resist
% the temptation to speak as Augustine does of word meanings?
his ideas (\S 2). The language consists of (only) words for
building materials: ``block,'' ``pillar,'' ``slab,'' and ``beam.'' This
language is in use between builder A and assistant B. When A says ``slab,'' B
passes a slab. Wittgenstein invites us to conceive of this as a complete
language. This language fits Augustine's picture well, and, as \citet{McGinn97}
points out, not only in the passage Wittgenstein quotes above. Earlier in his
\emph{Confessions} Augustine writes:
\begin{quote}
	Thus, little by little, I became conscious where I was; and to have a wish
	to express my wishes to those who could content them, and I could not; for
	the wishes were within me, and they without; nor could they by any sense of
	theirs enter within my spirit. So I flung about at random limbs and voice,
	making the few signs I could, and such as I could, like, though in truth
	very little like, what I wished. \citep[\S 1.6.8]{Pusey09}
\end{quote}
So, in both Augustine's conception and in the builder's game, a
primary function of language is to request objects. Additionally, since these
objects correspond to names, to words in the language, it is possible to learn
those names, and thus, learn the language.

Wittgenstein goes on to resist this picture in a number of ways. First, he
makes the obvious assertion that this is not all there is to language as we use
it. Augustine's account, like the builder's game ``does describe a system of
communication; only not everything that we call language is this system''
\citep[\S 3]{Wittgenstein53}. In \S 8, he expands the language-game to include
more words and concepts, including ``this'' and ``that'' (accompanied by
pointing) as well as analogs of elements from the grocer's game: numerals and
color samples. Immediately, it becomes obvious that Augustine's picture of
learning language cannot easily account for these new words.
% It does?  Why? This needs explanation.

Later, he goes on to resist Augustine's understanding of learning language 
% language learning
in addition to his conception of how language functions. If we imagine that the
language of \S 2 is the whole language of a tribe of builders, then children
``are brought up to perform \emph{these} actions, to use \emph{these} words
as they do so, and to react in \emph{this} way to the words of others''
\citep[\S 6]{Wittgenstein53}. He asserts that while training in this language
will consist of pointing at building materials and uttering their names, this
ought not be thought of as ``ostensive definition,'' because the child cannot
yet ask the name of the objects. Instead, he would prefer to call it
``ostensive teaching of words'' because it merely establishes a connection
between word and thing. The understanding of the word does not arise solely
from this ostensive teaching, he argues but also from other training, otherwise
the understanding at which the child arrived would be different. We can imagine
a child growing up ostensively taught a connection betweens slabs and the word
``slab,'' but not having these children's understanding of ``Slab!'' to
mean ``bring me a slab.''
% Needs more explanation. What grounds are there for saying that, despite the
% child having learned the association between “slab” and slabs, she still does
% not understand the word as it is used in the builder’s game?  What else is
% needed so that she would understand?  Cf. sec. 6 re. “uttering a word [being]
% like striking a note on the keyboard of the imagination.”

In \S 27, Wittgenstein returns to ostensive definition, and the critique to
which he has been building starts to solidify. He points out that the builder's
language has no way of asking something's name. Asking for a name is a language
game 
% n.b., a metalinguistic language-game.
that we are brought up to perform, but which might, conceivably, not be
available. 
% Indeed it is not available in the builder’s game as described in sec. 3.  So,
% given this absence, are slab, block, etc. names in the builder’s
% language-game? Can they be names in the absence of any metalinguistic
% practices, such as ostensive definition?  If not, why not?
Obviously, without learning names, one can still learn quite a lot about
\emph{usage}. In the builder's game, one can learn to produce a slab when she
hears ``slab,'' without it \emph{necessarily} being a name for the thing, 
% Of course, a dog can do much the same thing: go get his owner’s boots when
% the owner says “Walkies!”  
and in \S 31, Wittgenstein shows that one might learn chess without learning
the name of each piece. To ask for a definition of a chess piece, 
% Do you mean “To ask for the name of a chess piece…” (or “ask what it is
% called”)?
and to be told ``this is the king,'' tells one nothing about how to use it.
% What does “it” refer to here?  To the name “king” or to the piece itself. You
% have described the situation in terms of a child who has already learned
% chess, yet not the names of the pieces.  So, what is at issue is how to use
% the word, rather than how to use the piece (which the child already knows),
% right?
Likewise, knowing how to use it
% Again, what does “it” refer to: piece or name?
tells us nothing about its name. Wittgenstein says here that learning the name
of something can only tell us its use if ``the place for it is already
prepared,'' which is to say that we already understand to what the name refers.
To be told ``this is the king'' only helps if we already know what a king
\emph{is}.
% But, you’ve already established that the child has learned chess – so s/he
% knows what a king is in chess, what it can and cannot do, its role in the
% game, what being in ‘check’ is, etc.  So that isn’t LW’s point.  He says:
% “one has already to know (or be able to do) something before one can ask what
% something is called” (sec.30).  This ‘something’ is not apparently a matter
% of knowing how to use the king in chess – for s/he already does know this.
% What s/he doesn’t know yet is what it is to be a name or for something to be
% called N.  Such metalinguistic knowledge is quite different from merely
% knowing that X is associated with N.  (A dog looks at you when you say its
% name; he associates those sounds with himself or with some behavioral
% response.  But does he know that Rover is his name?  What kind of
% behavior/evidence would show that he does?) 
%
% Asking for an ostensive definition—e.g., “What’s this called?” or “What does
% king mean?”—only makes sense if the person asking already knows what use to
% make of the word as a name—how to ‘do something’ with the word that is
% ostensively defined: namely, use it in the ways that we use what we call
% ‘names’.  It is because he presupposes that the child is endowed with this
% (metalinguistic) knowledge that LW says that Augustine “describes the
% learning of human language as if the child came into a foreign country and …
% already had a language [and so already knew what a name is], only not this
% one”. 

A child does not know what a king is, nor does she know (how) to ask for
ostensive definitions. If she does not understand the concept already, 
% The concept?  That term hasn’t been involved in the discussion so far.
% Rather, it has so far only been about words and objects.
learning the name cannot help her make sense of it. Learning the signs for
objects, as Augustine describes, cannot give her the full set of tools she
needs to communicate. She must also have training of some other kind. 
% For one, she needs to know what a name is.  This is “the place” that already
% needs to be “prepared”, if she is to understand the ostensive explanation. An
% association between word W and object O is not the same thing as W being the
% name of O; so acquiring that association does not amount to learning that X
% is a name for O.  For that, as McGinn says, the child must already possess “a
% mastery of the techniques, or ways of operating with words”.
If Augustine's account were complete, that would mean that the child could
already ask about ostensive definition and guess about signs' meanings.
\citet{McGinn97} sums it up well in saying:
\begin{quote}
	Any sense that the account of language acquisition that Augustine presents
	somehow explains how we learn language is thus shown to be an illusion. For
	the picture actually presupposes what it purports to explain, by assuming
	that the child possesses a mastery of the techniques, or ways of operating
	with words, that provide the necessary background to his understanding what
	is meant when an adult points and utters a sound. (Ch.~2)
\end{quote}

% Good paper, Edward, despite all the (metaphorical) ink I have spilled in the
% margins.  You write well, which makes everything easier.  The prompt asks how
% the sections before 32 can be seen as leading to the point about Augustine
% describing “the learning of human language as if the child came into a
% foreign country and did not already understand the language of the country;
% that is, as if he already had a language, only not this one.”  I’m not sure
% how your paper does this.  At the same time, although you make some good
% points about the builder’s and grocer’s language-games, I think we need to
% look more closely at how Wittgenstein’s language-games function as crucial
% components in his philosophical method – especially in helping to resist the
% temptation to take the same first steps down the philosophical/theorizing
% road as Augustine does.  We ought also to talk a bit about his remarks in
% sec. 13 (“If we say, ‘Every word in the language signifies something’, we
% have so far said nothing whatever”) and in sec. 19 (“… to imagine a language
% means to imagine a form of life”).

\clearpage

\bibliography{course,extra}

\end{document}
