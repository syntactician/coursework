\documentclass[doc,12pt,apacite,biblatex]{apa6}
\usepackage{amssymb,amsmath,latexsym,apacite,dirtytalk}

% Page length commands go here in the preamble
%\setlength{\oddsidemargin}{-0.25in} % Left margin of 1 in + 0 in = 1 in
%\setlength{\textwidth}{7in}   % Right margin of 8.5 in - 1 in - 6.5 in = 1 in
%\setlength{\topmargin}{-.75in}  % Top margin of 2 in -0.75 in = 1 in
%\setlength{\textheight}{9.2in}  % Lower margin of 11 in - 9 in - 1 in = 1 in
\renewcommand{\baselinestretch}{1.5} % 1.5 denotes double spacing. Changing itwill change the spacing
\setlength{\parindent}{0in}

\usepackage{dirtytalk}

\begin{document} \title{Wittgenstein on Augustine}
\author{Edward Hern\'{a}ndez}
\date{\today}
\affiliation{College of William \& Mary}
\shorttitle{Wittgenstein on Augustine}

\maketitle

\citeA{Wittgenstein53} opens his \emph{Philosophical investigations} with a
passage from Augustine:\footnote{I am taking the liberty of producing a recent
translation, rather than the one Anscombe provides.} \begin{quote} When they
	named any thing, and as they spoke turned towards it, I saw and
	remembered that they called what they would point out by the name they
	uttered. And that they meant this thing and no other was plain from the
	motion of their body, the natural language, as it were, of all nations,
	expressed by the countenance, glances of the eye, gestures of the
	limbs, and tones of the voice, indicating the affections of the mind,
	as it pursues, possesses, rejects, or shuns. And thus by constantly
	hearing words, as they occurred in various sentences, I collected
	gradually for what they stood; and having broken in my mouth to these
	signs, I thereby gave utterance to my will.  \cite[\S~1.8.13]{Pusey09}
\end{quote} Wittgenstein then asserts that Augustine's view, as expressed here,
rests upon some assumptions: \say{Every word has a meaning. This meaning is
correlated with the word. It is the object for which the word stands.}
According to \cite{McGinn97}, these assumptions tempt not just Augustine, but
also Frege, Russell, and even Wittgenstein (in the \emph{Tractatus}. Here,
Wittgenstein centers Augustine's account in order to resist and react against
this temptation. He does so by constructing \say{language-games}, or situations
in which speakers use language situated in other activity.

The first language-game Wittgenstein offers is the grocer's game. It involves
sending an order to the grocer for \say{five red apples}. He imagines that upon
receiving this order, the grocer: \begin{quote} opens the drawer marked
	"apples"; then he looks up the word "red" in a table and finds a colour
	sample opposite it; then he says the series of cardinal numbers---I
	assume that he knows them by heart---up to the word "five" and for each
	number he takes an apple of the same colour as the sample out of the
drawer. \cite[\S~1]{Wittgenstein53} \end{quote} Already, we are moved away from
the assumption that that what is important is what is signified by each word.
We may be tempted to ask what the word \say{five} means here, but Wittgenstein
tells us that \say{No such thing was in question here, only how the word "five"
is used.}

Wittgenstein's next language-game, the builder's game, offers a small language
in which to test his ideas (\S~2). The language consists of (only) words for
building materials: \say{block,} \say{pillar,} \say{slab,} and \say{beam.} This
language is in use between builder A and assistant B. When A says \say{slab,} B
passes a slab. Wittgenstein invites us to conceive of this as a complete
language. This language fits Augustine's picture well, and, as \citeA{McGinn97}
points out, not only in the passage Wittgenstein quotes above. Earlier in his
\emph{Confessions} Augustine writes: \begin{quote} Thus, little by little, I
	became conscious where I was; and to have a wish to express my wishes
	to those who could content them, and I could not; for the wishes were
	within me, and they without; nor could they by any sense of theirs
	enter within my spirit. So I flung about at random limbs and voice,
	making the few signs I could, and such as I could, like, though in
	truth very little like, what I wished. \cite[\S 1.6.8]{Pusey09}
\end{quote} So, in both Augustine's conception and in the builder's game, a
primary function of language is to request objects. Additionally, since these
objects correspond to names, to words in the language, it is possible to learn
those names, and thus, learn the language.

Wittgenstein goes on to resist this picture in a number of ways. First, he
makes the obvious assertion that this is not all there is to language as we use
it. Augustine's account, like the builder's game \say{does describe a system of
communication; only not everything that we call language is this system}
\cite[\S~3]{Wittgenstein53}. In \S~8, he expands the language-game to include
more words and concepts, including \say{this} and \say{that} (accompanied by
pointing) as well as analogs of elements from the grocer's game: numerals and
color samples. Immediately, it becomes obvious that Augustine's picture of
learning language cannot easily account for these new words.

Later, he goes on to resist Augustine's understanding of learning language in
addition to his conception of how language functions. If we imagine that the
language of \S~2 is the whole language of a tribe of builders, then children
\say{are brought up to perform \emph{these} actions, to use \emph{these} words
as they do so, and to react in \emph{this} way to the words of others}
\cite[\S~6]{Wittgenstein53}. He asserts that while training in this language
will consist of pointing at building materials and uttering their names, this
ought not be thought of as \say{ostensive definition,} because the child cannot
yet ask the name of the objects. Instead, he would prefer to call it
\say{ostensive teaching of words} because it merely establishes a connection
between word and thing. The understanding of the word does not arise solely
from this ostensive teaching, he argues but also from other training, otherwise
the understanding at which the child arrived would be different. We can imagine
a child growing up ostensively taught a connection betweens slabs and the word
\say{slab,} but not having these children's understanding of \say{Slab!} to
mean \say{bring me a slab.}

In \S~27, Wittgenstein returns to ostensive definition, and the critique to
which he has been building starts to solidify. He points out that the builder's
language has no way of asking something's name. Asking for a name is a language
game that we are brought up to perform, but which might, conceivably, not be
available. Obviously, without learning names, one can still learn quite a lot
about \emph{usage}. In the builder's game, one can learn to produce a slab when
she hears \say{slab,} without it \emph{necessarily} being a name for the thing,
and in \S~31, Wittgenstein shows that one might learn chess without learning
the name of each piece. To ask for a definition of a chess piece, and to be
told \say{this is the king,} tells one nothing about how to use it. Likewise,
knowing how to use it tells us nothing about its name. Wittgenstein says here
that learning the name of something can only tell us its use if \say{the place
for it is already prepared,} which is to say that we already understand to what
the name refers. To be told \say{this is the king} only helps if we already
know what a king \emph{is}.

A child does not know what a king is, nor does she know (how) to ask for
ostensive definitions. If she does not understand the concept already, learning
the name cannot help her make sense of it. Learning the signs for objects, as
Augustine describes, cannot give her the full set of tools she needs to
communicate. She must also have training of some other kind. If Augustine's
account were complete, that would mean that the child could already ask about
ostensive definition and guess about signs' meanings. \citeA{McGinn97} sums it
up well in saying:
\begin{quote}
	Any sense that the account of language acquisition that
Augustine presents somehow explains how we learn language is thus shown to be
an illusion. For the picture actually presupposes what it purports to explain,
by assuming that the child possesses a mastery of the techniques, or ways of
operating with words, that provide the necessary background to his
understanding what is meant when an adult points and utters a sound. (Ch.~2)
\end{quote}

% \citeA{Wittgenstein53} opens his \emph{Philosophical Investigations} with a
%quote from Augustine:\footnote{I am taking the liberty of producing a recent
%translation, rather than the one Anscombe provides.} \begin{quote} When they
%named any thing, and as they spoke turned towards it, I saw and remembered
%that they called what they would point out by the name they uttered. And that
%they meant this thing and no other was plain from the motion of their body,
%the natural language, as it were, of all nations, expressed by the
%countenance, glances of the eye, gestures of the limbs, and tones of the
%voice, indicating the affections of the mind, as it pursues, possesses,
%rejects, or shuns. And thus by constantly hearing words, as they occurred in
%various sentences, I collected gradually for what they stood; and having
%broken in my mouth to these signs, I thereby gave utterance to my will.
%\cite[\S 1.8.13]{Pusey09} \end{quote} Wittgenstein goes on to describe the
%assumptions which he believes underlie this description of language:
%\begin{quote} Every word has a meaning. This meaning is correlated with the
%word. It is the object for which the word stands. \cite[\S 1]{Wittgenstein53}
%\end {quote} According to \citeA{McGinn97}, these assumptions tempt not only
%Augustine, but also Frege, Russell, and even Wittgenstein himself (as
%expressed in the \emph{Tractatus}). Because this conception of language comes
%so naturally to us, she argues, Wittgenstein centers it to critique it and
%examine the temptation to think about language in this way. He does so by
%constructing \say{language-games} to illustrate the flaws in this account of
%language.
% 
% Language-games are small examples of language \emph{in-situ}, which
% Wittgenstein invites us to imagine in order to explore the possible uses of
% language. According to \citeA{McGinn97}, because they are situated in the
% lives of speakers, they are not prone in the same way as Augustine's examples
% to oversimplification. This becomes evident immediately from Wittgenstein's
% first language-game, the grocer's game. In it, he describes the functioning
% of language in ordering five red apples from the grocer. Already, this
% situation of language use in practice leads us in a different direction than
% Augustine.
% 
% \citeA{McGinn97} points out that the passage Wittgenstein initially
% reproduces is not the only one in which Augustine displays these tendencies.
% In fact, earlier in the \emph{Confessions}, they are more overt:
% \begin{quote} Thus, little by little, I became conscious where I was; and to
% have a wish to express my wishes to those who could content them, and I could
% not; for the wishes were within me, and they without; nor could they by any
% sense of theirs enter within my spirit. So I flung about at random limbs and
% voice, making the few signs I could, and such as I could, like, though in
% truth very little like, what I wished. \cite[\S 1.6.8]{Pusey09}
%\end{quote} This passage even more clearly makes the child out to be somehow
%fully capable of having determinate wishes, thoughts, and desires, but simply
%lacking the signs to make them known. As Wittgenstein says, the child "could
%already \emph{think}, only not yet speak" (\S 32). It is also essential that
%the child understand what it is to name an object. Wittgenstein resists this
%assumption clearly in \S 6.
%
% To allow for this picture, the child must already have a language, or at
% least a language-game, internal to itself, in which it can express its
% desires. Then, as it observes others interact with objects, it learns the
% signs which they use, and fits them into its own language-game, thus learning
% how to express its wishes.
%This passage shows a clear picture of Augustine's view of the child. 

% Wittgenstein, to avoid characterizing language or its function incorrectly,
% invents and analyzes \say{language-games}, examples of language
% \emph{in-situ}, to analyze language as it might actually be used, and
% therefore as it actually exists.

% In \S 2, Wittgenstein asks us to imagine a case in which the entire language
% of a people consists in names for sorts of building materials: \say{block,}
% \say{pillar,} \say{slab,} \say{beam.} He argues that the naming of objects to
% a child ought not initially be thought of as \say{ostensive definition,}
% because the child does not yet understand what it is to name (\S 6). He would
% prefer, instead, to call the behaviour of drawing attention to an exemplar
% and uttering its name the \say{ostensive teaching of words.}

\clearpage

\bibliography{wittgenstein}{} \bibliographystyle{apacite} \printindex[autx]

\end{document}
