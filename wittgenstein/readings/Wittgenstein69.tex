% This is a typeset version of the copy of On Certainty found here:
% http://www.edtechpost.ca/readings/Ludwig%20Wittgenstein%20-%20On%20Certainty.html

\documentclass{book}
\usepackage[utf8]{inputenc}
\usepackage{ragged2e}
\usepackage{changepage}
\usepackage{hyperref}
\usepackage{soul}

\title{Uber Gewissheit \\ \emph{On Certainty}}
\author{Ludwig Wittgenstein \vspace{12pt} \\ G. E. M. Anscombe \& G. H. von Wright (eds.) \\ Denis Paul \& G. E. M. Anscombe (trans.)}
\date{1969}

\begin{document}
\frontmatter
\maketitle

\chapter{Preface}

What we publish here belongs to the last year and a half of Wittgenstein's
life. In the middle of 1949 he visited the United States at the invitation of
Norman Malcolm, staying at Malcolm's house in Ithaca. Malcolm acted as a goad
to his interest in Moore's `defence of common sense', that is to say his claim
to know a number of propositions for sure, such as ``Here is one hand, and here
is another'', and ``The earth existed for a long time before my birth'', and
``I have never been far from the earth's surface''. The first of these comes in
Moore's `Proof of the External World'. The two others are in his `Defence of
Common Sense'; Wittgenstein had long been interested in these and had said to
Moore that this was his best article. Moore had agreed. This book contains the
whole of what Wittgenstein wrote on this topic from that time until his death.
It is all first-draft material, which he did not live to excerpt and polish.

The material falls into four parts; we have shown the divisions at \S 65, \S
192, \S 299. What we believe to be the first part was written on twenty loose
sheets of lined foolscap, undated. These Wittgenstein left in his room in
G.E.M. Anscombe's house in Oxford, where he lived (apart from a visit to Norway
in the autumn) from April 1950 to February 1951. I (G.E.M.A.) am under the
impression that he had written them in Vienna, where he stayed from the
previous Christmas until March; but I cannot now recall the basis of this
impression. The rest is in small notebooks, containing dates; towards the end,
indeed, the date of writing is always given. The last entry is two days before
his death on April 29th 1951. We have left the dates exactly as they appear in
the manuscripts. The numbering of the single sections, however, is by the
Editors.

It seemed appropriate to publish this work by itself. It is not a selection;
Wittgenstein marked it off in his notebooks as a separate topic, which he
apparently took up at four separate periods during this eighteen months. It
constitutes a single sustained treatment of the topic.

\mainmatter

\chapter{}

\begin{enumerate}

\item
% 1.
If you do know that here is one hand, we'll grant you all the rest.  When one
says that such and such a proposition can't be proved, of course that does not
mean that it can't be derived from other propositions; any proposition can be
derived from other ones. But they may be no more certain than it is itself. (On
this a curious remark by H.Newman.)

\item
% 2.
From its seeming to me - or to everyone - to be so, it doesn't follow that it
is so.  What we can ask is whether it can make sense to doubt it.

\item
% 3.
If e.g. someone says ``I don't know if there's a hand here'' he might be told
``Look closer''. - This possibility of satisfying oneself is part of the
language-game. Is one of its essential features.

\item
% 4.
``I know that I am a human being.'' In order to see how unclear the sense of
this proposition is, consider its negation. At most it might be taken to mean
``I know I have the organs of a human''. (E.g. a brain which, after all, no one
has ever yet seen.) But what about such a proposition as ``I know I have a
brain''? Can I doubt it? Grounds for doubt are lacking! Everything speaks in
its favour, nothing against it. Nevertheless it is imaginable that my skull
should turn out empty when it was operated on.

\item
% 5.
Whether a proposition can turn out false after all depends on what I make count
as determinants for that proposition.

\item
% 6.
Now, can one enumerate what one knows (like Moore)? Straight off like that, I
believe not. - For otherwise the expression ``I know'' gets misused. And
through this misuse a queer and extremely important mental state seems to be
revealed.

\item
% 7.
My life shows that I know or am certain that there is a chair over there, or a
door, and so on. - I tell a friend e.g. ``Take that chair over there'', ``Shut
the door'', etc. etc.

\item
% 8.
The difference between the concept of `knowing' and the concept of `being
certain' isn't of any great importance at all, except where ``I know'' is meant
to mean: I can't be wrong. In a law-court, for example, ``I am certain'' could
replace ``I know'' in every piece of testimony. We might even imagine its being
forbidden to say ``I know'' there. \emph{A passage in ``Wilhelm Meister'', where
``You know'' or ``You knew'' is used in the sense ``You were certain'', the
facts being different from what he knew.}

\item
% 9.
Now do I, in the course of my life, make sure I know that here is a hand - my
own hand, that is?

\item
% 10.
I know that a sick man is lying here? Nonsense! I am sitting at his bedside, I
am looking attentively into his face. - So I don't know, then, that there is a
sick man lying here? Neither the question nor the assertion makes sense. Any
more than the assertion ``I am here'', which I might yet use at any moment, if
suitable occasion presented itself. - Then is ``2x2=4'' nonsense in the same
way, and not a proposition of arithmetic, apart from particular occasions?
``2x2=4'' is a true proposition of arithmetic - not ``on particular occasions''
nor ``always'' - but the spoken or written sentence ``2x2=4'' in Chinese might
have a different meaning or be out and out nonsense, and from this is seen that
it is only in use that the proposition has its sense. And ``I know that there's
a sick man lying here'', used in an unsuitable situation, seems not to be
nonsense but rather seems matter-of-course, only because one can fairly easily
imagine a situation to fit it, and one thinks that the words ``I know that...''
are always in place where there is no doubt, and hence even where the
expression of doubt would unintelligible.

\item
% 11.
We just do not see how very specialized the use of ``I know'' is.

\item
% 12.
- For ``I know'' seems to describe a state of affairs which guarantees what is
known, guarantees it as a fact. One always forgets the expression ``I thought I
knew''.

\item
% 13.
For it is not as though the proposition ``It is so'' could be inferred from
someone else's utterance: ``I know it is so''. Nor from the utterance together
with its not being a lie. - But can't I infer ``It is so'' from my own
utterance ``I know etc.''? Yes; and also ``There is a hand there'' follows from
the proposition ``He knows that there's a hand there''. But from his utterance
``I know...'' it does not follow that he does know it.

\item
% 14.
That he does know remains to be shown.

\item
% 15.
It needs to be shown that no mistake was possible. Giving the assurance ``I
know'' doesn't suffice. For it is after all only an assurance that I can't be
making a mistake, and it needs to be objectively established that I am not
making a mistake about that.

\item
% 16.
``If I know something, then I also know that I know it, etc.'' amounts to: ``I
know that'' means ``I am incapable of being wrong about that.'' But whether I
am so must admit of being established objectively.

\item
% 17.
Suppose now I say ``I'm incapable of being wrong about this: that is a book''
while I point to an object. What would a mistake here be like? And have I any
clear idea of it?

\item
% 18.
``I know'' often means: I have the proper grounds for my statement. So if the
other person is acquainted with the language-game, he would admit that I know.
The other, if he is acquainted with the language-game, must be able to imagine
how one may know something of the kind.

\item
% 19.
The statement ``I know that here is a hand'' may then be continued: ``for it's
my hand that I'm looking at.'' Then a reasonable man will not doubt that I
know. - Nor will the idealist; rather he will say that he was not dealing with
the practical doubt which is being dismissed, but there is a further doubt
behind that one. - That this is an illusion has to be shown in a different way.

\item
% 20.
``Doubting the existence of the external world'' does not mean for example
doubting the existence of a planet, which later observations proved to exist. -
Or does Moore want to say that knowing that here is his hand is different in
kind from knowing the existence of the planet Saturn? Otherwise it would be
possible to point out the discovery of the planet Saturn to the doubters and
say that its existence has been proved, and hence the existence of the external
world as well.

\item
% 21.
Moore's view really comes down to this: the concept `know' is analogous to the
concepts `believe', `surmise', `doubt', `be convinced' in that the statement
``I know...'' can't be a mistake. And if that is so, then there can be an
inference from such an utterance to the truth of an assertion. And here the
form ``I thought I knew'' is being overlooked. - But if this latter is
inadmissible, then a mistake in the assertion must be logically impossible too.
And anyone who is acquainted with the language-game must realize this - an
assurance from a reliable man that he knows cannot contribute anything.

\item
% 22.
It would surely be remarkable if we had to believe the reliable person who says
``I can't be wrong''; or who says ``I am not wrong''.

\item
% 23.
If I don't know whether someone has two hands (say, whether they have been
amputated or not) I shall believe his assurance that he has two hands, if he is
trustworthy. And if he says he knows it, that can only signify to me that he
has been able to make sure, and hence that his arms are e.g. not still
concealed by coverings and bandages, etc.etc. My believing the trustworthy man
stems from my admitting that it is possible for him to make sure. But someone
who says that perhaps there are no physical objects makes no such admission.

\item
% 24.
The idealist's question would be something like: ``What right have I not to
doubt the existence of my hands?'' (And to that the answer can't be: I know
that they exist.) But someone who asks such a question is overlooking the fact
that a doubt about existence only works in a language-game. Hence, that we
should first have to ask: what would such a doubt be like?, and don't
understand this straight off.

\item
% 25.
One may be wrong even about ``there being a hand here''. Only in particular
circumstances is it impossible. - ``Even in a calculation one can be wrong -
only in certain circumstances one can't.''

\item
% 26.
But can it be seen from a rule what circumstances logically exclude a mistake
in the employment of rules of calculation?  What use is a rule to us here?
Mightn't we (in turn) go wrong in applying it?

\item
% 27.
If, however, one wanted to give something like a rule here, then it would
contain the expression ``in normal circumstances''. And we recognize normal
circumstances but cannot precisely describe them. At most, we can describe a
range of abnormal ones.

\item
% 28.
What is `learning a rule'? - This.  What is `making a mistake in applying it'?
- This. And what is pointed to here is something indeterminate.

\item
% 29.
Practice in the use of the rule also shows what is a mistake in its employment.

\item
% 30.
When someone has made sure of something, he says: ``Yes, the calculation is
right'', but he did not infer that from his condition of certainty. One does
not infer how things are from one's own certainty.  Certainty is as it were a
tone of voice in which one declares how things are, but one does not infer from
the tone of voice that one is justified.

\item
% 31.
The propositions which one comes back to again and again as if bewitched -
these I should like to expunge from philosophical language.

\item
% 32.
It's not a matter of Moore's knowing that there's a hand there, but rather we
should not understand him if he were to say ``Of course I may be wrong about
this.'' We should ask ``What is it like to make such a mistake as that?'' -
e.g. what's it like to discover that it was a mistake?

\item
% 33.
Thus we expunge the sentences that don't get us any further.

\item
% 34.
If someone is taught to calculate, is he also taught that he can rely on a
calculation of his teacher's? But these explanations must after all sometime
come to an end. Will he also be taught that he can trust his senses - since he
is indeed told in many cases that in such and such a special case you cannot
trust them? - Rule and exception.

\item
% 35.
But can't it be imagined that there should be no physical objects? I don't
know. And yet ``There are physical objects'' is nonsense. Is it supposed to be
an empirical proposition? - And is this an empirical proposition: ``There seem
to be physical objects''?

\item
% 36.
``A is a physical object'' is a piece of instruction which we give only to
someone who doesn't yet understand either what ``A'' means, or what ``physical
object'' means. Thus it is instruction about the use of words, and ``physical
object'' is a logical concept. (Like colour, quantity,...) And that is why no
such proposition as: ``There are physical objects'' can be formulated.  Yet we
encounter such unsuccessful shots at every turn.

\item
% 37.
But is it adequate to answer to the scepticism of the idealist, or the
assurances of the realist, to say that ``There are physical objects'' is
nonsense? For them after all it is not nonsense. It would, however, be an
answer to say: this assertion, or its opposite is a misfiring attempt to
express what can't be expressed like that. And that it does misfire can be
shown; but that isn't the end of the matter. We need to realize that what
presents itself to us as the first expression of a difficulty, or of its
solution, may as yet not be correctly expressed at all. Just as one who has a
just censure of a picture to make will often at first offer the censure where
it does not belong, and an investigation is needed in order to find the right
point of attack for the critic.

\item
% 38.
Knowledge in mathematics: Here one has to keep on reminding oneself of the
unimportance of the `inner process' or `state' and ask ``Why should it be
important? What does it matter to me?'' What is interesting is how we use
mathematical propositions.

\item
% 39.
This is how calculation is done, in such circumstances a calculation is treated
as absolutely reliable, as certainly correct.

\item
% 40.
Upon ``I know that there is my hand'' there may follow the question ``How do
you know?'' and the answer to that presupposes that this can be known in that
way. So, instead of ``I know that here is my hand'', one might say ``Here is my
hand'', and then add how one knows.

\item
% 41.
``I know where I am feeling pain'', ``I know that I feel it here'' is as wrong
as ``I know that I am in pain''. But ``I know where you touched my arm'' is
right.

\item
% 42.
One can say ``He believes it, but it isn't so'', but not ``He knows it, but it
isn't so''. Does this stem from the difference between the mental states of
belief and knowledge? No. - One may for example call ``mental state'' what is
expressed by tone of voice in speaking, by gestures etc. It would thus be
possible to speak of a mental state of conviction, and that may be the same
whether it is knowledge or false belief. To think that different states must
correspond to the words ``believe'' and ``know'' would be as if one believed
that different people had to correspond to the word ``I'' and the name
``Ludwig'', because the concepts are different.

\item
% 43.
What sort of proposition is this: ``We cannot have miscalculated in
12x12=144''? It must surely be a proposition of logic. - But now, is it not the
same, or doesn't it come to the same, as the statement 12x12=144?

\item
% 44.
If you demand a rule from which it follows that there can't have been a
miscalculation here, the answer is that we did not learn this through a rule,
but by learning to calculate.

\item
% 45.
We got to know the nature of calculating by learning to calculate.

\item
% 46.
But then can't it be described how we satisfy ourselves of the reliability of a
calculation? O yes! Yet no rule emerges when we do so. - But the most important
thing is: The rule is not needed. Nothing is lacking. We do calculate according
to a rule, and that is enough.

\item
% 47.
This is how one calculates. Calculating is this. What we learn at school, for
example. Forget this transcendent certainty, which is connected with your
concept of spirit.

\item
% 48.
However, out of a host of calculations certain ones might be designated as
reliable once for all, others as not yet fixed. And now, is this a logical
distinction?

\item
% 49.
But remember: even when the calculation is something fixed for me, this is only
a decision for a practical purpose.

\item
% 50.
When does one say, I know that ... x ... = ....? When one has checked the
calculation.

\item
% 51.
What sort of proposition is: ``What could a mistake here be like?'' It would
have to be a logical proposition. But is it a logic that is not used, because
what it tells us is not taught by means of propositions. - It is a logical
proposition; for it does describe the conceptual (linguistic) situation.

\item
% 52.
This situation is thus not the same for a proposition like ``At this distance
from the sun there is a planet'' and ``Here is a hand'' (namely my own hand).
The second can't be called a hypothesis. But there isn't a sharp boundary line
between them.

\item
% 53.
So one might grant that Moore was right, if he is interpreted like this: a
proposition saying that here is a physical object may have the same logical
status as one saying that here is a red patch.

\item
% 54.
For it is not true that a mistake merely gets more and more improbable as we
pass from the planet to my own hand. No: at some point it has ceased to be
conceivable.  This is already suggested by the following: if it were not so, it
would also be conceivable that we should be wrong in every statement about
physical objects; that any we ever make are mistaken.

\item
% 55.
So is the hypothesis possible, that all the things around us don't exist? Would
that not be like the hypothesis of our having miscalculated in all our
calculations?

\item
% 56.
When one says: ``Perhaps this planet doesn't exist and the light-phenomenon
arises in some other way'', then after all one needs an example of an object
which does exist. This doesn't exist, - as for example does...  Or are we to
say that certainty is merely a constructed point to which some things
approximate more, some less closely? No. Doubt gradually loses its sense. This
language-game just is like that.  And everything descriptive of a language-game
is part of logic.

\item
% 57.
Now might not ``I know, I am not just surmising, that here is my hand'' be
conceived as a proposition of grammar? Hence not temporally. - But in that case
isn't it like this one: ``I know, I am not just surmising, that I am seeing
red''?  And isn't the consequence ``So there are physical objects'' like: ``So
there are colours''?

\item
% 58.
If ``I know etc'' is conceived as a grammatical proposition, of course the
``I'' cannot be important. And it properly means ``There is no such thing as a
doubt in this case'' or ``The expression `I do not know' makes no sense in this
case''. And of course it follows from this that ``I know'' makes no sense
either.

\item
% 59.
``I know'' is here a logical insight. Only realism can't be proved by means of
it.

\item
% 60.
It is wrong to say that the `hypothesis' that this is a bit of paper would be
confirmed or disconfirmed by later experience, and that, in ``I know that this
is a bit of paper'', the ``I know'' either relates to such an hypothesis or to
a logical determination.

\item
% 61.
...A meaning of a word is a kind of employment of it.  For it is what we learn
when the word is incorporated into our language.

\item
% 62.
That is why there exists a correspondence between the concepts `rule' and
`meaning'.

\item
% 63.
If we imagine the facts otherwise than as they are, certain language-games lose
some of their importance, while others become important. And in this way there
is an alteration - a gradual one - in the use of the vocabulary of a language.

\item
% 64.
Compare the meaning of a word with the `function' of an official. And
`different meanings' with `different functions'.

\item
% 65.
When language-games change, then there is a change in concepts, and with the
concepts the meanings of words change.

\chapter{}

\item
% 66.
I make assertions about reality, assertions which have different degrees of
assurance. How does the degree of assurance come out? What consequences has it?
We may be dealing, for example, with the certainty of memory, or again of
perception. I may be sure of something, but still know what test might convince
me of error. I am e.g. quite sure of the date of a battle, but if I should find
a different date in a recognized work of history, I should alter my opinion,
and this would not mean I lost all faith in judging.

\item
% 67.
Could we imagine a man who keeps on making mistakes where we regard a mistake
as ruled out, and in fact never encounter one?  E.g. he says he lives in such
and such a place, is so and so old, comes from such and such a city, and he
speaks with the same certainty (giving all the tokens of it) as I do, but he is
wrong.  But what is his relation to this error? What am I to suppose?

\item
% 68.
The question is: what is the logician to say here?

\item
% 69.
I should like to say: ``If I am wrong about this, I have no guarantee that
anything I say is true.'' But others won't say that about me, nor will I say it
about other people.

\item
% 70.
For months I have lived at address A, I have read the name of the street and
the number of the house countless times, have received countless letters here
and given countless people the address. If I am wrong about it, the mistake is
hardly less that if I were (wrongly) to believe I was writing Chinese and not
German.

\item
% 71.
If my friend were to imagine one day that he had been living for a long time
past in such and such a place, etc.etc., I should not call this a mistake, but
rather a mental disturbance, perhaps a transient one.

\item
% 72.
Not every false belief of this sort is a mistake.

\item
% 73.
But what is the difference between mistake and mental disturbance? Or what is
the difference between my treating it as a mistake and my treating it as mental
disturbance?

\item
% 74.
Can we say: a mistake doesn't only have a cause, it also has a ground? I.e.,
roughly: when someone makes a mistake, this can be fitted into what he knows
aright.

\item
% 75.
Would this be correct: If I merely believed wrongly that there is a table here
in front of me, this might still be a mistake; but if I believe wrongly that I
have seen this table, or one like it, every day for several months past, and
have regularly used it, that isn't a mistake?

\item
% 76.
Naturally, my aim must be to give the statements that one would like to make
here, but cannot make significantly.

\item
% 77.
Perhaps I shall do a multiplication twice to make sure, or perhaps get someone
else to work it over. But shall I work it over again twenty times, or get
twenty people to go over it? And is that some sort of negligence? Would the
certainty really be greater for being checked twenty times?

\item
% 78.
And can I give a reason why it isn't?

\item
% 79.
That I am a man and not a woman can be verified, but if I were to say I was a
woman, and then tried to explain the error by saying I hadn't checked the
statement, the explanation would not be accepted.

\item
% 80.
The truth of my statements is the test of my understanding of these statements.

\item
% 81.
That is to say: if I make certain false statements, it becomes uncertain
whether I understand them.

\item
% 82.
What counts as an adequate test of a statement belongs to logic. It belongs to
the description of the language-game.

\item
% 83.
The truth of certain empirical propositions belongs to our frame of reference.

\item
% 84.
Moore says he knows that the earth existed long before his birth. And put like
that it seems to be a personal statement about him, even if it is in addition a
statement about the physical world. Now it is philosophically uninteresting
whether Moore knows this or that, but it is interesting that, and how, it can
be known. If Moore had informed us that he knew the distance separating certain
stars, we might conclude from that that he had made some special
investigations, and we shall want to know what these were. But Moore chooses
precisely a case in which we all seem to know the same as he, and without being
able to say how. I believe e.g. that I know as much about this matter (the
existence of the earth) as Moore does, and if he knows that it is as he says,
then I know it too. For it isn't, either, as if he had arrived at this
proposition by pursuing some line of thought which, while it is open to me, I
have not in fact pursued.

\item
% 85.
And what goes into someone's knowing this? Knowledge of history, say? He must
know what it means to say: the earth has already existed for such and such a
length of time. For not any intelligent adult must know that. We see men
building and demolishing houses, and are led to ask:``How long has this house
been here?'' But how does one come on the idea of asking this about a mountain,
for example? And have all men the notion of the earth as a body, which may come
into being and pass away? Why shouldn't I think of the earth as flat, but
extending without end in every direction (including depth)? But in that case
one might still say ``I know that this mountain existed long before my birth.''
- But suppose I met a man who didn't believe that?

\item
% 86.
Suppose I replaced Moore's ``I know'' by ``I am of the unshakeable
conviction''?

\item
% 87.
Can't an assertoric sentence, which was capable of functioning as an
hypothesis, also be used as a foundation for research and action? I.e. can't it
simply be isolated from doubt, though not according to any explicit rule? It
simply gets assumed as a truism, never called in question, perhaps not even
ever formulated.

\item
% 88.
It may be for example that all enquiry on our part is set so as to exempt
certain propositions from doubt, if they were ever formulated. They lie apart
from the route travelled by enquiry.

\item
% 89.
One would like to say: ``Everything speaks for, and nothing against the earth's
having existed long before...'' Yet might I not believe the contrary after all?
But the question is: What would the practical effects of this belief be? -
Perhaps someone says: ``That's not the point. A belief is what it is whether it
has any practical effects or not.'' One thinks: It is the same adjustment of
the human mind anyway.

\item
% 90.
``I know'' has a primitive meaning similar to and related to ``I see''
(``wissen'', ``videre''). And ``I knew he was in the room, but he wasn't in the
room'' is like ``I saw him in the room, but he wasn't there''. ``I know'' is
supposed to express a relation, not between me and the sense of a proposition
(like ``I believe'') but between me and a fact. So that the fact is taken into
my consciousness. (Here is the reason why one wants to say that nothing that
goes on in the outer world is really known, but only what happens in the domain
of what are called sense-data.) This would give us a picture of knowing as the
perception of an outer event through visual rays which project it as it is into
the eye and the consciousness. Only then the question at once arises whether
one can be certain of this projection. And this picture does indeed show how
our imagination presents knowledge, but not what lies at the bottom of this
presentation.

\item
% 91.
If Moore says he knows the earth existed etc., most of us will grant him that
it has existed all that time, and also believe him when he says he is convinced
of it. But has he also got the right ground for this conviction? For if not,
then after all he doesn't know (Russell).

\item
% 92.
However, we can ask: May someone have telling grounds for believing that the
earth has only existed for a short time, say since his own birth? - Suppose he
had always been told that, - would he have any good reason to doubt it? Men
have believed that they could make the rain; why should not a king be brought
up in the belief that the world began with him? And if Moore and this king were
to meet and discuss, could Moore really prove his belief to be the right one? I
do not say that Moore could not convert the king to his view, but it would be a
conversion of a special kind; the king would be brought to look at the world in
a different way.  Remember that one is sometimes convinced of the correctness
of a view by its simplicity or symmetry, i.e., these are what induce one to go
over to this point of view. One then simply says something like: ``That's how
it must be.''

\item
% 93.
The propositions presenting what Moore `knows' are all of such a kind that it
is difficult to imagine why anyone should believe the contrary. E.g. the
proposition that Moore has spent his whole life in close proximity to the
earth. - Once more I can speak of myself here instead of speaking of Moore.
What could induce me to believe the opposite? Either a memory, or having been
told. - Everything that I have seen or heard gives me the conviction that no
man has ever been far from the earth. Nothing in my picture of the world speaks
in favour of the opposite.

\item
% 94.
But I did not get my picture of the world by satisfying myself of its
correctness; nor do I have it because I am satisfied of its correctness. No: it
is the inherited background against which I distinguish between true and false.

\item
% 95.
The propositions describing this world-picture might be part of a kind of
mythology. And their role is like that of rules of a game; and the game can be
learned purely practically, without learning any explicit rules.

\item
% 96.
It might be imagined that some propositions, of the form of empirical
propositions, were hardened and functioned as channels for such empirical
propositions as were not hardened but fluid; and that this relation altered
with time, in that fluid propositions hardened, and hard ones became fluid.

\item
% 97.
The mythology may change back into a state of flux, the river-bed of thoughts
may shift. But I distinguish between the movement of the waters on the
river-bed and the shift of the bed itself; though there is not a sharp division
of the one from the other.

\item
% 98.
But if someone were to say ``So logic too is an empirical science'' he would be
wrong. Yet this is right: the same proposition may get treated at one time as
something to test by experience, at another as a rule of testing.

\item
% 99.
And the bank of that river consists partly of hard rock, subject to no
alteration or only to an imperceptible one, partly of sand, which now in one
place now in another gets washed away, or deposited.

\item
% 100.
The truths which Moore says he knows, are such as, roughly speaking, all of us
know, if he knows them.

\item
% 101.
Such a proposition might be e.g. ``My body has never disappeared and reappeared
again after an interval.''

\item
% 102.
Might I not believe that once, without knowing it, perhaps is a state of
unconsciousness, I was taken far away from the earth - that other people even
know this, but do not mention it to me? But this would not fit into the rest of
my convictions at all. Not that I could describe the system of these
convictions. Yet my convictions do form a system, a structure.

\item
% 103.
And now if I were to say ``It is my unshakeable conviction that etc.'', this
means in the present case too that I have not consciously arrived at the
conviction by following a particular line of thought, but that it is anchored
in all my questions and answers, so anchored that I cannot touch it.

\item
% 104.
I am for example also convinced that the sun is not a hole in the vault of
heaven.

\item
% 105.
All testing, all confirmation and disconfirmation of a hypothesis takes place
already within a system. And this system is not a more or less arbitrary and
doubtful point of departure for all our arguments: no, it belongs to the
essence of what we call an argument. The system is not so much the point of
departure, as the element in which arguments have their life.

\item
% 106.
Suppose some adult had told a child that he had been on the moon. The child
tells me the story, and I say it was only a joke, the man hadn't been on the
moon; no one has ever been on the moon; the moon is a long way off and it is
impossible to climb up there or fly there. - If now the child insists, saying
perhaps there is a way of getting there which I don't know, etc. what reply
could I make to him? What reply could I make to the adults of a tribe who
believe that people sometimes go to the moon (perhaps that is how they
interpret their dreams), and who indeed grant that there are no ordinary means
of climbing up to it or flying there? - But a child will not ordinarily stick
to such a belief and will soon be convinced by what we tell him seriously.

\item
% 107.
Isn't this altogether like the way one can instruct a child to believe in a
God, or that none exists, and it will accordingly be able to produce apparently
telling grounds for the one or the other?

\item
% 108.
``But is there then no objective truth? Isn't it true, or false, that someone
has been on the moon?'' If we are thinking within our system, then it is
certain that no one has ever been on the moon. Not merely is nothing of the
sort ever seriously reported to us by reasonable people, but our whole system
of physics forbids us to believe it. For this demands answers to the questions
``How did he overcome the force of gravity?'' ``How could he live without an
atmosphere?'' and a thousand others which could not be answered. But suppose
that instead of all these answers we met the reply: ``We don't know how one
gets to the moon, but those who get there know at once that they are there; and
even you can't explain everything.'' We should feel ourselves intellectually
very distant from someone who said this.

\item
% 109.
``An empirical proposition can be tested'' (we say). But how? and through what?

\item
% 110.
What counts as its test? - ``But is this an adequate test? And, if so, must it
not be recognizable as such in logic?'' - As if giving grounds did not come to
an end sometime. But the end is not an ungrounded presupposition: it is an
ungrounded way of acting.

\item
% 111.
``I know that I have never been on the moon.'' That sounds different in the
circumstances which actually hold, to the way it would sound if a good many men
had been on the moon, and some perhaps without knowing it. In this case one
could give grounds for this knowledge. Is there not a relationship here similar
to that between the general rule of multiplying and particular multiplications
that have been carried out?  I want to say: my not having been on the moon is
as sure a thing for me as any grounds I could give for it.

\item
% 112.
And isn't that what Moore wants to say, when he says he knows all these things?
- But is his knowing it really what is in question, and not rather that some of
these propositions must be solid for us?

\item
% 113.
When someone is trying to teach us mathematics, he will not begin by assuring
us that he knows that a+b=b+a.

\item
% 114.
If you are not certain of any fact, you cannot be certain of the meaning of
your words either.

\item
% 115.
If you tried to doubt everything you would not get as far as doubting anything.
The game of doubting itself presupposes certainty.

\item
% 116.
Instead of ``I know...'', couldn't Moore have said: ``It stands fast for me
that...''? And further: ``It stands fast for me and many others...''

\item
% 117.
Why is it not possible for me to doubt that I have never been on the moon? And
how could I try to doubt it?  First and foremost, the supposition that perhaps
I have been there would strike me as idle. Nothing would follow from it,
nothing be explained by it. It would not tie in with anything in my life.  When
I say ``Nothing speaks for, everything against it,'' this presupposes a
principle of speaking for and against. That is, I must be able to say what
would speak for it.

\item
% 118.
Now would it be correct to say: So far no one has opened my skull in order to
see whether there is a brain inside; but everything speaks for, and nothing
against, its being what they would find there?

\item
% 119.
But can it also be said: Everything speaks for, and nothing against the table's
still being there when no one sees it? For what does speak of it?

\item
% 120.
But if anyone were to doubt it, how would his doubt come out in practice? And
couldn't we peacefully leave him to doubt it, since it makes no difference at
all?

\item
% 121.
Can one say: ``Where there is no doubt there is no knowledge either''?

\item
% 122.
Doesn't one need grounds for doubt?

\item
% 123.
Wherever I look, I find no ground for doubting that...

\item
% 124.
I want to say: We use judgments as principles of judgment.

\item
% 125.
If a blind man were to ask me ``Have you got two hands?'' I should not make
sure by looking. If I were to have any doubt of it, then I don't know why I
should trust my eyes. For why shouldn't I test my eyes by looking to find out
whether I see my two hands? What is to be tested by what? (Who decides what
stands fast?) And what does it mean to say that such and such stands fast?

\item
% 126.
I am not more certain of the meaning of my words that I am of certain
judgments. Can I doubt that this colour is called ``blue''?  (My) doubts form a
system.

\item
% 127.
For how do I know that someone is in doubt? How do I know that he uses the
words ``I doubt it'' as I do?

\item
% 128.
From a child up I learnt to judge like this. This is judging.

\item
% 129.
This is how I learned to judge; this I got to know as judgment.

\item
% 130.
But isn't it experience that teaches us to judge like this, that is to say,
that it is correct to judge like this? But how does experience teach us, then?
We may derive it from experience, but experience does not direct us to derive
anything from experience. If it is the ground for our judging like this, and
not just the cause, still we do not have a ground for seeing this in turn as a
ground.

\item
% 131.
No, experience is not the ground for our game of judging. Nor is its
outstanding success.

\item
% 132.
Men have judged that a king can make rain; we say this contradicts all
experience. Today they judge that aeroplanes and the radio etc. are means for
the closer contact of peoples and the spread of culture.

\item
% 133.
Under ordinary circumstances I do not satisfy myself that I have two hands by
seeing how it looks. Why not? Has experience shown it to be unnecessary? Or
(again): Have we in some way learnt a universal law of induction, and do we
trust it here too? - But why should we have learnt one universal law first, and
not the special one straight away?

\item
% 134.
After putting a book in a drawer, I assume it is there, unless... ``Experience
always proves me right. There is no well attested case of a book's (simply)
disappearing.'' It has often happened that a book has never turned up again,
although we thought we knew for certain where it was. - But experience does
really teach that a book, say, does not vanish away. (E.g. gradually
evaporates.) But is it this experience with books etc. that leads us to assume
that such a book has not vanished away? Well, suppose we were to find that
under particular novel circumstances books did vanish away. - Shouldn't we
alter our assumption? Can one give the lie to the effect of experience on our
system of assumption?

\item
% 135.
But do we not simply follow the principle that what has always happened will
happen again (or something like it)? What does it mean to follow this
principle? Do we really introduce it into our reasoning? Or is it merely the
natural law which our inferring apparently follows? This latter it may be. It
is not an item in our considerations.

\item
% 136.
When Moore says he knows such and such, he is really enumerating a lot of
empirical propositions which we affirm without special testing; propositions,
that is, which have a peculiar logical role in the system of our empirical
propositions.

\item
% 137.
Even if the most trustworthy of men assures me that he knows things are thus
and so, this by itself cannot satisfy me that he does know. Only that he
believes he knows. That is why Moore's assurance that he knows... does not
interest us. The propositions, however, which Moore retails as examples of such
known truths are indeed interesting. Not because anyone knows their truth, or
believes he knows them, but because they all have a similar role in the system
of our empirical judgments.

\item
% 138.
We don't, for example, arrive at any of them as a result of investigation.
There are e.g. historical investigations and investigations into the shape and
also the age of the earth, but not into whether the earth has existed during
the last hundred years. Of course many of us have information about this period
from our parents and grandparents; but maynt' they be wrong? - ``Nonsense!''
one will say. ``How should all these people be wrong?'' - But is that an
argument? Is it not simply the rejection of an idea? And perhaps the
determination of a concept? For if I speak of a possible mistake here, this
changes the role of ``mistake'' and ``truth'' in our lives.

\item
% 139.
Not only rules, but also examples are needed for establishing a practice. Our
rules leave loop-holes open, and the practice has to speak for itself.

\item
% 140.
We do not learn the practice of making empirical judgments by learning rules:
we are taught judgments and their connexion with other judgments. A totality of
judgments is made plausible to us.

\item
% 141.
When we first begin to believe anything, what we believe is not a single
proposition, it is a whole system of propositions. (Light dawns gradually over
the whole.)

\item
% 142.
It is not single axioms that strike me as obvious, it is a system in which
consequences and premises give one another mutual support.

\item
% 143.
I am told, for example, that someone climbed this mountain many years ago. Do I
always enquire into the reliability of the teller of this story, and whether
the mountain did exist years ago? A child learns there are reliable and
unreliable informants much later than it learns facts which are told it. It
doesn't learn at all that that mountain has existed for a long time: that is,
the question whether it is so doesn't arise at all. It swallows this
consequence down, so to speak, together with what it learns.

\item
% 144.
The child learns to believe a host of things. I.e. it learns to act according
to these beliefs. Bit by bit there forms a system of what is believed, and in
that system some things stand unshakeably fast and some are more or less liable
to shift. What stands fast does so, not because it is intrinsically obvious or
convincing; it is rather held fast by what lies around it.

\item
% 145.
One wants to say ``All my experiences show that it is so''. But how do they do
that? For that proposition to which they point itself belongs to a particular
interpretation of them.  ``That I regard this proposition as certainly true
also characterizes my interpretation of experience.''

\item
% 146.
We form the picture of the earth as a ball floating free in space and not
altering essentially in a hundred years. I said ``We form the picture etc.''
and this picture now helps us in the judgment of various situations.  I may
indeed calculate the dimensions of a bridge, sometimes calculate that here
things are more in favour of a bridge than a ferry, etc.etc., - but somewhere I
must begin with an assumption or a decision.

\item
% 147.
The picture of the earth as a ball is a good picture, it proves itself
everywhere, it is also a simple picture - in short, we work with it without
doubting it.

\item
% 148.
Why do I not satisfy myself that I have two feet when I want to get up from a
chair? There is no why. I simply don't. This is how I act.

\item
% 149.
My judgments themselves characterize the way I judge, characterize the nature
of judgment.

\item
% 150.
How does someone judge which is his right and which his left hand? How do I
know that my judgment will agree with someone else's? How do I know that this
colour is blue? If I don't trust myself here, why should I trust anyone else's
judgment? Is there a why? Must I not begin to trust somewhere? That is to say:
somewhere I must begin with not-doubting; and that is not, so to speak, hasty
but excusable: it is part of judging.

\item
% 151.
I should like to say: Moore does not know what he asserts he knows, but it
stands fast for him, as also for me; regarding it as absolutely solid is part
of our method of doubt and enquiry.

\item
% 152.
I do not explicitly learn the propositions that stand fast for me. I can
discover them subsequently like the axis around which a body rotates. This axis
is not fixed in the sense that anything holds it fast, but the movement around
it determines its immobility.

\item
% 153.
No one ever taught me that my hands don't disappear when I am not paying
attention to them. Nor can I be said to presuppose the truth of this
proposition in my assertions etc., (as if they rested on it) while it only gets
sense from the rest of our procedure of asserting.

\item
% 154.
There are cases such that, if someone gives signs of doubt where we do not
doubt, we cannot confidently understand his signs as signs of doubt.  I.e.: if
we are to understand his signs of doubt as such, he may give them only in
particular cases and may not give them in others.

\item
% 155.
In certain circumstance a man cannot make a mistake. (``Can'' is here used
logically, and the proposition does not mean that a man cannot say anything
false in those circumstances.) If Moore were to pronounce the opposite of those
propositions which he declares certain, we should not just not share his
opinion: we should regard him as demented.

\item
% 156.
In order to make a mistake, a man must already judge in conformity with
mankind.

\item
% 157.
Suppose a man could not remember whether he had always had five fingers or two
hands? Should we understand him? Could we be sure of understanding him?

\item
% 158.
Can I be making a mistake, for example, in thinking that the words of which
this sentence is composed are English words whose meaning I know?

\item
% 159.
As children we learn facts; e.g., that every human being has a brain, and we
take them on trust. I believe that there is an island, Australia, of
such-and-such a shape, and so on and so on; I believe that I had
great-grandparents, that the people who gave themselves out as my parents
really were my parents, etc. This belief may never have been expressed; even
the thought that it was so, never thought.

\item
% 160.
The child learns by believing the adult. Doubt comes after belief.

\item
% 161.
I learned an enormous amount and accepted it on human authority, and then I
found some things confirmed or disconfirmed by my own experience.

\item
% 162.
In general I take as true what is found in text-books, of geography for
example. Why? I say: All these facts have been confirmed a hundred times over.
But how do I know that? What is my evidence for it? I have a world-picture. Is
it true or false? Above all it is the substratum of all my enquiring and
asserting. The propositions describing it are not all equally subject to
testing.

\item
% 163.
Does anyone ever test whether this table remains in existence when no one is
paying attention to it?  We check the story of Napoleon, but not whether all
the reports about him are based on sense-deception, forgery and the like. For
whenever we test anything, we are already presupposing something that is not
tested. Now am I to say that the experiment which perhaps I make in order to
test the truth of a proposition presupposes the truth of the proposition that
the apparatus I believe I see is really there (and the like)?

\item
% 164.
Doesn't testing come to an end?

\item
% 165.
One child might say to another: ``I know that the earth is already hundred of
years old'' and that would mean: I have learnt it.

\item
% 166.
The difficulty is to realize the groundlessness of our believing.

\item
% 167.
It is clear that our empirical propositions do not all have the same status,
since one can lay down such a proposition and turn it from an empirical
proposition into a norm of description.  Think of chemical investigations.
Lavoisier makes experiments with substances in his laboratory and now he
concludes that this and that takes place when there is burning. He does not say
that it might happen otherwise another time. He has got hold of a definite
world-picture - not of course one that he invented: he learned it as a child. I
say world-picture and not hypothesis, because it is the matter-of-course
foundation for his research and as such also does unmentioned.

\item
% 168.
But now, what part is played by the presupposition that a substance A always
reacts to a substance B in the same way, given the same circumstances? Or is
that part of the definition of a substance?

\item
% 169.
One might think that there were propositions declaring that chemistry is
possible. And these would be propositions of a natural science. For what should
they be supported by, if not by experience?

\item
% 170.
I believe what people transmit to me in a certain manner. In this way I believe
geographical, chemical, historical facts etc. That is how I learn the sciences.
Of course learning is based on believing.  If you have learnt that Mont Blanc
is 4000 metres high, if you have looked it up on the map, you say you know it.
And can it now be said: we accord credence in this way because it has proved to
pay?

\item
% 171.
A principle ground for Moore to assume that he never was on the moon is that no
one ever was on the moon or could come there; and this we believe on grounds of
what we learn.

\item
% 172.
Perhaps someone says ``There must be some basic principle on which we accord
credence'', but what can such a principle accomplish? Is it more than a natural
law of `taking for true'?

\item
% 173.
Is it maybe in my power what I believe? or what I unshakeably believe?  I
believe that there is a chair over there. Can't I be wrong? But, can I believe
that I am wrong? Or can I so much as bring it under consideration? - And
mightn't I also hold fast to my belief whatever I learned later on?! But is my
belief then grounded?

\item
% 174.
I act with complete certainty. But this certainty is my own.

\item
% 175.
``I know it'' I say to someone else; and here there is a justification. But
there is none for my belief.

\item
% 176.
Instead of ``I know it'' one may say in some cases ``That's how it is - rely
upon it.'' In some cases, however ``I learned it years and years ago''; and
sometimes: ``I am sure it is so.''

\item
% 177.
What I know, I believe.

\item
% 178.
The wrong use made by Moore of the proposition ``I know...'' lies in his
regarding it as an utterance as little subject to doubt as ``I am in pain''.
And since from ``I know it is so'' there follows ``It is so'', then the latter
can't be doubted either.

\item
% 179.
It would be correct to say: ``I believe...'' has subjective truth; but ``I
know...'' not.

\item
% 180.
Or again ``I believe...'' is an `expression', but not ``I know...''.

\item
% 181.
Suppose Moore had said ``I swear...'' instead of ``I know...''.

\item
% 182.
The more primitive idea is that the earth never had a beginning. No child has
reason to ask himself how long the earth has existed, because all change takes
place on it. If what is called the earth really came into existence at some
time - which is hard enough to picture - then one naturally assumes the
beginning as having been an inconceivably long time ago.

\item
% 183.
``It is certain that after the battle of Austerlitz Napoleon... Well, in that
case it's surely also certain that the earth existed then.''

\item
% 184.
``It is certain that we didn't arrive on this planet from another one a hundred
years ago.'' Well, it's as certain as such things are.

\item
% 185.
It would strike me as ridiculous to want to doubt the existence of Napoleon;
but if someone doubted the existence of the earth 150 years ago, perhaps I
should be more willing to listen, for now he is doubting our whole system of
evidence. It does not strike me as if this system were more certain than a
certainty within it.

\item
% 186.
``I might suppose that Napoleon never existed and is a fable, but not that the
earth did not exist 150 years ago.''

\item
% 187.
``Do you know that the earth existed then?'' - ``Of course I know that. I have
it from someone who certainly knows all about it.''

\item
% 188.
It strikes me as if someone who doubts the existence of the earth at that time
is impugning the nature of all historical evidence. And I cannot say of this
latter that it is definitely correct.

\item
% 189.
At some point one has to pass from explanation to mere description.

\item
% 190.
What we call historical evidence points to the existence of the earth a long
time before my birth; - the opposite hypothesis has nothing on its side.

\item
% 191.
Well, if everything speaks for an hypothesis and nothing against it - is it
then certainly true? One may designate it as such. - But does it certainly
agree with reality, with the facts? - With this question you are already going
round in a circle.

\item
% 192.
To be sure there is justification; but justification comes to an end.

\chapter{}

\item
% 193.
What does this mean: the truth of a proposition is a certain?

\item
% 194.
With the word ``certain'' we express complete conviction, the total absence of
doubt, and thereby we seek to convince other people. That is subjective
certainty.  But when is something objectively certain? When a mistake is not
possible. But what kind of possibility is that? Mustn't mistake be logically
excluded?

\item
% 195.
If I believe that I am sitting in my room when I am not, then I shall not be
said to have made a mistake. But what is the essential difference between this
case and a mistake?

\item
% 196.
Sure evidence is what we accept as sure, it is evidence that we go by in acting
surely, acting without any doubt.  What we call ``a mistake'' plays a quite
special part in our language games, and so too does what we regard as certain
evidence.

\item
% 197.
It would be nonsense to say that we regard something as sure evidence because
it is certainly true.

\item
% 198.
Rather, we must first determine the role of deciding for or against a
proposition.

\item
% 199.
The reason why the use of the expression ``true or false'' has something
misleading about it is that it is like saying ``it tallies with the facts or it
doesn't'', and the very thing that is in question is what ``tallying'' is here.

\item
% 200.
Really ``The proposition is either true or false'' only means that it must be
possible to decide for or against it. But this does not say what the ground for
such a decision is like.

\item
% 201.
Suppose someone were to ask: ``Is it really right for us to rely on the
evidence of our memory (or our senses) as we do?''

\item
% 202.
Moore's certain propositions almost declare that we have a right to rely upon
this evidence.

\item
% 203.
\st{Everything that we regard as evidence indicates that the earth already existed
long before my birth. The contrary hypothesis has nothing to confirm it at all.
If everything speaks for an hypothesis and nothing against it, is it
objectively certain? One can call it that. But does it necessarily agree with
the world of facts? At the very best it shows us what ``agreement'' means. We
find it difficult to imagine it to be false, but also difficult to make use
of.} What does this agreement consist in, if not in the fact
that what is evidence in these language games speaks for our proposition?
(Tractatus Logico-Philosophicus)

\item
% 204.
Giving grounds, however, justifying the evidence, comes to an end; - but the
end is not certain propositions' striking us immediately as true, i.e. it is
not a kind of seeing on our part; it is our acting, which lies at the bottom of
the language-game.

\item
% 205.
If the true is what is grounded, then the ground is not true, not yet false.

\item
% 206.
If someone asked us ``but is that true?'' we might say ``yes'' to him; and if
he demanded grounds we might say ``I can't give you any grounds, but if you
learn more you too will think the same.'' If this didn't come about, that would
mean that he couldn't for example learn history.

\item
% 207.
``Strange coincidence, that every man whose skull has been opened had a
brain!''

\item
% 208.
I have a telephone conversation with New York. My friend tells me that his
young trees have buds of such and such a kind. I am now convinced that his tree
is... Am I also convinced that the earth exists?

\item
% 209.
The existence of the earth is rather part of the whole picture which forms the
starting-point of belief for me.

\item
% 210.
Does my telephone call to New York strengthen my conviction that the earth
exists?  Much seems to be fixed, and it is removed from the traffic. It is also
so to speak shunted onto an unused siding.

\item
% 211.
Now it gives our way of looking at things, and our researches, their form.
Perhaps it was once disputed. But perhaps, for unthinkable ages, it has
belonged to the scaffolding of our thoughts. (Every human being has parents.)

\item
% 212.
In certain circumstances, for example, we regard a calculation as sufficiently
checked. What gives us a right to do so? Experience? May that not have deceived
us? Somewhere we must be finished with justification, and then there remains
the proposition that this is how we calculate.

\item
% 213.
Our `empirical propositions' do not form a homogeneous mass.

\item
% 214.
What prevents me from supposing that this table either vanishes or alters its
shape and colour when on one is observing it, and then when someone looks at it
again changes back to its old condition? - ``But who is going to suppose such a
thing?'' - one would feel like saying.

\item
% 215.
Here we see that the idea of `agreement with reality' does not have any clear
application.

\item
% 216.
The proposition ``It is written''.

\item
% 217.
If someone supposed that all our calculations were uncertain and that we could
rely on none of them (justifying himself by saying that mistakes are always
possible) perhaps we would say he was crazy. But can we say he is in error?
Does he not just react differently? We rely on calculations, he doesn't; we are
sure, he isn't.

\item
% 218.
Can I believe for one moment that I have ever been in the stratosphere? No. So
do I know the contrary, like Moore?

\item
% 219.
There cannot be any doubt about it for me as a reasonable person. - That's it.
-

\item
% 220.
The reasonable man does not have certain doubts.

\item
% 221.
Can I be in doubt at will?

\item
% 222.
I cannot possibly doubt that I was never in the stratosphere. Does that make me
know it? Does it make it true?

\item
% 223.
For mightn't I be crazy and not doubting what I absolutely ought to doubt?

\item
% 224.
``I know that it never happened, for if it had happened I could not possibly
have forgotten it.'' But, supposing it did happen, then it just would have been
the case that you had forgotten it. And how do you know that you could not
possibly have forgotten it? Isn't that just from earlier experience?

\item
% 225.
What I hold fast to is not one proposition but a nest of propositions.

\item
% 226.
Can I give the supposition that I have ever been on the moon any serious
consideration at all?

\item
% 227.
``Is that something that one can forget?!''

\item
% 228.
``In such circumstances, people do not say `Perhaps we've all forgotten', and
the like, but rather they assume that...''

\item
% 229.
Our talk gets its meaning from the rest of our proceedings.

\item
% 230.
We are asking ourselves: what do we do with a statement ``I know...''? For it
is not a question of mental processes or mental states.  And that is how one
must decide whether something is knowledge or not.

\item
% 231.
If someone doubted whether the earth had existed a hundred years ago, I should
not understand, for this reason: I would not know what such a person would
still allow to be counted as evidence and what not.

\item
% 232.
``We could doubt every single one of these facts, but we could not doubt them
all.'' Wouldn't it be more correct to say: ``we do not doubt them all''.  Our
not doubting them all is simply our manner of judging, and therefore of acting.

\item
% 233.
If a child asked me whether the earth was already there before my birth, I
should answer him that the earth did not begin only with my birth, but that it
existed long, long before. And I should have the feeling of saying something
funny. Rather as if a child had asked if such and such a mountain were higher
than a tall house that it had seen. In answering the question I should have to
be imparting a picture of the world to the person who asked it.  If I do answer
the question with certainty, what gives me this certainty?

\item
% 234.
I believe that I have forebears, and that every human being has them. I believe
that there are various cities, and, quite generally, in the main facts of
geography and history. I believe that the earth is a body on whose surface we
move and that it no more suddenly disappears or the like than any other solid
body: this table, this house, this tree, etc. If I wanted to doubt the
existence of the earth long before my birth, I should have to doubt all sorts
of things that stand fast for me.

\item
% 235.
And that something stands fast for me is not grounded in my stupidity or
credulity.

\item
% 236.
If someone said ``The earth has not long been...'' what would he be impugning?
Do I know?  Would it not have to be what is called a scientific belief? Might
it not be a mystical one? Is there any absolute necessity for him to be
contradicting historical facts? or even geographical ones?

\item
% 237.
If I say ``an hour ago this table didn't exist'', I probably mean that it was
only made later on.  If I say ``this mountain didn't exist then'', I presumably
mean that it was only formed later on - perhaps by a volcano.  If I say ``this
mountain didn't exist an hour ago'', that is such a strange statement that it
is not clear what I mean. Whether for example I mean something untrue but
scientific. Perhaps you think that the statement that the mountain didn't exist
then is quite clear, however one conceives the context. But suppose someone
said ``This mountain didn't exist a minute ago, but an exactly similar one did
instead.'' Only the accustomed context allows what is meant to come through
clearly.

\item
% 238.
I might therefore interrogate someone who said that the earth did not exist
before his birth, in order to find out which of my convictions he was at odds
with. And then it might be that he was contradicting my fundamental attitudes,
and if that were how it was, I should have to put up with it.  Similarly if he
said he had at some time been on the moon.

\item
% 239.
I believe that every human being has two human parents; but Catholics believe
that Jesus only had a human mother. And other people might believe that there
are human beings with no parents, and give no credence to all the contrary
evidence. Catholics believe as well that in certain circumstances a wafer
completely changes its nature, and at the same time that all evidence proves
the contrary. And so if Moore said ``I know that this is wine and not blood'',
Catholics would contradict him.

\item
% 240.
What is the belief that all human beings have parents based on? On experience.
And how can I base this sure belief on my experience? Well, I base it not only
on the fact that I have known the parents of certain people but on everything
that I have learnt about the sexual life of human beings and their anatomy and
physiology: also on what I have heard and seen of animals. But then is that
really a proof?

\item
% 241.
Isn't this an hypothesis, which, as I believe, is again and again completely
confirmed?

\item
% 242.
Mustn't we say at every turn: ``I believe this with certainty''?

\item
% 243.
One says ``I know'' when one is ready to give compelling grounds. ``I know''
relates to a possibility of demonstrating the truth. Whether someone knows
something can come to light, assuming that he is convinced of it.  But if what
he believes is of such a kind that the grounds that he can give are no surer
than his assertion, then he cannot say that he knows what he believes.

\item
% 244.
If someone says ``I have a body'', he can be asked ``Who is speaking here with
this mouth?''

\item
% 245.
To whom does anyone say that he knows something? To himself, or to someone
else. If he says it to himself, how is it distinguished from the assertion that
he is sure that things are like that? There is no subjective sureness that I
know something. The certainty is subjective, but not the knowledge. So if I say
``I know that I have two hands'', and that is not supposed to express just my
subjective certainty, I must be able to satisfy myself that I am right. But I
can't do that, for my having two hands is not less certain before I have looked
at them than afterwards. But I could say: ``That I have two hands is an
irreversible belief.'' That would express the fact that I am not ready to let
anything count as a disproof of this proposition.

\item
% 246.
``Here I have arrived at a foundation of all my beliefs.'' ``This position I
will hold!'' But isn't that, precisely, only because I am completely convinced
of it? - What is `being completely convinced' like?

\item
% 247.
What would it be like to doubt now whether I have two hands? Why can't I
imagine it at all? What would I believe if I didn't believe that? So far I have
no system at all within which this doubt might exist.

\item
% 248.
I have arrived at the rock bottom of my convictions.  And one might almost say
that these foundation-walls are carried by the whole house.

\item
% 249.
One gives oneself a false picture of doubt.

\item
% 250.
My having two hands is, in normal circumstances, as certain as anything that I
could produce in evidence for it.  That is why I am not in a position to take
the sight of my hand as evidence for it.

\item
% 251.
Doesn't this mean: I shall proceed according to this belief unconditionally,
and not let anything confuse me?

\item
% 252.
But it isn't just that I believe in this way that I have two hands, but that
every reasonable person does.

\item
% 253.
At the foundation of well-founded belief lies belief that is not founded.

\item
% 254.
Any `reasonable' person behaves like this.

\item
% 255.
Doubting has certain characteristic manifestations, but they are only
characteristic of it in particular circumstances. If someone said that he
doubted the existence of his hands, kept looking at them from all sides, tried
to make sure it wasn't `all done by mirrors', etc., we should not be sure
whether we ought to call this doubting. We might describe his way of behaving
as like the behaviour of doubt, but this game would be not be ours.

\item
% 256.
On the other hand a language-game does change with time.

\item
% 257.
If someone said to me that he doubted whether he had a body I should take him
to be a half-wit. But I shouldn't know what it would mean to try to convince
him that he had one. And if I had said something, and that had removed his
doubt, I should not know how or why.

\item
% 258.
I do not know how the sentence ``I have a body'' is to be used.  That doesn't
unconditionally apply to the proposition that I have always been on or near the
surface of the earth.

\item
% 259.
Someone who doubted whether the earth had existed for 100 years might have a
scientific, or on the other hand philosophical, doubt.

\item
% 260.
I would like to reserve the expression ``I know'' for the cases in which it is
used in normal linguistic exchange.

\item
% 261.
I cannot at present imagine a reasonable doubt as to the existence of the earth
during the last 100 years.

\item
% 262.
I can imagine a man who had grown up in quite special circumstances and been
taught that the earth came into being 50 years ago, and therefore believed
this. We might instruct him: the earth has long... etc. - We should be trying
to give him our picture of the world.  This would happen through a kind of
persuasion.

\item
% 263.
The schoolboy believes his teachers and his schoolbooks.

\item
% 264.
I could imagine Moore being captured by a wild tribe, and their expressing the
suspicion that he has come from somewhere between the earth and the moon. Moore
tells them that he knows etc. but he can't give them the grounds for his
certainty, because they have fantastic ideas of human ability to fly and know
nothing about physics. This would be an occasion for making that statement.

\item
% 265.
But what does it say, beyond ``I have never been to such and such a place, and
have compelling grounds to believe that''?

\item
% 266.
And here one would still have to say what are compelling grounds.

\item
% 267.
``I don't merely have the visual impression of a tree: I know that it is a
tree''.

\item
% 268.
``I know that this is a hand.'' - And what is a hand? - ``Well, this, for
example''.

\item
% 269.
Am I more certain that I have never been on the moon than that I have never
been in Bulgaria? Why am I so sure? Well, I know that I have never been
anywhere in the neighbourhood - for example I have never been in the Balkans.

\item
% 270.
``I have compelling grounds for my certitude.'' These grounds make the
certitude objective.

\item
% 271.
What is a telling ground for something is not anything I decide.

\item
% 272.
I know = I am familiar with it as a certainty.

\item
% 273.
But when does one say of something that it is certain?  For there can be
dispute whether something is certain; I mean, when something is objectively
certain.  There are countless general empirical propositions that count as
certain for us.

\item
% 274.
One such is that if someone's arm is cut off it will not grow again. Another,
if someone's head is cut off he is dead and will never live again.  Experience
can be said to teach us these propositions. However, it does not teach us them
in isolation: rather, it teaches us a host of interdependent propositions. If
they were isolated I might perhaps doubt them, for I have no experience
relating to them.

\item
% 275.
If experience is the ground of our certainty, then naturally it is past
experience.  And it isn't for example just my experience, but other's people's,
that I get knowledge from.  Now one might say that it is experience again that
leads us to give credence to others. But what experience makes me believe that
the anatomy and physiology books don't contain what is false? Though it is true
that this trust is backed up by my own experience.

\item
% 276.
We believe, so to speak, that this great building exists, and then we see, now
here, now there, one or another small corner of it.

\item
% 277.
``I can't help believing...''

\item
% 278.
``I am comfortable that that is how things are.''

\item
% 279.
It is quite sure that motor cars don't grow out of the earth. We feel that if
someone could believe the contrary he could believe everything that we say is
untrue, and could question everything that we hold to be sure.  But how does
this one belief hang together with all the rest? We should like to say that
someone who could believe that does not accept our whole system of
verification.  This system is something that a human being acquires by means of
observation and instruction. I intentionally do not say ``learns''.

\item
% 280.
After he has seen this and this and heard that and that, he is not in a
position to doubt whether...

\item
% 281.
I, L.W., believe, am sure, that my friend hasn't sawdust in his body or in his
head, even though I have no direct evidence of my senses to the contrary. I am
sure, by reason of what has been said to me, of what I have read, and of my
experience. To have doubts about it would seem to me madness - of course, this
is also in agreement with other people; but I agree with them.

\item
% 282.
I cannot say that I have good grounds for the opinion that cats do not grow on
trees or that I had a father and a mother.  If someone has doubts about it -
how is that supposed to have come about? By his never, from the beginning,
having believed that he had parents? But then, is that conceivable, unless he
has been taught it?

\item
% 283.
For how can a child immediately doubt what it is taught? That could mean only
that he was incapable of learning certain language games.

\item
% 284.
People have killed animals since the earliest times, used the fur, bones
etc.etc. for various purposes; they have counted definitely on finding similar
parts in any similar beast.  They have always learnt from experience; and we
can see from their actions that they believe certain things definitely, whether
they express this belief or not. By this I naturally do not want to say that
men should behave like this, but only that they do behave like this.

\item
% 285.
If someone is looking for something and perhaps roots around in a certain
place, he shows that he believes that what he is looking for is there.

\item
% 286.
What we believe depends on what we learn. We all believe that it isn't possible
to get to the moon; but there might be people who believe that that is possible
and that it sometimes happens. We say: these people do not know a lot that we
know. And, let them be never so sure of their belief - they are wrong and we
know it.  If we compare our system of knowledge with theirs then theirs is
evidently the poorer one by far.

\item
% 287.
The squirrel does not infer by induction that it is going to need stores next
winter as well. And no more do we need a law of induction to justify our
actions or our predictions.

\item
% 288.
I know, not just that the earth existed long before my birth, but also that it
is a large body, that this has been established, that I and the rest of mankind
have forebears, that there are books about all this, that such books don't lie,
etc. etc. etc. And I know all this? I believe it. This body of knowledge has
been handed on to me and I have no grounds for doubting it, but, on the
contrary, all sorts of confirmation.  And why shouldn't I say that I know all
this? Isn't that what one does say?  But not only I know, or believe, all that,
but the others do too. Or rather, I believe that they believe it.

\item
% 289.
I am firmly convinced that others believe, believe they know, that all that is
in fact so.

\item
% 290.
I myself wrote in my book that children learn to understand a word in such and
such a way. Do I know that, or do I believe it? Why in such a case do I write
not ``I believe etc.'' but simply the indicative sentence?

\item
% 291.
We know that the earth is round. We have definitively ascertained that it is
round.  We shall stick to this opinion, unless our whole way of seeing nature
changes. ``How do you know that?'' - I believe it.

\item
% 292.
Further experiments cannot give the lie to our earlier ones, at most they may
change our whole way of looking at things.

\item
% 293.
Similarly with the sentence ``water boils at 100 C''.

\item
% 294.
This is how we acquire conviction, this is called ``being rightly convinced''.

\item
% 295.
So hasn't one, in this sense, a proof of the proposition? But that the same
thing has happened again is not a proof of it; though we do say that it gives
us a right to assume it.

\item
% 296.
This is what we call an ``empirical foundation'' for our assumptions.

\item
% 297.
For we learn, not just that such and such experiments had those and those
results, but also the conclusion which is drawn. And of course there is nothing
wrong in our doing so. For this inferred proposition is an instrument for a
definitive use.

\item
% 298.
`We are quite sure of it' does not mean just that every single person is
certain of it, but that we belong to a community which is bound together by
science and education.

\item
% 299.
We are satisfied that the earth is round. \emph{In English}

\chapter{}

\item
% 300.
Not all corrections of our views are on the same level.

\item
% 301.
Supposing it wasn't true that the earth had already existed long before I was
born - how should we imagine the mistake being discovered?

\item
% 302.
It's no good saying ``Perhaps we are wrong'' when, if no evidence is
trustworthy, trust is excluded in the case of the present evidence.

\item
% 303.
If, for example, we have always been miscalculating, and twelve times twelve
isn't a hundred and forty-four, why should we trust any other calculation? And
of course that is wrongly put.

\item
% 304.
But nor am I making a mistake about twelve times twelve being a hundred and
forty-four. I may say later that I was confused just now, but not that I was
making a mistake.

\item
% 305.
Here once more there is needed a step like the one taken in relativity theory.

\item
% 306.
``I don't know if this is a hand.'' But do you know what the word ``hand''
means? And don't say ``I know that it means now for me''. And isn't it an
empirical fact - that this word is used like this?

\item
% 307.
And here the strange thing is that when I am quite certain of how the words are
used, have no doubt about it, I can still give no grounds for my way of going
on. If I tried I could give a thousand, but none as certain as the very thing
they were supposed to be grounds for.

\item
% 308.
`Knowledge' and `certainty' belong to different categories. They are not two
`mental states' like, say `surmising' and `being sure'. (Here I assume that it
is meaningful for me to say ``I know what (e.g.) the word `doubt' means'' and
that this sentence indicates that the word ``doubt'' has a logical role.) What
interests us now is not being sure but knowledge. That is, we are interested in
the fact that about certain empirical propositions no doubt can exist if making
judgments is to be possible at all. Or again: I am inclined to believe that not
everything that has the form of an empirical proposition is one.

\item
% 309.
Is it that rule and empirical proposition merge into one another?

\item
% 310.
A pupil and a teacher. The pupil will not let anything be explained to him, for
he continually interrupts with doubts, for instance as to the existence of
things, the meaning of words, etc. The teacher says ``Stop interrupting me and
do as I tell you. So far your doubts don't make sense at all.''

\item
% 311.
Or imagine that the boy questioned the truth of history (and everything that
connects up with it) - and even whether the earth existed at all a hundred
years before.

\item
% 312.
Here it strikes me as if this doubt were hollow. But in that case - isn't
belief in history hollow too? No: there is so much that this connects up with.

\item
% 313.
So is that what makes us believe a proposition? Well - the grammar of
``believe'' just does hang together with the grammar of the proposition
believed.

\item
% 314.
Imagine that the schoolboy really did ask ``and is there a table there even
when I turn around, and even when no one is there to see it?'' Is the teacher
to reassure him - and say ``of course there is!''?  Perhaps the teacher will
get a bit impatient, but think that the boy will grow out of asking such
questions.

\item
% 315.
That is to say, the teacher will feel that this is not really a legitimate
question at all.  And it would be just the same if the pupil cast doubt on the
uniformity of nature, that is to say on the justification of inductive
arguments. - The teacher would feel that this was only holding them up, that
this way the pupil would only get stuck and make no progress. - And he would be
right. It would be as if someone were looking for some object in a room; he
opens a drawer and doesn't see it there; then he closes it again, waits, and
opens it once more to see if perhaps it isn't there now, and keeps on like
that. He has not learned to look for things. And in the same way this pupil has
not learned how to ask questions. He has not learned the game that we are
trying to teach him.

\item
% 316.
And isn't it the same as if the pupil were to hold up his history lesson with
doubts as to whether the earth really...?

\item
% 317.
This doubt isn't one of the doubts in our game. (But not as if we chose this
game!)

\item
% 318.
`The question doesn't arise at all.' Its answer would characterize a method.
But there is no sharp boundary between methodological propositions and
propositions within a method.

\item
% 319.
But wouldn't one have to say then, that there is no sharp boundary between
propositions of logic and empirical propositions? The lack of sharpness is that
of the boundary between rule and empirical proposition.

\item
% 320.
Here one must, I believe, remember that the concept `proposition' itself is not
a sharp one.

\item
% 321.
Isn't what I am saying: any empirical proposition can be transformed into a
postulate - and then becomes a norm of description. But I am suspicious even of
this. The sentence is too general. One almost wants to say ``any empirical
proposition can, theoretically, be transformed...'', but what does
``theoretically'' mean here? It sounds all to reminiscent of the Tractatus.

\item
% 322.
What if the pupil refused to believe that this mountain had been there beyond
human memory?  We should say that he had no grounds for this suspicion.

\item
% 323.
So rational suspicion must have grounds?  We might also say: ``the reasonable
man believes this''.

\item
% 324.
Thus we should not call anybody reasonable who believed something in despite of
scientific evidence.

\item
% 325.
When we say that we know that such and such..., we mean that any reasonable
person in our position would also know it, that it would be a piece of unreason
to doubt it. Thus Moore wants to say not merely that he knows that he etc.
etc., but also that anyone endowed with reason in his position would know it
just the same.

\item
% 326.
But who says what it is reasonable to believe in this situation?

\item
% 327.
So it might be said: ``The reasonable man believes: that the earth has been
there since long before his birth, that his life has been spent on the surface
of the earth, or near it, that he has never, for example, been on the moon,
that he has a nervous system and various innards like all other people, etc.,
etc.''

\item
% 328.
``I know it as I know that my name is L.W.''

\item
% 329.
`If he calls that in doubt - whatever ``doubt'' means here - he will never
learn this game'.

\item
% 330.
So here the sentence ``I know...'' expresses the readiness to believe certain
things.

\item
% 331.
If we ever do act with certainty on the strength of belief, should we wonder
that there is much we cannot doubt?

\item
% 332.
Imagine that someone were to say, without wanting to philosophize, ``I don't
know if I have ever been on the moon; I don't remember ever having been
there''. (Why would this person be so radically different from us?) In the
first place - how would he know that he was on the moon? How does he imagine
it? Compare: ``I do not know if I was ever in the village of X.'' But neither
could I say that if X were in Turkey, for I know that I was never in Turkey.

\item
% 333.
I ask someone ``Have you ever been in China?'' He replies ``I don't know''.
Here one would surely say ``You don't know? Have you any reason to believe you
might have been there at some time? Were you for example ever near the Chinese
border? Or were your parents there at the time when you were going to be
born?'' - Normally Europeans do know whether they have been in China or not.

\item
% 334.
That is to say: only in such-and-such circumstances does a reasonable person
doubt that.

\item
% 335.
The procedure in a court of law rests on the fact that circumstances give
statements a certain probability. The statement that, for example, someone came
into the world without parents wouldn't ever be taken into consideration there.

\item
% 336.
But what men consider reasonable or unreasonable alters. At certain periods men
find reasonable what at other periods they found unreasonable. And vice-versa.
But is there no objective character here?  Very intelligent and well-educated
people believe in the story of creation in the Bible, while others hold it as
proven false, and the grounds of the latter are well known to the former.

\item
% 337.
One cannot make experiments if there are not some things that one does not
doubt. But that does not mean that one takes certain presuppositions on trust.
When I write a letter and post it, I take it for granted that it will arrive -
I expect this.  If I make an experiment I do not doubt the existence of the
apparatus before my eyes. I have plenty of doubts, but not that. If I do a
calculation I believe, without any doubts, that the figures on the paper aren't
switching of their own accord, and I also trust my memory the whole time, and
trust it without any reservation. The certainty here is the same as that of my
never having been on the moon.

\item
% 338.
But imagine people who were never quite certain of these things, but said that
they were very probably so, and that it did not pay to doubt them. Such a
person, then, would say in my situation: ``It is extremely unlikely that I have
ever been on the moon'', etc., etc. How would the life of these people differ
from ours? For there are people who say that it is merely extremely probable
that water over a fire will boil and not freeze, and that therefore strictly
speaking what we consider impossible is only improbable. What difference does
this make in their lives? Isn't it just that they talk rather more about
certain things that the rest of us?

\item
% 339.
Imagine someone who is supposed to fetch a friend from the railway station and
doesn't simply look the train up in the time-table and go to the station at the
right time, but says:``I have no belief that the train will really arrive, but
I will go to the station all the same.'' He does everything that the normal
person does, but accompanies it with doubts or with self-annoyance, etc.

\item
% 340.
We know, with the same certainty with which we believe any mathematical
proposition, how the letters A and B are pronounced, what the colour of human
blood is called, that other human beings have blood and call it ``blood''.

\item
% 341.
That is to say, the questions that we raise and our doubts depend on the fact
that some propositions are exempt from doubt, are as it were like hinges on
which those turn.

\item
% 342.
That is to say, it belongs to the logic of our scientific investigations that
certain things are in deed not doubted.

\item
% 343.
But it isn't that the situation is like this: We just can't investigate
everything, and for that reason we are forced to rest content with assumption.
If I want the door to turn, the hinges must stay put.

\item
% 344.
My life consists in my being content to accept many things.

\item
% 345.
If I ask someone ``what colour do you see at the moment?'', in order, that is,
to learn what colour is there at the moment, I cannot at the same time question
whether the person I ask understands English, whether he wants to take me in,
whether my own memory is not leaving me in the lurch as to the names of
colours, and so on.

\item
% 346.
When I am trying to mate someone in chess, I cannot have doubts about the
pieces perhaps changing places of themselves and my memory simultaneously
playing tricks on me so that I don't notice.

\item
% 347.
``I know that that's a tree.'' Why does it strike me as if I did not understand
the sentence? though it is after all an extremely simple sentence of the most
ordinary kind? It is as if I could not focus my mind on any meaning. Simply
because I don't look for the focus where the meaning is. As soon as I think of
an everyday use of the sentence instead of a philosophical one, its meaning
becomes clear and ordinary.

\item
% 348.
Just as the words ``I am here'' have a meaning only in certain contexts, and
not when I say them to someone who is sitting in front of me and sees me
clearly, - and not because they are superfluous, but because their meaning is
not determined by the situation, yet stands in need of such determination.

\item
% 349.
``I know that that's a tree'' - this may mean all sorts of things: I look at a
plant that I take for a young beech and that someone else thinks is a
black-currant. He says ``that's a shrub''; I say it is a tree. - We see
something in the mist which one of us takes for a man, and the other says ``I
know that that's a tree''. Someone wants to test my eyes etc.etc. - etc.etc.
Each time the `that' which I declare to be a tree is of a different kind.  But
what when we express ourselves more precisely? For example: ``I know that that
thing there is a tree, I can see it quite clearly.'' - Let us even suppose I
had made this remark in the context of a conversation (so that it was relevant
when I made it); and now, out of all context, I repeat it while looking at the
tree, and I add ``I mean these words as I did five minutes ago''. If I added,
for example, that I had been thinking of my bad eyes again and it was a kind of
sigh, then there would be nothing puzzling about the remark.  For how a
sentence is meant can be expressed by an expansion of it and may therefore be
made part of it.

\item
% 350.
``I know that that's a tree'' is something a philosopher might say to
demonstrate to himself or to someone else that he knows something that is not a
mathematical or logical truth. Similarly, someone who was entertaining the idea
that he was no use any more might keep repeating to himself ``I can still do
this and this and this.'' If such thoughts often possessed him one would not be
surprised if he, apparently out of all context, spoke such a sentence out loud.
(But here I have already sketched a background, a surrounding, for this remark,
that is to say given it a context.) But if someone, in quite heterogeneous
circumstances, called out with the most convincing mimicry: ``Down with him!'',
one might say of these words (and their tone) that they were a pattern that
does indeed have familiar applications, but that in this case it was not even
clear what language the man in question was speaking. I might make with my hand
the movement I should make if I were holding a hand-saw and sawing through a
plank; but would one have any right to call this movement sawing, out of all
context? - (It might be something quite different!)

\item
% 351.
Isn't the question ``have these words a meaning?'' similar to ``Is that a
tool?'' asked as one produces, say, a hammer? I say ``Yes, it's a hammer.'' But
what if the thing that any of us would take for a hammer were somewhere else a
missile, for example, or a conductor's baton? Now make the application
yourself.

\item
% 352.
If someone says, ``I know that that's a tree'' I may answer: ``Yes, that is a
sentence. An English sentence. And what is it supposed to be doing?'' Suppose
he replies: ``I just wanted to remind myself that I know thing like that''? -

\item
% 353.
But suppose he said ``I want to make a logical observation''? - If a forester
goes into a wood with his men and says ``This tree has got to be cut down, and
this one and this one'' -- what if he then observes ``I know that that's a
tree''? - But might not I say of the forester ``He knows that that's a tree -
he doesn't examine it, or order his men to examine it''?

\item
% 354.
Doubting and non-doubting behavior. There is the first only if there is the
second.

\item
% 355.
A mad-doctor (perhaps) might ask me ``Do you know what that is?'' and I might
reply ``I know that it's a chair; I recognize it, it's always been in my
room''. He says this, possibly, to test not my eyes but my ability to recognize
things, to know their names and their functions. What is in question here is a
kind of knowing one's way about. Now it would be wrong for me to say ``I
believe that it's a chair'' because that would express my readiness for my
statement to be tested. While ``I know that it...'' implies bewilderment if
what I said was not confirmed.

\item
% 356.
My ``mental state'', the ``knowing'', gives me no guarantee of what will
happen. But it consists in this, that I should not understand where a doubt
could get a foothold nor where a further test was possible.

\item
% 357.
One might say: `` `I know' expresses comfortable certainty, not the certainty
that is still struggling.''

\item
% 358.
Now I would like to regard this certainty, not as something akin to hastiness
or superficiality, but as a form of life. (That is very badly expressed and
probably badly thought as well.)

\item
% 359.
But that means I want to conceive it as something that lies beyond being
justified or unjustified; as it were, as something animal.

\item
% 360.
I know that this is my foot. I could not accept any experience as proof to the
contrary. - That may be an exclamation; but what follows from it? At least that
I shall act with a certainty that knows no doubt, in accordance with my belief.

\item
% 361.
But I might also say: It has been revealed to me by God that it is so. God has
taught me that this is my foot. And therefore if anything happened that seemed
to conflict with this knowledge I should have to regard that as deception.

\item
% 362.
But doesn't it come out here that knowledge is related to a a decision?

\item
% 363.
And here it is difficult to find the transition from the exclamation one would
like to make, to its consequences in what one does.

\item
% 364.
One might also put this question: ``If you know that that is your foot, - do
you also know, or do you only believe, that no future experience will seem to
contradict your knowledge?'' (That is, that nothing will seem to you yourself
to do so.)

\item
% 365.
If someone replied: ``I also know that it will never seem to me as if anything
contradicted that knowledge'', - what could we gather from that, except that he
himself had no doubt that it would never happen? -

\item
% 366.
Suppose it were forbidden to say ``I know'' and only allowed to say ``I believe
I know''?

\item
% 367.
Isn't it the purpose of construing a word like ``know'' analogously to
``believe'' that then opprobrium attaches to the statement ``I know'' if the
person who makes it is wrong?  As a result a mistake becomes something
forbidden.

\item
% 368.
If someone says that he will recognize no experience as proof of the opposite,
that is after all a decision. It is possible that he will act against it.

\item
% 369.
If I wanted to doubt whether this was my hand, how could I avoid doubting
whether the word ``hand'' has any meaning? So that is something I seem to know
after all.

\item
% 370.
But more correctly: The fact that I use the word ``hand'' and all the other
words in my sentence without a second thought, indeed that I should stand
before the abyss if I wanted so much as to try doubting their meanings - shows
that absence of doubt belongs to the essence of the language-game, that the
question ``How do I know...'' drags out the language-game, or else does away
with it.

\item
% 371.
Doesn't ``I know that that's a hand'', in Moore's sense, mean the same, or more
or less the same, as: I can make statements like ``I have a pain in this hand''
or `this hand is weaker than the other'' or ``I once broke this hand'', and
countless others, in language-games where a doubt as to the existence of this
hand does not come in?

\item
% 372.
Only in certain cases is it possible to make an investigation ``is that really
a hand?'' (or ``my hand''). For ``I doubt whether that is really my (or a)
hand'' makes no sense without some more precise determination. One cannot tell
from these words alone whether any doubt at all is meant - nor what kind of
doubt.

\item
% 373.
Why is it supposed to be possible to have grounds for believing something if it
isn't possible to be certain?

\item
% 374.
We teach a child ``that is your hand'', not ``that is perhaps (or ``probably'')
your hand''. That is how a child learns the innumerable language-games that are
concerned with his hand. An investigation or question, `whether this is really
a hand' never occurs to him. Nor, on the other hand, does he learn that he
knows that this is a hand.

\item
% 375.
Here one must realize that complete absence of doubt at some point, even where
we would say that `legitimate' doubt can exist, need not falsify a
language-game. For there is also something like another arithmetic.  I believe
that this admission must underlie any understanding of logic.

\item
% 376.
I may claim with passion that I know that this (for example) is my foot.

\item
% 377.
But this passion is after all something very rare, and there is no trace of it
when I talk of this foot in the ordinary way.

\item
% 378.
Knowledge is in the end based on acknowledgement.

\item
% 379.
I say with passion ``I know that this is a foot'' - but what does it mean?

\item
% 380.
I might go on: ``Nothing in the world will convince me of the opposite!'' For
me this fact is at the bottom of all knowledge. I shall give up other things
but not this.

\item
% 381.
This ``Nothing in the world'' is obviously an attitude which one hasn't got
towards everything one believes or is certain of.

\item
% 382.
That is not to say that nothing in the world will in fact be able to convince
me of anything else.

\item
% 383.
The argument ``I may be dreaming'' is senseless for this reason: if I am
dreaming, this remark is being dreamed as well - and indeed it is also being
dreamed that these words have any meaning.

\item
% 384.
Now what kind of sentence is ``Nothing in the world...''?

\item
% 385.
It has the form of a prediction, but of course it is not one that is based on
experience.

\item
% 386.
Anyone who says, with Moore, that he knows that so and so... - gives the degree
of certainty that something has for him. And it is important that this degree
has a maximum value.

\item
% 387.
Someone might ask me: ``How certain are you that that is a tree over there;
that you have money in your pocket; that that is your foot?'' And the answer in
one case might be ``not certain'', in another ``as good as certain'', in the
third ``I can't doubt it''. And these answers would make sense even without any
grounds. I should not need for example, to say: ``I can't be certain whether
that is a tree because my eyes aren't sharp enough.'' I want to say: it made
sense for Moore to say ``I know that that is a tree'', if he meant something
quite particular by it.
\\
\emph{I believe it might interest a philosopher, one who can think himself, to
read my notes. For even if I have hit the mark only rarely, he would recognize
what targets I had been ceaselessly aiming at.}

\item
% 388.
Every one of us often uses such a sentence, and there is no question but that
it makes sense. But does that mean it yields any philosophical conclusion? Is
it more of a proof of the existence of external things, that I know that this
is a hand, than that I don't know whether that is gold or brass?

\item
% 389.
Moore wanted to give an example to show that one really can know propositions
about physical objects. - If there were a dispute whether one could have a pain
in such and such a part of the body, then someone who just then had a pain in
that spot might say: ``I assure you, I have a pain there now.'' But it would
sound odd if Moore had said: ``I assure you, I know that's a tree.'' A personal
experience simply has no interest for us here.

\item
% 390.
All that is important is that it makes sense to say that one knows such a
thing; and consequently the assurance that one does know it can't accomplish
anything here.

\item
% 391.
Imagine a language-game ``When I call you, come in through the door.'' In any
ordinary case, a doubt whether there really is a door there will be impossible.

\item
% 392.
What I need to show is that a doubt is not necessary even when it is possible.
That the possibility of the language-game doesn't depend on everything being
doubted that can be doubted. (This is connected with the role of contradiction
in mathematics.)

\item
% 393.
The sentence ``I know that that's a tree'' if it were said outside its
language-game, might also be a quotation (from an English grammar-book
perhaps). - ``But suppose I mean it while I am saying it?'' The old
misunderstanding about the concept `mean'.

\item
% 394.
``This is one of the things that I cannot doubt.''

\item
% 395.
``I know all that.'' And that will come out in the way I act and in the way I
speak about the things in question.

\item
% 396.
In the language-game (2), can he say that he knows that those are building
stones? - ``No, but he does know it.''
\\
\emph{Philosophical Investigations I,2: ... and write with confidence ``In the
beginning was the deed.'' Goethe, Faust I.}

\item
% 397.
Haven't I gone wrong and isn't Moore perfectly right? Haven't I made the
elementary mistake of confusing one's thoughts with one's knowledge? Of course
I do not think to myself ``The earth already existed for some time before my
birth'', but do I know it any the less? Don't I show that I know it by always
drawing its consequences?

\item
% 398.
And don't I know that there is no stairway in this house going six floors deep
into the earth, even though I have never thought about it?

\item
% 399.
But doesn't my drawing the consequences only show that I accept this
hypothesis?

\item
% 400.
Here I am inclined to fight windmills, because I cannot yet say the thing I
really want to say.

\item
% 401.
I want to say: propositions of the form of empirical propositions, and not only
propositions of logic, form the foundation of all operating with thoughts (with
language). - This observation is not of the form ``I know...''. ``I know...''
states what I know, and that is not of logical interest.

\item
% 402.
In this remark the expression ``propositions of the form of empirical
propositions'' is itself thoroughly bad; the statements in question are
statements about material objects. And they do not serve as foundations in the
same way as hypotheses which, if they turn out to be false, are replaced by
others.

\item
% 403.
To say of man, in Moore's sense, that he knows something; that what he says is
therefore unconditionally the truth, seems wrong to me. - It is the truth only
inasmuch as it is an unmoving foundation of his language-games.

\item
% 404.
I want to say: it's not that on some points men know the truth with perfect
certainty. No: perfect certainty is only a matter of their attitude.

\item
% 405.
But of course there is still a mistake even here.

\item
% 406.
What I am aiming at is also found in the difference between the casual
observation ``I know that that's a...'', as it might be used in ordinary life,
and the same utterance when a philosopher makes it.

\item
% 407.
For when Moore says ``I know that that's...'' I want to reply ``you don't know
anything!'' - and yet I would not say that to anyone who was speaking without
philosophical intention. That is, I feel (rightly?) that these two mean to say
something different.

\item
% 408.
For if someone says he knows such-and-such, and this is part of his philosophy
- then his philosophy is false if he has slipped up in this statement.

\item
% 409.
If I say ``I know that that's a foot'' - what am I really saying? Isn't the
whole point that I am certain of the consequences - that if someone else had
been in doubt I might say to him ``you see - I told you so''? Would my
knowledge still be worth anything if it let me down as a clue in action? And
can't it let me down?

\item
% 410.
Our knowledge forms an enormous system. And only within this system has a
particular bit the value we give it.

\item
% 411.
If I say ``we assume that the earth has existed for many years past'' (or
something similar), then of course it sounds strange that we should assume such
a thing. But in the entire system of our language-games it belongs to the
foundations. The assumption, one might say, forms the basis of action, and
therefore, naturally, of thought.

\item
% 412.
Anyone who is unable to imagine a case in which one might say ``I know that
this is my hand'' (and such cases are certainly rare) might say that these
words were nonsense. True, he might also say ``Of course I know - how could I
not know?'' - but then he would possibly be taking the sentence ``this is my
hand'' as an explanation of the words ``my hand''.

\item
% 413.
For suppose you were guiding a blind man's hand, and as you were guiding it
along yours you said ``this is my hand''; if he then said ``are you sure?'' or
``do you know it is?'', it would take very special circumstances for that to
make sense.

\item
% 414.
But on the other hand: how do I know that it is my hand? Do I even here know
exactly what it means to say it is my hand? - When I say ``how do I know?'' I
do not mean that I have the least doubt of it. What we have here is a
foundation for all my action. But it seems to me that it is wrongly expressed
by the words ``I know''.

\item
% 415.
And in fact, isn't the use of the word ``know'' as a preeminently philosophical
word altogether wrong? If ``know'' has this interest, why not ``being
certain''? Apparently because it would be too subjective. But isn't ``know''
just as subjective? Isn't one misled simply by the grammatical peculiarity that
``p'' follows from ``I know p''?  ``I believe I know'' would not need to
express a lesser degree of certainty. - True, but one isn't trying to express
even the greatest subjective certainty, but rather that certain propositions
seem to underlie all questions and all thinking.

\item
% 416.
And have we an example of this in, say, the proposition that I have been living
in this room for weeks past, that my memory does not deceive me in this?  -
``certain beyond all reasonable doubt'' -

\item
% 417.
``I know that for the last month I have had a bath every day.'' What am I
remembering? Each day and the bath each morning? No. I know that I bathed each
day and I do not derive that from some other immediate datum. Similarly I say
``I felt a pain in my arm'' without this locality coming into my consciousness
in any other way (such as by means of an image).

\item
% 418.
Is my understanding only blindness to my own lack of understanding? It often
seems so to me.

\item
% 419.
If I say ``I have never been in Asia Minor'', where do I get this knowledge
from? I have not worked it out, no one told me; my memory tells me. - So I
can't be wrong about it? Is there a truth here which I know? - I cannot depart
from this judgment without toppling all other judgments with it.

\item
% 420.
Even a proposition like this one, that I am now living in England, has these
two sides: it is not a mistake - but on the other hand, what do I know of
England? Can't my judgment go all to pieces?  Would it not be possible that
people came to my room and all declared the opposite? - even gave me `proofs'
of it, so that I suddenly stood there like a madman alone among people who were
all normal, or a normal person alone among madmen? Might I not then suffer
doubts about what at present seems at the furthest remove from doubt?

\item
% 421.
I am in England. - Everything around me tells me so; wherever and however I let
my thoughts turn, they confirm this for me at once. - But might I not be shaken
if things such as I don't dream of at present were to happen?

\item
% 422.
So I am trying to say something that sounds like pragmatism.  Here I am being
thwarted by a kind of Weltanschauung.

\item
% 423.
Then why don't I simply say with Moore ``I know that I am in England?'' Saying
this is meaningful in particular circumstances, which I can imagine. But when I
utter the sentence outside these circumstances, as an example to show that I
can know truths of this kind with certainty, then it at once strikes me as
fishy. - Ought it to?

\item
% 424.
I say ``I know p'' either to assure people that I, too, know the truth p, or
simply as an emphasis of |-p. One says too, ``I don't believe it, I know it''.
And one might also put it like this (for example): ``That is a tree. And that's
not just surmise.'' But what about this: ``If I were to tell someone that that
was a tree, that wouldn't be just surmise.'' Isn't this what Moore was trying
to say?

\item
% 425.
It would not be surmise and I might tell it to someone else with complete
certainty, as something there is no doubt about. But does that mean that it is
unconditionally the truth? May not the thing that I recognize with complete
certainty as the tree that I have seen here my whole life long - may this not
be disclosed as something different? May it not confound me?  And nevertheless
it was right, in the circumstances that give this sentence meaning, to say ``I
know (I do not merely surmise) that that's a tree.'' To say that in strict
truth I only believe it, would be wrong. It would be completely misleading to
say: ``I believe my name is L.W.'' And this too is right: I cannot be making a
mistake about it. But that does not mean that I am infallible about it.

\item
% 426.
But how can we show someone that we know truths, not only about sense-data but
also about things? For after all it can't be enough for someone to assure us
that he knows this.  Well, what must our starting point be if we are to show
this?

\item
% 22.

427. We need to show that even if he never uses the words ``I know...'', his
conduct exhibits the thing we are concerned with.

\item
% 428.
For suppose a person of normal behavior assured us that he only believed his
name was such-and-such, he believed he recognized the people he regularly lived
with, he believed that he had hands and feet when he didn't actually see them,
and so on. Can we show him it is not so from the things he does (and says)?

\item
% 429.
What reason have I, now, when I cannot see my toes, to assume that I have five
toes on each foot?  Is it right to say that my reason is that previous
experience has always taught me so? Am I more certain of previous experience
than that I have ten toes?  That previous experience may very well be the cause
of my present certitude; but is it its ground?

\item
% 430.
I meet someone from Mars and he asks me ``How many toes have human beings
got?'' - I say ``Ten. I'll show you'', and take my shoes off. Suppose he was
surprised that I knew with such certainty, although I hadn't looked at my toes
- ought I to say: ``We humans know how many toes we have whether we can see
them or not''?

\item
% 431.
``I know that this room is on the second floor, that behind the door a short
landing leads to the stairs, and so on.'' One could imagine cases where I
should come out with this, but they would be extremely rare. But on the other
hand I show this knowledge day in, day out by my actions and also in what I
say.  Now what does someone else gather from these actions and words of mine?
Won't it be just that I am sure of my ground? - From the fact that I have been
living here for many weeks and have gone up and down the stairs every day he
will gather that I know where my room is situated. - I shall give him the
assurance ``I know'' when he does not already know things which would have
compelled the conclusion that I knew.

\item
% 432.
The utterance ``I know...'' can only have its meaning in connection with the
other evidence of my `knowing'.

\item
% 433.
So if I say to someone ``I know that that's a tree'', it is also as if I told
him ``that is a tree; you can absolutely rely on it; there is no doubt about
it''. And a philosopher could only use the statement to show that this form of
speech is actually used. But if his use of it is not to be merely an
observation about English grammar, he must give the circumstances in which this
expression functions.

\item
% 434.
Now does experience teach us that in such-and-such circumstances people know
this and that? Certainly, experience shows us that normally after so-and-so
many days a man can find his way about a house he has been living in. Or even:
experience teaches us that after such-and-such a period of training a man's
judgment is to be trusted. He must, experience tells us, have learnt for so
long in order to be able to make a correct prediction. But -----.

\item
% 435.
One is often bewitched by a word. For example, by the word ``know''.

\item
% 436.
Is God bound by our knowledge? Are a lot of our statements incapable of
falsehood? For that is what we want to say.

\item
% 437.
I am inclined to say: ``That cannot be false.'' That is interesting. But what
consequences has it?

\item
% 438.
It would not be enough to assure someone that I know what is going on at a
certain place - without giving him grounds that satisfy him that I am in a
position to know.

\item
% 439.
Even the statement ``I know that behind this door there is a landing and the
stairway down to the ground floor'' only sounds so convincing because everyone
takes it for granted that I know it.

\item
% 440.
There is something universal here; not just something personal.

\item
% 441.
In a court of law the mere assurance ``I know...'' on the part of a witness
would convince no one. It must be shown that he was in a position to know.
Even the assurance ``I know that that's a hand'', said while someone looked at
his own hand, would not be credible unless we knew the circumstances in which
it was said. And if we do know them, it seems to be an assurance that the
person speaking is normal in this respect.

\item
% 442.
For may it not happen that I imagine myself to know something?

\item
% 443.
Suppose that in a certain language there were no word corresponding to our
``know''. - The people simply make assertions. (``That is a tree'', etc.)
Naturally it can occur for them to make mistakes. And so they attach a sign to
the sentence which indicates how probable they take a mistake to be - or should
I say, how probable a mistake is in this case? This latter can also be
indicated by mentioning certain circumstances. For example ``Then A said to B
'...' I was standing quite close to them and my hearing is good'', or ``A was
at such-and-such a place yesterday. I saw him from a long way off. My eyes are
not very good'', or ``There is a tree over there: I can see it clearly and I
have seen it innumerable times before''.

\item
% 444.
``The train leaves at two o'clock. Check it once more to make certain'' or
``The train leaves at two o'clock. I have just looked it up in a new
time-table.'' One may also add ``I am reliable in such matters''. The
usefulness of such additions is obvious.

\item
% 445.
But if I say ``I have two hands'', what can I add to indicate reliability? At
the most that the circumstances are the ordinary ones.

\item
% 446.
But why am I so certain that this is my hand? Doesn't the whole language-game
rest on this kind of certainty?  Or: isn't this `certainty' (already)
presupposed in the language-game? Namely by virtue of the fact that one is not
playing the game, or is playing it wrong, if one does not recognize objects
with certainty.

\item
% 447.
Compare with this 12x12=144. Here too we don't say ``perhaps''. For, in so far
as this proposition rests on our not miscounting or miscalculating and on our
senses not deceiving us as we calculate, both propositions, the arithmetical
one and the physical one, are on the same level.  I want to say: The physical
game is just as certain as the arithmetical. But this can be misunderstood. My
remark is a logical and not a psychological one.

\item
% 448.
I want to say: If one doesn't marvel at the fact that the propositions of
arithmetic (e.g. the multiplication tables) are `absolutely certain', then why
should one be astonished that the proposition ``This is my hand'' is so
equally?

\item
% 449.
Something must be taught us as a foundation.

\item
% 450.
I want to say: our learning has the form ``that is a violet'', ``that is a
table''. Admittedly, the child might hear the word ``violet'' for the first
time in the sentence ``perhaps that is a violet'', but then he could ask ``what
is a violet?'' Now this might of course be answered by showing him a picture.
But how would it be if one said ``that is a...'' only when showing him a
picture, but otherwise said nothing but ``perhaps that is a...'' - What
practical consequences is that supposed to have?  A doubt that doubted
everything would not be a doubt.

\item
% 451.
My objection against Moore, that the meaning of the isolated sentence ``That is
a tree'' is undetermined, since it is not determined what the ``that'' is that
is said to be a tree - doesn't work, for one can make the meaning more definite
by saying, for example: ``The object over there that looks like a tree is not
an artificial imitation of a tree but a real one.''

\item
% 452.
It would not be reasonable to doubt if that was a real tree or only...  My
finding it beyond doubt is not what counts. If a doubt would be unreasonable,
that cannot be seen from what I hold. There would therefore have to be a rule
that declares doubt to be unreasonable here. But there isn't such a rule,
either.

\item
% 453.
I do indeed say: ``Here no reasonable person would doubt.'' - Could we imagine
learned judges being asked whether a doubt was reasonable or unreasonable?

\item
% 454.
There are cases where doubt is unreasonable, but others where it seems
logically impossible. And there seems to be no clear boundary between them.

\item
% 455.
Every language-game is based on words `and objects' being recognized again. We
learn with the same inexorability that is a chair as that 2x2=4.

\item
% 456.
If, therefore, I doubt or am uncertain about this being my hand (in whatever
sense), why not in that case about the meaning of these words as well?

\item
% 457.
Do I want to say, then, that certainty resides in the nature of the
language-game?

\item
% 458.
One doubts on specific grounds. The question is this: how is doubt introduced
into the language-game?

\item
% 459.
If the shopkeeper wanted to investigate each of his apples without any reason,
for the sake of being certain about everything, why doesn't he have to
investigate the investigation? And can one talk of belief here (I mean belief
as in `religious belief', not surmise)? All psychological terms merely distract
us from the thing that really matters.

\item
% 460.
I go to the doctor, show him my hand and say ``This is a hand, not...; I've
injured it, etc.,etc.'' Am I only giving him a piece of superfluous
information? For example, mightn't one say: supposing the words ``This is a
hand'' were a piece of information - how could you bank on his understanding
this information? Indeed, if it is open to doubt `whether that is a hand', why
isn't it also open to doubt whether I am a human being who is informing the
doctor of this? - But on the other hand one can imagine cases - even if they
are very rare ones - where this declaration is not superfluous, or is only
superfluous but not absurd.

\item
% 461.
Suppose that I were the doctor and a patient came to me, showed me his hand and
said: ``This thing that looks like a hand isn't just a superb imitation - it
really is a hand'' and went on to talk about his injury - should I really take
this as a piece of information, even though a superfluous one? Shouldn't I be
more likely to consider it nonsense, which admittedly did have the form of a
piece of information? For, I should say, if this information really were
meaningful, how can he be certain of what he says? The background is lacking
for it to be information.

\item
% 462.
Why doesn't Moore produce as one of the things that he knows, for example, that
is such-and-such a part of England there is a village called so-and-so? In
other words: why doesn't he mention a fact that is known to him and not to
every one of us?

\item
% 463.
This is certainly true, that the information ``That is a tree'', when no one
could doubt it, might be a kind of joke and as such have meaning. A joke of
this kind was in fact made once by Renan.

\item
% 464.
My difficulty can also be shown like this: I am sitting talking to a friend.
Suddenly I say: ``I knew all along that you were so-and-so.'' Is that really
just a superfluous, though true, remark?  I feel as if these words were like
``Good morning'' said to someone in the middle of a conversation.

\item
% 465.
How would it be if we had the words ``They know nowadays that there are
over...species of insects'' instead of ``I know that that's a tree''? If
someone were suddenly to utter the first sentence out of all context one might
think: he has been thinking of something else in the interim and is now saying
out loud some sentence in his train of thought. Or again: he is in a trance and
is speaking without understanding what he is saying.

\item
% 466.
Thus it seems to me that I have known something the whole time, and yet there
is no meaning in saying so, in uttering this truth.

\item
% 467.
I am sitting with a philosopher in the garden; he says again and again ``I know
that that's a tree'', pointing to a tree that is near us. Someone else arrives
and hears this, and I tell him: ``This fellow isn't insane. We are only doing
philosophy.''

\item
% 468.
Someone says irrelevantly ``That's a tree''. He might say this sentence because
he remembers having heard it in a similar situation; or he was suddenly struck
by the tree's beauty and the sentence was an exclamation; or he was pronouncing
the sentence to himself as a grammatical example;etc.,etc. And now I ask him
``How did you mean that?'' and he replies ``It was a piece of information
directed at you.'' Shouldn't I be at liberty to assume that he doesn't know
what he is saying, if he is insane enough to want to give me this information?

\item
% 469.
In the middle of a conversation, someone says to me out of the blue: ``I wish
you luck.'' I am astonished; but later I realize that these words connect up
with his thoughts about me. And now they do not strike me as meaningless any
more.

\item
% 470.
Why is there no doubt that I am called L.W.? It does not seem at all like
something that one could establish at once beyond doubt. One would not think
that it is one of the indubitable truths.
\\
\emph{Here there is still a big gap in my thinking. And I doubt whether it will
be filled now.}

\item
% 471.
It is so difficult to find the beginning. Or, better: it is difficult to begin
at the beginning. And not try to go further back.

\item
% 472.
When a child learns language it learns at the same time what is to be
investigated and what not. When it learns that there is a cupboard in the room,
it isn't taught to doubt whether what it sees later on is still a cupboard or
only a kind of stage set.

\item
% 473.
Just as in writing we learn a particular basic form of letters and then vary it
later, so we learn first the stability of things as the norm, which is then
subject to alterations.

\item
% 474.
This games proves its worth. That may be the cause of its being played, but it
is not the ground.

\item
% 475.
I want to regard man here as an animal; as a primitive being to which one
grants instinct but not ratiocination. As a creature in a primitive state. Any
logic good enough for a primitive means of communication needs no apology from
us. Language did not emerge from some kind of ratiocination
\emph{Raisonnement}.

\item
% 476.
Children do not learn that books exist, that armchairs exist, etc.,etc. - they
learn to fetch books, sit in armchairs, etc.,etc.  Later, questions about the
existence of things do of course arise, ``Is there such a thing as a unicorn?''
and so on. But such a question is possible only because as a rule no
corresponding question presents itself. For how does one know how to set about
satisfying oneself of the existence of unicorns? How did one learn the method
for determining whether something exists or not?

\item
% 477.
``So one must know that the objects whose names one teaches a child by an
ostensive definition exist.'' - Why must one know they do? Isn't it enough that
experience doesn't later show the opposite?  For why should the language-game
rest on some kind of knowledge?

\item
% 478.
Does a child believe that milk exists? Or does it know that milk exists? Does a
cat know that a mouse exists?

\item
% 479.
Are we to say that the knowledge that there are physical objects comes very
early or very late?

\item
% 480.
A child that is learning to use the word ``tree''. One stands with it in front
of a tree and says ``Lovely tree!'' Clearly no doubt as to the tree's existence
comes into the language-game. But can the child be said to know: `that a tree
exists'? Admittedly it's true that `knowing something' doesn't involve thinking
about it - but mustn't anyone who knows something be capable of doubt? And
doubting means thinking.

\item
% 481.
When one hears Moore say ``I know that that's a tree'', one suddenly
understands those who think that that has by no means been settled.  The matter
strikes one all at once as being unclear and blurred. It is as if Moore had put
it in the wrong light.  It is as if I were to see a painting (say a painted
stage-set) and recognize what it represents from a long way off at once and
without the slightest doubt. But now I step nearer: and then I see a lot of
patches of different colours, which are all highly ambiguous and do not provide
any certainty whatever.

\item
% 482.
It is as if ``I know'' did not tolerate a metaphysical emphasis.

\item
% 483.
The correct use of the expression ``I know''. Someone with bad sight asks me:
``do you believe that the thing we can see there is a tree?'' I reply ``I know
it is; I can see it clearly and am familiar with it.'' - A: ``Isn't N.N. at
home?'' - I: ``I believe he is.'' - A: ``Was he at home yesterday?'' - I;
``Yesterday he was - I know he was; I spoke to him.'' - A: ``Do you know or
only believe that this part of the house is built on later than the rest?'' -
I: ``I know it is; I got it from so and so.''

\item
% 484.
In these cases, then, one says ``I know'' and mentions how one knows, or at
least one can do so.

\item
% 485.
We can also imagine a case where someone goes through a list of propositions
and as he does so keeps asking ``Do I know that or do I only believe it?'' He
wants to check the certainty of each individual proposition. It might be a
question of making a statement as a witness before a court.

\item
% 486.
``Do you know or do you only believe that your name is L.W.?'' Is that a
meaningful question?  Do you know or do you only believe that what you are
writing down now are German words? Do you only believe that ``believe'' has
this meaning? What meaning?

\item
% 487.
What is the proof that I know something? Most certainly not my saying I know
it.

\item
% 488.
And so, when writers enumerate all the things they know, that proves nothing
whatever.  So the possibility of knowledge about physical objects cannot be
proved by the protestations of those who believe that they have such knowledge.

\item
% 489.
For what reply does one make to someone who says ``I believe it merely strikes
you as if you knew it''?

\item
% 490.
When I ask ``Do I know or do I only believe that I am called...?'' it is no use
to look within myself.  But I could say: not only do I never have the slightest
doubt that I am called that, but there is no judgement I could be certain of if
I started doubting about that.

\item
% 491.
``Do I know or do I only believe that I am called L.W.?'' - Of course, if the
question were ``Am I certain or do I only surmise...?'', then my answer could
be relied on.

\item
% 492.
``Do I know or do I only believe...?'' might also be expressed like this: What
if it seemed to turn out that what until now has seemed immune to doubt was a
false assumption? Would I react as I do when a belief has proved to be false?
or would it seem to knock from under my feet the ground on which I stand in
making any judgements at all? - But of course I do not intend this as a
prophecy.  Would I simply say ``I should never have thought it!'' - or would I
(have to) refuse to revise my judgement - because such a `revision' would
amount to annihilation of all yardsticks?

\item
% 493.
So is this it: I must recognize certain authorities in order to make judgements
at all?

\item
% 494.
``I cannot doubt this proposition without giving up all judgement.'' But what
sort of proposition is that? (It is reminiscent of what Frege said about the
law of identity.) It is certainly no empirical proposition. It does not belong
to psychology. It has rather the character of a rule.

\item
% 495.
One might simply say ``O, rubbish!'' to someone who wanted to make objections
to the propositions that are beyond doubt. That is, not reply to him but
admonish him.

\item
% 496.
This is a similar case to that of showing that it has no meaning to say that a
game has always been played wrong.

\item
% 497.
If someone wanted to arouse doubts in me and spoke like this: here your memory
is deceiving you, there you've been taken in, there again you have not been
thorough enough in satisfying yourself, etc., and if I did not allow myself to
be shaken but kept to my certainty - then my doing so cannot be wrong, even if
only because this is just what defines a game.

\item
% 498.
The queer thing is that even though I find it quite correct for someone to say
``Rubbish!'' and so brush aside the attempt to confuse him with doubts at
bedrock, - nevertheless, I hold it to be incorrect if he seeks to defend
himself (using, e.g., the words ``I know'').

\item
% 499.
I might also put it like this: the `law of induction' can no more be grounded
than certain particular propositions concerning the material of experience.

\item
% 500.
But it would also strike me as nonsense to say ``I know that the law of
induction is true''.  Imagine such a statement made in a court of law! It would
be more correct to say ``I believe in the law of...'' where `believe' has
nothing to do with surmising.

\item
% 501.
Am I not getting closer and closer to saying that in the end logic cannot be
described? You must look at the practice of language, then you will see it.

\item
% 502.
Could one say ``I know the position of my hands with my eyes closed'', if the
position I gave always or mostly contradicted the evidence of other people?

\item
% 503.
I look at an object and say ``That is a tree'', or ``I know that that's a
tree''. - Now if I go nearer and it turns out that it isn't, I may say ``It
wasn't a tree at all'' or alternatively I say ``It was a tree but now it isn't
any longer''. But if all the others contradicted me, and said it never had been
a tree, and if all the other evidences spoke against me - what good would it do
to me to stick to my ``I know''?

\item
% 504.
Whether I know something depends on whether the evidence backs me up or
contradicts me. For to say one knows one has a pain means nothing.

\item
% 505.
It is always by favour of Nature that one knows something.

\item
% 506.
``If my memory deceives me here it can deceive me everywhere.'' If I don't know
that, how do I know if my words mean what I believe they mean?

\item
% 507.
``If this deceives me, what does `deceive' mean any more?''

\item
% 508.
What can I rely on?

\item
% 509.
I really want to say that a language-game is only possible if one trusts
something (I did not say ``can trust something'').

\item
% 510.
If I say ``Of course I know that that's a towel'' I am making an utterance. I
have no thought of a verification. For me it is an immediate utterance.  I
don't think of past or future. (And of course it's the same for Moore, too.) It
is just like directly taking hold of something, as I take hold of my towel
without having doubts.

\item
% 511.
And yet this direct taking-hold corresponds to a sureness, not to a knowing.
But don't I take hold of a thing's name like that, too?

\item
% 512.
Isn't the question this: ``What if you had to change your opinion even on these
most fundamental things?'' And to that the answer seems to me to be: ``You
don't have to change it. That is just what their being `fundamental' is.''

\item
% 513.
What if something really unheard-of happened? - If I, say, saw houses gradually
turning into steam without any obvious cause, it the cattle in the fields stood
on their heads and laughed and spoke comprehensible words; if trees gradually
changed into men and men into trees. Now, was I right when I said before all
these things happened ``I know that that's a house'' etc., or simply ``that's a
house'' etc.?

\item
% 514.
This statement appeared to me fundamental; if it is false, what are `true' and
`false' any more?!

\item
% 515.
If my name is not L.W., how can I rely on what is meant by ``true'' and
``false''?

\item
% 516.
If something happened (such as someone telling me something) calculated to make
me doubtful of my own name, there would certainly also be something that made
the grounds of these doubts themselves seem doubtful, and I could therefore
decide to retain my old belief.

\item
% 517.
But might it not be possible for something to happen that threw me entirely off
the rails? Evidence that made the most certain thing unacceptable to me? Or at
any rate made me throw over my most fundamental judgements? (Whether rightly or
wrongly is beside the point.)

\item
% 518.
Could I imagine observing this in another person?

\item
% 519.
Admittedly, if you are obeying the order ``Bring me a book'', you may have to
check whether the thing you see over there really is a book, but then you do at
least know what people mean by a ``book''; and if you don't you can look it up,
- but then you must know what some other word means. And the fact that a word
means such-and-such, is used in such-and-such a way, is in turn an empirical
fact, like the fact that what you see over there is a book.  Therefore, in
order for you to be able to carry out an order there must be some empirical
fact about which you are not in doubt. Doubt itself rests only on what is
beyond doubt.  But since a language-game is something that consists in the
recurrent procedures of the game in time, it seems impossible to say in any
individual case that such-and-such must be beyond doubt if there is to be a
language-game - though it is right enough to say that as a rule some empirical
judgment or other must be beyond doubt.

\item
% 520.
Moore has every right to say he knows there's a tree there in front of him.
Naturally he may be wrong. (For it is not the same as with the utterance ``I
believe there is a tree there''.) But whether he is right or wrong in this case
is of no philosophical importance. If Moore is attacking those who say that one
cannot really know such a thing, he can't do it by assuring them that he knows
this and that. For one need not believe him. If his opponents had asserted that
one could not believe this and that, then he could have replied: ``I believe
it''.

\item

521.
Moore's mistake lies in this - countering the assertion that one cannot know
that, by saying ``I do know it''.

\item
% 522.
We say: if a child has mastered language - and hence its application - it must
know the meaning of words. It must, for example, be able to attach the name of
its colour to white, black, red or blue object without the occurrence of any
doubt.

\item
% 523.
And indeed no one misses doubt here; no one is surprised that we do not merely
surmise the meaning of our words.

\item
% 524.
Is it essential for our language-games ('ordering and obeying' for example)
that no doubt appears at certain points, or is it enough if there is the
feeling of being sure, admittedly with a slight breath of doubt?  That is, is
it enough if I do not, as I do now, call something `black', `green', `red',
straight off, without any doubt at all interposing itself - but do instead say
``I am sure that is red'', as one may say ``I am sure that he will come today''
(in other words with the `feeling of being sure')?  The accompanying feeling is
of course a matter of indifference to us, and equally we have no need to bother
about the words ``I am sure that'' either. - What is important is whether they
go with a difference in the practice of the language.  One might ask whether a
person who spoke like this would always say ``I am sure'' on occasions where
(for example) there is sureness in the reports we make ( in an experiment, for
example, we look through a tube and report the colour we see through it). If he
does, our immediate inclination will be to check what he says. But if he proves
to be perfectly reliable, one will say that his way of talking is merely a bit
perverse, and does not affect the issue. One might for example suppose that he
has read sceptical philosophers, become convinced that one can know nothing,
and that is why he has adopted this way of speaking. Once we are used to it, it
does not infect practice.

\item
% 525.
What, then, does the case look like where someone really has got a different
relationship to the names of colours, for example, from us? Where, that is,
there persists a slight doubt or a possibility of doubt in their use.

\item
% 526.
If someone were to look at an English pillar-box and say ``I am sure that it's
red'', we should have to suppose that he was colour-blind, or believe he had no
mastery of English and knew the correct name for the colour in some other
language.  If neither was the case we should not quite understand him.

\item
% 527.
An Englishman who calls this colour ``red'' is not `sure it is called ``red''
in English'.  A child who has mastered the use of the word is not `sure that in
his language this colour is called...'. Nor can one say of him that when he is
learning to speak he learns that the colour is called that in English; not yet
: he knows this when he has learnt the use of the word.

\item
% 528.
And in spite of this: if someone asked me what the colour was called in German
and I tell him, and now he asks me ``are you sure?'' - then I shall reply ``I
know it is; German is my mother tongue''.

\item
% 529.
And one child, for example, will say, of another or of himself, that he already
knows what such-and-such is called.

\item
% 530.
I may tell someone ``this colour is called `red' in English'' (when for example
I am teaching him English). In this case I should not say ``I know that this
colour...'' - I would perhaps say that if I had just now learned it, or by
contrast with another colour whose English name I am not acquainted with.

\item
% 531.
But now, isn't it correct to describe my present state as follows: I know what
this colour is called in English? And if that is correct, why then should I not
describe my state with the corresponding words ``I know etc.''?

\item
% 532.
So when Moore sat in front of a tree and said ``I know that that's a tree'', he
was simply stating the truth about this state at the time.
\\
\emph{I do philosophy now like an old woman who is always mislaying something
and having to look for it again: now her spectacles, now her keys.}

\item
% 533.
Well, if it was correct to describe his state out of context, then it was just
as correct to utter the words ``that's a tree'' out of context.

\item
% 534.
But is it wrong to say: ``A child that has mastered a language-game must know
certain things''?  If instead of that one said ``must be able to do certain
things'', that would be a pleonasm, yet this is just what I want to counter the
first sentence with. - But : ``a child acquires a knowledge of natural
history''. That presupposes that it can ask what such and such a plant is
called.

\item
% 535.
The child knows what something is called if he can reply correctly to the
question ``what is that called?''

\item
% 536.
Naturally, the child who is just learning to speak has not yet got the concept
is called at all.

\item
% 537.
Can one say of someone who hasn't this concept that he knows what such-and-such
is called?

\item
% 538.
The child, I should like to say, learns to react in such-and-such a way; and in
so reacting it doesn't so far know anything. Knowing only begins at a later
lever.

\item
% 539.
Does it go for knowing as it does for collecting?

\item
% 540.
A dog might learn to run to N at the call ``N'', and to M at the call ``M'', -
but would that mean he knows what these people are called?

\item
% 541.
``He only knows what this person is called - not yet what that person is
called''. That is something one cannot, strictly speaking, say of someone who
simply has not yet got the concept of people's having names.

\item
% 542.
``I can't describe this flower if I don't know that this colour is called
`red'. ''

\item
% 543.
A child can use the names of people long before he can say in any form
whatever: ``I know this one's name; I don't know that one's yet.''

\item
% 544.
Of course I may truthfully say ``I know what this colour is called in
English'', at the same time as I point (for example) to the colour of fresh
blood. But ---

\item
% 545.
`A child knows which colour is meant by the word ``blue''.' What he knows here
is not all that simple.

\item
% 546.
I should say ``I know what this colour is called'' if e.g. what is in question
is shades of colour whose name not everybody knows.

\item
% 547.
One can't yet say to a child who is just beginning to speak and can use the
words ``red'' and ``blue'': ``Come on, you know what this colour is called!''

\item
% 548.
A child must learn the use of colour words before it can ask for the name of a
colour.

\item
% 549.
It would be wrong to say that I can only say ``I know that there is a chair
there'' when there is a chair there. Of course it isn't true unless there is,
but I have a right to say this if I am sure there is a chair there, even if I
am wrong.
\\
\emph{Pretensions are a mortgage which burdens a philosopher's capacity to
think.}

\item
% 550.
If someone believes something, we needn't always be able to answer the question
`why he believes it'; but if he knows something, then the question ``how does
he know?'' must be capable of being answered.

\item
% 551.
And if one does answer this question, one must do so according to generally
accepted axioms. This is how something of this sort may be known.

\item
% 552.
Do I know that I am now sitting in a chair? - Don't I know it?! In the present
circumstances no one is going to say that I know this; but no more will he say,
for example, that I am conscious. Nor will one normally say that of the
passers-by in the street.  But now, even if one doesn't say it, does that make
it untrue??

\item
% 553.
It is queer: if I say, without any special occasion, ``I know'' - for example,
``I know that I am now sitting in a chair'', this statement seems to me
unjustified and presumptuous. But if I make the same statement where there is
some need for it, then, although I am not a jot more certain of its truth, it
seems to me to be perfectly justified and everyday.

\item
% 554.
In its language-game it is not presumptuous. There, it has no higher position
than, simply, the human language-game. For there it has its restricted
application.  But as soon as I say this sentence outside its context, it
appears in a false light. For then it is as if I wanted to insist that there
are things that I know. God himself can't say anything to me about them.

\item
% 555.
We say we know that water boils when it is put over a fire. How do we know?
Experience has taught us. - I say ``I know that I had breakfeast this
morning''; experience hasn't taught me that. One also says ``I know that he is
in pain''. The language-game is different every time, we are sure every time,
and people will agree with us that we are in a position to know every time. And
that is why the propositions of physics are found in textbooks for everyone.
If someone says he know something, it must be something that, by general
consent, he is in a position to know.

\item
% 556.
One doesn't say: he is in a position to believe that.  But one does say: ``It
is reasonable to assume that in this situation'' (or ``to believe that'').

\item
% 557.
A court-martial may well have to decide whether it was reasonable in
such-and-such a situation to have assumed this or that with confidence (even
thought wrongly).

\item
% 558.
We say we know that water boils and does not freeze under such-and-such
circumstances. Is it conceivable that we are wrong? Wouldn't a mistake topple
all judgment with it? More: what could stand if that were to fall? Might
someone discover something that made us say ``It was a mistake''?  Whatever may
happen in the future, however water may behave in the future, - we know that up
to now it has behaved thus in innumerable instances.  This fact is fused into
the foundations of our language-game.

\item
% 559.
You must bear in mind that the language-game is so to say something
unpredictable. I mean: it is not based on grounds. It is not reasonable (or
unreasonable).  It is there - like our life.

\item
% 560.
And the concept of knowing is coupled with that of the language-game.

\item
% 561.
``I know'' and ``You can rely on it''. But one cannot always substitute the
latter for the former.

\item
% 562.
At any rate it is important to imagine a language in which our concept
`knowledge' does not exist.

\item
% 563.
One says ``I know that he is in pain'' although one can produce no convincing
grounds for this. - Is this the same as ``I am sure that he...''? - No, ``I am
sure'' tells you my subjective certainty. ``I know'' means that I who know it,
and the person who doesn't are separated by a difference in understanding.
(Perhaps based on a difference in degree of experience.) If I say ``I know'' in
mathematics, the justification for this is a proof.
\\
If in these two cases instead of ``I know'', one says ``you can rely on it'' then the
substantiation is of a different kind in each case.  And substantiation comes
to an end.

\item
% 564.
A language-game: bringing building stones, reporting the number of available
stones. The number is sometimes estimated, sometimes established by counting.
Then the question arises ``Do you believe there are as many stones as that?'',
and the answer ``I know there are - I've just counted them''. But here the ``I
know'' could be dropped. If, however, there are several ways of finding
something out for sure, like counting, weighing, measuring the stack, then the
statement ``I know'' can take the place of mentioning how I know.

\item
% 565.
But here there isn't yet any question of any ``knowledge'' that this is called
``a slab'', this ``a pillar'', etc.

\item
% 566.
Nor does a child who learns my language-game (PI No.2) learn to say ``I know
that this is called `a slab' ''.  Now of course there is a language-game in
which the child uses that sentence. This presupposes that the child is already
capable of using the name as soon as he is given it. (As if someone were to
tell me ``this colour is called...'') - Thus, if the child has learnt a
language-game with building stones, one can say something like ``and this stone
is called...'', and in this way the original language-game has been expanded.

\item
% 567.
And now, is my knowledge that I am called L.W. of the same kind as knowledge
that water boils at 100C? Of course, this question is wrongly put.

\item
% 568.
If one of my names were used only very rarely, then it might happen that I did
not know it. It goes without saying that I know my name, only because, like
anyone else, I use it over and over again.

\item
% 569.
An inner experience cannot show me that I know something.  Hence, if in spite
of that I say, ``I know that my name is...'', and yet it is obviously not an
empirical proposition,---

\item
% 570.
``I know this is my name; among us any grown-up knows what his name is.''

\item
% 571.
``My name is... - you can rely on that. If it turns out to be wrong you need
never believe me in the future.''

\item
% 572.
Don't I seem to know that I can't be wrong about such a thing as my own name?
This comes out in the words: ``If that is wrong, then I am crazy''. Very well,
but those are words; but what influence has it on the application of language?

\item
% 573.
Is it through the impossibility of anything's convincing me of the contrary?

\item
% 574.
The question is, what kind of proposition is: ``I know I can't be mistaken
about that'', or again ``I can't be mistaken about that''?  This ``I know''
seems to prescind from all grounds: I simply know it. But if there can be any
question at all of being mistaken here, then it must be possible to test
whether I know it.

\item
% 575.
Thus the purpose of the phrase ``I know'' might be to indicate where I can be
relied on; but where that's what it's doing, the usefulness of this sign must
emerge from experience.

\item
% 576.
One might say ``How do I know that I'm not mistaken about my name?'' - and if
the reply was ``Because I have used it so often'', one might go on to ask ``How
do I know that I am not mistaken about that?'' And here the ``How do I know''
cannot any longer have any significance.

\item
% 577.
``My knowledge of my name is absolutely definite.'' I would refuse to entertain
any argument that tried to show the opposite!  And what does ``I would refuse''
mean? Is it the expression of an intention?

\item
% 578.
But mightn't a higher authority assure me that I don't know the truth? So that
I had to say ``Teach me!'' ? But then my eyes would have to be opened.

\item
% 579.
It is part of the language-game with people's names that everyone knows his
name with the greatest certainty.

\item
% 580.
It might surely happen that whenever I said ``I know'' it turned out to be
wrong. (Showing up.)

\item
% 581.
But perhaps I might nevertheless be unable to help myself, so that I kept on
declaring ``I know...''. But ask yourself: how did the child learn the
expression?

\item
% 582.
``I know that'' may mean; I am quite familiar with it - or again: it is
certainly so.

\item
% 583.
``I know that the name of this in...is...'' - How do you know? - ``I have
learnt...''.  Could I substitute ``In...the name of this is...'' for ``I know
etc'' in this example?

\item
% 584.
Would it be possible to make use of the verb ``know'' only in the question
``How do you know?'' following a simple assertion? - Instead of ``I already
know that'' one says ``I am familiar with that''; and this follows only upon
being told the fact. But what does one say instead of ``I know what that is''?

\item
% 585.
But doesn't ``I know that that's a tree'' say something different from ``that
is a tree''?

\item
% 586.
Instead of ``I know what that is'' one might say ``I can say what that is''.
And if one adopted this form of expression what would then become of ``I know
that that is a...''?

\item
% 587.
Back to the question whether ``I know that that's a...'' says anything
different from ``that is a...''. In the first sentence a person is mentioned,
in the second, not. But that does not show that they have different meanings.
At all events one often replaces the first form by the second, and then often
gives the latter a special intonation. For one speaks differently when one
makes an uncontradicted assertion from when one maintains an assertion in face
of contradiction.

\item
% 588.
But don't I use the words ``I know that...'' to say that I am in a certain
state, whereas the mere assertion ``that is a...'' does not say this? And yet
one often does reply to such an assertion by asking ``how do you know?'' -
``But surely, only because the fact that I assert this gives to understand that
I think I know it.'' - This point could be made in the following way: in a zoo
there might be a notice ``this is a zebra''; but never ``I know that this is a
zebra''.  ``I know'' has meaning only when it is uttered by a person. But,
given that, it is a matter of indifference whether what is uttered is ``I
know...'' or ``That is...''.

\item
% 589.
For how does a man learn to recognize his own state of knowing something?

\item
% 590.
At most one might speak of recognizing a state, where what is said is ``I know
what that is''. Here one can satisfy oneself that one really is in possession
of this knowledge.

\item
% 591.
``I know what kind of tree that is. - It is a chestnut.'' ``I know what kind of
tree that is. - I know it's a chestnut.'' The first statement sounds more
natural than the second. One will only say ``I know'' a second time if one
wants especially to emphasize certainty; perhaps to anticipate being
contradicted. The first ``I know'' means roughly: I can say.  But in another
case one might begin with the observation ``that's a...'', and then, when this
is contradicted, counter by saying: ``I know what sort of tree it is'', and by
this means lay emphasis on being sure.

\item
% 592.
``I can tell you what kind of a... that is, and no doubt about it.''

\item
% 593.
Even when one can replace ``I know'' by ``It is...'' still one cannot replace
the negation of the one by the negation of the other.  With ``I don't know...''
a new element enters our language-games.

\item
% 594.
My name is ``L.W.'' And if someone were to dispute it, I should straightaway
make connexions with innumerable things which make it certain.

\item
% 595.
``But I can still imagine someone making all these connexions, and none of them
corresponding with reality. Why shouldn't I be in a similar case?'' If I
imagine such a person I also imagine a reality, a world that surrounds him; and
I imagine him as thinking (and speaking) in contradiction to this world.

\item
% 596.
If someone tells me his name is N.N., it is meaningful for me to ask him ``Can
you be mistaken?'' That is an allowable question in the language-game. And the
answer to it, yes or no, makes sense. - Now of course this answer is not
infallible either, i.e., there might be a time when it proved to be wrong, but
that does not deprive the question ``Can you be...'' and the answer ``No'' of
their meaning.

\item
% 597.
The reply to the question ``Can you be mistaken?'' gives the statement a
definite weight. The answer may also be: ``I don't think so.''

\item
% 598.
But couldn't one reply to the question ``Can you...'' by saying: ``I will
describe the case to you and then you can judge for yourself whether I can be
mistaken''?  For example, if it were a question of someone's own name, the fact
might be that he had never used this name, but remembered he had read it on
some document, - but on the other hand the answer might be: ``I've had this
name my whole life long, I've been called it by everybody.'' If that is not
equivalent to the answer ``I can't be mistaken'', then the latter has no
meaning whatever. And yet quite obviously it points to a very important
distinction.

\item
% 599.
For example one could describe the certainty of the proposition that water
boils at circa 100C. That isn't e.g. a proposition I have once heard (like this
or that, which I could mention). I made the experiment myself at school. The
proposition is a very elementary one in our text-books, which are to be trusted
in matters like this because... - Now one can offer counter-examples to all
this, which show that human beings have held this and that to be certain which
later, according to our opinion, proved false. But the argument is worthless.
\emph{May it not also happen that we believe we recognize a mistake of earlier
times and later come to the conclusion that the first opinion was the right
one? etc.} To say: in the end we can only adduce such grounds as we hold to be
grounds, is to say nothing at all.  I believe that at the bottom of this is a
misunderstanding of the nature of our language-games.

\item
% 600.
What kind of grounds have I for trusting text-books of experimental physics?  I
have no grounds for not trusting them. And I trust them. I know how such books
are produced - or rather, I believe I know. I have some evidence, but it does
not go very far and is of a very scattered nature. I have heard, seen and read
various things.

\item
% 601.
There is always the danger of wanting to find an expression's meaning by
contemplating the expression itself, and the frame of mind in which one uses
it, instead of always thinking of the practice. That is why one repeats the
expression to oneself so often, because it is as if one must see what one is
looking for in the expression and in the feeling it gives one.

\item

602.
Should I say ``I believe in physics'', or ``I know that physics is true''?

\item
% 603.
I am taught that under such circumstances this happens. It has been discovered
by making the experiment a few times. Not that that would prove anything to us,
if it weren't that this experience was surrounded by others which combine with
it to form a system. Thus, people did not make experiments just about falling
bodies but also about air resistence and all sorts of other things.  But in the
end I rely on these experiences, or on the reports of them, I feel no scruples
about ordering my own activities in accordance with them. - But hasn't this
trust also proved itself? So far as I can judge - yes.

\item
% 604.
In a court of law the statement of a physicist that water boils at about 100C
would be accepted unconditionally as truth.  If I mistrusted this statement
what could I do to undermine it? Set up experiments myself? What would they
prove?

\item
% 605.
But what if the physicist's statement were superstition and it were just as
absurd to go by it in reaching a verdict as to rely on ordeal by fire?

\item
% 606.
That to my mind someone else has been wrong is no ground for assuming that I am
wrong now. - But isn't it a ground for assuming that I might be wrong? It is no
ground for any unsureness in my judgement, or my actions.

\item
% 607.
A judge might even say ``That is the truth - so far as a human being can know
it.'' But what would this rider \emph{Zusatz} achieve? (``beyond all reasonable
doubt'').

\item
% 608.
Is it wrong for me to be guided in my actions by the propositions of physics?
Am I to say I have no good ground for doing so? Isn't precisely this what we
call a `good ground'?

\item
% 609.
Supposing we met people who did not regard that as a telling reason. Now, how
do we imagine this? Instead of the physicist, they consult an oracle. (And for
that we consider them primitive.) Is it wrong for them to consult an oracle and
be guided by it? - If we call this ``wrong'' aren't we using our language-game
as a base from which to combat theirs?

\item
% 610.
And are we right or wrong to combat it? Of course there are all sorts of
slogans which will be used to support our proceedings.

\item
% 611.
Where two principles really do meet which cannot be reconciled with one
another, then each man declares the other a fool and heretic.

\item
% 612.
I said I would `combat' the other man, - but wouldn't I give him reasons?
Certainly; but how far do they go? At the end of reasons comes persuasion.
(Think what happens when missionaries convert natives.)

\item
% 613.
If I now say ``I know that the water in the kettle in the gas-flame will not
freeze but boil'', I seem to be as justified in this ``I know'' as I am in any.
`If I know anything I know this'. - Or do I know with still greater certainty
that the person opposite me is my old friend so-and-so? And how does that
compare with the proposition that I am seeing with two eyes and shall see them
if I look in the glass? - I don't know confidently what I am to answer here. -
But still there is a difference between cases. If the water over the gas
freezes, of course I shall be as astonished as can be, but I shall assume some
factor I don't know of, and perhaps leave the matter to physicists to judge.
But what could make me doubt whether this person here is N.N., whom I have
known for years? Here a doubt would seem to drag everything with it and plunge
it into chaos.

\item
% 614.
That is to say: If I were contradicted on all sides and told that this person's
name was not what I had always known it was (and I use ``know'' here
intentionally), then in that case the foundation of all judging would be taken
away from me.

\item
% 615.
Now does that mean: ``I can only make judgements at all because things behave
thus and thus (as it were, behave kindly)''?

\item
% 616.
Why, would it be unthinkable that I should stay in the saddle however much the
facts bucked?

\item
% 617.
Certain events would me into a position in which I could not go on with the old
language-game any further. In which I was torn away from the sureness of the
game.  Indeed, doesn't it seem obvious that the possibility of a language-game
is conditioned by certain facts?

\item
% 618.
In that case it would seem as if the language-game must `show' the facts that
make it possible. (But that's not how it is.) Then can one say that only a
certain regularity in occurrences makes induction possible? The `possible'
would of course have to be `logically possible'.

\item
% 619.
Am I to say: even if an irregularity in natural events did suddenly occur, that
wouldn't have to throw me out of the saddle, I might make inferences then just
as before, but whether one would call that ``induction'' is another question.

\item
% 620.
In particular circumstances one says ``you can rely on this''; and this
assurance may be justified or unjustified in everyday language, and it may also
count as justified even when what was foretold does not occur. A language-game
exists in which this assurance is employed.

\item
% 621.
If anatomy were under discussion I should say: ``I know that twelve pairs of
nerves lead from the brain.'' I have never seen these nerves, and even a
specialist will only have observed them in a few specimens. - This just is how
the word ``know'' is correctly used here.

\item
% 622.
But now it is also correct to use ``I know'' in the contexts which Moore
mentioned, at least in particular circumstances. (Indeed, I do not know what
``I know that I am a human being'' means. But even that might be given a
sense.) For each one of these sentences I can imagine circumstances that turn
it into a move in one of our language-games, and by that it loses everything
that is philosophically astonishing.

\item
% 623.
What is odd is that in such a case I always feel like saying (although it is
wrong): ``I know that - so far as one can know such a thing.'' That is
incorrect, but something right is hidden behind it.

\item
% 624.
``Can you be mistaken about this colour's being called `green' in English?'' My
answer to this can only be ``No''. If I were to say ``Yes, for there is always
the possibility of delusion'', that would mean nothing at all.  For is that
rider \emph{Nachsatz} something unknown to the other? And how is it known to
me?

\item
% 625.
But does that mean that it is unthinkable that the word ``green'' should have
been produced here by a slip of the tongue or a momentary confusion? Don't we
know of such cases? - One can also say to someone ``Mightn't you perhaps have
made a slip?'' That amounts to: ``Think about it again.'' - But these rules of
caution only make sense if they come to an end somewhere.  A doubt without an
end is not even a doubt.

\item
% 626.
Nor does it mean anything to say: ``The English name of this colour is
certainly `green', - unless, of course, I am making a slip of the tongue or am
confused in some way.''

\item
% 627.
Wouldn't one have to insert this clause into all language-games? (Which shows
its senselessness.)

\item
% 628.
When we say ``Certain propositions must be excluded from doubt'', it sounds as
if I ought to put these propositions - for example, that I am called L.W. -
into a logic-book. For if it belongs to the description of a language-game, it
belongs to logic. But that I am called L.W. does not belong to any such
description. The language-game that operates with people's names can certainly
exist even if I am mistaken about my name, - but it does presuppose that it is
nonsensical to say that the majority of people are mistaken about their names.

\item
% 629.
On the other hand, however, it is right to say of myself ``I cannot be mistaken
about my name'', and wrong if I say ``perhaps I am mistaken''. But that doesn't
mean that it is meaningless for others to doubt what I declare to be certain.

\item
% 630.
It is simply the normal case, to be incapable of mistake about the designation
of certain things in one's mother tongue.

\item
% 631.
``I can't be making a mistake about it'' simply characterizes one kind of
assertion.

\item
% 632.
Certain and uncertain memory. If certain memory were not in general more
reliable than uncertain memory, i.e., if it were not confirmed by further
verification more often than uncertain memory was, then the expression of
certainty and uncertainty would not have its present function in language.

\item
% 633.
``I can't be making a mistake'' - but what if I did make a mistake then, after
all? For isn't that possible? But does that make the expression ``I can't be
etc.'' nonsense? Or would it be better to say instead ``I can hardly be
mistaken''? No; for that means something else.

\item
% 634.
``I can't be making a mistake; and if the worst comes to the worst I shall make
my proposition into a norm.''

\item
% 635.
``I can't be making a mistake; I was with him today.''

\item
% 636.
``I can't be making a mistake; but if after all something should appear to
speak against my proposition I shall stick to it, despite this appearance.''

\item
% 637.
``I can't etc.'' shows my assertion its place in the game. But it relates
essentially to me, not to the game in general.  If I am wrong in my assertion
that doesn't detract from the usefulness of the language-game.

\item
% 638.
``I can't be making a mistake'' is an ordinary sentence, which serves to give
the certainty-value of a statement. And only in its everyday use it is
justified.

\item
% 639.
But what the devil use is it if - as everyone admits - I may be wrong about it,
and therefore about the proposition it was supposed to support too?

\item
% 640.
Or shall I say: the sentence excludes a certain kind of failure?

\item
% 641.
``He told me about it today - I can't be making a mistake about that.'' - But
what if it does turn out to be wrong?! - Mustn't one make a distinction between
the ways in which something `turns out wrong'? - How can it be shown that my
statement was wrong? Here evidence is facing evidence, and it must be decided
which is to give way.

\item
% 642.
But suppose someone produced the scruple: what if I suddenly as it were woke up
and said ``Just think, I've been imagining I was called L.W.!'' ---- well, who
says that I don't wake up once again and call this an extraordinary fancy, and
so on?

\item
% 643.
Admittedly one can imagine a case - and cases do exist - where after the
`awakening' one never has any more doubt which was imagination and which was
reality. But such a case, or its possibility, doesn't discredit the proposition
``I can't be wrong''.

\item
% 644.
For otherwise, wouldn't all assertion be discredited in this way?

\item
% 645.
I can't be making a mistake, - but some day, rightly or wrongly, I may think I
realize that I was not competent to judge.

\item
% 646.
Admittedly, if that always or often happened it would completely alter the
character of the language-game.

\item
% 647.
There is a difference between a mistake for which, as it were, a place is
prepared in the game, and a complete irregularity that happens as an exception.

\item
% 648.
I may also convince someone else that I `can't be making a mistake'.  I say to
someone ``So-and-so was with me this morning and told me such-and-such''. If
this is astonishing he may ask me: ``You can't be mistaken about it?'' That may
mean: ``Did that really happen this morning?'' or on the other hand: ``Are you
sure you understood him properly?'' It is easy to see what details I should add
to show that I was not wrong about the time, and similarly to show that I
hadn't misunderstood the story. But all that can not show that I haven't
dreamed the whole thing, or imagined it to myself in a dreamy way. Nor can it
show that I haven't perhaps made some slip of the tongue throughout. (That sort
of thing does happen.)

\item
% 649.
(I once said to someone - in English - that the shape of a certain branch was
typical of the branch of an elm, which my companion denied. Then we came past
some ashes, and I said ``There, you see, here are the branches I was speaking
about.'' To which he replied ``But that's an ash'' - and I said ``I always
meant ash when I said elm''.)

\item
% 650.
This surely means: the possibility of a mistake can be eliminated in certain
(numerous) cases. - And one does eliminate mistakes in calculation in this way.
For when a calculation has been checked over and over again one cannot then say
``Its rightness is still only very probable - for an error may always still
have slipped in''. For suppose it did seem for once as if an error had been
discovered - why shouldn't we suspect an error here?

\item
% 651.
I cannot be making a mistake about 12x12 being 144. And now one cannot contrast
mathematical certainty with the relative uncertainty of empirical propositions.
For the mathematical proposition has been obtained by a series of actions that
are in no way different from the actions of the rest of our lives, and are in
the same degree liable to forgetfulness, oversight and illusion.

\item
% 652.
Now can I prophesy that men will never throw over the present arithmetical
propositions, never say that now at last they know how the matter stands? Yet
would that justify a doubt on our part?

\item
% 653.
If the proposition 12x12=144 is exempt from doubt, then so too must
non-mathematical propositions be.

\item


6
4. But against this there are plenty of objections. - In the first place there
is the fact that ``12x12 etc.'' is a mathematical proposition, and from this
one may infer that only mathematical propositions are in this situation. And if
this inference is not justified, then there ought to be a proposition that is
just as certain, and deals with the process of this calculation, but isn't
itself mathematical. I am thinking of such a proposition as: ``The
multiplication `12x12', when carried out by people who know how to calculate,
will in the great majority of cases give the result `144'.'' Nobody will
contest this proposition, and naturally it is not a mathematical one. But has
it got the certainty of the mathematical proposition?

\item
% 655.
The mathematical proposition has, as it were officially, been given the stamp
of incontestability. I.e.: ``Dispute about other things; this is immovable - it
is a hinge on which your dispute can turn.''

\item
% 656.
And one can not say that of the propositions that I am called L.W. Nor of the
proposition that such-and-such people have calculated such-and-such a problem
correctly.

\item
% 657.
The propositions of mathematics might be said to be fossilized. - The
proposition ``I am called....'' is not. But it too is regarded as
incontrovertible by those who, like myself, have overwhelming evidence for it.
And this not out of thoughtlessness. For, the evidence's being overwhelming
consists precisely in the fact that we do not need to give way before any
contrary evidence. And so we have here a buttress similar to the one that makes
the propositions of mathematics incontrovertible.

\item
% 658.
The question ``But mightn't you be in the grip of a delusion now and perhaps
later find this out?'' - might also be raised as an objection to any
proposition of the multiplication tables.

\item
% 659.
``I cannot be making a mistake about the fact that I have just had lunch.'' For
if I say to someone ``I have just eaten'' he may believe that I am lying or
have momentarily lost my wits but he won't believe that I am making a mistake.
Indeed, the assumption that I might be making a mistake has no meaning here.
But that isn't true. I might, for example, have dropped off immediately after
the meal without knowing it and have slept for an hour, and now believe I have
just eaten.  But still, I distinguish here between different kinds of mistake.

\item
% 660.
I might ask: ``How could I be making a mistake about my name being L.W.?'' And
I can say: I can't see how it would be possible.

\item
% 661.
How might I be mistaken in my assumption that I was never on the moon?

\item
% 662.
If I were to say ``I have never been on the moon - but I may be mistaken'',
that would be idiotic.  For even the thought that I might have transported
there, by unknown means, in my sleep, would not give me any right to speak of a
possible mistake here. I play the game wrong if I do.

\item
% 663.
I have a right to say ``I can't be making a mistake about this'' even if I am
in error.

\item
% 664.
It makes a difference: whether one is learning in school what is right and
wrong in mathematics, or whether I myself say that I cannot be making a mistake
in a proposition.

\item
% 665.
In the latter case I am adding something special to what is generally laid
down.

\item
% 666.
But how is it for example with anatomy (or a large part of it)? Isn't what it
describes, too, exempt from all doubt?

\item
% 667.
Even if I came to a country where they believed that people were taken to the
moon in dreams, I couldn't say to them: ``I have never been to the moon. - Of
course I may be mistaken''. And to their question ``Mayn't you be mistaken?'' I
should have to answer: No.

\item
% 668.
What practical consequences has it if I give a piece of information and add
that I can't be making a mistake about it?  (I might also add instead: ``I can
no more be wrong about this than about my name's being L.W.'') The other person
might doubt my statement nonetheless. But if he trusts me he will not only
accept my information, he will also draw definite conclusions from my
conviction, as to how I shall behave.

\item
% 669.
The sentence ``I can't be making a mistake'' is certainly used in practice. But
we may question whether it is then to be taken in a perfectly rigorous sense,
or is rather a kind of exaggeration which perhaps is used only with a view to
persuasion.

\item
% 670.
We might speak of fundamental principles of human enquiry.

\item
% 671.
I fly from here to a part of the world where the people have only indefinite
information, or none at all, about the possibility of flying. I tell them I
have just flown there from... They ask me if I might be mistaken. - They have
obviously a false impression of how the thing happens. (If I were packed up in
a box it would be possible for me to be mistaken about the way I had
travelled.) If I simply tell them that I can't be mistaken, that won't perhaps
convince them; but it will if I describe the actual procedure to them. Then
they will certainly not bring the possibility of a mistake into the question.
But for all that - even if they trust me - they might believe I had been
dreaming or that magic had made me imagine it.

\item
% 672.
``If I don't trust this evidence why should I trust any evidence?''

\item
% 673.
Is it not difficult to distinguish between the cases in which I cannot and
those in which I can hardly be mistaken? Is it always clear to which kind a
case belongs? I believe not.

\item
% 674.
There are, however, certain types of case in which I rightly say I cannot be
making a mistake, and Moore has given a few examples of such cases.  I can
enumerate various typical cases, but not give any common characteristic. (N.N.
cannot be mistaken about his flown from America to England a few days ago. Only
if he is mad can he take anything else to be possible.)

\item
% 675.
If someone believes that he has flown from America to England in the last few
days, then, I believe, he cannot be making a mistake.  And just the same if
someone says that he is at this moment sitting at a table and writing.

\item
% 676.
``But even if in such cases I can't be mistaken, isn't it possible that I am
drugged?'' If I am and if the drug has taken away my consciousness, then I am
not now really talking and thinking. I cannot seriously suppose that I am at
this moment dreaming. Someone who, dreaming, says ``I am dreaming'', even if he
speaks audibly in doing so, is no more right than if he said in his dream ``it
is raining'', while it was in fact raining. Even if his dream were actually
connected with the noise of the rain.

\end{enumerate}
\end{document}
