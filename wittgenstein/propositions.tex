\documentclass[man,12pt,natbib]{apa6}
\usepackage{amssymb,amsmath,latexsym,times}
\linespread{1.5}

% Page length commands go here in the preamble
%\setlength{\oddsidemargin}{-0.25in} % Left margin of 1 in + 0 in = 1 in
%\setlength{\textwidth}{7in}   % Right margin of 8.5 in - 1 in - 6.5 in = 1 in
%\setlength{\topmargin}{-.75in}  % Top margin of 2 in -0.75 in = 1 in
%\setlength{\textheight}{9.2in}  % Lower margin of 11 in - 9 in - 1 in = 1 in
\renewcommand{\baselinestretch}{1.5} % 1.5 denotes double spacing. Changing it will change the spacing
\setlength{\parindent}{0in}

\begin{document}

\title{Wittgenstein on Propositions}
\shorttitle{Wittgenstein on Propositions}
\author{Edward Hern\'{a}ndez}
\date{\today}
\affiliation{College of William \& Mary}

\maketitle

% \vspace{-20pt}
\begin{quote}
	Although what the Tractatus says about the world, facts, and thoughts
	appears before its discussion of propositions, it is clear that the
	priority is the other way round.  What the Tractatus says about these other
	things depends upon what it says about propositions.  Indeed, roughly 90\%
	of the book is about the nature of the proposition: \S\S 3.1-6.  So, what
	is the book's account of a proposition?  (Let's leave until next time the
	book's remarks about sense and nonsense, saying and showing, tautologies,
	ethics, aesthetics, religion, philosophical `problems', philosophical
	method, etc.) 

	In tackling this question, try to make sense of how the concept of a
	proposition relates to (at least some of) the following concepts: fact,
	object, world, logical form, picture/model, truth/falsity, a thought, a
	sense, a possible state of affairs, an elementary proposition, an atomic
	fact, inference, a name,... 
\end{quote}
\clearpage

 \citet{Wittgenstein22} puts forward in his \emph{Tractatus
 Logico-Philosophicus} a view of the proposition that is complex and barely
 penetrable. To even discuss the ideas on his terms, we must be clear on what
 his terms are. Since he begins the \emph{Tractatus} with discussion of facts,
 we will begin there as well.

 Wittgenstein begins the \emph{Tractatus} by asserting that the world is
 composed of and determined by facts (\S\S 1, 1.1). These facts are independent
 of each other, as each could be false without any other consequently being
 false (\S\S 1.2, 1.21). Facts are, for Wittgenstein, the existence of “atomic
 facts,” \footnote{At least, Ogden translates this term as ``atomic fact.''
 Other translations call them ``states of affairs,'' which seems much clearer.}
 and an ``atomic fact''  is a combination of objects (\S\S 2, 2.01), meaning
 that any particular fact is identical to the existence of a combination of
 objects.  This also means that the fact is, itself, composed of atomic facts
 (\S 2.034).  Since atomic facts are combinations of objects, and facts are
 composed of atomic facts, the fact must always be nothing more than an
 elaboration of relations of objects. Sets of facts compose reality, and the
 ``total reality,'' which is the world, is then the set of all object
 relationships (\S 2.063), not those objects themselves (\S 1.1).

According to Wittgenstein, the way that we interact with these object
relationships that make up the world is to ``make ourselves pictures of facts''
which are models of reality (\S\S 2.1, 2.12). Such a picture of the facts is a
thought (\S 3). The proposition is the (perceptible) expression of such a
thought (\S 3.1). Thus, propositions are expressions of facts, in that they
express thoughts, which are representations of the object relations identical
to those facts.

Propositions are divided into ``elementary'' propositions and non-elementary
proposition, mirroring the division of facts into atomic and non-atomic. The
elementary proposition is the assertion of an atomic fact (4.21), and cannot
contradict any other elementary proposition (\S 4.211) in the same way that no
atomic fact determines the existence (or non-existence) of any other atomic
fact. As \citet{Speaks07}\footnote{This is a useful handout from the PHIL 43904
History of Analytic Philosophy course at the University of Notre Dame,
available online.} notes, this means that Wittgenstein's elementary
propositions are nothing like the propositions discussed in the work of other
philosophers. Other conceptions of propositions, like Russell's allow them to
be contradictory \cite{Klement15}. Here are two of the example propositions
from \citet{Speaks07} which look like the propositions with which we are
familiar:    
\begin{equation}
\text{That is exactly 6 feet tall.}
\end{equation}
\begin{equation}
\text{That is exactly 5 feet tall.}
\end{equation}
These propositions are contradictory. The truth of (1) dictates the falsity of
(2), or vice versa. Thus, neither can be ``elementary'' by Wittgenstein's
definition. Further, an elementary preposition is not even a grammatical
assertion, like these examples; it is nothing but a concatenation or list of
names of objects (\S 4.22). This is the case because they assert atomic facts,
which are nothing but combinations of objects. As a result, whether or not an
elementary proposition is true is determined by whether the atomic fact it
asserts does exist (\S 4.25), which seems to line up with other philosopher's
ideas about whether a fact or state of affairs ``obtains'' \cite{Mulligan13}.
Since the set of existent atomic facts is the world, the set of all true
elementary propositions “describes the world completely” (\S 4.26) in that it
``specify'' every object relation which makes up the world.

It is important for Wittgenstein that all propositions are truth functions, not
just elementary propositions. The truth values of an elementary proposition
corresponds directly to whether or not the atomic fact it asserts exists (\S
4.2). If the atomic fact exists, the elementary proposition that asserts it is
true; if the atomic fact is false, its corresponding elementary proposition is
false. This makes their evaluation relatively straightforward, given the
pertinent facts. But what about other propositions? Wittgenstein asserts that
all propositions are truth-functions of elementary propositions (\S 5) and can
be determined from the set of elementary propositions (\S 4.51). A proposition
has a truth value, which is determined by the truth values of all of the
elementary propositions of which it is composed (and by nothing else). This
means that drawing out a truth table, as Wittgenstein discusses in \S\S 4-5, of
elementary propositions can determine the truth value of any proposition.

Since it is the case that the truth possibilities of all propositions are truth
functions of elementary propositions, it follows that all propositions can be
derived from the set of elementary propositions. Wittgenstein goes so far as to
say that all propositions can be derived via application of a single operation.
He argues that because of the nature of truth-possibilities, all truth
functions have the general form: 
\begin{equation}
	[ \overline{p}, \overline{\xi}, N(\overline{\xi}) ]
\end{equation}
Here, if I understand correctly,\footnote{I acknowledge that I may not. This is
quite confusing to me, and many secondary sources, including \emph{The Stanford
Encyclopedia of Philosophy} gloss over what the individual components of this
form represent.} the variable $ \overline{p} $ represents the set of elementary
propositions, and $ \overline{\xi} $ represents some subset of propositions
\cite{Speaks07}. The operation $ N( \xi ) $, which \citet{Geach81} describes as
``joint denial,'' yields the conjunction of the negation of each proposition in
the set to which it is applied. That is to say, if $\xi$ has only the value
$p$, then $N(\xi) = \neg p$; applying $N(\overline{\xi})$ yields negation. If
$\xi$ has the values $p$ and $q$, then $N(\overline{\xi}) = \neg p. \neg q$;
$N(\xi)$ yields the negation of each value of $\xi$, conjoined (\S 5.51). On
Wittgenstein's view, applying $N$ to some subset of elementary propositions
yields another non-elementary proposition. Applying it yet again produces
another. Applying it to produce every possible combination of truth values will
produce the set of all propositions.

I do not have anything else I understand about the topic yet. Wittgenstein
tells me it is time to be silent (\S 7).

\clearpage

\bibliography{wittgenstein}

\end{document}
