\documentclass[doc,12pt,natbib]{apa6}
\usepackage[colorlinks=false]{hyperref}
\usepackage{amsmath}
\usepackage{amssymb}
\usepackage{multirow}
\usepackage{times}
\usepackage{gb4e}

\linespread{1.5}

\begin{document}

\title{Homework 4: Absolutive Nouns in Classical Nahuatl}
\shorttitle{Homework 4}
\author{Edward Hern\'{a}ndez}
\date{19 September 2016}
\affiliation{College of William \& Mary}
\maketitle

	% Write about 1.5 pages (double spaced) on a grammatical topic related to
	% nouns (possession, noun compounds, noun plurals, case, diminutives,
	% etc.). Present the author's description and data, being careful to
	% evaluate strengths or weaknesses of the author's treatment. You want to
	% give the reader a feel for a major area of grammar in relatively little
	% space, so don't get bogged down by details or by trying to cover too many
	% subtopics. List your source in a bibliography or in a footnote. Refer to
	% the page numbers where you found the information (e.g., Smith
	% (2009:145)). Explain the abbreviations you use in glosses in a footnote.

	% I will grade this assignment based on the following: Did you give the
	% source of the work? Did you rise above the details of the book to give a
	% concise overview of this topic? Did you distinguish the author's analysis
	% from your own point of view, commenting on the former? Is your
	% description structured (organized into paragraphs that relate to each
	% other in a logical way)? Did you use appropriate style and is your
	% wording clear? Did you turn in the assignment on time?


\citet{Sullivan88} describes Classical Nahuatl as having ``word stems,'' (p.~1)
which may be derived, via affixes, into any part of speech. Sullivan does not
offer a definition of a noun (nor any explicit or fuctional definition
whatsoever for any part of speech), but she does demonstrate that nouns have
possessedness and number marked in most cases. She terms nouns which are marked
as neither possessed nor plural \emph{absolutive} (p.~15). This appears to be a
non-standard use of the term, since on this definition, it appears that no
plural noun can be \emph{absolutive}, even if it is not marked as possessed.

To form an absolutive noun, most stems must take a suffix. There are
two\footnote{Sullivan lists four, but concedes (though not explicitly) that
	three are allomorphs of the same suffix.} absolutive suffixes: \textit{-in}
and \textit{-tli}.

Some stems take the suffix \textit{-tli}, which has the allomorphs
\textit{-tl}, \textit{-tli}, and \textit{-li}.  Of stems which take this
suffix, those which end with a vowel take \textit{-tl}. Stems which end with a
consonant other than /l/ take \textit{-tli}, and those which end with /l/ take
\textit{-li}.
\begin{exe}
	\ex
	\begin{tabbing}
		\hspace*{2cm}\= \kill
		\textit{a-tl} \> water \\
		\textit{teuc-tli} \> lord \\
		\textit{xal-li} \> sand
	\end{tabbing}
\end{exe}

Some stems form the absolutive by taking the suffix \textit{-in}.
\begin{exe}
	\ex 
	\begin{tabbing}
		\hspace*{2cm}\= \kill
		\textit{zol-in} \> quail \\
		\textit{ocul-in} \> worm
	\end{tabbing}
\end{exe}
Many, but not all, plant and animal names form the absolutive in this way.
Additionally, some stems which take \textit{-in} may alternately take
\textit{-tli} (though perhaps not any of its other allomorphs).
\begin{exe}
	\ex
	\begin{tabbing}
		\hspace*{2cm}\= \kill
		\textit{toch-in} \> rabbit \\
		\textit{toch-tli} \> rabbit
	\end{tabbing}
\end{exe}

Some stems may stand alone without a derivational affix, which Sullivan
interprets as equivalent to the absolutive suffixed forms of other stems
(p.~16).
\begin{exe}
	\ex
	\begin{tabbing}
		\hspace*{2cm}\= \kill
		\textit{ilama} \> old woman \\
		\textit{chichi} \> dog
	\end{tabbing}
\end{exe}

\nocite{Sullivan88}

\clearpage
\bibliography{../extra.bib}

\end{document}
