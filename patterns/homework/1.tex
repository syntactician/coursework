\documentclass[doc,12pt]{apa6}
\usepackage[colorlinks=false]{hyperref}
\usepackage{amssymb}
\usepackage{amsmath}
\usepackage{textgreek}
\usepackage{times}
\usepackage{gb4e}

\linespread{1.5}

\begin{document}

\title{Homework 1: Tetelcingo Nahuatl}
\shorttitle{Homework 1}
\author{Edward Hern\'{a}ndez}
\date{29 August 2016}
\affiliation{College of William \& Mary}
\maketitle

\setcounter{secnumdepth}{3}

\section{Verbs and their Affixes}

Verbs in Tetelcingo Nahuatl have at least two affix slots: a prefix slot which
marks for subject, and a suffix slot which marks for tense and progressive
aspect.

\begin{exe}
	\ex \textit{n$\iota$-\v{c}uka-t$\iota$ka} \\
	1SG-cry-PRS.PROG \\
	``I am crying.''
\end{exe}

\subsection{Subject}

Verbs have a prefix slot which mark them for subject, including person and
number. This slot may mark a verb for person, or both person and
number.

\begin{exe}
	\ex \textit{n$\iota$-kwika} \\
		1SG-sing \\
		``I sing.'' \\
	\ex \textit{t$\iota$-kwika} \\
		2-sing \\
		``You sing.''
\end{exe}

Verbs with this slot filled can constitute complete, single word sentences, as
shown in the translation of (1) and (2). This affix slot may be obligatory, as
there is as yet no evidence that it may be unfilled.

\subsection{Tense and Aspect}

\textsc{Progressive aspect} (\textsc{prog}) is a gramatical aspect of a verb which
indicates that an action is continuing or incomplete. Tetelcingo Nahuatl verbs
are marked for progressive aspect by the same suffix slot as tense.

\begin{exe}
	\ex \textit{t$\iota$-\v{c}uka} \\
		2-cry \\
		``You cry.''
	\ex \textit{n$\iota$-\v{c}uka-t$\iota$ka} \\
		1SG-cry-PRS.PROG \\
		``I am crying.''
	\ex \textit{n$\iota$-\v{c}uka-k} \\
		1SG-cry-PST \\
		``I cried.''
	\ex \textit{t$\iota$-\v{c}uka-taya} \\
		2-cry-PST.PROG \\
		``You were crying.''
	\ex \textit{t$\iota$-kwika-s} \\
		2-sing-FUT \\
		``You will sing.''
\end{exe}

As seen in (2), (3), and (4), if this slot is unfilled, the verb is translated
as present tense and as lacking progressive aspect. A verb may alternately be
marked as present progressive (5), past (6), past progressive (7), or future
(8).

\end{document}
