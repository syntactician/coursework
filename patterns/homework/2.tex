\documentclass[doc,12pt]{apa6}
\usepackage[colorlinks=false]{hyperref}
\usepackage{amssymb}
\usepackage{amsmath}
\usepackage{multirow}
\usepackage{textgreek}
\usepackage{times}
\usepackage{gb4e}

\linespread{1.5}

\begin{document}

\title{Homework 2: Pocomch\'{i}}
\shorttitle{Homework 2}
\author{Edward Hern\'{a}ndez}
\date{29 August 2016}
\affiliation{College of William \& Mary}
\maketitle

\setcounter{secnumdepth}{3}

Verbs in Pocomch\'{i} have two prefix slots which mark the person and number of
the objects and subject.\footnote{
	In the glosses, \textsc{1pl.obj} = first person plural object;
	\textsc{3pl.sbj} = third person plural subject, etc.
}
Both may be obligatory.

The first prefix slot marks the person and number of the object. There are at
least three prefixes which may fill this spot, enumerated in Table
\ref{table:obj}.
\begin{table}
	\begin{tabular}{ c c c }
		\textbf{number} & \textbf{person} & \textbf{prefix} \\ \hline
		\multirow{ 3}{*}{\textsc{pl}} & 1 & \textit{qo-} \\ \cline{2-3}
		& 2 & \textit{ti-} \\ \cline{2-3}
		& 3 & \textit{ki-} \\ \hline \\
	\end{tabular} 
	\caption{Object Prefixes}
	\label{table:obj}
\end{table}
\begin{exe}
	\ex
	\gll \textit{\underline{qo}-r-il} \\
	\textsc{1.pl.obj-3.sg.sbj-}see \\
	\trans `he sees \underline{us}'
\end{exe}

The second prefix slot marks the person and number of the subject. There are at
least four prefixes which may fill the slot, enumerated in Table
\ref{table:sbj}.
\begin{table}
	\begin{tabular}{ c c c }
		\textbf{number} & \textbf{person} & \textbf{prefix} \\ \hline
		\multirow{ 2}{*}{\textsc{sg}} & 1 & \textit{-w-} \\ \cline{2-3}
		& 3 & \textit{-r-} \\ \hline
		\multirow{ 2}{*}{\textsc{pl}} & 1 & \textit{-q-} \\ \cline{2-3}
		& 3 & \textit{-k-} \\ \hline \\
	\end{tabular} 
	\caption{Subject Prefixes}
	\label{table:sbj}
\end{table}
\begin{exe}
	\ex
	\gll \textit{ki-\underline{k}-eht'al} \\
	\textsc{3pl.obj-3pl.sbj}-recognize \\
	\trans `\underline{they} recognize them'
\end{exe}

\end{document}
