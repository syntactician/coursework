\documentclass[12pt]{article}
\usepackage{etoolbox,keyval,ifthen,url,csquotes}
\usepackage[notes,strict,backend=biber,autolang=other,%
bibencoding=inputenc]{biblatex-chicago}
\usepackage{setspace}

\addbibresource{midterm.bib}

\setlength{\oddsidemargin}{-0.25in} % Left margin of 1 in + 0 in = 1 in
\setlength{\textwidth}{7in}   % Right margin of 8.5 in - 1 in - 6.5 in = 1 in
\setlength{\topmargin}{-.75in}  % Top margin of 2 in -0.75 in = 1 in
\setlength{\textheight}{9.2in}  % Lower margin of 11 in - 9 in - 1 in = 1 in
\linespread{1.5}

\renewcommand{\footnoterule}{%
  \kern -3pt
  \hrule width 3in height 1pt
  \kern 5pt
}

\begin{document}
\title{Shared Trauma and Healing in Toni Morrison's \emph{Home}}
\author{Edward Hern\'{a}ndez}
\date{\today}

\maketitle

\abstract{This paper traces the narrative arc of Toni Morrison's \emph{Home},
highlighting Frank and Cee's joint healing process and applying this process as
a framework under which to practice narrative medicine and soul repair with
respect to moral injury.}

\vspace{12pt}

\section{Introduction}

Toni Morrison, in the novel \emph{Home}\autocite{Morrison12}, explores the
themes of shared suffering and shared healing. In the text, Frank Money, a
Korean War veteran, struggles with moral injury both from the war and from his
childhood. At the same time, he must rescue and care for his sister Cee, who is
physically ill, having been subjected to inhumane experimentation.  This paper
will examine this narrative arc and its implications for healing through the
narrative medicine lens provided by Rita Charon.\autocite[ch.~6]{Charon06}
Charon identifies five elements of a text to which a close reading ought to
attend: frame, form, time, plot, and desire. Each is critical to a full
appreciation of \emph{Home}., but for the sake of depth, I will focus on two in
this paper: form and desire.

\section{Form}

The form in which the narrative of \emph{Home} is told is striking. Even
assigning a genre to the text is tricky. As Leah Hager Cohen describes it, it
is ``on the basis of its publisher’s description a novel, on the basis of its
length a novella, and on the basis of its stripped-down, symbol-laden plot
something of an allegory.''\autocite{Cohen12} Beyond the genre confusion, the
text is odd in that it is quilted together of dissimilar pieces, told by two
narrators of different types, with wildly different styles. Some chapters
feature a seemingly omniscient narrator (who can somehow be misled by Frank's
narrative) telling Frank and Cee's stories in the third person with smooth,
controlled prose. Others are told by Frank himself, in short, jagged, halting,
first-personal sentences. While Frank is not an omniscient narrator, he speaks
directly with the reader and the other narrator.

These stylistic choices serve to highlight Frank's complex relation to his own
story. It is his story in that it is about him. It is the story of his life, of
his childhood, his experience in the war, his journey to rescue his sister. But
no matter how personal it is, the story does not belong solely to him. It is
also Cee's story, of their shared childhood, of her rescue by Frank, of how his
experiences impact her. It is the story of the killed man, of his death, of the
children witnessing his burial, of his ritual re-burial. These stories cannot
be fully told without being told together.

In the same way that the story cannot be individual, Frank and Cee do not
suffer in isolation. Frank suffers in the same way as many veterans of combat.
His time on the battlefield in Korea damages his mind and his spirit. He
experiences what Rita Nakashima Brock calls ``moral injury,'' which she
describes as ``the collapse of your moral meaning system and your moral
foundations because of things that you've experienced.''\autocite{Brock15} He
killed a Korean girl during the war, knowing full well that the act was
horrible. This produced cognitive dissonance between his positive, righteous
self image, feeling ``so proud grieving over [his] dead friends'',\autocite[p.
133]{Morrison12} and the guilt and shame he felt for for his actions. While
Frank's suffering stems from very personal experiences, many veterans commit or
experience similar horrors, and subsequently have to deal with the same sort of
dissonance and injury. Frank's moral pathology is shared by many who fought
with him.

Similarly, Cee's encounter with the eugenicist doctor is part of a larger
pattern, of exploitative, non-beneficent research on Black bodies and on Black
women's reproductive systems specifically. White science and medicine have long
been attracted to the study of the Other. Throughout the 19th century, Khoikhoi
women like Saartjie Baartman were studied and exhibited while they were alive,
and dissected after they were dead, to discover the secrets of their
physiological (especially sexual) differences from the white
body.\autocite{Qureshi04} As recently as 1972, the Public Health Service
studied the effects of syphilis on black men's reproductive systems without
offering them any treatment or even notifying them that they had the disease.
Perhaps the most striking similarity to Cee's case is the work of J. Marion
Sims, considered ``the father of modern gynecology,'' whose work consisted
primarily of inhumane exploratory surgery on enslaved black women without
anesthesia.\autocite{Spettel11} Cee, like Frank, suffers together with others
who have similar traumatic experiences.

The sharing of particular sorts of experiences is not the only way that the
sufferers of trauma are tied together, however. This is perhaps more visible in
the case of the killed man. Neither Frank nor Cee are killed with him, nor do
they see him killed. They do not interact with him while he is alive, nor do
they knowingly interact with his killers. Nonetheless, they are affected by his
death. Seeing him buried is a traumatic experience which haunts both of them
for years. Being in the same community as those who undergo trauma affects all
members of the community in some way, however small. Additionally, recent
research is beginning to show that this sort of trauma can be passed down, from
one generation to the next,\autocite{Dias14,Love10} meaning that our stories
are inseparable not only from everyone currently in our communities, but all
those who came before us. This is part of why the text must be written from
multiple different perspectives.

I also argue that the distinct narrative style in which the novel is written
indicates a particular relationship between the narrators: that the text is a
retelling of Frank's story by the second narrator, as he confessed it. In work
about the effects of war, moral injury, and the healing power of narratives,
many authors retell the stories of their
patients.\autocite{Shay95,vanDernoot09,Brock15,Charon06} This amplifies those
patients' voices, allowing their narratives to reach (and be healing tools for)
a wider readership. Often, the authors of these retellings will quote their
patients, to allow them to retain their agency and express their ideas directly
to the reader. I hold that it is exactly this sort of relationship which exists
between Frank and the narrator. Frank even notes in Chapter 1 that the other
narrator is ``set on telling [his] story.''\autocite[p.~6]{Morrison12} This
relationship explains both the dual narration and the fallibility of the
seemingly omniscient narrator. Frank tells the story, the narrator retells it,
sometimes quoting him directly to show the emotional power of his account. This
account explains the style of the novel down to the typography: Frank's
chapters are italicized: they are quoted. As Frank is unreliable as a source of
information, the other narrator, who relies on his account, is likewise
unreliable.

\section{Desire}

Frank, as a narrator, desires to admit his experiences to the reader. We see
this in his hedging comments to the narrator; he wants to ensure that she gets
it right, that she understands.\autocite[p.~6,~42,~70,~84]{Morrison12} He
struggles throughout the novel to tell us the truth, and as the novel
progresses, this produces in the reader a growing desire to know the truth. In
chapter 15,\autocite[p.~133]{Morrison12} he finally gives in, and tells us what
he has been withholding, that he killed the Korean girl. It is hard to say that
this realization gives the reader any ``pleasure,'' but it certainly releases
much of the pent-up anticipation, anxiety, and curiosity about Frank's wartime
experiences. Anxiety has been building as Frank's wartime imagery gets
increasingly graphic and violent with each chapter, starting with the outright
denial of violence in Chapter 1 and ending with a graphic description of the
dead Korean girl in Chapter 9.\autocite[Ch.~1,~3,~5,~7,~9]{Morrison12} Reaching
the culmination and catharsis with Frank as he finds the resolve to tell the
truth satisfies both the narrator's and reader's desires at once. As unpleasant
as it is to read, it is a release, a satisfaction, and thus a ``pleasure of the
text'' as Roland Barthes discusses them.\autocite[p.  124]{Charon06}

Likewise, Frank. as a character, wants closure for the killed man. After seeing
him buried, he is so traumatized that he either forgets the experience or
claims to (much like the story of shooting the girl).\autocite[p.
5-6]{Morrison12} I would argue that seeing the man unceremoniously buried
constituted a moral injury undermining Frank's moral beliefs about the value of
life. Whether or not he remembered, and whether or not the injury was moral, he
is haunted by the experience, and cannot make peace with it until years later,
when he is able to return to the burial site and give the man a proper, marked
grave. This resolution, propelled by the desire of Frank the character, is a
\emph{pleasure} to the reader as well. It is the happiest ending we can hope
for in a story of trauma: the characters who survive make peace with it and
begin to heal. This ritual heals not only Frank, whose desire propels it, but
Cee as well, who, as we see in the final scene, was also haunted by the killed
man. Cohen points out that this realization, that Cee also sees the man,
``underscores the book's most powerful proposition: that there is no such thing
as individual pathology.''\autocite{Cohen12}

Nakashima Brock briefly discusses this idea as well in an interview. She
describes the ways in which her father's wartime experience was a shared trauma
among her family, to whom he returned almost unrecognizable.\autocite{Brock15}
Since we as people and communities are so intertwined, we inevitably share our
trauma and our pathology. We share our desire for healing just as narrator,
character, and reader share the desire in \emph{Home}. We must harness this
connectedness, our shared desire, to heal as communities, just as we are hurt
as communities.

\section{Conclusion}

Moral injury is pernicious. It not only harms us deeply, but it harms those
around us, and largely, it harms us without us knowing. Since most of us have
little to no understanding of what it is to be morally injured, we have no
understanding of how to heal from these injuries. Some work is being done to
give a psychological account of moral injury, and suggest clinical practices
for its treatment,\autocite{Litz09} but the work is far from complete, and what
sparse recommendations there are for treatment rely on the presence of
``benevolent moral authority.''\autocite[\S 7.2]{Litz09} Treatment under this
model may include ``an imaginary conversation with another person who [the
patient has] great respect for and who can weigh in as a relevant and generous
moral authority.''\autocite[\S 7.2.5]{Litz09} This recommendation a step in the
right direction, as it acknowledges the importance of human interaction, of
conversation, of confession, in the healing of moral injury. The idea is good;
however, the implementation is not sufficient.

Can we believe that Frank would heal, if he were to only \emph{imagine} telling
the truth? What if he were not to confess to the reader and the other narrator,
but rather pretend to? What if he were to imagine asking the men on the porch
about the ``dogfight?'' Could he have found any closure without their input?
To suggest that he could is to underestimate the importance and power of
community.  This mistake is typical of a psychological paradigm which embraces
controlled environments and clinical tests and fails to incorporate and embrace
the complexities of spirits and the communities to which they are tied.  This
sort of focus leads us to envision healing happening internally, person by
person. This is inherently limiting. Even if Frank might find some solace in
imagining conversations, would that help Cee to heal? She needs the community
to participate in her healing, both physically and psychologically. The women
are essential to her healing physically. She could not heal by imagining their
ministrations. Likewise, she would not begin to heal her trauma from witnessing
the man killed in the ``men-treated-like-dog-fights''\autocite[137]{Morrison12}
without the ritual she shares with Frank. If Frank did not perform the burial
ritual with Cee (even if he had imagined it), she would not have found closure.

Moreover, healing from moral injury is not only to alleviate our individual
distress, but also to ``[renew] our relationship to our own humanity, to
each other, to the rest of the world, and to all that sustains
life.''\autocite[p.~116]{Brock12} We cannot renew connections by imagining them.
We cannot connect by imagining connection.  Brock and Lettini, in \textit{Soul
Repair: Recovering from Moral Injury after War},\autocite{Brock12} suggest that
healing requires a reclamation of moral integrity that is impossible alone, and
that dialogues \say{mine a deeper level of moral questioning in which language
moves from being descriptive to being deeply
transformative.}\autocite[p.~112]{Brock12} Instead of \emph{imagined}
conversations, we ought to encourage \emph{actual} conversations and
rituals,\autocite{Brock15} so that those affected by trauma can heal together,
as a community. Toni Morrison's \emph{Home} points us toward the answer: we
desire to tell the truth to each other, to hear the truth from each other, and
when we do, we can begin to heal together.

\clearpage
\printbibliography

% A
% Outstanding revision!

\end{document}
