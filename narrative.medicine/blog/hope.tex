\documentclass[doc,12pt]{apa6}
\usepackage[colorlinks=false]{hyperref}
\usepackage{lmodern,amssymb,amsmath}
\linespread{1.5}

\begin{document}

\title{Narrative Medicine is Hope}
\shorttitle{Narrative Medicine is Hope}
\author{Edward Hern\'{a}ndez}
\date{17 September 2015}
\affiliation{College of William \& Mary}
\maketitle

I feel largely failed by the medical system. I have a condition (or, likely,
multiple conditions) whose effects I feel every day. Just walking to class is
enough to exhaust me. Sometimes, it's a struggle to even stand. I can barely
remember what it's like to not be in pain, because it's been years. These
experiences obviously play a large role in my life, not only limiting what I
can do, and defining who I am.

I share this information vaguely because I have no way to be more precise. I
have never had a formal, written diagnosis of anything but asthma (and
nearsightedness, and that only accidentally). I believe that this is because
doctors have failed to listen to me. I tell them my symptoms. This summer I
went into the ER, telling the responding physician that I had tightness in my
chest, pain spreading into my left arm, difficulty breathing. These symptoms
are worrisome to me. As a layperson, they sound like a textbook heart attack.
The attending orders an ECG, but it feels like he's simply humoring me. When he
comes back fifteen minutes later, he doesn't even bother telling me whether the
results are normal. I can only assume that they are, since he seems in a rush
to send me home. When I ask what could be wrong, why I could be having this
chest pain, he tells me not to worry. ``You're a young healthy guy,'' he says.
``Nothing to worry about.'' This doesn't keep me from worrying. This isn't the
first time I've felt scary chest pain. They never ask. They don't even ask for
a family history. I leave the ER empty-handed and scared.

Before this, my pediatrician several times told me he didn't believe my reports
of my symptoms. He said that the symptoms I described were ``impossible.'' On
another occasion he accused me of bringing my symptoms on myself. He told me I
was probably suffering from ``Sunday Morning Palsy,'' meaning I had drunkenly
fallen asleep or unconscious in a bad position or over furniture. I hadn't even
been drinking. When I told him I hadn't, he said it was ``normal to lie'' about
that sort of thing to a doctor. He then referred me to a physical therapist.
The therapist gave me exercises to level my hips, without telling me that's
what the exercises were for. I have scoliosis; leveling my hips is dangerous
without dealing with the curvature of the rest of the spine. The therapy
injured me, and years later I am still in pain.

These are just a few of many unsuccessful medical experiences for me. Each one
erodes my hope of diagnosis and relief a little more. When I heard of Narrative
Medicine, I was excited. For me, it seems to suggest the answers to the
problems I've always had with medicine. I think that these doctors, had they
listened to me -- truly listened, that is, not just checking boxes or filling
forms -- would have taken me seriously. I wouldn't have been brushed off as a
``young healthy guy'' this whole time. I might have a diagnosis by now. I
certainly wouldn't have been injured by my therapist. When I hear Charon
talking about uncovering the lives and stories of her patients to facilitate
their care, I'm excited, because I imagine that people like me, with largely
invisible illness, have a hope of receiving treatment. People like me might not
be discredited and hurt by those who should be healing them. To me, it means
extending treatment to everyone, based on their story, rather than treating
only easy, visible cases. To me, this is a revelation.

\end{document}
