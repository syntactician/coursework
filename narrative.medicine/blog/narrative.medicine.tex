\documentclass[doc,12pt]{apa6}
\usepackage[colorlinks=false]{hyperref}
\usepackage{lmodern,amssymb,amsmath,apacite}
\linespread{1.5}

\begin{document}

\title{In Defense of Narrative Medicine}
\shorttitle{Narrative Medicine Defense}
\author{Edward Hern\'{a}ndez}
\date{24 September 2015}
\affiliation{College of William \& Mary}
\maketitle

Roland Barthes gives us the idea of le lisible and le scriptible, the readerly
and the writerly. For Barthes, the readerly text is static.  There is only one
way to read a readerly text, and it largely does not change over time. In
contrast, the writerly text is living, dynamic, co-authored by the reader, of
whom creative action is required, to perform the text for herself. He goes so
far as to say that ``the writerly text is ourselves writing.'' This performance
and writing of the text, which the writerly text requires of each reader,
allows the reader to break out of her subject-position and explore and
understand the subject matter in whole new way. For Barthes, the outcome is
jouissance, or bliss; the writerly exploration of a text yields bliss in the
way a readerly text cannot.

For \citeA{Charon06}, the stakes are life and death. Narrative Medicine in treating the
experience of patients as a narrative to be explored, allows us to use
narrative skills and knowledge in order to understand and heal our patients'
illness. Charon directly teaches her students these skills:

\begin{quote}
	We teach our students fundamental skills of close reading and disciplined
	and considered reflective writing. We equip them with the skills to receive
	and critique respectfully and honestly what colleagues write. We introduce
	them to great literary texts and give them the tools to make authentic
	contact with works of fiction, poetry, and drama. We present complex theory
	from literary studies and the narrative disciplines. In settings as diverse
	as ward medicine attending rounds, staff meetings on the adult oncology
	in-patient service, the AIDS clinic, and home visit programs, we meet with
	health care professionals to read and to write, to attend to and to
	represent all that occurs in these lives led among the sick. As a result,
	we deepen our students' capacity to hear what their patients tell them.
	\cite[Preface]{Charon06}
\end{quote}

Charon asserts that this allows us to turn the patient's narrative into
a writerly text, allowing the physician to break out of her
subject-position and truly empathize with and understand the patient.
This has positive effects on four distinct areas of patient-physician
interaction: empathy, active listening, professionalism, and ethics.

\begin{enumerate}
\def\labelenumi{\arabic{enumi}.}

\item
  Empathy

  Charon asserts that health care cannot be effective without empathy, without
  genuine connection between physician and patient:

\begin{quote}
	For the sick patient to accept the care of well strangers, those strangers
	have to form a link, a passage between the sick and the healthy who tender
	care. \cite[p.~21]{Charon06}
\end{quote}

  For a patient to benefit from our care, she must accept it. For her to accept
  it, we must present it empathetically. To do so, we must understand the
  patient, we must listen. We cannot hope to do this without narrative
  competence, without exploring the patient's narrative of her life. Without
  that understanding empathy is impossible.

\item
  Active Listening

  In addition to empathy, Charon asserts that listening is key to diagnosis and
  treatment.

\begin{quote}
	health professionals have to learn to hear the body and the self telling
	of illness, in whatever forms, dictions, and discourses they find
	themselves giving utterance to their reality. If patients' reports do not
	limit themselves to answers in our reviews of systems, then we must be
	prepared to comprehend all that is contained in the patient's words,
	silences, metaphors, genres, and allusions. \cite[p.~107]{Charon06}
\end{quote}

  If we are unable to listen to our patient's explanations as they are, we can
  never hope to understand our treat the patients themselves. To develop
  narrative skill is to develop active listening, to render the speech act
  itself writerly and explore the patient's experiences in new ways. This can
  do nothing but help in communication.

\item
  Professionalism

  Communicating about health can be difficult. The stakes are high, and the
  topic is delicate. One wrong move could spell disaster. Saying something less
  than tactfully to the family of a terminally ill patient, for example, might
  permanently hurt them emotionally, sour their interaction with you a
  physician, and damage the quality of care you are able to provide to them. To
  communicate professionally, one has to understand what is appropriate to say
  and how to be tactful.  This cannot be done without an understanding of the
  patient and her family, as these concepts differ from person to person.
  Charon attests that narrative insight directly impacts this competence:

\begin{quote}
	At the same time that individual health care professionalsdeveloped
	insight into their practice and strengthened specific skills of attention
	and representation, we also found ourselves growing in honesty, altruism,
	collegiality, and duty, the hallmarks of health care professionalism.
	\cite[p.~225]{Charon06}
\end{quote}

\item
  Ethics

  Ethics in medicine has long been dominated by the idea that the
  patient-physician interaction is adversarial. It often has been, almost
  always at the expense of the patient. Informed consent and other advances in
  medical ethics have been attempts at mediating this interaction to protect
  the patient, but the assumption that patients must be protected from doctors
  has also done lasting damage to health care as a whole. Charon writes:

\begin{quote}
	In their zeal to protect patients' autonomy, some bioethicists designated
	as paternalism any expression of personal opinion or clinical counsel on
	the part of health professionals. So as not to manipulate patients, some
	doctors have ended up withholding their own viewpoints from confused
	patients, leaving patients and families to make their treatment choices
	alone. Protecting patients' autonomy, in the extreme, constitutes
	abandonment. \cite[p.~206]{Charon06}
\end{quote}

  To allow ourselves to embrace patients' stories, and allow ourselves to be
  co-authors of them, means to organically be a part of the healing process,
  including decision-making. While we should never obviously never take a
  patient's autonomy away, physicians ought to bring their understandings and
  competences to bear on their patient's care, helping them make the best
  decisions instead of leaving them to fend for themselves. Narrative Medicine
  allows for this necessary change.
\end{enumerate}

In these areas, and potentially in many others, Narrative Medicine gives us the
tools to be better empaths, better listeners, better doctors, and better
people. To reject Narrative Medicine as a method to explore the patient's
experience of her symptoms is to refuse our patients the best quality of care
that we are capable of giving them.

\bibliography{blog}
\bibliographystyle{apacite}

\end{document}
