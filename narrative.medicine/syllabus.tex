\documentclass[12pt]{article}
\usepackage{lmodern}
\usepackage{amssymb,amsmath}
\usepackage{ifxetex,ifluatex}
\usepackage{fixltx2e} % provides \textsubscript
\usepackage[T1]{fontenc}
\usepackage[utf8]{inputenc}
  
\usepackage[usenames,dvipsnames]{color}

\usepackage{graphicx,grffile}
\graphicspath{ {.resources/} }
\makeatletter
\def\maxwidth{\ifdim\Gin@nat@width>\linewidth\linewidth\else\Gin@nat@width\fi}
\def\maxheight{\ifdim\Gin@nat@height>\textheight\textheight\else\Gin@nat@height\fi}
\makeatother
% Scale images if necessary, so that they will not overflow the page
% margins by default, and it is still possible to overwrite the defaults
% using explicit options in \includegraphics[width, height, ...]{}
\setkeys{Gin}{width=\maxwidth,height=\maxheight,keepaspectratio}
\setlength{\parindent}{0pt}
\setlength{\parskip}{6pt plus 2pt minus 1pt}
\setlength{\emergencystretch}{3em}  % prevent overfull lines
\providecommand{\tightlist}{%
  \setlength{\itemsep}{0pt}\setlength{\parskip}{0pt}}
\setcounter{secnumdepth}{0}

\usepackage[colorlinks=false]{hyperref}

% Redefines (sub)paragraphs to behave more like sections
\ifx\paragraph\undefined\else
\let\oldparagraph\paragraph
\renewcommand{\paragraph}[1]{\oldparagraph{#1}\mbox{}}
\fi
\ifx\subparagraph\undefined\else
\let\oldsubparagraph\subparagraph
\renewcommand{\subparagraph}[1]{\oldsubparagraph{#1}\mbox{}}
\fi

\usepackage{dirtytalk}

\begin{document}

\title{Narrative Medicine, Social Justice, \& the Arts Syllabus}
%\shorttitle{Narrative Medicine Syllabus}
\author{Joanne Braxton}
%\affiliation{William \& Mary}
%\date{2015-09-31}

\maketitle

\noindent
\subsubsection{What is Narrative
Medicine?}\label{what-is-narrative-medicine}

\say{Practicing medicine involves the
intersection of science and art. A patient has not only symptoms but a
story which includes and helps explain these symptoms. The W\&M-EVMS
pilot has exciting possibilities for the training and practice of
healthcare providers, and demonstrates the value of the partnership
between our two institutions.} ---
\href{https://www.wm.edu/news/stories/2015/narrative-medicine-examines-charts-and-hearts.php}{W\&M
Provost Michael R. Halleran}

``Medicine, Arts and Social Justice'' explores routes to freedom and
equality for those who have been voiceless and whose communities are
often under-represented in the medical professions. Our work brings
together faculty and students from across the disciplines---both at
William and Mary and Eastern Virginia Medical School to increase
awareness of inequities in the American health care system. It also
seeks to augment the creativity and the reflective capacity of students
who may be exploring entering medical and health professions and to
improve their abilities to ``read,'' listen to, and respond to others as
well as to become reflective and resilient practitioners. Together we
will examine the nature of inequalities both in our communities and in
the health professions and explore selected modalities of intervention
currently being researched and taught at Columbia, Stanford and Duke
Universities, among others. We will further explore pathways for
literary scholars, storytellers, activists, helpers and healers to
address these issues through science, technology, engineering,
mathematics and the arts in the spirit of what Dr.~Kelly Crace has
defined as ``authentic excellence.''

In addition to reaching out to possible pre-med students and those
exploring careers in nursing and public health, this course offers
opportunities for creative writing, literary and film students to
explore emerging fields in the healthcare professions, for all
participants to cultivate cultural competency through engagement with
literature, film and rituals of diverse cultures. Visiting lectures
include faculty from Eastern Virginia Medical Schools and Emory
University. Course materials include materials from Ethics, Neuroscience
and Biology as well as literature and the arts, especially creative
writing.

\subsubsection{Who is this course for?}\label{who-is-this-course-for}

This course addresses issues of health equity and originates in Africana
Studies. It has been developed as part of the W\&M-Eastern Virginia
Medical School Narrative Medicine for Excellence Project. ``Medicine,
Arts and Social Justice'' is for those who crave diversity and
intellectual stimulation in a classroom setting that is authentically
multi-disciplinary and inter-cultural. It is for artists seeking careers
that overlap with the healing arts as well as pre-med types who love
stories, especially stories from the experiences of African,
Asian-American and African-American people. It is for scholars of Ethics
and/or Religious Studies exploring careers they haven't fully imagined
yet. It is for scientists who want to discover how literature can inform
methodology, for folks who are ready to be creative and to think and
play outside the box. It is for inquiring minds seeking solutions to
health problems that plague our diverse and varied communities and who
want to be in conversation with other like-minded people grappling with
similar theoretical and practical problems.

\subsubsection{Course Requirements:}\label{course-requirements}

Attendance and Participation, including in-class creative writing
assignments 10\%\\
Sustained Cumulative Assessment in the form of 6 blog posts 30\%\\
5 Page mid-term paper or personal blog site 20\%\\
Final Project: Paper, Verbatim or Creative Project 20\%\\
Final Examination (in class presentation of paper, personal blog or
project) 20\%

The six cumulative assessment blogs are to be posted to the class site
to which you have been invited by the instructor. Mid-term blogs and
final blogs presented in lieu of traditional papers are to be posted to
your personal blog site. If you wish to present a personal blog as your
final creative project, you must do a personal blog in lieu of your
mid-term paper. However, if you do a mid-term blog in lieu of your
mid-term, you may choose a different project for your final. Personal
blogs submitted for major requirements are in addition to the 6 blog
posts required of everyone for cumulative assessment. Taken together,
these 6 posts, which must be a minimum of 600 words each and show
engagement with course material, replace the traditional mid-term exam.
Students are encouraged to blog on their personal sites throughout the
semester and to revise, edit and update frequently. It is recommended
that students blog at least once weekly. You may expand shorter blogs
from your personal site for posting as one or more of the required 600
word blog posts. Final projects should teach, illuminate, or raise a
call to action.

\textbf{Unique Opportunities for Engagement:} Students are encouraged to
seek a conference with the professor at least 2-3 times during the
semester. The purpose of these conferences is twofold: 1) Students can
ask questions, explore new directions and deepen their understanding of
the materials in the course. 2) The professor values student feedback
for quality improvement throughout the semester. The professor's weekly
office hours are posted, and you may drop in for a conference during
office hours. In addition, W\&M-EVMS Narrative Medicine for Excellence
Team Members Gemeda, Sriraman, Babineau, Hinton, Sher, Green and
Tanglao-Aguas may be available for special consultation and mentorship
on request. Students are encouraged to explore their creativity.

\subsubsection{Objectives:}\label{objectives}

\begin{itemize}
\tightlist
\item
  Application of the methodologies of New Criticism to understanding
  specific texts.
\item
  An improved their ability to ``read,'' listen to, and to respond to
  others.
\item
  Increasing creativity and expressiveness, as well as the capacity for
  analysis and reflection across the disciplines.
\item
  Engaging students with scientists, artists and medical professionals
  who appreciate the practical necessity of narrative competence and the
  healing power of story.
\item
  Cultivating cultural competency through engagement with literature,
  film and rituals of diverse cultures, with an emphasis on, but not
  limited to, Asian-American and African-American contexts and
  inter-cultural dialogues.
\item
  Theorizing and Practice of the Ethical Principles of Social and
  Restorative Justice.
\end{itemize}

\subsubsection{Guest Lecturers:}\label{guest-lecturers}

Brian Dias, Postdoctoral Research Scholar, Neuroscience, Emory
University\\
Mekbib Gemeda, Vice President for Diversity, EVMS\\
Natasha Sriraman, M.D., Pediatrics Faculty, EVMS\\
Theresa Babineau, M.D., Family Medicine, Director of the HOPES Clinic,
EVMS\\
Others, TBA

\subsubsection{Required Readings:}\label{required-readings}

Alexi Auld, Canto Tonto Pocahontas (coming of age story)\\
Rita Charon, Narrative Medicine (e-book recommended)\\
Toni Morrison, Home\\
David Small, Stitches\\
Rosalyn Story, Wading Home\\
Abraham Verghese, Cutting for Stone

\subsubsection{Short Texts on
Blackboard:}\label{short-texts-on-blackboard}

Tim Cunningham, ``Fatima,'' blog post\\
Brian Dias, Ph.D. , 3 Scientific Essays (Neuroscience)\\
Anne Fadiman, The Spirit Catches You and You Fall Down, excerpt\\
Sayantani Das Gupta, Her Own Medicine: A Woman's Journey, excerpt\\
Henry Dumas, ``Ark of Bones,'' title story, Ark of Bones\\
Jennifer Lee, ``Terminal Device,'' short story\\
Toni Morrison, ``Baby Suggs Sermon'' from Beloved\\
Jacques P. Thiroux and Keith W. Krasserman, Ethics, excerpt\\
Quincy Troupe---3 poems, ``The Times We Live In,'' ``A Poem for All
So-Called Half-Breeds,'' and ``Memory as a Circle''\\
Laura van Dernoot , Trauma Stewardship, excerpt\\
Alice Walker, ``The Revenge of Hannah Kemhuff,'' short story\\
Walton and Cohen, ``A Question of Belonging: Race, Social Fit and
Achievement.''

\paragraph{Recommended}\label{recommended}

Laurie Kaye Abraham, Mama Might Be Better Off Dead: The Failure of
Health Care in Urban America\\
Belinda, ``The Cruelty of Men Whose Faces Were Like the Moon''\\
Alison Bechdel, Fun Home (graphic novel)\\
Tom Feelings, Middle Passage\\
Mindy Fulllove, Root Shock\\
Mindy Fullilove, Urban Alchemy\\
Robin Wall Kimmerer, Braiding Sweetgrass\\
Rebecca Skloot, The Immortal Life of Henrietta Lacks\\
A.K. Summers, Pregnant Butch\\
Howard Thurman, Jesus and the Disinherited\\
Abraham Verghese, My Own Country\\
Abraham Verghese, The Tennis Partner

\subsubsection{Schedule}\label{schedule}

\textbf{August 27 ---} ``Narrative Healing and Social Justice,'' Course
Introduction.

September 1 ``Writing in the Shadow/Writing in the Clearing: How Do
Stories Heal?''

\begin{itemize}
\tightlist
\item
  Charon, ``What is Narrative Medicine?'' and ``Close Reading,''
\item
  https://www.youtube.com/watch?v=24kHX2HtU3o.
\end{itemize}

\textbf{September 3 ---} Understanding Difference.

\begin{itemize}
\tightlist
\item
  Walton and Cohen, ``A Question of Belonging: Race, Social Fit and
  Achievement.'' Journal of Social Personality, 2007, Vol. 92, 82-96
\item
  ``Langston Hughes on the White Campus,'' Braxton
\end{itemize}

\textbf{September 8 ---} ``Understanding Health Equity and Why it
Matters to All of Us.'' Mekbib Gemeda, VP for Diversity at Eastern
Virginia Medical School.

\begin{itemize}
\tightlist
\item
  Gemeda, ``Crossing Cultures: Reflections on Language,''
\item
  Charon, ``Bridging Health Care's Divides''
\item
  \textbf{Note:} If you are struggling with the amount of reading, read
  the Gemeda essay first and come back to the remaining material when
  you can.
\end{itemize}

\textbf{September 10 ---} ``How Does Narrative Medicine Heal the
Healer?'' Natasha Sriraman, M.D., Pediatrics Faculty, EVMS.

\begin{itemize}
\tightlist
\item
  \textbf{Note:} The readings for this day are all review, and
  Dr.~Sriraman will be reviewing some of the material you have already
  covered.
\item
  Review Charon's ``Narrative Medicine'' and ``Close Reading.''
\item
  Also review first day PP.
\end{itemize}

\textbf{First blog post due to class site. Consider posting your revised
600 word engaged definition of NM. Respond to at least 3 other posts.
Responses should be at least 100 words. Complete the responses any time
before September 15.}

\textbf{September 15 ---} Required Close Reading: Verghese, Cutting for
Stone, Parts One and Two. Write one post ``in the shadow of'' Part One
and a second ``in the shadow of'' Part Two.

\textbf{September 17 ---} Required Close Reading: Verghese, Cutting for
Stone, Parts Three and Four.

\textbf{September 22 ---} Guest Lecturer, Timmy Cunningham.

\begin{itemize}
\tightlist
\item
  Cunningham, ``Fatima''
\end{itemize}

\textbf{Second blog post due to class site. Respond to at least 3 other
posts.}

\textbf{September 24 and September 29 ---} ``Close Readings of Toni
Morrison's Home: Intersectionality and Intersubjectivity.''

\begin{itemize}
\tightlist
\item
  Toni Morrison, Home. ``Writing in the shadow''
\item
  Soul Repair by Lettini and Brock
\end{itemize}

\textbf{October 1 ---} Class Visit by Dr.~Brian Dias, Reves Center
Distinguished Visiting Lecturer and Prof.~Shanta' Hinton, W\&M Biology
Department and Narrative Medicine for Excellence team member.

\begin{itemize}
\tightlist
\item
  Essays by Brian Dias:

  \begin{itemize}
  \tightlist
  \item
    ``Towards New Approaches to Disorders of Fear and Anxiety,''
  \item
    ``Parental Olfactory Experience Influences Behavior and Neural
    Structure in Subsequent Generations''
  \end{itemize}
\item
  Dr.~Dias' blog about his work with Tibetan Monks.
\item
  ``Blackburn, et. al. ``Can Meditation Slow the Rate of Cellular
  Aging?''
\item
  ``DNA and Trauma: Epigenetic Approaches to Understanding Fear.''
\item
  ``\href{https://www.youtube.com/watch?v=fMxgkSgZoJs}{The Ghost in Your
  Genes}''
\end{itemize}

\textbf{October 6 ---}

\begin{itemize}
\tightlist
\item
  Henry Dumas ``Ark of Bones''
\item
  Walker's `The Revenge of Hannah Kemhuff.''
\end{itemize}

\textbf{October 8 --- TBA.} Mid-term paper or project due.

\textbf{October 9-12 --- Enjoy fall Break! Optional visit, Hopes Free
Clinic, EVMS (TBA)}

\textbf{October 15 ---} Post your ``writing in the shadow'' blogs by
October 14. Your critical commentary and questions will shape our
classroom discussions.

\begin{itemize}
\tightlist
\item
  Rosalyn Story's Wading Home. \emph{What are the multiple meanings of
  ``wading'' and ``home'' in this work?}
\item
  Spike Lee's ``When the Levees Broke''
\end{itemize}

\textbf{Third blog post due to class site. Respond to at least 3 other
posts.}

\textbf{October 20 ---} ``Each One Teach One: How to Develop a Creative
Writing Prompt From Any Text. The professor will model Wading Home
writing prompts. Post your second ``writing in the shadow blog'' by
October 19.

\begin{itemize}
\tightlist
\item
  Rosalyn Story's Wading Home.
\end{itemize}

\textbf{October 22 ---}

\begin{itemize}
\tightlist
\item
  Stitches, by David Small.
\item
  ``Welcoming the Unbidden: The Case for Conserving Human
  Biodiversity,'' Rosemarie Garland-Thompson.
\end{itemize}

\textbf{October 27 ---}

\begin{itemize}
\tightlist
\item
  Stitches, by David Small.
\item
  Summers, Pregnant Butch.
\end{itemize}

\textbf{Fourth blog post due to class site. Respond to at least 3 other
posts.}

\textbf{October 29 --- TBA.}

\textbf{November 3 ---}

\begin{itemize}
\tightlist
\item
  Laurie Kaye Abraham, Mama Might Be Better Off Dead: The Failure of
  Health Care in Urban America, Chapters 6 and 7.
\item
  Abraham Verghese, Cutting for Stone, chapter 40, ``Salt and Pepper.''
\end{itemize}

\textbf{November 5 ---}

\begin{itemize}
\tightlist
\item
  The Spirit Catches You and You Fall Down
\end{itemize}

\textbf{Fifth blog post due to class site. Respond to at least 3 other
posts.}

\textbf{November 10 ---} Short stories and poems from the required list,
TBA.

\textbf{November 17 and 19 ---}

\begin{itemize}
\tightlist
\item
  Alexi Auld, Tonto Canto Pocahontas.
\end{itemize}

\textbf{December 1 ---} Short stories and poems from the required list,
TBA.

\textbf{Sixth and final blog post due to site. Respond to at least 3
other posts.}

\textbf{December 3 ---} Closing Discussion. Final paper or project due.

\end{document}
