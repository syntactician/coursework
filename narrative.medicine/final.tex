\documentclass[12pt]{article}
% \usepackage{etoolbox,keyval,ifthen,url,csquotes}
\usepackage[notes,strict,backend=biber,autolang=other,%
bibencoding=inputenc]{biblatex-chicago}
\usepackage{setspace,doi}

\addbibresource{course.bib}
\addbibresource{extra.bib}

\setlength{\oddsidemargin}{-0.25in} % Left margin of 1 in + 0 in = 1 in
\setlength{\textwidth}{7in}   % Right margin of 8.5 in - 1 in - 6.5 in = 1 in
\setlength{\topmargin}{-.75in}  % Top margin of 2 in -0.75 in = 1 in
\setlength{\textheight}{9.2in}  % Lower margin of 11 in - 9 in - 1 in = 1 in
% \linespread{1.6}
\doublespacing


\renewcommand{\footnoterule}{%
  \kern -3pt
  \hrule width 3in height 1pt
  \kern 5pt
}

\begin{document}
\title{I \emph{am} my Identity: \\ A Cultural Critique of \emph{Narrative Medicine: Honoring the Stories of Illness}}
\author{Edward Hern\'{a}ndez}
\date{15 December 2015}
\maketitle

\abstract{This paper introduces concerns with Charon's \emph{Narrative
Medicine: Honoring the Stories of Illness}, particularly its conceptions of the
self and the role of narrative skills. It presents an alternative
interpretation of the self, the physician, and their interaction via narrative
medicine.}

\section{Introduction}

Charon asserts that patients, in a clinical setting, are reminded of their
``mortality and frailty and ultimate end,''\autocite[p.~78]{Charon06} and in
view of this, reveal ``the core of who they are.''\autocite[p.~78]{Charon06}
She asserts that they reveal this ``core'' by ``par[ing] away the optional
layers . . . ---occupation, habits, even history and
culture,''\autocite[p.~78]{Charon06} because, on her view, the ``self-who-tells
in the medical context . . . is eclipsed by bodily
concerns.''\autocite[p.~78]{Charon06}. She uses this claim to motivate
narrative medicine: since the patient expresses their core in the clinical
setting, we must train health care professionals to receive her narrative.%
	\footnote{I contest her secondary claim here, that the physician
	``replac[es] the confessors or spiritual advisors of former times.''
	Religion is by no means over. Spiritual advisors are still a reality in
	many patients' lives. The physician should not presume to have taken over
	any such role.}

Charon does hold that narratives are ``frame[d]''\autocite[p.~114]{Charon06}
by the reader, and that each `reading'%
	\footnote{The written narrative is used as a metaphor for the spoken one
	throughout the text (presumably to facilitate direct application of
	narratological theory), and will likewise be used throughout this paper.}
of a narrative ``has its own frame,''\autocite[p.~115]{Charon06} and that that
frame ``depend[s] on the text . . . but also on the situation of the reader and
writer---personally, temporally, and culturally.''\autocite[p.~115]{Charon06}
Charon seems to here occupy a relatively typical postmodern position on
reading.\autocite[p.~40]{Charon06} From this position, she is able to employ
Roland Barthes' notions of the `readerly' (\emph{le lisable}) and the
`writerly' (\emph{le scriptable}).\autocite{Barthes70} Readerly texts can only 
be read one way, and the reader cannot contribute to that meaning. In contrast,
writerly texts ``require active creation from each
reader''\autocite[p.~46]{Charon06}, making each reader a coauthor ``not by
virtue of observing what its author did but by virtue of performing what the
text compels.''\autocite[p.~46]{Charon06}

The aim of narrative medicine, as she describes it, is to bring narrative
skills to bear on the practice of medicine. This will equip doctors with tools
to better evaluate their patients' expressions of their symptoms in order to
``absorb . . . patients' multiple, often contradictory, stories of
illness''\autocite[p.~4]{Charon06} in order to understand the patients and
their conditions better. This will ``bridge some of the divides between the
sick and the well,''\autocite[p.~12]{Charon06} which she holds is key to
effective healthcare.

\section{Embodiment}

I argue that Charon's view of the self---as some core identity with other
`optional' identities attached, but somehow removable---is mistaken, especially
as it applies to culture and history.
Neither culture nor history can be taken off. They are part of the core of who we are.
They cannot be, as Charon claims, ``eclipsed by bodily concerns,'' because they
are the very same sort of concern as illness and other physical threats.

Culture is a large part of who we are. This is borne out in narratives but also
in the research of cultural psychology. Culture informs conceptions of
self,\autocite{Markus10} understandings of disease,\autocite{Fadiman97}
orientation to authority (like doctors)\autocite{Tyler00}, concepts of
politeness,\autocite{Yin09} among many other medically consequential beliefs
and behaviors. Cultures are also tied to religions, which come with sets of
consequential beliefs about the self, the body, illness, and mortality. A 
patient cannot `pare away' these beliefs,%
	\footnote{Neither can the physician pare away her cultural beliefs.}
because they are at the core of who they are. Culture and mind are ``mutually
constituted,''\autocite[p.~1423]{Heine10} and they cannot be meaningfully
separated.
% Psychology tell us that this is impossible, that ``culture and the mind can
% be said to be mutually constituted,''\autocite[p.~1423]{Heine10} that these
% ``layers'' are not, in fact, optional.

Neither can history be pared away. Patients of color, black women in
particular, cannot simply take off their history of being subjects of medical
experimentation, rather than treatment. Bettina Judd in
\emph{Patient},\autocite{Judd14} describes her ``ordeal'' with medicine, 
dealing not only with her physical symptoms, but also with racist
microagressions and the ever-present knowledge that ``gynecology was built on
the backs of black women.''\autocite{Judd15} This awareness is necessary
consequential for patient-physician interaction.

Even if the patient is not explicitly aware of this history, or aware of it in
its totality, it is not pared away (and cannot be). History still effects them
directly; aggressions, micro- and macro- alike, are still routinely leveled at
black women in clinical contexts, from Judd's experience with probing,
incredulous questioning about her sexual history to Terry Ragland's experience
being diagnosed with ``ghetto booty.''\autocite{Grenoble13,McCowan13} Judd
notes that microagressions make it clear that you are not welcome, and, ``what
microaggressions always point to is a history.''\autocite{Judd15} This history
looms large over those who feel its effects. It cannot be pared away.

Culture and history are both inseparable from the self and from social
interactions. Moreover, they are also inseparable from the body. Charon calls
the body ``a common destination on the pilgrimage of
identity,''\autocite[p.~86]{Charon06} and ``the passport, the warrant, the seal
of one's identity.''\autocite[p.~86]{Charon06} She describes efforts to create,
claim, or attain identity via ``tattoo[s], embellish[ments], and
scarif[ication].''\autocite[p.~86]{Charon06} While these actions are
interpretable as assertions of identities (or histories), it is important to
consider other identities are inherently perceptible, without intentional
assertion.%
	\footnote{It is also important to consider in what direction this causation
	flows. Charon seems to conceive of body modification as a project to create
	identity, to ``fix'' the self by fixing the body. But does the punk truly
	get tattoos in order to become a punk? Or does she get tattoos to display
	(or affirm) her pre-existing punk identity. This causative order is of
	particular importance when discussing queer identities. Charon seems to
	claim that ``[t]ranssexual surgery'' is a biological attempt to ``fix a
	failed [identity],'' a claim I would very much like a challenge.}
Charon is right that ``identities are clearly, directly, and
irrevocably tied to their bodies,''\autocite[p.~87]{Charon06} but this extends
to identities other than those we \emph{choose} to portray.

% She is correct in her assertion that identity has visible correlates, but her
% ideas about causation appear to be backwards. Is it the case that ``punks
% tattoo . . . their bodies''\autocite[p.~86]{Charon06} \emph{in order to
% become} punks? Or do they do so to \emph{reflect} their punk identity? 

My skin is always visible. It shows my racial and ethnic identities. The way I
move is visible. Often, it shows me to be queer. In a clinical context, my scars
are visible. They show my history. I cannot avoid this visibility. I cannot
`pare away' my skin to express some disembodied self to a physician. Even if I
could, it would be worse than useless; these things might well be relevant to
treating me.  Like my skin shows my race, my language shows my culture. It is
always audible.

Language (and variety within that language) mark cultural identity.  Even if
everyone involved speaks the same language, word choice,\autocite{Pullum99}
pronunciation,\autocite[p.~57-58]{Thomas07,Anzaldua87} greetings, politeness
strategies,\autocite{Maha14} even names\autocite{Fryer04b} are all influenced
by and reflect particular cultures. As Gloria Anzald\'{u}a so eloquently puts
it: ``Ethnic identity is twin skin to linguistic identity---I am my
language.''\autocite[p.~59]{Anzaldua87} Like my skin, constantly visible, shows
my race, my language shows my culture. I \emph{am} the culture that my language
shows. I \emph{am} my language. I \emph{am} my culture. These identities are
embodied in my speech, in my skin, in my self. I cannot pare them away, or hide
them, nor would I if I could. If medicine is to treat me, it must accept me as
I am, as I speak, with these identities intact.

These cultural identities and historical positions in large part comprise me.
To limit my expression of them, or to devalue them, is to devalue me. In
Anzald\'{u}a's words, ``[I]f you want to really hurt me, talk bad about my
language,''\autocite[p.~59]{Anzaldua87} or about my culture, or my history.
These identities are embodied, and when they are under threat, I am under 
threat. They cannot be eclipsed by bodily concerns, because they \emph{are}
my bodily concerns.

A previous critique of narrative medicine, written by Esther Jones, asks
whether ``narrative medicine adequately engage[s] with the problem of
difference and the pernicious operation of pathological stereotyping which
likely accounts for the breaches in ethical behavior enacted by medical
practitioners upon blacks and others who embody
difference.''\autocite[p.~190]{Jones15a} I argue that Charon, in fact, does not
adequately engage with difference. She does certainly does not engage with
embodied difference, as I discuss above. For her, difference is still a problem
to be solved, a divide to be bridged. 
Jones brings up in particular the history of medical racism that I discuss
above, citing Terry Ragland and her diagnosis with ``ghetto booty''
specifically. She also brings up religious difference and the ways in which it
effects health (and may conflict with science or medicine). I agree that Charon
fails to engage with these issues, and that medicine \emph{must} engage with
them to be successful.
% Jones calls for an embrace of science fiction (particularly science fiction by
% black women writers) as a tool for ``interrogat[ing] our naturalized
% assumptions about our social patters and behaviors, how we construct
% difference,''\autocite[p.~191]{Jones15a} etc., asserting that it has the
% capacity to challenge our most fundamental assumptions.
However, I do not agree with Jones that this failure makes narrative medicine
untenable.%
	\footnote{I do leave open the possibility that her focus on black science
	fiction and its denaturalizing power is useful in conceptualizing the
	future of teaching narrative skills.}
Charon is not the only scholar in the field. Her work has been
expanded by doctors like Sayantani DasGupta, whose ideas about narrative
humility give us a hopeful starting place for building a new, liberatory vision
of narrative medicine.


\section{Cultural Humility}

Since culture cannot be separated from the patient, physicians must be prepared
to engage with the patient as a cultural being. This is, generally speaking,
the goal of `cultural competence.' Such efforts are absolutely
essential to treating patients of diverse backgrounds.
Cross, who coined the term `cultural competence,' envisioned it as a set of
behaviors, attitudes, and policies aimed at effective cross-cultural
communication.\autocite{Cross89} This is a lofty goal, and many people have
taken it up. While many of those efforts have been successful, others have not.
Cultural  training is now widespread, but it is less than ideal.

Sayantani DasGupta recounts her experience in a classroom at Johns Hopkins, 
being taught cultural , of ``being given a list of ten things that
Dominican Americans believe, or ten things that Southeast Asians
do.''\autocite{DasGupta13} If you memorize the list, you're `competent' to deal
with Dominican Americans or Southeast Asians. This sort of list is still very
much in use.\autocite{HealthCareChaplaincy13} Learning beliefs or practices
likely to be shared by particular cultural groups is not a \emph{bad} idea. It
is useful information to have in interacting with members of those groups. But
it is not the only tool one needs, and to try to distill cultures to lists to
be memorized rather than engaging with and learning about those cultures
reinforces their position as \emph{other}. It reinforces a hierarchy with the
dominant culture (and the doctor who represents it in the clinical setting)
firmly on top.

DasGupta brings up an alternate framework: ``cultural humility.'' This shifts
the focus from lists to listening.  It has been defined as the ``ability to
maintain an interpersonal stance that is other-oriented (or open to the other)
in relation to aspects of cultural identity that are most important to the
[patient]''\autocite{Hook13} This is a promising improvement over cultural
 as it is currently practiced. If you have only learned lists, what
do you do when a patient's culture is not on the list? You would not have been
taught tools for dealing with that patient. If you have training in listening,
rather than just in facts, you have tools to engage with the patient, to derive
meaning from what they have to say. This is where narrative medicine shines. In
conjunction with a framework for valuing patients' cultures and identities.

Cultural competence presumes the experience of the patient to be a readerly
text. Beliefs or practices in the culture of the patient determine in advance
the meaning of their narrative. When they say \emph{this}, they mean
\emph{that}.  Such training guides a physician through the reading of a
patient's narrative, not allowing for alternate readings, for writerly
exploration, for \emph{joissance}. Cultural humility, on the other hand, does
note presume meaning. It allows the patient to express all of her embodied, 
cultural identities. It allows the `self-who-tells' to be a full being.  Charon
asserts ``The sick person needs to continue to be, somehow, the self . .  . she
was before illness struck.''\autocite[p.~21]{Charon06} Insofar as this is true,
we must allow the patient to express herself fully and culturally. If we expect
the patient to `pare away' layers, to express herself in a way that we expect
(or a way that we understand), we ask her to cease to be this self. If we
presume her meaning based on our limited training in her culture, we fail to
listen to her complex narrative, we fail to meet her where she is. Either way,
we ask her to change. We ask her to risk her healing process. We cannot, in
good conscience ask this of a patient.  We need to be humble. We need to
exercise cultural humility, allow the patient to express the identities that
are important to them, and truly listen.

Not only does cultural humility strengthen narrative medicine, narrative
medicine supports cultural humility. It offers a new way to teach cultural
humility. DasGupta, in her work with the Columbia Pediatrics Program, used
narrative medicine to teach cultural  in a way that stressed
humility.\autocite{DasGupta06} Rather than putting the culture of doctors at
the top of a hierarchy, her teaching method involved bidirectional exchange of
cultural narratives between doctors, patients, and community workers. The
program revealed disconnects in culture and understanding, which had been
previously undressed, pointing toward possible resolution in the future.
Narrative skills paired with focus on humility allow for dialog which previous
cultural competence models do not. That dialog is key to both narrative
medicine and cultural humility.


\section{Liberation}

Cultural competence (even with its current weaknesses) serves to improve the
health care of those commonly failed by the healthcare system. Culturally
competent doctors serve black patients better---so much so that a physician's
cultural competence has been shown to be a better predictor of successful
interactions with black patients than the physicians
race.\autocite{Michalopoulou09} Cultural competence has been shown to improve
the rate of successful treatment plans and improve health care for underserved
groups, including Latinas, Cambodian refugees, and Muslims.\autocite{Padela08}
Cultural humility promises to be even more successful, setting its sights
explicitly on ``redressing . . . power imbalances''\autocite{Tervalon98} and
improving healthcare equity.

However, we need much more than cultural competence, more, even, than cultural
humility.  We need a system of practicing healthcare which aims to liberate
those ordinarily failed by the healthcare system.  We need not just to orient
ourselves in a particular way to patients, as cultural humility suggests, but
also to allow them agency.


% better medicine for the oppressed\autocite{Michalopoulou09,Ohana15}

% Reactive theology.\autocite{}

To be truly effective, truly inclusive, narrative medicine must provide the
possibility of healthcare not only in English, but in the languages spoken in
the communities which it serves. Lack of translation (and mistranslation by
overworked or undertrained medical translators) causes disastrous
miscommunication between physicians and
patients.\autocite{Flores06} If we are to respect difference and accept that
patients come with various cultures and languages, we cannot expect them to 
speak as we do; we cannot expect that they can (or should) translate themselves
to be comprehensible to us. To ask them to translate themselves is to limit
them, limit what they can express. As Anzald\'{u}a puts it:
\begin{quote}
	Until I am free to write bilingually and to switch codes without having
	always to translate, while I have to speak English or Spanish when I would
	rather speak Spanglish, and as long as I have to accommodate the English
	speakers rather than having them accommodate me, my tongue will be
	illegitimate.\autocite[p.~59]{Anzaldua87}
\end{quote}
We cannot pretend to listen to patients unless we allow them to express 
themselves freely, in their own language. We cannot invalidate that part of
them and still be culturally competent, culturally humble. We must accommodate
speakers of all tongues, as we must welcome members of all cultures.
We must allow patients' narratives to come to us truly writerly, full of
cultural codes which we do not understand, and we need to develop the skills
to commune with the patient and make sense of them. Anything less is 
insufficient to liberate the patient from the hierarchy that is inherent in
current medicine.

\section{Conclusion}

I think that narrative medicine provides the tools for medical liberation.
While Charon's book does fail to adequately address these issues, the seeds
planted by her and DasGupta's work are bearing fruit, teaching cultural
competence and humility. There is now discussion about the need for narrative
skills, listening to patients in order to provide quality care. There is 
discussion of providing culturally humble care, and linguistically appropriate
care.\autocite{Koh14} With concerted effort, these movements could push forward
together, paving the way for a culturally inclusive, equitable healthcare
system, which foregrounded the patient and her story. I am hopeful that just
such an outcome is in sight.

% ``Ethnic identity is twin skin to linguistic identity---I am my
% language.''\autocite[p.~59]{Anzaldua87}

% ``The sick person needs to continue to be, somehow, the self . . . she was
% before illness struck.''\autocite[p.~21]{Charon06}

% \footnote{I also take issue with her assertion, in the same paragraph, that
% health care professionals ``replac[e] the confessors or spiritual advisors of
% former times''.}
% \footnote{I take issue with Charon's deployment of metaphors of blindness and
% deafness, as well.}
% \footnote{Transsexual}

% If we think of the ``self-who-tells'' as a cultural being, a necessarily
% cultural being, we must conceive of narrative medicine entirely differently.
% The skills central to narrative medicine, valuing narrative, listening
% deeply, are still valuable, still essential. They are merely applied
% differently and get us farther.

% DasGupta uses Narrative Medicine to teach Cultural
% Competency.\autocite{DasGupta06}

% \section{Conclusion}


% \clearpage
% Here are the rest of my sources so far:

% Cultural competence:
% \autocite{Cross89}
% \autocite{LavizzoMourey96}
% \autocite{Tervalon98}

% Narrative Medicine/Humility:
% \autocite{Charon11}
% \autocite{DasGupta03}
\nocite{DasGupta08}

% Language + Medicine:
% \autocite{Koh14}
% \autocite{Sullivan95}

% Liberation Theology:
% \autocite{Eiesland05}
% \autocite{Eiesland94}

% Jones:
% \autocite{Jones15a}
\nocite{Jones15b}
\nocite{Lockhart07}
% \autocite{Jones15b}
% % \autocite{Grenoble13,McCowan13}

% Misc:
% \autocite{Fadiman97}
% \autocite{Morris00}
% \autocite{Perakis13}
% \autocite{Judd15}

\clearpage
\printbibliography

\end{document}
