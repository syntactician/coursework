\documentclass[12pt]{article}
\usepackage{etoolbox,keyval,ifthen,url,csquotes}
\usepackage[notes,strict,backend=biber,autolang=other,%
bibencoding=inputenc]{biblatex-chicago}
\usepackage{setspace}

\addbibresource{course.bib}
\addbibresource{extra.bib}

\setlength{\oddsidemargin}{-0.25in} % Left margin of 1 in + 0 in = 1 in
\setlength{\textwidth}{7in}   % Right margin of 8.5 in - 1 in - 6.5 in = 1 in
\setlength{\topmargin}{-.75in}  % Top margin of 2 in -0.75 in = 1 in
\setlength{\textheight}{9.2in}  % Lower margin of 11 in - 9 in - 1 in = 1 in
\linespread{1.5}

\renewcommand{\footnoterule}{%
  \kern -3pt
  \hrule width 3in height 1pt
  \kern 5pt
}

\begin{document}
\title{A Cultural Critique of \emph{Narrative Medicine}}
\author{Edward Hern\'{a}ndez}
\date{\today}

\maketitle

\abstract{}

\section{Introduction}

\begin{itemize}
	\item give a definition of narrative medicine (as Charon defines/uses it)
	\item express misgivings
	\item throw in an explanatory foot/endnote about other concerns (disability, queerness, separation of ill and well)
	\item outline three part system of cultural competence, empathy?, liberation, to explore the failings of Charon's system and the possibility which a reconceptualization opens up

\end{itemize}

\section{Cultural Competence}

\begin{itemize}
	\item explain the lack of cultural competence
	\item explain how Charon does not adequately respond to The Spirit Catches You and You Fall Down
	\item use cultural psychology to emphasize that the self is cultural
	\item bring in Linguistics to show that culture is not an optional layer -- it's who we are
	\item use DasGupta's work to illustrate that narrative medicine \emph{can} respond to this concern
\end{itemize}

Charon asserts that ``when patients talk about themselves to their doctors and
nurses, they are revealing aspects of the self closest to the skin, having
pared away the optional layers, if you will—occupation, habits, even history
and culture—to get down to the core of who they
are.''\autocite[p.~78]{Charon06} She conceives of the ``self-who-tells'' as an
acultural being, with those ``optional layers'' of the self ``eclipsed by
bodily concerns.''\autocite[p.~78]{Charon06}
However, Social and Cultural Psychology tell
us that this is impossible, that ``culture and the mind can be said to be
mutually constituted,''\autocite[p.~1423]{Heine10} that these ``layers'' are
not, in fact, optional.

If we think of the ``self-who-tells'' as a cultural being, a necessarily
cultural being, we must conceive of narrative medicine entirely differently.
The skills central to narrative medicine, valuing narrative, listening deeply,
are still valuable, still essential. They are merely applied differently and
get us farther.

DasGupta uses Narrative Medicine to teach Cultural Competency.\autocite{DasGupta06}

\section{Embodiment}

\begin{itemize}
	\item make sure that this was the second point of the three we talked about
	\item talk about Charon's idea of narrative imagination
	\item argue that we can't imagine the situation of other's that we can't understand, but we can listen
	\item use DasGupta's idea Narrative Humility\autocite{DasGupta13} here
\end{itemize}

\section{Liberation}

\begin{itemize}
	\item we can't be liberated as acultural beings
	\item conceiving of patients as without culture (or as having their cultures ``eclipsed'') privileges those whose cultural ``layers'' are like ours
	\item tie to the Disabled God?
\end{itemize}

\section{Conclusion}

\begin{itemize}
	\item where to address Jones?
	\item make sure that it doesn't come off as an attack, but a \emph{hopeful} critique
\end{itemize}

\clearpage

Here are the rest of my sources so far.
\autocite{Eiesland05}
\autocite{Eiesland94}
\autocite{Fadiman97}
\autocite{Jones15a}
\autocite{Jones15b}
\autocite{Morris00}
\autocite{Sullivan95}

\clearpage
\printbibliography

\end{document}
