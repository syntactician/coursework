\documentclass[doc,12pt]{apa6}
\usepackage[colorlinks=false]{hyperref}
\usepackage{amssymb,amsmath,times,gb4e}
\linespread{1.5}

\begin{document}

\title{Homework 2}
\shorttitle{Homework 2}
\author{Edward Hern\'{a}ndez}
\date{26 February}
\affiliation{Computational Methods}
\maketitle

\begin{verbatim}
[CST*]|who|which [N*]|[P*] [V*]
[CST*]|who|which [V*] [N*]|[P*]
\end{verbatim}

The central problem is that COCA simply tags by part of speech. There is
no way to ensure that a query returns only sets of words which are truly
in these grammatical structures; we must guess

For instance, this is a concordance line from the corpus, showing a sentence
from the film Clerks which contains two phrases caught by the query
\texttt{who\ {[}V*{]}\ {[}N*{]}}:
\begin{exe}
	\ex . . . Ass, Play with my Puss, Three on a Dildo, Girls Who Crave Cock, Girls Who Crave Cunt, Men Alone Two-The K.Y. Connection, Pink Pussy Lips . . .
\end{exe}
Obviously, neither `Who Crave Cock' or `Who Crave Cunt' are here relative
clauses (though they usually would be).

WHO MADE IT

\begin{verbatim}
	which|who [V*]
	which|who [V*] [N*]|[P*]
	which|who [V*] [AT*]|[APPGE*] [N*]|[P*]
	[CST*] [V*]
	[CST*] [V*] [N*]|[P*]
	[CST*] [V*] [AT*]|[APPGE*] [N*]|[P*]

	which|who [N*]|[P*] [V*]
	which|who [AT*]|[APPGE*] [N*]|[P*] [V*]
	[CST*] [N*]|[P*] [V*]
	[CST*] [AT*|[APPGE*] [N*]|[P*] [V*]
	
	[N*]|[P*] [N*]|[*P] [V*]   ********
\end{verbatim}

\end{document}
