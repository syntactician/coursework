\documentclass[doc,12pt]{apa6}
\usepackage[colorlinks=false]{hyperref}
\usepackage{amssymb}
\usepackage{amsmath}
% \usepackage{ellipsis}
\usepackage{times}
\usepackage{gb4e}
\linespread{1.5}

\begin{document}

\title{Homework 2}
\shorttitle{Homework 2}
\author{Edward Hern\'{a}ndez}
\date{26 February}
\affiliation{Computational Methods}
\maketitle

To understand the differences between subject and object relative clauses, we
might want to begin by determining the (relative) frequencies of these
structures. If it is the case that subject relative clauses are less frequent
in the input, that might begin to explain why they are harder for us to
process.

Unfortunately, COCA tags only parts of speech. There is no way to search
explicitly for a particular grammatical construction, or even to ensure that a
query returns only sets of words which are truly in these grammatical
structures. 
To illustrate the problem, let's form some initial queries and see what they
return. Consider the following sentence:
\begin{exe}
	\ex The employee \textbf{that liked the manager} was promoted.
\end{exe}
How should we construct a query to catch phrases like this one? Maybe
\texttt{that {[}v*{]} {[}at*{]} {[}n*{]}}? Though this seems right, if you
check the concordance lines, it produces mostly results which are not relative
clauses:
\begin{exe}
	\ex Megan (ph) she stood on her head afterwards... OK. \textbf{That 's the way} you get a boy. Swear to God.
	\ex You see them two little ratty-ass dudes standing behind him? \textbf{That 's the problem}. Those two little ratty-ass dudes
\end{exe}
Most of these are common phrases starting with \emph{demonstrative} `that'.

Luckily, we can restrict \texttt{that} down to only its uses as a conjunction
using the tag \texttt{{[}cst*{]}}. This does not by any means solve the
problem, but it certainly makes the results more relevant. Most of the results
are, in fact, relative clauses.

However, relative clauses can also begin with other complementizers:
\texttt{that which who}. While these do not seem to have the same problem as
\texttt{that}, of being used as demonstratives, they are used to ask questions,
which may often begin \texttt{who|which {[}v*{]} {[}at*{]} {[}n*{]}}:
\begin{exe}
	\ex And when people commuted back with her, they said, Who 's the father?' She said, I will never tell.'
\end{exe}
It's not clear to me how to eliminate these results, or whether they constitute
a meaningful fraction of results.\footnote{I opt to include them in my count
since they seemed from the first several hundred concordance lines to be
relatively low percentages.} Relative clauses can also occur with or without an
article \texttt{{[}at*{]}}, with a pronoun \texttt{{[}p*{]}} instead of a noun,
with or without a complementizer (in the case of object relative clauses), or
even without a noun (in the case of subject relative clauses).

In light of this variation, it is hard to decide what queries to run, and
absolutely impossible to run a single query to catch all relative clauses of a
particular type. These clauses can be a variable number of words, which COCA is
very bad about supporting. I think to begin we'll have to try to catch the
patterns we expect to be simplest or most common.  To catch the simplest
relative clauses of each type, I'd like to run the following queries:
\begin{verbatim}
[cst*]|who|which [n*]|[p*] [v*]

[cst*]|who|which [v*] [n*]|[p*]
\end{verbatim}
COCA returns nothing for these queries. This is not because nothing matches,
but rather because \texttt{{[}cst*{]}{|}which{|}who} is not a legal slot, since
it combines determinate words with a tag.

Finally in the list of problems, the complementizer can be omitted for object 
relative clauses. This is the hardest thing to search for. The simplest should
be simply \texttt{{[}n*{]}|{[}p*{]} {[}v*{]}}, but clearly this will return
subjects and matrix verbs in sentences as well. Additionally, every slot in
this query occurs more than 30,000,000 times in COCA, which is the upper limit
for searchable queries. I attempted to add context (the preceding subject noun
or matrix verb), but that neither conceptually limited results to only relative
clauses nor lowered a slot below 30,000,000 occurrences.

% For instance, this is a concordance line from the corpus, showing a sentence
% from the film Clerks which contains two phrases caught by the query
% \texttt{who\ {[}V*{]}\ {[}N*{]}}:
% \begin{exe}
% 	\ex \dots{} Ass, Play with my Puss, Three on a Dildo, Girls Who Crave Cock, Girls Who Crave Cunt, Men Alone Two-The K.Y. Connection, Pink Pussy Lips \dots{}
% \end{exe}
% Obviously, neither `Who Crave Cock' or `Who Crave Cunt' are here relative
% clauses (though they usually would be).

% WHO MADE IT

In the end, this is what I searched:
\begin{verbatim}
	[cst*] [v*]
	which|who [v*]

	[cst*] [n*]|[p*] [v*]
	[cst*] [at*]|[appge*] [n*]|[p*] [v*]
	which|who [n*]|[p*] [v*]
	which|who [at*]|[appge*] [n*]|[p*] [v*]
\end{verbatim}
I only issued two queries for subject relative clauses, because all of the
other queries I could think of returned subsets of the above two, and I didn't
want to count data twice. The results of these queries seemed to be
overwhelmingly composed of relative clauses, as expected.

If my numbers are anything to go by, subject relative clauses are \textit{much}
more common than object relative clauses. My queries returned 3,069,821 subject
relative clauses and 1,604,332 object relative clauses. Certainly, this
distribution cannot explain our difficulty in processing subject relative
clauses.

The reason that this search was difficult is the same reason our results cannot
explain our processing difficulties: we rely on hierarchical grammar. The
difficulty with subject relative clauses is not processing an unusual construct,
but rather forming a correct hierarchical representation of the sentence. If the
corpus could parse these grammatical structures, it would be possible to easily
and correctly return this data.

	% * [CST*] [V*] [N*]|[P*]
	% * [CST*] [V*] [AT*]|[APPGE*] [N*]|[P*]
	% * which|who [V*] [N*]|[P*]
	% * which|who [V*] [AT*]|[APPGE*] [N*]|[P*]


\end{document}
