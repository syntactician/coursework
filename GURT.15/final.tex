\documentclass[man,12pt,natbib]{apa6}
\usepackage[colorlinks=false]{hyperref}
\usepackage{amssymb,amsmath,times}
\linespread{1.5}

\begin{document}

\title{GURT Final}
\shorttitle{GURT Final}
\author{Edward Hern\'{a}ndez}
\date{May 2015}
\affiliation{College of William \& Mary}
\maketitle

At GURT 2015, there were two papers about rap.  They were quite different, with
one focusing on story rounds and overlapping frames in songs with two artists,
and the other focusing on the behaviors and identities of migrant Finnish
rappers.  Despite being so different, they notably both dealt with stereotypes
and their role in identity construction.  Curiously, it seems that the papers
made incomplete arguments about the ways in which the rappers under study
interact with stereotypes and their impacts. 

\citet{Hodge15} discusses stereotypes as they are subverted by Kanye West and
Jay Z in their collaborative album, \emph{Watch the Throne}.  She asserts that
in the songs \emph{Ni**as in Paris}\footnote{This spelling of the title is used
	by Kanye West and Jay Z for the album version of the song, and will be used
throughout this paper.} \citep{West11a} and \emph{Otis} \citep{West11b}, the
artists intentionally call to mind and immediately subvert stereotypes, to
construct their own identities and to boast. For instance, in \emph{Ni**as in
Paris} , when Jay Z says ``This shit weird, we ain't even s'posed to be here,''
Hodge asserts that Jay Z and West are contrasting their position of fame and
opulence with a disenfranchised, impoverished position which may be
stereotypically expected of black men.  She further asserts that every time Jay
Z and West discuss their wealth and possessions, they are intentionally
separating themselves from the stereotype that black people are poor.
Additionally, as they subvert these negative stereotypes about their identities
they co-construct positive identities for themselves.  Hodge claims that this
behavior qualifies as boasting as it is often discussed in the literature on
hip hop \citep{Rickford00}. 

\citet{Westinen15} discusses identity construction largely in the scope of the
song \emph{Prinsille morsian} \citep{Palmunen14} by Prinssi Jusuf.  She asserts
that Jusuf embraces stereotypes about his country of origin, Ethiopia, to
construct his identity as a migrant and a Finnish outsider.  For example, when
he says `` Tulin Suomeen ilman mun kaverii koska Finnair ei kuljeta kamelii ''
(I came to Finland without my friend, because Finnair does not transport
camels), Westinen claims that he is invoking stereotypes about Ethiopia and its
people and applying them to himself to communicate that he is African, in order
to construct an identity as an authentic black migrant rapper.  Additionally,
the fonts in the video for the song are stereotypical ``African'' patterns, and
the video is modeled after scenes in the Eddie Murphy film Coming to America .
Westinen argues that this also serves to construct an identity for Prinssi
Jusuf in contrast with what is stereotypically Finnish, allowing him to remain
outside of the dominant culture, lending authenticity to his various identities
as a black, migrant rapper. 

These two approaches to interacting with stereotypes are quite different, and I
think that each view would be strengthened by input from the other.  There also
seem to be some issues with both positions. Hodge's claim that West and Jay Z's
are subverting stereotypes is potentially in conflict with the claim that they
are using the stereotypes to boast.  Similarly, it seems Westinen fails to
address the full impact of Prinssi Jusuf's stereotyping.  It seems he does more
than she gives him credit for. 

There is an explanatory gap in Hodge's analysis of Jay Z and Wests' interaction
with stereotypes.  There are two things that the artists might be doing, in
subverting stereotypes.  It might be the case that West and Jay Z are
subverting these stereotypes entirely, by showing that they have no basis, or
they may simply be showing that the stereotypes do not apply to them
specifically.  These two things are very different, but Hodges does not engage
with this issue.  If they are showing the stereotypes to be inaccurate for all
black people, then they are doing no work to boast or set themselves apart.  If
Hodges is right that in interacting with stereotypes the rappers are boasting,
then they must be establishing that they are better than other people.  To say
that you are rich is not boasting if everyone is rich, and saying that you're
black and rich is not boasting if all black people are rich.  It seems that to
use these stereotypes to boast, they have to reify them, or at the least leave
them largely unchallenged. 

Conversely, it seems that Prinssi Jusuf is doing more than just embrace and
utilize stereotypes in the way Westinen imagines.  He is also subverting them,
but not in the way that Hodge discusses Jay Z and Kanye West doing.  Jusuf
invokes the stereotypical association of Ethiopians and camels, but in a way
that should not be read any way but satyrical.  Clearly, his camel is not his
best friend, and there is no reason to expect that an airline would admit a
camel.  Moreover, in the video is is accompanied by many of his friends,
ensuring that this is read solely as a punchline.  It is not even truly
plausible that this would be a reasonable utterance from any Ethiopian migrant.
The absurd video, modeled after an equally absurd Eddie Murphy movie, shows
Prinssi Jusuf, who is in name and in scene a prince, getting off a plane with
an entourage (who are inexplicably carrying an umbrella inside a hangar) and
immediately wooing a blonde woman who happens to be on the tarmac, declaring to
``prinssi on paikalla ja osaa parantaa viidakkokuumeen '' (Prince is here and
can cure the jungle fever).  Nothing about this is reasonable, and this seems
to subvert not only the application of these stereotypes to Jusuf himself but
also the question the basis of the stereotypes. 

While both Hodges and Westinen offer a good analysis of the role of stereotype
in the construction of the identity of individual rappers, both fail to address
the implications that these strategies have on the stereotypes themselves.
While Kanye West and Jay Z do seem to successfully boast by putting themselves
in contrast with stereotypical images of poor black men, in doing so they
affirm the idea that black people are indeed poor.  In saying that they ``ain't
even s'posed to be here'' or are ``supposed to be locked up too,'' they concede
that other black people truly fit these stereotypes, and that they are
exceptions.  They accomplish their goal in boasting, but in doing so reify the
stereotypes they invoke.  Prinssi Jusuf, on the other hand, in invoking Finnish
stereotypes about Ethiopians, reproduces them in such a way that they cannot be
taken at face value.  One cannot take his utterances to be literally true, or
to be affirmations of the stereotypes.  He accomplishes his goal of
establishing his migrant identity solely by associating himself with these
stereotypes, doing nothing to affirm or reinforce them. 

\clearpage
\bibliography{extra.bib}

\end{document}
