\documentclass[man,12pt,natbib]{apa6}
\usepackage[colorlinks=false]{hyperref}
\usepackage{amssymb,amsmath,times}
\linespread{1.75}

\begin{document}

\title{Quiz}
\shorttitle{Quiz}
\author{Edward Hern\'{a}ndez}
\date{21 April 2016}
\affiliation{College of William \& Mary}
\maketitle

% The Norton editors, in referring to Emerson's ``proclamations on the
% individual's ability to break from institutions and traditional Christian
% beliefs,'' refer also to ``the difficulties of \emph{making sense} of a
% universe in which \emph{meaning} derives from individual creative insights
% rather than received authority'' [my emphasis] and to Hawthorne and
% Melville's works as grappling with ``the problematics of interpretation''
% (13).  One such problem:  things/``texts'' mean (or may mean) different
% things for different observers/``readers,'' may be interpreted differently.
% This is an obvious phenomenon, and is obviously illustrated by NH via the
% birthmark, the meteor ``A'' in the sky as well as the scarlet letter ``A''
% that Hester wears, and the ``A'' on his breast that Dimmesdale reveals in
% chapter XXIV.  More interesting, perhaps, is another fact that NH points out:
% certain ``readers'' share the ``same'' interpretation of X because they share
% the same (or a similar) attitude about X, and agree about X, because (as we
% might say today) they are ``on the same page'' regarding  X.  Thus those
% women jealous of Georgiana’s beauty regard the birthmark as rendering her
% ``hideous,'' and Dimmesdale’s ``friends'' deny that ``there was any mark
% whatever on his breast,'' and deny as well that his ``dying words
% acknowledged'' any guilt on his part, or any connection with Hester.  [We,
% the readers of Hawthorne’s text, The Scarlet Letter, know better because we
% know more about Dimmesdale, from NH, than his friends do.]
%
% The Norton editors also refer to Poe's ``articles on ``cryptography---the art
% and science of code-breaking'' (631).  All languages, of course, are
% ``codes'' in which certain letters put together mean/refer to something:
% thus c-a-t [or c-h-a-t, or g-a-t-o] refers to you-know-what.  However, in
% non-expository prose (that is, fiction) it can mean more than that furry
% animal.  Put oversimply, the author of a story often, or usually, employs the
% language-code to ``encode'' a meaning beyond that conveyed by the literal
% meaning of the words she uses, a ``hidden'' or ``secret'' additional meaning.
% The job of the reader/critic is to ``decode'' the story, to interpret its
% encrypted ``message''/meaning—which is, when you think about it, what the FBI
% recently wanted Apple to do with the terrorists' phones and encrypted text
% messages!  And, as with the examples mentioned above, we find that certain
% readers/critics agree, more or less, on a certain interpretation of a story,
% or of X in a story.  Why?  Because they share a particular ``critical or
% interpretive perspective'' [psychological, or Marxist, or feminist, or
% Christian, or whatever] that enables them to ``make sense'' of a text, or X,
% in a certain way that differs from the decodings of other critics employing
% other perspectives.  Usually, practitioners of a particular perspective will
% have already read certain seminal/canonical texts [from Freud, or Jung, or
% Adam Smith or Marx, or various feminists or the Bible, etc.], texts which
% they employ as their particular ``de-coding machine.''
%
% So:  as a quiz on your grasp of the above [which is not that cryptic or
% arcane], and as prep for the Final, answer the following (rather
% straightforward) questions [and again, no introductory or concluding
% paragraphs necessary, and use in-text, page numbers only, citations].

\section{The Black Cat}
% Discuss ``Black Cat'' from a feminist perspective AND from a
% psychological/Freudian perspective:
\nocite{Poe12b}

``The Black Cat'' offers a perfect case for the exploration of the problematics
of male narration of female experience. The narrator's wife is an incredibly
flat character. The reader is provided with almost no information about her.
We know the narrator marries her early, having found her ``disposition not
uncongenial'' with his own. She supplies him with birds, fish, a dog, rabbits,
a monkey, and a cat. This, despite his proclaimed desire for the ``the
unselfish . . . love of brute'' does little to make him happy. He is unappeased
by the effort which she puts into supplying him with these pets. Her
labor---whatever labor results in her acquiring the pets---is thus devalued,
unpaid. Despite this clear issue in the narrator's interaction with his wife,
he makes no mention of this financial tension (or of his apparent unemployment,
or his failure to father children), focusing instead on his feelings about his
wife, absent the causes of those feelings. His ability to live in such a way,
without working, is likely due to exploiting the labor of his wife, but he does
not feel the need to give us this information. To uncover the true details of
his life --- or any details at all about his wife --- we must engage in such
reasoning about his financial position, and even when we do, our picture is far
from complete.

A Freudian might explore the psyche of the narrator in terms of competing
drives.  On this reading, the narrator is possessed almost entirely of
\emph{Thanatos}, the death drive, to the exclusion of \emph{Eros}, or the drive
to life, propagation, sex, and creation. That Thanatos dominates Eros explains
his lack of children, as his sex drive is directly overcome, and his apparent
unemployment, since he has no drive to create. It is this death drive turned
outwards which results in the violence he visits on his pets and wife. He
wishes not only for the cessation of his life, but those of others. The drive
turned inwards motivates his self-destructive drinking, explains his desires
for the perverse, and sheds light on his eventual compulsion to reveal himself
(and, perhaps, be executed). Such an analysis is especially useful to explain
his desire to do the perverse. His description of the ``spirit of
P\textsc{erverseness}'' lines up perfectly with Freud's concept of the id: ``Of
this spirit philosophy takes no account. . .  perverseness is one of the
primitive impulses of the human heart'' (p. 696).  Philosophy and reason,
housed as they are in the superego, cannot account for the death drive
dominating his id, so he acts on it, and it exclusively.

\section{My Kinsman, Major Molineux}
% Discuss ``My Kinsman'' from a historical/political perspective AND from a
% psychological/Freudian perspective:
\nocite{Hawthorne12a,Lesser55}

``My Kinsman'' is easily read as a historical-political allegory. Robin, ``a
youth of barely eighteen years'' (p. 374), is then a representation of a young
America, and his kinsman Major Molineux, a ``large and majestic person'' (p.
385), is Great Britain (for whom he is literally a governor). It is here
significant that ``one of'' (p. 374) the protagonist's names is Robin. His
other name is not given, and there are two meaningfully different birds both
called ``robin.'' The American Robin, a bird native to New England, is named
after its distant relative, the European robin due to superficial similarities.
The European robin is a sedentary ``Old World flycatcher'' found throughout
Europe, while the American Robin is a migratory songbird native to North
America.  They are as distantly related as two birds can be, belonging not only
to different genii but different families. Their only similarity is their
coloration: red breasts and black or gray bodies (the complexions of the
painted man). The American Robin was so named on the basis of this similarity
alone, almost certainly without investigation of or regard for any more
substantive difference between the birds.  This relationship maps well onto the
political allegory: Britain, sedentary and Old World, lacks the energy and
beauty of the new America. When Robin is forced to choose between his old
kinsman and the New World to which he's migrating, his shout, his birdsong is
``the loudest there'' (p. 385).

Equally available is a Freudian reading in which Major Molineux is a father
figure whose authority Robin seeks to overcome. Robin's desire for his
kinsman's ``housekeeper'' (p. 378) is quite like an Oedipus complex, in that he
desires the object of his older kinsman's attraction. This necessarily leads to
ambivalent feelings toward the kinsman himself and, on Freud's view, castration
anxiety, the fear his kinsman will damage his genitals in retaliation. His
``flourishing'' and ``thrusting'' `` his cudgel'' (p. 380) are thus his phallic
attempts at self-assertion. On this reading, Robin is delighted to find his
kinsman tarred and feathered, as he is now practically impotent, powerless to
harm him or deny him sexual pleasure. The final scene, in which a gentleman
reveals that Robin may be dreaming, lends this reading considerable support. If
Robin \emph{was} dreaming, his witnessing his kinsman's destruction was a
manifestation of a subconscious desire of exactly the sort for which Freudian
readers argue, and the gentleman's assurance that he may ``rise'' (p. 386)
without (or in the absence of) his kinsman is an affirmation of the victory of
his threatened phallus.

\section{Paradise of Bachelors}
% Discuss ``Paradise of Bachelors'' from a Marxist perspective and from a
% psychological/Freudian perspective.
\nocite{Melville12}

On Marx's view, lawyers are ``unproductive'' insofar as they they do not
produce a good. They necessarily live off the surplus value produced by others.
They also wield and represent the power of the law, which is, by Marx's
thinking, a mechanism for keeping the oppressed classes in their subordinate
positions. This puts them in an extremely powerful position, and the Bachelors,
perhaps more than most of their peers have incredible social mobility, with
their close associations with M.P.s and other figures of bourgeois power. The
Templars who came before them, though ostensibly monastic soldiers (as lawyers
are ostensibly representatives), were an invasion force, formed, equipped, and
(soon) paid handsomely to extend Christian influence in the Outremer. They
received donations from other bourgeois church officials and were granted
immunity from any law below the word of the Pope. As the Templars were make to
look like servants of the people, protecting them on their pilgrimages abroad,
despite their corruption, lawyers have been made to appears as impartial or
even beneficent protectors (using the arm of the law, not of the lance) of the
people. This strategy legitimizes not only some members of the bourgeoisie (the
lawyers and judges themselves), but the very laws in which they function,
ultimately reinforcing the position of the oppressed classes. The difference
between the lawyers and common people is obvious not only from their lodging
and meals, but from the contrast Melville offers in the form of the Maids. The
maids, despite their unquestionably higher labor outputs (and almost certainly
higher value produced), are treated horribly in comparison, as are all laborers
who possess none of the means of production --- the only thing of value which
they possess under Capitalism is their labor potential, which must be sold to
subsist, forcing them into exploitation.

The Bachelors are orally fixated. Pages are spent recounting the exact details
of their shared feast: the ox-tail soup, roast beef, mutton, turkey,
chicken-pie, ``endless other savory things'' (p. 1513), and the innumerable
rounds of various alcohols. Talking (also crucially an activity of the mouth)
and dining with other bachelors is their primary source of pleasure. They do
not, as other men do, enjoy erotic genital pleasures with women, nor even the
thanatotic proxy-phallic pleasures of running other men through with lances, as
did their Templar forebears. Instead of genital exploration, they derive their
pleasure from oral exploration: new tastes, new topics. For Freud the oral
stage is crucially ended by weaning, by delaying of gratification. These
Bachelors seem never to have been weaned from immediate gratification. With
their wealth they can always produce new oral pleasures. As a result, they
suffer from the unrealistic optimism and malformed egos characteristic of the
orally fixated, wrapped up in their own insular, self-aggrandizing world,
refusing to see the trouble outside their enclave.

\clearpage
\bibliography{course,extra}


\end{document}
