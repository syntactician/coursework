\documentclass[man,12pt,natbib]{apa6}
\usepackage[colorlinks=false]{hyperref}
\usepackage{amssymb,amsmath,times}
\linespread{1.75}

\begin{document}

\title{Paper 2}
\shorttitle{Paper 2}
\author{Edward Hern\'{a}ndez}
\date{29 March 2016}
\affiliation{College of William \& Mary}
\maketitle

% Write on both questions: two (2) typed double spaced pages or so per
% question.  Do NOT write ``5-paragraph essays''---that is, you do NOT need
% ``introductory'' and ``concluding'' paragraphs for these responses.  [The
% prompt is your `introduction,' so just get to the discussion; `concluding'
% paragraphs are obviously unnecessary for such short discussions.]

\section{Fetters}

% In ``Economy,'' the first chapter of \emph{Walden, or Life in the Woods},
% Thoreau provides an insightful (and sometimes amusing) critique of
% ``civilization'' in mid-nineteenth-century America.  Early in this chapter he
% ``sometimes'' wonders that ``we can be so frivolous, I may almost say, as to
% attend to that gross but somewhat foreign [i.e., to his New England readers]
% form of servitude called Negro Slavery, there are so many keen and subtle
% masters that enslave both north and south.  It is hard to have a southern
% overseer; it is worse to have a northern one; but worst of all when you are
% the slave-driver of yourself.''  And throughout this chapter [see, esp.,
% 982-88] he acidly comments on the ``inherited encumbrances'' and ``golden or
% silver fetters'' that constrain us, on the various ways we have ``servitude''
% imposed on us, or, worse, impose upon ourselves.    

% Some of these constraints/``fetters'' that he discusses are
% religious/cultural/social, some are economic, and some are ``personal''
% (biological/ psychological in origin).  Discuss the relevance of some of
% these constraints/``fetters'' to two (2) of these stories:
% ``Paradise/Tartarus,'' ``May-pole,'' ``Birth-Mark,'' ``Ligeia'' [you can use
% short titles to save space].  How are these constraints depicted in the
% stories, and do they fully support HDT’s points, or is this ``support''
% qualified in some way by the author(s) of the stories.  NOTE: in all of these
% stories, characters may be ``enslaved'' by more than one ``master.''

In \emph{The Paradise of Bachelors and the Tartarus of Maids}, Melville
describes a group of lawyers who at first glance appear to be truly free ---
free of want, of pain, of obligation. They are, nevertheless, fettered, by
their wealth, their social relations, and their own outlooks.

The bachelors are all quite wealthy. While by many analyses, this makes them
free, Thoreau describes a class of people who ``have accumulated dross'' and
``thus have forged their own . . . fetters.'' These bachelors dine on a
sumptuous, meals of multiple courses ``with innumerable niceties,'' share an
incredible amount of liquor and snuff. This is pure luxury, of exactly the sort
Thoreau despises.  These excesses separate them from Thoreau's
``philosophers,'' who live who live ``[poor] in outward riches, [but] rich in
inward,'' and their forbears, the Templars, before ``the worm of luxury crawled
beneath their guard.''

Additionally, they are constrained by the social conventions of their
gatherings to avoid all thought (or at least talk) of pain or trouble. They do
not, as Thoreau describes ``entertain the true problems of life'' --- though
they have the prerequisite Food, Shelter, Clothing, and Fuel --- instead
dwelling on ``pleasant stories'' and ``[c]hoice experiences,'' the history of
the ancient Flemish, the British Museum, etc. 

The Maids, in contrast are fettered quite literally by overseers and by the
nature of their work. They ``serve'' the machinery, worked until they are
``sheet-white,'' with a ``consumptive pallor.''

By Thoreau's reckoning, the condition of the Bachelors is especially
condemnable insofar as they have no slave-drivers but themselves. The Maids,
while still fettered, enslaved much more literally, are not the slave-drivers
of themselves.  Thoreau would seem to say that the condition of the Bachelors
is worse.  Melville seems very much to disagree, as the narrator exclaims, in
summary: ``Oh! Paradise of Bachelors! and oh! Tartarus of Maids!'' There is no
indication that Melville considers the Maids' fetters in any way preferable.

In \emph{The May-Pole of Merry Mount}, Hawthorne describes a settlement where
sadness as ``high treason.'' The colonists of Merry Mount are socially
obligated to ignore problems, to keep up an air of gaiety. Edith and Edgar,
their hearts ``glow[ing] with real passion,'' are aware of the encroaching
problems of the world, and ``ha[ve] no more a home at Merry Mount.''

The elder revelers at Merry Mount ``kn[o]w that mirth was but the counterfeit
of happiness, yet followed the false shade wilfully.'' They allow themselves to
be fettered to the social practices of the settlement, despite their knowledge
that it is in error.

The Puritans, in stark contrast, are obligated --- both socially and
religiously --- to resist the gaiety of Merry Mount. They labor from daybreak
to dusk, with no pastimes but singing hymns and hearing sermons. These are
social obligations quite similar in nature to those in Merry Mount. Gaiety is
prohibited among the Puritans as sadness is prohibited around the May-Pole.

Endicott, ``the Puritan of Puritans'' cuts down the May-Pole, disbands Merry
Mount, and forces all the ``mirth-makers'' to join Puritan society. Thus, the
settlers of Merry Mount trade their social fetters for new, Puritan fetters,
which are not the fetters they apply themselves, but rather, ones that are
applied at a whipping post.

Despite what reads as a scathing critique of Puritan social mores, Hawthorne
here seems to hold the fetters of Merry Mount as worse than the Puritan social
fetters. The ending seems to imply that Edith and Edgar are, in fact, better
off. While it is ``their lot'' to tread a difficult path, they go ``heavenward,
supporting each other,'' never wasting ``regretful thought on the vanities of
Merry Mount.''

\nocite{Thoreau12, Melville12, Hawthorne12b}

\clearpage

\section{Women}

% Eleven years before \emph{Walden}, Lydia Maria Child wrote: ``I have said
% enough to show that I consider prevalent opinions and customs highly
% unfavorable to the moral and intellectual development of women. . . .  True
% culture, in them, as in men, consists in the full and free development of
% individual character, regulated by their own perceptions of what is true, and
% their own love of what is good'' [Letter XXXIV].  In this letter on ``women’s
% rights'' she discourses on ``the present position of women in society,'' on
% how (in various ways) and why (for various reasons) women are NOT free, and
% ``society is on a false foundation.''  Are her observations/complaints
% supported, or qualified, or refuted by the characters and what happens in two
% (2) of these stories: ``Black Cat,'' ``May-Pole,'' ``Paradise/Tartarus,''
% ``Rappaccini’s Daughter''?  Discuss.

In \emph{The Paradise of Bachelors and the Tartarus of Maids}, Melville
imagines a clear distinction in the ways in which men and women are treated.
The Bachelors are individuals, each having his own character, his own pursuits,
his own stories. They are free to develop their their individual characters,
both in life and in the narrative:
\begin{quote}
	A third was a great frequenter of the British Museum, and knew all about
	scores of wonderful antiquities, of Oriental manuscripts, and costly books
	without a duplicate.  A fourth had lately returned from a trip to Old
	Granada, and, of course, was full of Saracenic scenery. A fifth had a funny
	case in law to tell. A sixth was erudite in wines. A seventh had a strange
	character anecdote of the private life of the Iron Duke, never
	printed, and never before announced in any public or private company. An
	eighth had lately been amusing his evenings, now and then, with translating
	a comic poem of Pulci's. He quoted for us the more amusing passages. 
\end{quote}
Theses bachelors are free to fully develop their ``individual character[s].''

The Maids of the paper mill, on the other hand, have no such individual
characterization, and no such freedom. They are not allowed to develop, morally and intellectually,
as individuals, as are the Bachelors. They are ``blank-looking,'' and they do
not even speak. In fact, they appear to be, despite their small superficial differences, be largely interchangeable:
\begin{quote}
	   I looked upon the first girl's brow, and saw it was young and fair; I
	   looked upon the second girl's brow, and saw it was ruled and wrinkled.
	   Then, as I still looked, the two -- for some small variety to the
	   monotony -- changed places; and where had stood the young, fair brow,
	   now stood the ruled and wrinkled one. 
\end{quote}

This difference in the way that men's and women's characters are treated
mirrors the way their duties are treated and respected. The work of law, in
which the Bachelors engage, affords them luxury, comfort, and the further
development of their character through discourse with other characters. The
Maids, of course, are not afforded these luxuries. They work at their duties in
the paper-mill, and ``the human voice [is] banished.'' They cannot develop
their characters (in life or the narrative) as the Bachelors do. Their duties
preclude it.  Insofar as Child is correct that culture consists in the ``free
development of individual character,'' the Bachelors are granted it in
abundance, and the Maids are denied it altogether. Melville appears to be
expressing a similar stance: that women are denied this freedom.

In \emph{The Black Cat}, Poe describes a man whose nature is far from
masculine. He is ``noted for the docility and humanity of [his] disposition.''
This is regularly a trait praised in women, but serves to ``make [him] the jest
of [his] companions.'' Though this might mean that there is ``less separation''
between his character and pursuits than that of his wife (as Childs describes),
his disposition does nothing to support his marriage or eliminate ``separation
and discord.''

While Child argues that ``society is on a false foundation,'' and that there
need not be division of duties --- more women ought ``become rational
companions, partners in business and thought,'' Poe offers a pessimistic view
of masculinity. Instead of becoming ``finding [himself] ennobled'' by his more
feminine nature, Poe's protagonist ``gr[ows], day by day, more moody, more
irritable, more regardless of the feelings of others,'' eventually visiting
violence on his pets and his wife.

Child claims that women ``will not neglect'' their duties and that men will
share them as the roles of men and women grow closer, but Poe's protagonist
seems to abandon his duties altogether.  He does not father children. At no
point is there is no indication that he holds any sort of job. His only focuses
are liquor and cats --- not his duties, and certainly not his wife's.

Poe's story seems to contradict Child's argument that the sharing of duties and
natures between men and women will make men more gentle. It does appear,
however, to highlight an unfavorable positioning of women, as targets and
victims of violence.

\nocite{Child12b, Melville12, Poe12b}

\clearpage
\bibliography{course,extra}


\end{document}
