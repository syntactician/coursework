\documentclass[man,12pt,natbib]{apa6}
\usepackage[colorlinks=false]{hyperref}
\usepackage{amssymb,amsmath,times}
\linespread{1.5}

\begin{document}

\title{Paper 2}
\shorttitle{Paper 2}
\author{Edward Hern\'{a}ndez}
\date{29 March 2016}
\affiliation{College of William \& Mary}
\maketitle

\section{Fetters}

% In ``Economy,'' the first chapter of \emph{Walden, or Life in the Woods},
% Thoreau provides an insightful (and sometimes amusing) critique of
% ``civilization'' in mid-nineteenth-century America.  Early in this chapter he
% ``sometimes'' wonders that ``we can be so frivolous, I may almost say, as to
% attend to that gross but somewhat foreign [i.e., to his New England readers]
% form of servitude called Negro Slavery, there are so many keen and subtle
% masters that enslave both north and south.  It is hard to have a southern
% overseer; it is worse to have a northern one; but worst of all when you are
% the slave-driver of yourself.''  And throughout this chapter [see, esp.,
% 982-88] he acidly comments on the ``inherited encumbrances'' and ``golden or
% silver fetters'' that constrain us, on the various ways we have ``servitude''
% imposed on us, or, worse, impose upon ourselves.    

% Some of these constraints/``fetters'' that he discusses are
% religious/cultural/social, some are economic, and some are ``personal''
% (biological/ psychological in origin).  Discuss the relevance of some of
% these constraints/``fetters'' to two (2) of these stories:
% ``Paradise/Tartarus,'' ``May-pole,'' ``Birth-Mark,'' ``Ligeia'' [you can use
% short titles to save space].  How are these constraints depicted in the
% stories, and do they fully support HDT’s points, or is this ``support''
% qualified in some way by the author(s) of the stories.  NOTE: in all of these
% stories, characters may be ``enslaved'' by more than one ``master.''

\section{Women}

% Eleven years before \emph{Walden}, Lydia Maria Child wrote: ``I have said
% enough to show that I consider prevalent opinions and customs highly
% unfavorable to the moral and intellectual development of women. . . .  True
% culture, in them, as in men, consists in the full and free development of
% individual character, regulated by their own perceptions of what is true, and
% their own love of what is good'' [Letter XXXIV].  In this letter on ``women’s
% rights'' she discourses on ``the present position of women in society,'' on
% how (in various ways) and why (for various reasons) women are NOT free, and
% ``society is on a false foundation.''  Are her observations/complaints
% supported, or qualified, or refuted by the characters and what happens in two
% (2) of these stories: ``Black Cat,'' ``May-Pole,'' ``Paradise/Tartarus,''
% ``Rappaccini’s Daughter''?  Discuss.

\nocite{*}

\clearpage
\bibliography{course,extra}

\end{document}
