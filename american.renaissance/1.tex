\documentclass[man,12pt,natbib]{apa6}
\usepackage[colorlinks=false]{hyperref}
\usepackage{amssymb,amsmath,times}
\linespread{1.5}

\begin{document}

\title{Paper 1: The Machine in the Garden}
\shorttitle{Paper 1}
\author{Edward Hern\'{a}ndez}
\date{\today}
\affiliation{College of William \& Mary}
\maketitle

\section{\emph{The Machine in the Garden}}

% Explain the metaphors out of \citet{Marx64}, making mention of
% \citet{Freud62}.

\citet{Marx64} argues that in the literature of the ninteenth century there
exists a tension between a pastoral ideal of America --- tied to the mythos and
metaphor of the Garden of Eden --- and the reality of rapidly proliferating
technology (the ``Machine''). In many cases, this tension is expressed by a
machine quite literally present in the text. For instance, in Walden, the
locomotive interlopes in the garden of Walden Pond \citep[ch.~4]{Thoreau12},
and in Adventures of Huckleberry Finn, the steamboat destroys the raft and the
idyllic beauty of the river \citep[ch.~16]{Twain85}.  I argue that this tension
is also expressed metaphorically, and that civilization --- more specifically
the economy it necessitates --- itself constitutes a machine which disrupts the
garden.


\section{\emph{Economy}}

% Economy should probably be the first work out of the gate. It's explicit.
% Men labor under a mistake about labor ? ? ?
% They do too much, they work as machines, when really, they can rely on the
% garden for sustenence (nature provides much that we need).
% In the end, the garden wins, since it frees Thoreau from having to labor as a machine.
% Also, he solves types 1, 2, and 3 of alienation.

In the opening chapter of \emph{Walden}, \citet{Thoreau12} describes his
motivation for living on Walden Pond. He describes ``the mass of men lead[ing]
lives of quiet desperation'' (p.~984), ``labor[ing] under a mistake . . .
laying up treasures'' (p.~982).
Thoreau sets himself up in contrast with these desperate workers, living off
the land in incredible frugality, laying up no physical treasures whatsoever.

I argue that the laborer with whom Thoreau is concerned is desperate insofar as
her labor is exploited and she is alienated from her human nature
(\emph{Gatungwesen}). \citet{Marx44} argues that in an economically stratified
society, those who possess the majority of the wealth\footnote{Rather, his body
of theory deals most explicitly with the possession of the means of
production.} dictate the actions of less prosperous laborers, violating those
workers' autonomy in order to extract surplus value from their labor. In this
relation, workers' autonomy is lessened, they are alienated from their nature
as humans; in the extreme, workers are reduced to mere cogs in a Machine
wrought to produce profit. In Thoreau's time, this Machine was, on the heels of
the Industrial Revolution, reaching fever pitch, prompting him to escape the
alienation he observed in his peers.

Thoreau, in order to retain \emph{his} nature, returns to nature. He moves to
Walden Pond, to live in nature rather than lead a life ``of quiet desperation''
or ``labor under a mistake.'' He lives without the ``treasures'' his peers seek
to lay up. He returns to a simpler existence, closer to nature, living directly
from the land. This fits well into the metaphor of the Garden.

Thoreau is able to sustain his life in his new Garden, living outside of the
Machine. He is able to stave off the hold which the Machine has on him. This,
on my view, constitutes a resolution of the conflict in favor of the Garden.
However, the resolution is not permanent. Thoreau eventually has to return to
civilization, to the Machine. In the end, his victory is temporary, and the
conflict continues beyond the end of his account.


\section{\emph{Rip Van Winkle}}

In \emph{Rip Van Winkle}, \citet{Irving12} describes a man struggling to attend
to his business and maintain his own farm.  He is not averse to work, being
entirely willing to ``make playthings'' for the children of the village,
``[husk] Indian corn,'' ``[build] stone fences,'' run errands, hunt, or fish
(p.~31). However, he ``[finds] it impossible'' to maintain his own farm because
he is averse to specifically ``to all kinds of \emph{profitable} [emphasis
added] labor'' (p.~31).  Instead of economically viable work, Rip prefers to
wander, fish and fowl out in nature, but he is married to a woman who has no
patience for his unprofitable pursuits.

Rip prefers to live in nature, habitually wandering and hunting rather than
working. For him, the Catskills are the Garden. The Machine which disrupts this
garden is not a literal piece of technology, but rather the machinery of
Economy. Farm work stands in contrast with the freedom of hunting and fowling.
Working for profit contrasts with working for pleasure. The Machine of the
economy pushes him to remain on his farm, out of the Garden, and work without
pleasure.
% Despite his penchant to ``starve on a penny than work for a pound'' (p.~32),
% he

I argue that Van Winkle resists his work on the farm because it is profitable
rather than fulfilling.  As \citet{Marx44} describes, being forced to labor ---
as part of a mechanism --- alienates the worker from her \emph{Gatungwesen}
(``species-essence'' or human nature).  For Marx, the worker is oppressed and
thus estranged from her nature by means of her productive forces being directed
and controlled by the bourgeoisie to extract surplus value. In Rip's case, his
productive effort is doubly thus controlled --- by the economy and by Dame Van
Winkle.  Dame Van Winkle embodies the priorities of the economy to which Rip
stands in contrast.\footnote{Dame Van Winkle is female, which would appear to
fly in the face of the typical masculine expression of the Machine.  However,
she is notably aggressive rather than submissive, embodying masculine values as
she reproduces the value system of the Machine.} She attempts to keep Rip
focused on his farm-work, which elevates this conflict to a visible,
interpersonal level.

Eventually, Rip, to escape the nagging of this manifestation of the Machine,
retreats into the Garden of the Catskills, and falls asleep. When Rip awakes,
Dame Van Winkle is dead, and his labor is no longer controlled. He is free of
``the yoke of matrimony'' (and the economic influence of Britain), and free to
live, unproductive and unrestrained, in the garden for the rest of his days.
The tension is resolved with Rip and the Garden triumphing unequivocally.

% His work on the farm estranges his \emph{Gattungswesen} or whatever, and he
% resists being a machine, so he goes off into the womb-garden and the garden
% wins, because he doesn't have to do it any more.

% Rip fights against type 4 by working for others but not himself. also 2 --- he
% gets satisfaction from the stuff he does.

% Also Dame Van Winkle might be a machine --- just saying. Her henpecking
% \emph{is} masculine assertiveness.


\section{\emph{The Quadroon's Story}}

\citet{Stowe12}, in \emph{The Quadroon's Story}, Chapter 34 of her novel
\emph{Uncle Tom's Cabin} \citep{Stowe52}, gives an account of a slave named
Cassy, whose childhood was idyllic. She was  ``brought up in luxury,'' playing
``under the orange trees'' in ``a garden opening from the saloon windows,''
\citep[p.~895]{Stowe12}. This living situation is somewhat insulated from the
racial injustices of the nineteenth century, and Cassy is treated well despite
being the daughter of an enslaved woman. When her father dies, that insulation
is suddenly and brutally removed, and Cassy is put on the market. She is
repeatedly bought and sold, eventually ending up in a brothel.

Cassy's childhood occurs in a bounded area of New Orleans, where she can live
in beauty, surrounded by a literal garden. Like in the Garden of Eden, she
lives here in ignorance: she is unaware of the evil of slavery because it is
not visible within the Garden. When she is forcibly evicted from the Garden,
she gains this knowledge and enters the Machine of the economy --- unlike Rip
Van Winkle, not as a worker --- as a product. Not only is her labor exploited
as a commodity, but her body and her children are as well. Her labor and her
body are not her own, and she has next to no autonomy, which under
\citet{Marx44} alienates her completely from her \emph{Gatungwesen}.

Cassy and her children are abused and exploited in the Machine. Cassy, upon 
discovering that her children are being abused, loses hope. When she later 
has a child at the brothel, she poisons it with laudanum, so that that child
would not have to experience the life of bondage and misery that it would
have been born into. For Cassy, this is a violation of her belief in God
and against her nature as a human and as a mother. This further alienates 
her from her nature.

In this case, the Machine triumphs. It subjugates Cassy completely. She is
permanently cut off from the Garden of her youth, from her youthful ignorance
of her and her mother's legal status as property, from her very human nature.
There is little she is able to do to resist this Machine. The best she can do
is prevent a new body from being assimilated into the machine, dehumanized
and abused as she has been. That action concedes not only that the Garden was
unattainable for her, but that no Garden could exist for her child. It would
have experienced a world of pure Machine, a fate Cassy could not accept.


\section{\emph{The Quadroons}}

\citet{Child12} tells the story of Xarifa, who much like Cassy, is raised in a
Garden, somewhat removed from the realities of slavery, ``like a flower deep
hid in a rocky cleft'' (p.~188). Like Cassy, this Garden keeps her largely
ignorant of her position in the system of racial injustice outside its bounds.
When her parents die, she is enslaved, her lover is killed, and she is driven
to suicide.

Xarifa begins her life in a Garden, in the best sort of life a non-White person
could have at the time, not being herself enslaved. She is treated well until
her family situation falls apart and her parents die. When she is enslaved,
she, like Cassy, enters the economy as both laborer and inherent commodity. She
does not immediately lose hope --- or her nature.  She attempts to escape, out
of this Machine, back to a Garden. Her lover comes to take her away, but he is
killed. After her lover is killed (and she is subsequently raped by her owner),
her nature is utterly alienated. She no longer retains bodily autonomy, or
mental autonomy, having become ``a raving lunatic'' (p.~190).

In this case, as in Cassy's, the conflict is resolved in a total victory of the
Machine over the Garden. Xarifa simply has no recourse against the power of the
Machine, no way to reclaim the Garden of her youth.

% But then she is subjected to the Machine of the free market, reduced to a body to do labor (sexually and otherwise).

% This one is the biggest stretch, and should go last, as it will rely on my argumentation in all the above cases.

\section{Conclusion}

% White dudes can even fuck up the ingenious inexorable dehumanizing processes of
% their own making. Whoa.

In all four of these cases, the central character has rejected the Machine,
attempting to reclaim the Garden. Thoreau found some measure of success 
temporarily, but his conflict remained largely unresolved. Rip Van Winkle 
won a total victory for the Garden. Cassy and Xarifa lost their Gardens, 
and in doing so, lost their connection to their natures.


% \nocite{*}
\clearpage
\bibliography{course,extra}

\end{document}
